%%                                    %%
%% Generated from MathBook XML source %%
%%    on 2015-01-15T14:24:54Z    %%
%%                                    %%
%%   http://mathbook.pugetsound.edu   %%
%%                                    %%
\documentclass[10pt,]{book}
%% Load geometry package to allow page margin adjustments
\usepackage{geometry}
\geometry{letterpaper,total={5.0in,9.0in}}
%% Custom Preamble Entries, early (use latex.preamble.early)
%% Inline math delimiters, \(, \), made robust with next package
\usepackage{fixltx2e}
%% Page Layout Adjustments (latex.geometry)
%% For unicode character support, use the "xelatex" executable
%% If never using xelatex, the next three lines can be removed
\usepackage{ifxetex}
\ifxetex\usepackage{xltxtra}\fi
%% Symbols, align environment, bracket-matrix
\usepackage{amsmath}
\usepackage{amssymb}
%% extpfeil package for certain extensible arrows,
%% as also provided by MathJax extension of the same name
\usepackage{extpfeil}
%% allow more columns to a matrix
%% can make this even bigger by overiding with  latex.preamble.late  processing option
\setcounter{MaxMatrixCols}{30}
%% XML, MathJax Conflict Macros
%% Two nonstandard macros that MathJax supports automatically
%% so we always define them in order to allow their use and
%% maintain source level compatibility
%% This avoids using two XML entities in source mathematics
\newcommand{\lt}{<}
\newcommand{\gt}{>}
%% Semantic Macros
%% To preserve meaning in a LaTeX file
%% Only defined here if required in this document
%% Used for inline definitions of terms
\newcommand{\terminology}[1]{\textbf{#1}}
%% Subdivision Numbering, Chapters, Sections, Subsections, etc
%% Subdivision numbers may be turned off at some level ("depth")
%% A section *always* has depth 1, contrary to us counting from the document root
%% The latex default is 3.  If a larger number is present here, then
%% removing this command may make some cross-references ambiguous
%% The precursor variable $numbering-maxlevel is checked for consistency in the common XSL file
\setcounter{secnumdepth}{3}
%% Environments with amsthm package
%% Theorem-like enviroments in "plain" style, with or without proof
\usepackage{amsthm}
\theoremstyle{plain}
%% Numbering for Theorems, Conjectures, Examples, Figures, etc
%% Controlled by  numbering.theorems.level  processing parameter
%% Always need a theorem environment to set base numbering scheme
%% even if document has no theorems (but has other environments)
\newtheorem{theorem}{Theorem}[section]
\renewcommand*{\proofname}{Proof}%% Only variants actually used in document appear here
%% Numbering: all theorem-like numbered consecutively
%% i.e. Corollary 4.3 follows Theorem 4.2
\newtheorem{proposition}[theorem]{Proposition}
%% Definition-like environments, normal text
%% Numbering for definition, examples is in sync with theorems, etc
%% also for free-form exercises, not in exercise sections
\theoremstyle{definition}
\newtheorem{definition}[theorem]{Definition}
%% Raster graphics inclusion, wrapped figures in paragraphs
\usepackage{graphicx}
%% Colors for Sage boxes and author tools (red hilites)
\usepackage[usenames,dvipsnames,svgnames,table]{xcolor}
%% hyperref driver does not need to be specified
\usepackage{hyperref}
%% Hyperlinking active in PDFs, all links solid and blue
\hypersetup{colorlinks=true,linkcolor=blue,citecolor=blue,filecolor=blue,urlcolor=blue}
\hypersetup{pdftitle={Part III Homotopy Theory 2014}}
%% If you manually remove hyperref, leave in this next command
\providecommand\phantomsection{}
%% Custom Preamble Entries, late (use latex.preamble.late)
\usepackage[all]{xy}
%% Convenience macros
\DeclareMathOperator{\GL}{GL}
\DeclareMathOperator{\Map}{Map}
\DeclareMathOperator{\Hom}{Hom}
\DeclareMathOperator{\Tor}{Tor}
\DeclareMathOperator{\Ext}{Ext}
\DeclareMathOperator{\im}{im}
\DeclareMathOperator{\id}{id}
\DeclareMathOperator{\coker}{coker}
\DeclareMathOperator{\CC}{\mathbf{C}}
\DeclareMathOperator{\QQ}{\mathbf{Q}}
\DeclareMathOperator{\RR}{\mathbf{R}}
\DeclareMathOperator{\ZZ}{\mathbf{Z}}
\DeclareMathOperator{\RP}{\mathbf{RP}}
\DeclareMathOperator{\dd}{\partial}
%% Title page information for book
\title{Part III Homotopy Theory 2014}
\author{}
\date{}
\begin{document}
\frontmatter
%% half-title
\thispagestyle{empty}
\vspace*{\stretch{1}}
\begin{center}
{\Huge Part III Homotopy Theory 2014}
\end{center}\par
\vspace*{\stretch{2}}
\clearpage
\thispagestyle{empty}
\clearpage
\maketitle
\clearpage
\thispagestyle{empty}
\vspace*{\stretch{2}}
\vspace*{\stretch{1}}
\clearpage
\setcounter{tocdepth}{1}
\renewcommand*\contentsname{Contents}
\tableofcontents
\mainmatter
\typeout{************************************************}
\typeout{Chapter 1 Homotopy Theory}
\typeout{************************************************}
\chapter[Homotopy Theory]{Homotopy Theory}\label{chap-homotopy}
\typeout{************************************************}
\typeout{Section 1.1 Introduction}
\typeout{************************************************}
\section[Introduction]{Introduction}\label{sec-introduction}
These are lecture notes for the 2014 Part III Homotopy Theory course taught by Dr. Oscar Randal-Williams, these notes are part of \href{https://alexjbest.github.io/mjolnir/}{MJOLNIR}.%
\par
The recommended books are: 
          \begin{itemize}
\item{} - ??\end{itemize}

        %
\par

          Generated: January 15, 2015, 14:24:54 (Z)
        %
\typeout{************************************************}
\typeout{Section 1.2 Homotopy groups}
\typeout{************************************************}
\section[Homotopy groups]{Homotopy groups}\label{sec-homotopy}
The goal of the course is to introduce tools to compute higher homotopy groups, though it will turn out that the tools developed are more interesting than the groups themselves.%
\par
Let \((X,x_0)\) be a based space and write \(I = [0,1]\), \(I^n\) for the \(n\)-cube.
          We then have \(\dd I^n = \{(x_1,\ldots,x_n)\in I^n : \text{some } x_i \in\{0,1\}\}\).
        %
\begin{definition}[Homotopy groups]\label{definition-1}
\(\pi_n(X,x_0)\) is the set of homotopy classes of maps \(f\colon I^n\to X\) such that \(f(\dd I^n) = \{x_0\}\) and homotopies are taken through such maps.\end{definition}
\par
 If \(A\) is a subspace we write \((X,A)\).
          Then a map \(f\colon (X,A)\to (Y,B)\) is a map \(f\colon X\to Y\) such that \(f(A)\subset B\).
          A homotopy of maps of pairs is a homotopy \(H\colon X\times I \to Y\) such that \(H(A\times I) \subset B\).
          \newline{}
          Thus \(\pi_n(X,x_0)\) is the set of homotopy classes of maps \(f\colon (I^n,\dd I^n) \to (X,\{x_0\})\).
        %
\par
For \(n= 1\) this is the usual fundamental ggroup of \((X,x_0)\), for \(n= 0\) let \(I^n = \{*\}\), \(\dd I^n = \emptyset\) and so \(\pi_0(X,x_0)\) is the set of path components of \(X\).%
\par
We may define a composition law on \(\pi_n(X,x_0)\) by \[f\cdot g(x_1,\ldots,x_n)= \begin{cases}f(2x_1,\ldots,x_n) & 0\le x_1\le \frac12\\g(2x-1, \ldots,x_n)&\frac12\le x_1\le 1.\end{cases}\]
          Just as for \(\pi_1(X,x_0)\) this composition law makes \(\pi_n(X,x_0)\) into a group for \(n\ge 1\).
          For \(n\ge 2\) this group is abelian.
        %
\par
If \(u\) is a path from \(x_0\) to \(x_1\) in \(X\) we obtain a map \(u_\#\colon \pi_n(X,x_1) \to \pi_n(X,x_0)\) given by using the path to bridge the outer boundary of the cube to an inner cube of half the size.
          This map satisfies
          \begin{enumerate}
\item{}If \(u\simeq u'\) as paths then \(u_\# = u'_\#\).\item{}\((C_{x_0})_\# = \mathrm{id}\).\item{}\(u_\#\) is a homomorphism.\item{}\(v_\#(u_\#(f)) = (v\cdot u)_\# (f)\).\end{enumerate}

        %
\par
We have the following consequences of this definition.
          \begin{enumerate}
\item{}If \(x_0,x_1\) are in the smae path component then \(\pi_n(X,x_0) \cong \pi_n(X,x_1)\), but not canonically so.\item{}Taking \(x_1 = x_0\) in the above we get a left action of \(\pi_1(X,x_0)\) on \(\pi_n(X,x_0)\) (i.e. for \(n\ge 2\) \(pi_n(X,x_0)\) is a \(\ZZ \pi_1(X,x_0)\)-module.\item{}\(\pi_n(-)\) is a functor \[\text{based spaces} \to \begin{cases} \text{groups}&n=1,\\\text{abelian groups}& n\ge 2.\end{cases}\] and if \(f\colon (X,x_0) \to (Y,y_0)\) then we have \(f_*([\gamma]\cdot x) = f_*([\gamma])\cdot f_*(x)\).\item{}If \(f\colon (X,x_0)\to (Y,y_0)\) is  a based homotopy equivalence then \(f_*\) is  an isomorphism on all \(\pi_n\). (This is still true if \(f\) is just a naked homotopy equivalence.)\end{enumerate}

        %
\typeout{************************************************}
\typeout{Section 1.3 Relative homotopy groups}
\typeout{************************************************}
\section[Relative homotopy groups]{Relative homotopy groups}\label{sec-rel-homotopy}
Relative homotopy groups are defined for a space \(X\) a subspace \(A\subset X\) and a point \(x_0\in A\).%
\par
Let \(\sqcap^{n-1} \subset \dd I^n \subset I^n\) be the closure of the complement of \(I^{n-1}\times\{0\}\).%
\begin{definition}[Relative homotopy groups]\label{definition-2}
The \terminology{relative homotopy group}\(\pi_n(X,A,x_0)\) is the set of homotopy classes of maps \(f\colon I^n\to X\) such that
            \begin{enumerate}
\item{}\(f(\dd I^n)\subset A\).\item{}\(f(\sqcap^{n-1}) = \{x_0\}\).\end{enumerate}

            For \(n\ge 2\) the usual formula defines a composition law on \(\pi_n(X,A,x_0)\).
            For \(n\ge 3\) the usual argument shows \(\pi_n(X,A,x_0)\) is an abelian group.
          \end{definition}
\par
Observe that as \((I^n/\sqcap^{n-1}, \dd I^n/\sqcap^{n-1},\sqcap^{n-1}/\sqcap^{n-1})\cong (D^n,\dd D^n, *)\) we can define \(\pi_n(X,A,x_0) = \{\text{homotopy classes of maps }f\colon (D^n,\dd D^n, *)\to (X,A,x_0)\}\).%
\par
If we have \(f\colon (X,A, x_0) \to (Y,B,y_0)\) then \(f\) induces a map \(f_*\colon \pi_n(X,A,x_0)\to \pi_n(Y,B,y_0)\) and if \(f\simeq g\) through such maps then \(f_* = g_*\).%
\begin{proposition}[Compression criterion]\label{proposition-1}
A map \(f\colon (D^n, \dd D^n, *)\to (X,A, x_0)\) is trivial in \(\pi_n(X,A,x_0)\iff f\) is homotopic relative to \(\dd D^n\) to a map into \(A\).\end{proposition}
\begin{proof}
(\(\Rightarrow\)) Let \(f\) be homotopic relative to \(\dd D^n\) to a map into \(A\), call the homotopy \(H\colon D^n\times I \to X\).
            So \(H(D^n\times \{1\}\subset A\) and \(H(\dd D^n \times I\subset A\), projection from the point \((0,\ldots,0,-1)\) then gives a deformation retract of \(D^n \times I\) to \(B = (\dd D^n \times I)\cup (D^n\times\{0\})\) so it gives a homotopy from \(H\) to \(H'\) relative to \(B\) such that \(H'\) lands in \(A\).
            Restriction this homotopy to \(D^n\times \{0\}\) gives a homotopy from \(f\) to a \(f'\colon D^n \to A\) relative to \(\dd D^n\).
            \newline{}
            (\(\Leftarrow\)) If \(f\simeq g\colon( D^n,\dd D^n, *)\to (X,A,x_0) \) and \(g(D^n)\subset A\) then \([f] = [g]\), now consider \(g\colon (D^n , *) \to (A,x_0)\).
            The deformation of \(D^n\) to \(*\) gives a based homotopy from \(g\) to \(C_{x_0}\), i.e. letting \(r\colon D^n \times I \to D^n\) be the linear deformation retract then \(g\circ r \colon D^n \times I \to A\) is a based homotopy which at \(t = 1\) is \(C_{x_0}\).
          %
\end{proof}
%
\backmatter
%
\typeout{************************************************}
\typeout{Section 1 Notation}
\typeout{************************************************}
\section[Notation]{Notation}\label{sec-notation}
\begin{longtable}[l]{llr}
\textbf{Symbol}&\textbf{Description}&\textbf{Page}\\[1em]
\endfirsthead
\textbf{Symbol}&\textbf{Description}&\textbf{Page}\\[1em]
\endhead
\multicolumn{3}{r}{(Continued on next page)}\\
\endfoot
\endlastfoot
\end{longtable}
\end{document}