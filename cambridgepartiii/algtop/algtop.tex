%%                                    %%
%% Generated from MathBook XML source %%
%%    on 2014-11-19T01:27:43Z    %%
%%                                    %%
%%   http://mathbook.pugetsound.edu   %%
%%                                    %%
\documentclass[10pt,]{book}
%% Load geometry package to allow page margin adjustments
\usepackage{geometry}
\geometry{letterpaper,total={5.0in,9.0in}}
%% Custom Preamble Entries, early (use latex.preamble.early)
%% Inline math delimiters, \(, \), made robust with next package
\usepackage{fixltx2e}
%% Page Layout Adjustments (latex.geometry)
%% For unicode character support, use the "xelatex" executable
%% If never using xelatex, the next three lines can be removed
\usepackage{ifxetex}
\ifxetex\usepackage{xltxtra}\fi
%% Symbols, align environment, bracket-matrix
\usepackage{amsmath}
\usepackage{amssymb}
%% extpfeil package for certain extensible arrows,
%% as also provided by MathJax extension of the same name
\usepackage{extpfeil}
%% allow more columns to a matrix
%% can make this even bigger by overiding with  latex.preamble.late  processing option
\setcounter{MaxMatrixCols}{30}
%% XML, MathJax Conflict Macros
%% Two nonstandard macros that MathJax supports automatically
%% so we always define them in order to allow their use and
%% maintain source level compatibility
%% This avoids using two XML entities in source mathematics
\newcommand{\lt}{<}
\newcommand{\gt}{>}
%% Semantic Macros
%% To preserve meaning in a LaTeX file
%% Only defined here if required in this document
%% Used for inline definitions of terms
\newcommand{\terminology}[1]{\textbf{#1}}
%% Subdivision Numbering, Chapters, Sections, Subsections, etc
%% Subdivision numbers may be turned off at some level ("depth")
%% A section *always* has depth 1, contrary to us counting from the document root
%% The latex default is 3.  If a larger number is present here, then
%% removing this command may make some cross-references ambiguous
%% The precursor variable $numbering-maxlevel is checked for consistency in the common XSL file
\setcounter{secnumdepth}{3}
%% Environments with amsthm package
%% Theorem-like enviroments in "plain" style, with or without proof
\usepackage{amsthm}
\theoremstyle{plain}
%% Numbering for Theorems, Conjectures, Examples, Figures, etc
%% Controlled by  numbering.theorems.level  processing parameter
%% Always need a theorem environment to set base numbering scheme
%% even if document has no theorems (but has other environments)
\newtheorem{theorem}{Theorem}[section]
\renewcommand*{\proofname}{Proof}%% Only variants actually used in document appear here
%% Numbering: all theorem-like numbered consecutively
%% i.e. Corollary 4.3 follows Theorem 4.2
\newtheorem{corollary}[theorem]{Corollary}
\newtheorem{lemma}[theorem]{Lemma}
\newtheorem{proposition}[theorem]{Proposition}
%% Definition-like environments, normal text
%% Numbering for definition, examples is in sync with theorems, etc
%% also for free-form exercises, not in exercise sections
\theoremstyle{definition}
\newtheorem{definition}[theorem]{Definition}
\newtheorem{example}[theorem]{Example}
\newtheorem{exercise}[theorem]{Exercise}
%% Equation Numbering
%% Controlled by  numbering.equations.level  processing parameter
\numberwithin{equation}{section}
%% Figures, Tables, Floats
%% The [H]ere option of the float package fixes floats in-place,
%% in deference to web usage, where floats are totally irrelevant
%% We redefine the figure and table environments, if used
%%   1) New mbxfigure and/or mbxtable environments are defined with float package
%%   2) Standard LaTeX environments redefined to use new environments
%%   3) Standard LaTeX environments redefined to step theorem counter
%%   4) Counter for new enviroments is set to the theorem counter before caption
%% You can remove all this figure/table setup, to restore standard LaTeX behavior
%% HOWEVER, numbering of figures/tables AND theorems/examples/remarks, etc
%% WILL ALL de-synchronize with the numbering in the HTML version
%% You can remove the [H] argument of the \newfloat command, to allow flotation and 
%% preserve numbering, BUT the numbering may then appear "out-of-order"
\usepackage{float}
\newfloat{mbxfigure}{H}{lof}[section]
\floatname{mbxfigure}{Figure}
\renewenvironment{figure}%
{\begin{mbxfigure}\setcounter{mbxfigure}{\value{theorem}}\stepcounter{theorem}}%
{\end{mbxfigure}}
\newfloat{mbxtable}{H}{lot}[section]
\floatname{mbxtable}{Table}
\renewenvironment{table}%
{\begin{mbxtable}\setcounter{mbxtable}{\value{theorem}}\stepcounter{theorem}}%
{\end{mbxtable}}
%% Raster graphics inclusion, wrapped figures in paragraphs
\usepackage{graphicx}
%% Colors for Sage boxes and author tools (red hilites)
\usepackage[usenames,dvipsnames,svgnames,table]{xcolor}
%% Package for tables spanning several pages
\usepackage{longtable}
%% hyperref driver does not need to be specified
\usepackage{hyperref}
%% Hyperlinking active in PDFs, all links solid and blue
\hypersetup{colorlinks=true,linkcolor=blue,citecolor=blue,filecolor=blue,urlcolor=blue}
\hypersetup{pdftitle={Part III Algebraic Topology 2014}}
%% If you manually remove hyperref, leave in this next command
\providecommand\phantomsection{}
%% Custom Preamble Entries, late (use latex.preamble.late)
\usepackage[all]{xy}
%% Convenience macros
\DeclareMathOperator{\Map}{Map}
\DeclareMathOperator{\Hom}{Hom}
\DeclareMathOperator{\Tor}{Tor}
\DeclareMathOperator{\Ext}{Ext}
\DeclareMathOperator{\im}{im}
\DeclareMathOperator{\id}{id}
\DeclareMathOperator{\coker}{coker}
\DeclareMathOperator{\CC}{\mathbf{C}}
\DeclareMathOperator{\QQ}{\mathbf{Q}}
\DeclareMathOperator{\RR}{\mathbf{R}}
\DeclareMathOperator{\ZZ}{\mathbf{Z}}
\DeclareMathOperator{\RP}{\mathbf{RP}}
\DeclareMathOperator{\C}{C_*}
\DeclareMathOperator{\U}{\mathcal{U}}
%% Title page information for book
\title{Part III Algebraic Topology 2014}
\author{}
\date{}
\begin{document}
\frontmatter
%% half-title
\thispagestyle{empty}
\vspace*{\stretch{1}}
\begin{center}
{\Huge Part III Algebraic Topology 2014}
\end{center}\par
\vspace*{\stretch{2}}
\clearpage
\thispagestyle{empty}
\clearpage
\maketitle
\clearpage
\thispagestyle{empty}
\vspace*{\stretch{2}}
\vspace*{\stretch{1}}
\clearpage
\setcounter{tocdepth}{1}
\renewcommand*\contentsname{Contents}
\tableofcontents
\mainmatter
\typeout{************************************************}
\typeout{Chapter 1 Homology}
\typeout{************************************************}
\chapter[Homology]{Homology}\label{chapter-1}
\typeout{************************************************}
\typeout{Section 1.1 Introduction}
\typeout{************************************************}
\section[Introduction]{Introduction}\label{section-1}

          These are lecture notes for the 2014 Part III Algebraic Topology course taught by Dr. Jacob Rasmussen.
        %
\par

          The recommended books are:
          \begin{itemize}
\item{}\href{http://www.math.cornell.edu/~hatcher/AT/ATpage.html}{Algebraic Topology} - Allen Hatcher,\item{}Homology Theory - James W. Vick,\item{}Differential Forms in Algebraic Topology - Raoul Bott and Loring W. Tu.\end{itemize}

        %
\par

          Generated: November 19, 2014, 01:27:43 (Z)
        %
\typeout{************************************************}
\typeout{Section 1.2 Homotopy}
\typeout{************************************************}
\section[Homotopy]{Homotopy}\label{section-2}
\typeout{************************************************}
\typeout{Subsection 1.2.1 Homotopies}
\typeout{************************************************}
\subsection[Homotopies]{Homotopies}\label{subsection-1}
\begin{definition}[Homotopic maps]\label{definition-1}

              Maps \(f_0,f_1\colon X \to Y\) are said to be \terminology{homotopic} if there is a
              continuous map \(F\colon X\times I \to Y\) such that
              \[
                F(x,0) = f_0(x)\text{ and }F(x,1) = f_1(x)\ \forall x\in X.
              \]\end{definition}

            We let \(\Map(X,Y) = \{f\colon X \to Y \text{ continuous}\}\).
            Then letting \(f_t(x) = F(x,t)\) in the above definition we see that \(f_t\) is a
            path from \(f_0\) to \(f_1\) in \(\Map(X,Y)\).
          %
\begin{example}\label{example-1}
\begin{enumerate}
\item{}\(X = Y = \RR^n\), \(f_0(\overline{x}) = \overline{0}\) and \(f_1(\overline{x}) = \overline{x}\) are homotopic via \(f_t(\overline{x}) = t\overline{x}\).\item{}\(S^1 = \{z\in \CC : |z| = 1\}\) then \item{}\(S^n = \{ \overline{x} \in \RR^n : |\overline{x}| = 1\}\) \end{enumerate}
\end{example}
\begin{lemma}\label{lemma-1}

              Homotopy is an equivalence relation on \(\Map(X,Y)\).
            \end{lemma}
\label{notation-1}
\begin{lemma}\label{lemma-2}

              If \(f_0 \sim f_1\colon X \to Y\) and \(g_0 \sim g_1\colon Y \to Z\) then
              \(g_0\circ f_0 \sim g_1\circ f_1\).
            \end{lemma}
\begin{corollary}\label{corollary-1}

              For any space \(X\) the set \([X,\RR^n]\) has one element.
            \end{corollary}
\begin{proof}

              Define \(0_X\colon X \to \RR^n\) by \(0_X(x) = 0 \in \RR^n\) for any \(x\in X\).
            \end{proof}
\begin{definition}[Contractible space]\label{definition-2}
\(X\) is \terminology{contractible} if \(1_X\) is homotopic to a constant map.
            \end{definition}
\begin{proposition}\label{proposition-1}
\(Y\) is contractible \(\iff\)\([X,Y]\) has one element for any space \(X\).
            \end{proposition}
\begin{proof}

              (\(\Rightarrow\)) as in corollary.  (\(\Leftarrow\)) \([X,Y]\) has one element so
              \(1_Y \sim \) a constant map.
            \end{proof}
\par

            Given a space \(X\) how can we tell if \(X\) is contractible? If \(X\) is
            contractible then it must be path connected for one.

            \begin{proof}

              Contractible implies that \([S^0, X]\) has one element and so \(f \colon
              S^0 \to X\) extends to \(D^1\), and therefore \(X\) is path connected. \end{proof}


            Similarly if \([S^1, X]\) has more than one element then \(X\) is not
            contractible.
          %
\begin{definition}[Simply connected]\label{definition-3}

              We say \(X\) is \terminology{simply connected} if \([S^1, X]\) has only one
              element.

              We say two space \(X\) and \(Y\) are \emph{homotopy equivalent} if there
              exists \(f\colon X \to Y\) and \(g\colon Y \to X\) such that \(g\circ f
              \sim 1_X\) and \(f\circ g \sim 1_Y\).
            \end{definition}
\begin{example}\label{example-2}

              \(X\) is contractible if and only if \(X \sim \{p\}\).
            %
\begin{proof}
\(X\) contractible \(\implies 1_X \sim c\), a constant map.
              Choose \(f\colon X \to \{p\}\), \(f(x) = p\) and \(g\colon\{p\} \to X\),
              \(g(p) = c\). Then \(g\circ f = c \sim 1\) and \(f\circ g = 1_{\{p\}}\).
              Converse: exercise.
            \end{proof}
\end{example}
\begin{exercise}\label{exercise-1}
\end{exercise}
\par

            Given \(X\) and \(Y\) how can we determine if \(X\sim Y\)?
            How do we determine \([X,Y]\)?
            For example is \(S^n \sim S^m\).
          %
\typeout{************************************************}
\typeout{Subsection 1.2.2 Homotopy groups}
\typeout{************************************************}
\subsection[Homotopy groups]{Homotopy groups}\label{subsection-2}
\begin{definition}[Map of pairs]\label{definition-4}

              A \terminology{map of pairs}\(f\colon (X, A) \to (Y, B)\) is a map \(f\colon X
              \to Y\) with sets \(A\subset X\) and \(B\subset Y\) such that \(f(A)\subset
              B\).

              If we have maps of pairs \(f_0, f_1\colon (X,A) \to (Y,B)\) then we
              write \(f_0\sim f_1\) if there exists \(F\colon(X\times I, A\times I) \to
              (Y,B)\) such that \(F(x,0) = f_0(x)\) and \(F(x,1) = f_1(x)\).
            \end{definition}
\begin{definition}[Homotopy groups]\label{definition-5}

              If \(*\in X\) then the \terminology{\(n\)th homotopy group} is
              \[\pi_n(X, *) = [(D^n, S^{n-1}) \to (X, \{*\})].\]\end{definition}

            We now note several properties of this definition:
            \begin{enumerate}
\item{}\(\pi_0(X, *) =\) set of path components of \(X\).\item{}\(\pi_1(X, *)\) is a group.\(\pi_n(X, *)\) is an abelian group. \item{}\(\pi_n\) is a functor
                \[
                  \left\{ \begin{subarray}{l}\text{pointed spaces} \\ \text{pointed
                  maps} \end{subarray}\right\} \to \left\{\begin{subarray}{l}
                  \text{groups}\\ \text{group homomorphisms}\end{subarray}\right\}.
                \]
                So given
                \[f\colon(X, p)\to(Y,q)\]
                we get
                \[f_*\colon \pi_n(X,p)\to\pi_n(y,q)\]
                defined by
                \[f_*(\gamma) = f\circ \gamma.\]
                
              \end{enumerate}

          %
\begin{example}[Homotopy groups of \(S^2\)]\label{example-3}
\begin{table}
\centering
\begin{tabular}{*{8}{c}}
\(n\)&1&2&3&4&5&6&7\\
\(\pi_n(S^2)\)&0&\(\ZZ\)&\(\ZZ\)&\(\ZZ/2\)&\(\ZZ/2\)&\(\ZZ/12\)&\(\ZZ/15\)\\
\end{tabular}
\end{table}
\end{example}
\typeout{************************************************}
\typeout{Section 1.3 Homology}
\typeout{************************************************}
\section[Homology]{Homology}\label{section-3}

          Our goal is to construct a functor \(H_*\) from the category of topological spaces and continuous maps to the category of \(\ZZ\)-modules and \(\ZZ\)-linear maps.
          This means to each space \(X\) we associate an abelian group \(H_*(X) = \bigoplus_{n\ge 0}H_n(X)\), and to each map \(f\colon X \to Y\) a function \(f_*\colon H_n(X) \to H_n(Y)\) satisfying \((1_X)_* = 1_{H_n(X)}\) and \((f\circ g)_* = f_* \circ g_*\).
        %
\par

          Some properties we would like to have for our construction are:
          \begin{enumerate}
\item{}Homotopy invariance, if \(f \sim g \colon X \to Y\) then \(f_* = g_*\).\item{}The dimension axiom, \(H_n(X) = 0\) for any \(n > \dim X\).\end{enumerate}

        %
\typeout{************************************************}
\typeout{Subsection 1.3.1 Chain complexes}
\typeout{************************************************}
\subsection[Chain complexes]{Chain complexes}\label{subsection-3}
\begin{definition}[Chain complex]\label{definition-6}

              If \(R\) is a commutative ring then a \terminology{chain complex} over \(R\) is a pair \((C,d)\) satisfying:
              \begin{enumerate}
\item{}\(C = \bigoplus_{n\in\ZZ} C_n\) for \(R\)-modules \(C_n\).\item{}\(d\colon C \to C\) where \(d = \bigoplus d_n\) for \(R\)-linear maps \(d_n\).\item{}\(d\circ d = 0\).\end{enumerate}

              The indexing by \(n\) is called a \terminology{grading}.
              Usually we take \(C_n = 0\) for \(n \lt 0\).
              An element of \(\ker d_n\) is called \terminology{closed} or a \terminology{cycle}.
              An element of \(\im d_n\) is called a \terminology{boundary}.
              \(d\) is the \terminology{boundary map} or \terminology{differential}.
            \end{definition}
\begin{definition}[Homology groups]\label{definition-7}

              If \((C,d)\) is a chain complex, its \terminology{\(n\)th homology group} is
              \[H_n(C,d) = \ker d_n/\im d_{n+1}.\]
              If \(x \in \ker d_n\) we write \([x]\) for its image in \(H_n(C)\).
            \end{definition}
\begin{example}\label{example-4}
\begin{enumerate}
\item{}
                \(C_0 = C_1 = \ZZ\), \(C_i = 0\) otherwise,
                \[
                  0\to \ZZ \xrightarrow{\cdot 3} \ZZ \to 0.
                \]
                Then \(H_1 = 0\), \(H_0 = \ZZ/3\).
              \item{}
                \[
                  \ZZ=\langle e\rangle\to\ZZ^2 = \langle f_1,f_2\rangle\to\ZZ=\langle g\rangle \to 0
                \]
                with \(d(e) = f_1-f_2\), \(d(f_1) = d(f_2) = g\), then \(H_*(C) = 0\)(exercise).
              \end{enumerate}
\end{example}
\typeout{************************************************}
\typeout{Subsection 1.3.2 The chain complex of a simplex}
\typeout{************************************************}
\subsection[The chain complex of a simplex]{The chain complex of a simplex}\label{subsection-4}
\begin{definition}[\(n\)-simplex]\label{definition-8}

              The \terminology{\(n\)-dimensional simplex}\(\Delta^n\) is \[\Delta^n = \left\{ (x_0,\ldots,x_n)\in\RR^n : \sum_{i} x_i = 1,\, x_i \ge 0 \forall i\right\}.\]\(\Delta^n\) has \terminology{vertices}\(v_0,\ldots,v_n\) which are the intersections with the coordinate axes. The \(k\)-dimensional \terminology{faces} are in bijection with the \(k+1\)element subsets of \(\{0,\ldots,n\}\).
            \end{definition}
\begin{definition}[Simplicial chain complex]\label{definition-9}
\(S_*(\Delta^n)\) is the chain complex with \(S_k(\Delta^n)\) the free \(\ZZ\)-module generated by the \(k\)-dimensional faces of \(\Delta^n\).
              So
              \[
                S_k(\Delta^n) = \langle e_I : I = \{i_0,\ldots,i_k : 0 \le i_0 \le \cdots \le i_k \le n \}\rangle.
              \]
              To define \(d\) it suffices to define \(d(e_I)\), we let \[d(e_I) = \sum_{j = 0}^{k}(-1)^j e_{i_0,\ldots,i_{j-1}, i_{j+1},\ldots,i_k}\in S_{k-1}(\Delta^n).\]\end{definition}
\begin{example}\label{example-5}
\begin{enumerate}
\item{}For \(\Delta^1\) we have \(d(e_{0,1}) = e_1 - e_0\).\item{}For \(\Delta^2\) we have \(d(e_{0,1,2}) = e_{12} - e_{02} + e_{01}\) and so \(d^2(e_{I} = (e_2 - e_1) - (e_2 - e_0) + (e_1 - e_0) = 0\).\end{enumerate}
\end{example}
\begin{lemma}\label{lemma-3}
\(d^2 = 0\)\end{lemma}
\begin{proof}

              It suffices to show \(d^2(e_I) = 0\) for all \(I\).
              \begin{align*}
d^2(e_I) &= d\left(\sum_{j = 0}^{k} (-1)^j e_{i_0,\ldots,\hat{i}_j,\ldots,i_k}\right)\\
&= \sum_{j = 0}^{k} (-1)^j d\left(e_{i_0,\ldots,\hat{i}_j,\ldots,i_k} \right)\\
&= \sum_{j = 0}^{k} (-1)^j\left( \sum_{l \lt j} (-1)^l e_{i_0,\ldots,\hat{i}_l,\ldots,\hat{i}_j,\ldots,i_k} + \sum_{l \gt j} (-1)^{l-1} e_{i_0,\ldots,\hat{i}_j,\ldots,\hat{i}_l,\ldots,i_k}\right)\\
&= \sum_{j = 0}^{k} \left( \sum_{l \lt j} (-1)^{l+j} e_{I - i_l - i_j} - \sum_{l \gt j} (-1)^{l+j} e_{I - i_j - i_l}\right) = 0.
\end{align*}\end{proof}
\begin{example}[Computing \(H_*(S_*(\Delta^2))\)]\label{example-6}

              \[
                0\to \ZZ = \langle e_{012}\rangle \to \ZZ^3 = \langle e_{01}, e_{02},e_{12}\rangle \to \ZZ^3 = \langle e_0,e_1,e_2 \rangle \to 0
              \]
              with \(d(e_{012}) = e_{12} - e_{02} + e_{01}\).
              So \(\ker d_2 = 0 \implies H_2 = 0\).
              \[
                d(ae_{01} + be_{02} + ce_{12}) = a(e_1 - e_0) + b (e_2 - e_1)  + c(e_2 - e_0)
              \]
              so \(ae_{01} + be_{02} + ce_{12}\in \ker d_1 \iff -a-b=0,\,a-c = 0,\, c+b = 0 \iff a = -b = c\) hence \(\ker d_1 = \langle e_{01} - e_{02} + e_{12} = \im d_2\) and so \(H_1 = 0\).
              \(\ker(d_0) = 0\), \(\im d_1 = \langle e_1 - e_0, e_2 - e_1, e_2 - e_0\rangle\) so \(H_0 = \ZZ = \langle [e_0]\rangle\).
            %
\end{example}
\begin{exercise}\label{exercise-2}

              Show that \(H_*(S_*(\Delta^n)) = 0\) if \(k \ne 0\) and \(\ZZ\) if \(k = 0\).
            \end{exercise}
\typeout{************************************************}
\typeout{Subsection 1.3.3 The singular chain complex}
\typeout{************************************************}
\subsection[The singular chain complex]{The singular chain complex}\label{subsection-5}
\begin{definition}[Singular chain complex]\label{definition-10}

              If \(X\) is a space, the \terminology{singular chain complex} of \(X\), \(C_*(X)\) is defined by
              \[C_n(x) = \langle e_\sigma : \sigma \colon \Delta^n \to X \text{ any continuous map}\rangle.\]
              Where 
              \[d(e_\sigma) = \sum_{j=0}^{n} (-1)^j e_{\sigma\circ F_j} \in C_{n-1}(X)\]
              where \(F_j\colon \Delta^{n-1} \to \Delta^n\) is given by \(F_j(x_0,\ldots,x_{n_1} = (x_0,\ldots,0,\ldots,x_{n-1})\) with the \(0\) in the \(j\)th place.
            \end{definition}
\typeout{************************************************}
\typeout{Subsubsection 1.3.3.1 Homotopy invariance}
\typeout{************************************************}
\subsubsection[Homotopy invariance]{Homotopy invariance}\label{subsubsection-1}
\begin{definition}[Chain homotopic maps]\label{definition-11}

              Suppose \(\phi,\psi\colon C_* \to C_*'\) are chain maps, we say that \(\phi\) are \terminology{chain homotopic} if there exists an \(R\)-linear map \(h\colon C_* \to C_{*+1}'\) such that \(d'\circ h + h \circ d' = \phi - \psi\).
              We denote this relation by \(\phi \sim \psi\).
              \end{definition}
\begin{lemma}\label{lemma-4}

                If \(\phi \sim \psi\) then \(\phi_* = \psi_*\).
              \end{lemma}
\begin{proof}
\begin{align*}
\phi_*([x]) - \psi_*([x]) &= [\phi(x) - \psi(x)]\\
&= [d'hx + hdx] = [d'hx] = 0 \in H_*(C').
\end{align*}\end{proof}
\begin{theorem}\label{theorem-1}

                Suppose \(f\sim g\colon X\to Y\) via \(H\) then \(f_\# \sim g_\# \implies f_* = g_*\).
              \end{theorem}
\begin{proof}
\end{proof}
\begin{corollary}\label{corollary-2}

                If \(X\sim Y\) then \(H_*(X) \cong H_*(Y)\).
              \end{corollary}
\begin{corollary}\label{corollary-3}

                If \(X\) is contractible then \(H_*(X) \cong H_*(\{p\}) \cong \ZZ\) if \(* = 0\), \(0\) otherwise.
              \end{corollary}
\typeout{************************************************}
\typeout{Section 1.4 Homology of a pair}
\typeout{************************************************}
\section[Homology of a pair]{Homology of a pair}\label{section-4}
\typeout{************************************************}
\typeout{Subsection 1.4.1 Exact sequences}
\typeout{************************************************}
\subsection[Exact sequences]{Exact sequences}\label{subsection-6}
\begin{definition}[Exact sequence]\label{definition-12}

              A sequence
              \[
                \cdots \to A_{n+1} \to A_n \to A_{n-1} \to \cdots
              \]
              of \(R\)-modules and \(R\)-linear maps is \terminology{exact at \(A_n\)} if \(\ker f_n = \im f_{n+1}\).
              \newline{}
              A sequence is \terminology{exact} if it is exact at all \(A_n\), then \((A_*,f)\) is known as a \terminology{acyclic} chain complex (the homology is zero).
            \end{definition}
\begin{example}\label{example-7}
\begin{enumerate}
\item{}
                \(0 \to A \xrightarrow{f} B\) is exact if and only if \(f\) is surjective.
              \item{}
                \(B \xrightarrow{g} C \to 0\) is exact if and only if \(g\) is injective.
              \item{}
                \(0 \to A \xrightarrow{f} A' \to 0\) is exact if and only if \(f\) is an isomorphism.
              \item{}
                \(0 \to A \to B \to C \to 0\) is exact if and only if \(A\subset B\) and \(C\cong B/A\).
              \end{enumerate}
\end{example}
\begin{definition}[Short exact sequence]\label{definition-13}

              A sequence 
              \[
                0 \to A_* \xrightarrow i B_* \xrightarrow{\pi} C_* \to 0
              \]
              is a \terminology{short exact sequence} of chain complexes if
              \begin{enumerate}
\item{}
                  \(A_*,B_*,C_*\) are chain complexes.
                \item{}
                  \(i, \pi\) are chain maps.
                \item{}
                  \[
                    0 \to A_n \xrightarrow i B_n \xrightarrow{\pi} C_n \to 0
                  \]
                  is exact for all \(n\).
                \end{enumerate}
\end{definition}
\begin{lemma}[Snake lemma]\label{lemma-5}

              If
              \[
                0 \to A_* \xrightarrow i B_* \xrightarrow{\pi} C_* \to 0
              \]
              is a short exact sequence of chain complexes then there is an associated \terminology{long exact sequence} of homology groups
              \begin{figure}
\centering
\[
                  \xymatrix{ \cdots\ar[r] & H_n(A) \ar[r]&
                  H_n(B) \ar[r] & H_n(C)\ar `r[d] `[l]
                  `[llld]_{\partial_n} `[dll] [dll]\\
                  & H_{n-1}(A) \ar[r] & H_{n-1}(B)
                  \ar[r] & H_{n-1}(C)\ar[r] & {\dots} }
                \]\end{figure}

              The map \(\partial\) is called the \terminology{boundary map} in the exact sequence.
            \end{lemma}
\begin{proof}

              We first define \(\partial\), given \([c] \in H_n(C)\) pick \(b\in B_n\) such that \(\pi(b) = c\).
              Now \(\pi db = d\pi b = dc = 0\) and so \(db = ia\) for some \(a \in A_{n-1}\).
              We have that \(ida = dia = ddb = 0\) and so \(da = 0\) and we can define \(\partial [c] = [a] \in H_{n-1} (A)\). \newline{}
              We must check that the definition of \(\partial\) does not depend on the choice of \(b\) or \(c\) and also check exactness at each term.
              Here we prove exactness at \(H_n(C)\).
              With notation as above suppose \([c] = \pi_*([x])\) then we can take \(b = x\), \(db = 0\) implying that \(a =0\) and so \(\partial [c] = [a] = 0\).
              We have then that \(\im \pi_* \subset \ker \partial\), so it remains to show that \(\ker \partial\subset \im \pi_*\).
              If \([a] = da',\,a'\in A_n\) we let \(b' = b - i(a')\) then \(d(b') = db - d(ia') = db - i(da') = db - db = 0\).
              Then \([b'] \in H_n(B)\) so \(\pi_*([b']) = [\pi(b')] = [\pi(b) - \pi(i(a'))] = [\pi(b)] = [c]\).
            \end{proof}
\begin{exercise}\label{exercise-3}

              Suppose we have a map of short exact sequences  commuting, then show there is a map of long exact sequences of homology commuting .
            \end{exercise}
\begin{example}[Reduced homology]\label{example-8}

              If \(X\) is a space let
              \[
                \tilde C_*(X) = \begin{cases}C_*(X), &* \ne -1,\\\ZZ, & * = -1,\end{cases}
              \]
              then define \(d(e_p)= e\) for \(p \in X\).
              We have \(d^2e_\gamma = d(e_{\gamma(1)} - e_{\gamma(0)}) = e - e = 0\).
              So \(\tilde C_*\) is a chain complex.
              \newline{}
              This construction can be motivated by thinking of \(\Delta^{-1} = \{\}\) and then considering \(\Map(\{\},X) = \{e\}\).
              If
              \[
                A_* = \begin{cases} \ZZ = \langle e \rangle, & * = -1, \\ 0, & * = -1.\end{cases}
              \]
              Then we have a short exact sequence
              \[
                0 \to A_* \to \tilde C_*(X) \to C_*(X) \to 0.
              \]
              We have \(H_*(A) = A_*\) and so the long exact sequence of homology then says that for \(n \ge 0\)\[
                \cdots \to H_n(A) = 0 \to H_n(\tilde C(X))\to H_n(C(X)) \to H_{n-1}(A)= 0 \to \cdots
              \]
              giving that \(H_n(\tilde C_*(X)) \cong H_n(C_*(X))\) for \(n \gt 0\).
              We write \(\tilde H_*(X)\) for \(H_*(\tilde C_*(X))\). 
              At \(n = 0\) we get
              \[
                \cdots \to 0 \to \tilde H_0(X)\to H_0(X) \to \ZZ \to 0 \to\cdots
              \]
              giving that \(\tilde H_0(X)\) has one fewer copies of \(\ZZ\) than \(H_0(X)\).
            \end{example}
\begin{example}[Homology of a pair]\label{example-9}

              Suppose \(A\subset X\), then \(C_*(A) \subset C_*(X)\) is a subcomplex.
              Define \(C_*(X,A) = C_*(X)/C_*(A)\) and so we have a short exact sequence
              \[
                0 \to C_*(A) \to C_*(X) \to C_*(X,A) \to 0.
              \]\end{example}
\typeout{************************************************}
\typeout{Subsection 1.4.2 Homology of a pair}
\typeout{************************************************}
\subsection[Homology of a pair]{Homology of a pair}\label{subsection-7}
\begin{definition}[Homology of a pair]\label{definition-14}
\(H_*(X,A) = H_*(C_*(X,A))\) is the \terminology{homology of the pair \((X,A)\)}.
            \end{definition}

            From this we obtain:
          %
\begin{definition}[Long exact sequence of a pair]\label{definition-15}

              The \terminology{long exact sequence of the pair \((X,A)\)} is
              \[ \cdots \to H_n(A) \to H_n(X) \to H_n(X,A) \xrightarrow{\partial} H_{n-1} (A) \to \cdots.\]\end{definition}
\par

            If \(f\colon (X,A) \to (Y,B)\) is a map of pairs then we get an \terminology{induced map} \(f_\# \colon C_*(X) \to C_*(Y)\) defined by
            \[
              (\sigma\colon \Delta^n \to A) \mapsto f\circ \sigma.
            \]
            Observe that \(f_\#(C(A)) \subset C_*(B)\) and so \(f_\#\) descends to a map
            \[
              f_\#\colon C_*(X)/C_*(A) \to C_*(Y)/C_*(B)
            \]
            or equivalently \(f_\#\colon C_*(X,A) \to C_*(Y,B)\) this then induces \(f_*\colon H_*(X,A) \to H_*(Y,B)\).
          %
\begin{proposition}[Homotopy invariance]\label{proposition-2}

              If \(f,g \colon (X,A) \to (Y,B)\) are homotopic as maps of pairs then \(f_* = g_* \colon H_*(X,A) \to H_*(Y,B)\).
            \end{proposition}
\begin{proof}

              Let \(H\colon (X\times I, A\times I) \to (Y,B)\) be the homotopy, \(H\) induces a chain homotopy \(h \colon C_*(X) \to C_{*+1}(Y)\) where \(dh + hd = f_\# - g_\#\).
              \(H(A\times I) \subset B\) so \(h(C_*(A)) \subset C_{*+1}(B)\) this implies that \(h\) descends to a map
              \[
                h \colon C_*(X)/C_*(A) \to C_{*+1}(Y)/C_{*+1}(B)
              \]
              with \(hd + dh = f_\# - g_\#\) as any relation satisfied will remain satisfied in the quotient.
              So we have \(h \colon C_*(X,A) \to C_{*+1}(Y,A)\) and hence \(f_\#,g_\#\colon C_*(X,A) \to C_*(Y,B)\) are chain homotopic.
            \end{proof}
\typeout{************************************************}
\typeout{Subsubsection 1.4.2.1 Visualising relative homology classes}
\typeout{************************************************}
\subsubsection[Visualising relative homology classes]{Visualising relative homology classes}\label{subsubsection-2}

              If \(W^{n+1}\) is a connected oriented compact manifold then we'll show that \(H_{n+1}(W,\partial W) \cong \ZZ = \langle [W,\partial] \rangle\) (where the \(\partial\) notation means relative to boundary).
              So given \(f\colon (W, \partial) \to (X,A)\) we get \(f_*([W,\partial W]) \in H_{n+1}(X,A)\).
            %
\begin{example}\label{example-10}

                Let \(X = \RR^3\) and \(A= S^1\) then \(W = D^2\) defines a class in \(H_2(\RR^3,S^1)\) (the boundary of \(W\) lies inside of \(A\)).
              \end{example}
\typeout{************************************************}
\typeout{Subsection 1.4.3 Good pairs}
\typeout{************************************************}
\subsection[Good pairs]{Good pairs}\label{subsection-8}
\begin{definition}[Good pair]\label{definition-16}
\((X,A)\) is a \terminology{good pair} if
              \begin{enumerate}
\item{}
                  \(\exists U \subset X\) open and \(A \subset U\)
                \item{}
                  \(\exists \pi \colon U \to A\) with \(\pi|_A = \id_A\).
                \item{}
                  \(\pi \sim 1_{(U,A)}\) as maps of pairs from \((U,A)\) to itself.
                \end{enumerate}

              (i.e. \(A\) is a deformation retract of \(U\)).
            \end{definition}
\begin{example}[Good pairs]\label{example-11}
\begin{enumerate}
\item{}
                  \((\text{smooth manifold}, \text{closed submanifold})\).
                \item{}
                  \((\text{simplicial complex}, \text{subcomplex})\).
                \end{enumerate}
\end{example}
\begin{example}[Not good pairs]\label{example-12}
\begin{enumerate}
\item{}
                  \((\RR, \QQ)\).
                \item{}
                  Letting \(H\subset \RR^2\) be the Hawaiian earring then \((\RR^2, H)\) is not a good pair.
                \end{enumerate}
\end{example}
\begin{theorem}\label{theorem-2}

              Take \(A\subset X\) and let \(\pi\) be the natural map \(\pi\colon(X,A)\to(X/A,A/A)\cong (X/A,*)\).
              Then if \((X,A)\) is a good pair the induced map \(\pi_*\colon H_*(X,A)\to H_*(X/A,*)\) is an isomorphism.
            \end{theorem}
\begin{proof}

              Postponed.
            \end{proof}
\begin{exercise}\label{exercise-4}

              The composite map \(\phi\) in
              \[
                \tilde H_*(X) \to H_*(X) \to H_*(X,*)
              \]
              is an isomorphism.
              i.e.
              \[
                H_*(X,*) = \begin{cases} H_*(X), &* \gt 0,\\H_0(X)/\ZZ, &* \le 0.\end{cases}
              \]\end{exercise}
\begin{proposition}\label{sphere-homology}
\[
                \tilde H_*(S^n) = \begin{cases} \ZZ, &* = n,\\0, &* \ne n.\end{cases}
              \]\end{proposition}
\begin{proof}

              We proceed by induction on \(n\).
              For \(n = 0\) we have \(S^0 = \{p_+, p_-\}\) implying
              \[
                H_*(S^0) = H_*(p_+) \oplus H_*(p_-) = \begin{cases} \ZZ^2, &* = 0,\\0, &* \ne 0,\end{cases}
              \]
              giving
              \[
                \tilde H_*(S^0) = \begin{cases} \ZZ, &* = 0,\\0, &* \ne 0.\end{cases}
              \]
              Now consider the long exact sequence of the pair \((D^n, S^{n-1})\)\[
                H_*(D^n) \to H_*(D^n, S^{n-1}) \to H_{*-1}(S^{n-1}) \xrightarrow{\phi} H_{*-1}(D^n),
              \]
              we can break this up using the kernel and cokernel to get
              \begin{equation}
                0 \to \coker \phi_m \to H_m(D^n, S^{n-1}) \to \ker \phi_{m-1} \to 0.
              \label{sphere-homology-coker-sequence}\end{equation}\(D^n\) is contractible and so we get
              \[
                H_*(D^n) = \begin{cases} \ZZ, &* = 0,\\0, &* \ne 0.\end{cases}
              \]
              We have \(\phi\colon H_0(S^{n-1}) \to H_0(D^n) \cong \ZZ = \langle e_p\rangle\) given by \(\phi(e_p) = e_p\).
              We see that \(\ker \phi = \tilde H_*(S^{n-1})\) and \(\coker \phi = 0\).
              Now by looking at \eqref{sphere-homology-coker-sequence} we see that
              \[
                0 \to \tilde H_*(S^n) = H_*(S^n,*) = H_*(D^n, S^{n-1}) \to \ker\phi = \tilde H_*(S^{n-1}) \to 0
              \]
              is exact giving \(\tilde H_*(S^n) \cong \tilde H_*(S^{n-1})\).
              The claim then follows by induction.
            \end{proof}
\begin{corollary}\label{corollary-4}
\[
                S^n \sim S^m \implies m = n.
              \]\end{corollary}
\begin{example}[Chains generating \(H_n(S^n, *)\)]\label{example-13}
\begin{enumerate}
\item{}
                  Choose \(f\colon \Delta^n \to \Delta^n\) a homeomorphism and let \(e\colon \Delta^n \to \Delta^n\) be the identity map.
                  Then \(a_{n-1} = f_\#(de) \in C_*(S^{n-1})\) and we have \(d a_{n-1} = df_\#(de) = f_\# (d^2 e) = 0\) and so \(a_{n-1}\) is closed.
                \item{}
                  Choose \(g\colon (D^n/S^{n-1}, S^{n-1}/S^{n-1}) \to (S^n, *)\).
                  Now let \(b_n = g_\#(f_\#(e)) \in C_n(S^n,*)\).
                  Then \(db_n = g_\#(a_{n-1}) \in C_*(*)\) implies \(b_n\) is closed in \(C_*(S^n, *)\).
                  \newline{}
                  For example if \(n = 2\) we are crushing the boundary of the \(2\)-simplex to a point.
                \end{enumerate}
\end{example}
\begin{proposition}\label{proposition-4}
\([a_n]\) generates \(\tilde H_n(S^n) = \ZZ\) (statement \((A_n)\)). \newline{}\([b_n]\) generates \(\tilde H_n(S^n, *) = \ZZ\) (statement \((B_n)\)). \newline{}\end{proposition}
\begin{proof}

              Statement \((A_0)\): \(S^0 = \{p_+, p_-\}\), \(de = e_{p_+} - e_{p_-}\) so \(de\) generates \(\tilde H_0(S^0)\).\newline{}
              We'll show that \(A_{n-1}\implies B_n\).
              Here \(\partial\colon H_n(S^n, *)=H_n(D^n, S^{n-1}) \xrightarrow{\sim} \tilde H_{n-1}(S^{n-1})\) (by \ref{sphere-homology}).
              So it suffices to check that \(\partial [b_n] = [a_{n-1}]\)\end{proof}
\begin{definition}[Wedge product]\label{definition-17}

              If \((X_i, p_i)\) are pointed spaces the \terminology{wedge product}\[
                \bigvee_{i\in I} (X_i,p_i) \text{ is } \coprod_{i\in I} X_i\Big/\{p_i : i\in I\}.
              \]
              If \(X_i\) is such that for any \(p,q\in X_i\) there exists a homeomorphism \(f\colon X_i \to X_i\) with \(f(p) = q\) we can drop \(p_i\) from the notation (for example in the case of \(X_i\) a connected manifold we can do this).
            \end{definition}
\begin{example}\label{example-14}
\end{example}
\begin{corollary}\label{corollary-5}

              If \((X_i, p_i)\) are good pairs then
              \[
                \tilde H_*\left(\bigvee_{i}(X_i, p_i)\right) = \bigoplus_i \tilde H_*(X_i).
              \]\end{corollary}
\begin{proof}
\begin{align*}
\tilde H_*\left(\bigvee_{i\in I} X_i\right) &\cong H_*\left(\bigvee_{i\in I} X_i, p\right) \cong H_*\left(\coprod_{i\in I} X_i, \{p_i : i\in I\}\right)\\
&\cong \bigoplus_{i\in I} H_*\left(X_i, p_i\right) \cong \bigoplus_{i\in I} \tilde H_*(X_i).
\end{align*}\end{proof}
\begin{example}\label{example-15}
\[
                H_*(S^1 \vee S^2) = \begin{cases}\ZZ, &* = 0,1,2\\0, &\text{otherwise}.\end{cases}
              \]\end{example}
\typeout{************************************************}
\typeout{Section 1.5 Subdivision and Excision}
\typeout{************************************************}
\section[Subdivision and Excision]{Subdivision and Excision}\label{section-5}
\begin{definition}\label{definition-18}

            If \(\U = \{U_i\}_{i\in I}\) is an open cover of \(X\), let
            \[
              C^{\U}_n(X) = \langle e_\sigma : \sigma\colon \Delta^n \to X,\,\im\sigma\subset U_i\text{ for some }i\rangle \subset C_n(X).
            \]
            Observe that \(\im \sigma \subset U_i\) implies \(\im \sigma \circ F_j \subset U_i\) and so \(C^{\U}_*\) is a subcomplex of \(C_*\).
            Let \(H^{\U}_*\) be the homology of this complex, then we have a map
            \[
              i\colon C^{\U}_*(X) \hookrightarrow C_*(X).
            \]\end{definition}
\begin{lemma}[Subdivision]\label{lemma-6}
\(C_*\colon H^{\U}_*(X) \to H_*(X)\) is an isomorphism.
          \end{lemma}
\typeout{************************************************}
\typeout{Section 1.6 Degree and Orientations}
\typeout{************************************************}
\section[Degree and Orientations]{Degree and Orientations}\label{section-6}
\typeout{************************************************}
\typeout{Section 1.7 Cell Complexes}
\typeout{************************************************}
\section[Cell Complexes]{Cell Complexes}\label{section-7}
\begin{definition}[Attaching of cells]\label{definition-19}

            If \(f\colon S^{n-1} \to X\) then
            \[
              X\cup_f D^n = X\amalg D^n/\sim
            \]
            is the space obtained by \terminology{attaching} an \(n\)-dimensional cell to \(X\) via the map \(f\).
          \end{definition}
\begin{example}\label{example-16}

            If \(X = \{p\}\) and \(f\colon S^{n-1} \to X\) then
            \[
              X\cup_f D^n\cong D^n/S^{n-1} \cong S^n.
            \]\end{example}
\begin{definition}[Finite cell complex]\label{definition-20}

            A 0-dimensional \terminology{finite cell complex} is a finite disjoint union of points.
            \newline{}
            A \(k\)-dimensional finite cell complex is a space obtained by attaching finitely many \(k\)-cells to a \((k-1)\)-dimensional finite cell complex.
          \end{definition}
\begin{example}\label{example-17}
\end{example}
\begin{example}\label{example-18}

            If a finite cell complex \(X\) has one 0-cell and one \(n\)-cell then \(X \cong S^n\).
            Similarly if \(X\) has one 0-cell and \(k\)\(n\)-cells then \(X \cong \bigvee_{i=1}^{k}S^n\).
          \end{example}
\begin{example}\label{example-19}
\(T^2\) is a finite cell complex with one 0-cell, two 1-cells and one 2-cell.\end{example}
\begin{example}\label{example-20}

            Any simplicial complex is a finite cell complex and any closed manifold can be given the structure of a finite cell complex.
          \end{example}
\typeout{************************************************}
\typeout{Chapter 2 Cohomology and Products}
\typeout{************************************************}
\chapter[Cohomology and Products]{Cohomology and Products}\label{chapter-2}
\typeout{************************************************}
\typeout{Section 2.1 Homology with Coefficients and Cohomology}
\typeout{************************************************}
\section[Homology with Coefficients and Cohomology]{Homology with Coefficients and Cohomology}\label{section-8}
\typeout{************************************************}
\typeout{Subsection 2.1.1 Hom and \(\otimes\) for modules}
\typeout{************************************************}
\subsection[Hom and \(\otimes\) for modules]{Hom and \(\otimes\) for modules}\label{subsection-9}
\begin{definition}[Tensor product of \(R\)-modules]\label{definition-21}

              Let \(M,N\) be \(R\)-modules.
              Then the \terminology{tensor product }\(M\otimes N\) is the \(R\)-modules generated by all pairs \(m\otimes n\) for \(m\in M,\,n\in N\) modulo the relations:
              \begin{enumerate}
\item{}
                  \((m_1+m_2)\otimes (n_1 + n_2) = \sum m_i \otimes n_j.\)
                \item{}
                  \(r(m\otimes n) = (rm)\otimes n = m\otimes (rn).\)
                \end{enumerate}
\end{definition}

            We have the following properties of this definition:
            \begin{enumerate}
\item{}\((M_1\oplus M_2) \otimes (N_1\oplus N_2) = \bigoplus M_i\otimes N_j\).\item{}\(M\otimes N \cong N \otimes M\).\item{}\(M\otimes R = M\)\end{enumerate}

          %
\begin{example}[Tensor products]\label{example-21}
\begin{enumerate}
\item{}\(R^n \otimes R^m \cong R^{mn}\).\item{}Letting \(R = \ZZ\), \( \QQ \otimes \ZZ/a\cong 0\).\item{}\(\ZZ/a\otimes\ZZ/b\cong \ZZ/(a,b)\).\end{enumerate}
\end{example}
\par
If \(f\colon M_1 \to M_2\) and \(g\colon N_1 \to N_2\) then there is a map
            \begin{align*}
f\otimes g \colon M_1 &\otimes N_1 \to M_2 \otimes N_2,\\
m&\otimes n \mapsto f(m) \otimes g(n).
\end{align*}
          %
\begin{example}\label{example-22}

              If \( f\colon R^n \to R^m\) is given by multiplication by \(A\in \operatorname{Mat}_{n\times m}(R)\) then \(f\otimes 1_M\colon M^n \to M^m\) is given by multiplication by \(A\).
            \end{example}
\begin{definition}[Hom]\label{definition-22}
\[\Hom(M, N) = \{f\colon M\to N : f \text{ is } R \text{-linear}\}\]
              is an \(R\)-module, via \((f + rg)(m) = f(m) + rg(m)\).
            \end{definition}
\par
From this definition we see that
            \begin{enumerate}
\item{}\(\Hom(\bigoplus M_i, \bigoplus N_j)\cong \bigoplus_{i,j} \Hom(M_i, N_j)\).\item{}\(\Hom(R, M) \cong M\).\end{enumerate}

          %
\par
Note however that we do not have \(\Hom(M,N) = \Hom(N,M)\) as for example \(\Hom(\ZZ/2,\ZZ) = 0\) but \(\Hom(\ZZ, \ZZ/2) = \ZZ/2\).%
\begin{definition}[Dual module]\label{definition-23}

              Given an \(R\)-module \(M\) the \terminology{dual} of \(M\) is \(M^* = \Hom(M,R)\).
            \end{definition}
\par
Now if we have \(f\colon M \to N\) we get a map
            \[f^*\colon \Hom(N, O) \to \Hom(M, O)\]
            given by \(f^*g = g\circ f\).
          %
\begin{example}\label{example-23}

              If \(f\colon R^n \to R^m\) is multiplication by \(A\) then
              \[f^*\colon \Hom(R^m , O) \cong O^m \to \Hom(R^n, O) \cong O^n)\]
              is multiplication by \(A^\top\).
            \end{example}
\typeout{************************************************}
\typeout{Subsection 2.1.2 Hom and \(\otimes\) for chain complexes}
\typeout{************************************************}
\subsection[Hom and \(\otimes\) for chain complexes]{Hom and \(\otimes\) for chain complexes}\label{subsection-10}
If \((C, d)\) is a chain complex defined over \(R\) then so are \((C_*\otimes M, d\otimes 1_M) = C_* \otimes M\) and \((\Hom(C_*,M),d^*) = \Hom(C_*, M)\).%
\typeout{************************************************}
\typeout{Section 2.2 Universal Coefficient Theorems}
\typeout{************************************************}
\section[Universal Coefficient Theorems]{Universal Coefficient Theorems}\label{section-9}
\typeout{************************************************}
\typeout{Section 2.3 Products}
\typeout{************************************************}
\section[Products]{Products}\label{section-10}
\begin{definition}[Product of chain complexes]\label{definition-24}
If \(A,B\) are chain complexes the \(C=A \otimes B\) is the chain complex with
            \[C_i = \bigoplus_{j+k = i}A_j \otimes B_k\]
            and
            \[d(a\otimes b) = (da)\otimes b + (-1)^{|a|} a\otimes (db)\]
            where \(|a| = j\) if \(a \in A_j\).
          \end{definition}
The following theorem is true but we will not prove it.%
\begin{theorem}[Eilenberg-Zilber]\label{theorem-3}
\[C_*(X\times Y) \sim C_*(X) \otimes C_*(Y).\]\end{theorem}
\par
Given \(H_*(A)\) and \(H_*(B)\) how can we compute \(H_*(A\otimes B)\).%
\begin{lemma}\label{lemma-7}
Suppose \(A\sim A'\). Then \(A\otimes B \sim A'\otimes B\).\end{lemma}
\begin{proof}
Easy.\end{proof}
\begin{corollary}\label{corollary-6}
Suppose \(A,B\) are finitely generated chain complexes, defined over a field. Then
            \[H_*(A\otimes B) = \bigoplus_{j+k = i} H_j(A) \otimes H_k(B)\]
            i.e.
            \[H_*(A\otimes B) \cong H_*(A) \otimes H_*(B).\]\end{corollary}
\begin{proof}
\(A\) is a chain complex over a field, so \(A\sim A'\) where \(A'_i = H_i(A)\) and \(d \equiv 0\) on \(A'\).
            Similarly for \(B\).
            Then
            \[H_*(A\otimes B) \cong H_*(A'\otimes B') \cong A'\otimes B' = H_*(A)\otimes H_*(B).\]\end{proof}
\begin{corollary}\label{homology-product-is-tensor}
If \(k\) is a field and \(X,\,Y\) are finite chain complexes then
            \[H_*(X\times Y; k)\cong H_*(X; k) \otimes_k H_*(Y; k).\]\end{corollary}
\begin{definition}[Poincare polynomial]\label{definition-25}
The \terminology{Poincare polynomial} of \(X\) with coefficients in \(k\) is
            \[P_k(X) = \sum_{i\ge 0} \dim H_i(X; k) t^i.\]\end{definition}
\begin{example}\label{example-24}
\[P_{\ZZ/2} (\RP^2) = 1 + t + t^2.\]\end{example}
\par
\ref{homology-product-is-tensor} tells us that \[P_k(X\times Y) = P_k(X)P_k(Y).\]%
\begin{example}\label{example-25}
\[P_{\ZZ/2}(\RP^3 \times \RP^2) = (1+t+t^2)(1+t+t^2+t^3).\]\end{example}
\par
For now we will suppose that \(A,B\) are finitely generated chain complexes over a PID \(R\).%
\begin{lemma}\label{free-homology-homotopic-zero-boundary}
If \(H_*(A)\) is a free \(R\)-module then \(A\sim A'\) where \(A_i' = H_i(A)\) and \(d\equiv 0\) on \(A'\).
          \end{lemma}
\begin{proof}
\(A\) is the direct sum of short and stupid complexes. \(H_*(A)\) being torsion free implies that every short complex is of the form
            \[
              \xymatrix{ C\colon R\ar@<.5ex>[r]^{\cdot a} & R \ar@<.5ex>[l]^{\cdot a^{-1}}}
            \]
            with \(a\in R^\times\) such a \(C \sim 0\) (same proof as for a field).
          \end{proof}
\begin{corollary}\label{corollary-8}
If \(H_*(A)\) is free then \(H_*(A\otimes B) \cong H_*(A) \otimes H_*(B)\).\end{corollary}
\begin{proof}
\(A\otimes B \cong A' \otimes B\) where \(A'\) is as in \ref{free-homology-homotopic-zero-boundary}.
            \(A'\otimes B\) is a direct sum of copies of \(B\), one for each generator of \(A'\).
            This gives that \(H_*(A'\otimes B)\) is  a direct sum of copies of \(H_*(B)\), one for each generator of \(H_*(A)\) which implies that
            \[H_*(A\otimes B) \cong H_*(A) \otimes H_*(B)\]
            (since \(H_*(A)\) is free).
            \[R^n \otimes H_*(B) \cong (H_*(B))^n.\]\end{proof}
\begin{corollary}\label{corollary-9}
If \(X\) and \(Y\) are finite cell complexes and \(H_*(X)\) is free over \(\ZZ\), then \(H_*(X\times Y) \cong H_*(X) \otimes H_*(Y)\).\end{corollary}
\par
Fact: If \(C_1\) is a free resolution of \(M\) and \(C_2\) is a free resolution of \(N\).
          Then \(\Tor_i(M,N) \cong H_i(C_1 \otimes C_2) \cong H_i(C_1 \otimes N) \cong H_i(M\otimes C_2)\).
        %
\begin{theorem}[Kunneth formula]\label{theorem-4}
If \(A,B\) are free (finitely generated) chain complexes over a PID \(R\) then
            \[H_i(A\otimes B) \cong \bigoplus_{j + k = i} \Tor_0(H_j(A), H_k(B)) \oplus\bigoplus_{j+k = i-1} \Tor_1(H_j(A), H_k(B)).\]\end{theorem}
\begin{proof}
Suffices to check the statement where \(A,B\) are short and/or stupid.
            We just did the case where both are short.
            The case where one or more is stupid was covered in the statement that
            \[H_*(A\otimes B) \cong H_*(A) \otimes H_*(B)\]
            if \(H_*(A)\) is free.
          \end{proof}
\begin{corollary}\label{corollary-10}
We can compute \(H_*(X\times Y)\) from \(H_*(X)\) and \(H_*(Y)\).\end{corollary}
\typeout{************************************************}
\typeout{Section 2.4 Cup product}
\typeout{************************************************}
\section[Cup product]{Cup product}\label{section-11}
Motivation: if \(k\) is a field
          \begin{align*}
H^*(X\times X; k) &\cong (H_*(X\times X; k))^*\\
 &\cong (H_*(X; k)\otimes H_*(X; k))^*\\
 &\cong H^*(X; k)\otimes H^*(X; k)
\end{align*}
          
        %
\begin{definition}[Cup product]\label{definition-26}
Suppose \(a\in C^i(X)\) and \(b\in C^j(X)\).
            If \(\sigma\colon \Delta^{i+j} \to X\) then we define \(a\smile b\in C^{i+j}(X)\) by
            \[(a\smile b)(\sigma) = a(\sigma\circ F_i^+)b(\sigma\circ F_j^-)\]
            where
            \begin{align*}
F_i^+\colon \Delta^i &\to \Delta^{i+j}\\
(x_0,\ldots,x_i) &\mapsto (x_0,\ldots,x_i,0,\ldots,0)
\end{align*}
            and
            \begin{align*}
F_j^-\colon \Delta^j &\to \Delta^{i+j}\\
(x_0,\ldots,x_j) &\mapsto (0,\ldots,0,x_0,\ldots,x_j).
\end{align*}\end{definition}
\begin{lemma}\label{lemma-9}
\[d(a\smile b) = da\smile b + (-1)^{|a|} a\smile db.\]\end{lemma}
\begin{proof}
Denote by \([v_0,\ldots,v_k]\) the singular \(k\)-simplex that sends
            \[\Delta^k\to\Delta^{i+j}\xrightarrow{\sigma} X\]\[\text{vertices of }\Delta^k\to\text{vertices }v_0,\ldots,v_k\text{ of }\Delta^{i+j}.\]
            Then taking \(\sigma\colon \Delta^{i+j+1} \to X\)\begin{align*}
d(a\smile b)(\sigma) &=(a\smile b)(d\sigma)\\
&=\sum_{k=0}^{i+j+1}(-1)^k a\smile b([v_0,\ldots,\hat v_k,\ldots,v_n])\\
&=\sum_{k\le i}(-1)^k a([v_0,\ldots,\hat v_k,\ldots,v_{i+1}])b([v_{i+1},\ldots,v_n])\\
&+\sum_{k\gt i}(-1)^k a([v_0,\ldots,v_{i}])b([v_{i},\ldots,\hat v_k,\ldots,v_n])\\
&=\sum_{k\le i + 1}(-1)^k a([v_0,\ldots,\hat v_k,\ldots,v_{i+1}])b([v_{i+1},\ldots,v_n])\\
&+\sum_{k\ge i}(-1)^k a([v_0,\ldots,v_{i}])b([v_{i},\ldots,\hat v_k,\ldots,v_n])\\
&=a(d([v_0,\ldots,v_{i+1}]))b([v_{i+1},\ldots,v_n])\\
&+ (-1)^i a([v_0,\ldots,v_{i}])b(d([v_{i},\ldots,v_n]))\\
&= da\smile b([v_0,\ldots,v_n]) + (-1)^i a\smile db([v_0,\ldots,v_n]).
\end{align*}\end{proof}
%
\backmatter
%
\typeout{************************************************}
\typeout{Section 1 Notation}
\typeout{************************************************}
\section[Notation]{Notation}\label{section-12}
\begin{longtable}[l]{llr}
\textbf{Symbol}&\textbf{Description}&\textbf{Page}\\[1em]
\endfirsthead
\textbf{Symbol}&\textbf{Description}&\textbf{Page}\\[1em]
\endhead
\multicolumn{3}{r}{(Continued on next page)}\\
\endfoot
\endlastfoot
$$&&\pageref{notation-1}\\
\end{longtable}
\end{document}