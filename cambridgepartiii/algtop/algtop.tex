\title{Part III Algebraic Topology - Michelmas 2014}
\author{Based on lectures by Dr. Jacob Rasmussen \\ Notes by Alex J. Best}
\date{\today}

\documentclass{article}
\usepackage{amsmath}
\usepackage{amsfonts}
\usepackage{amssymb}
\usepackage{amsthm}
\usepackage{upgreek}
\usepackage{tikz}
\usepackage{hyperref}
\usepackage{stmaryrd}
\usetikzlibrary{arrows}

\newtheorem*{thm}{Theorem}
\newtheorem*{lem}{Lemma}
\newtheorem*{cor}{Corollary}
\newtheorem*{prop}{Proposition}
\theoremstyle{definition}
\newtheorem*{defn}{Definition}
\newtheorem*{defns}{Definitions}
\newtheorem*{ex}{Example}
\newtheorem*{exs}{Examples}
\newtheorem*{exer}{Exercise}
\newtheorem*{rem}{Remark}
\newtheorem*{nota}{Notation}
\newtheorem*{alg}{Algorithm}
\newtheorem*{fact}{Fact}

\DeclareMathOperator{\Map}{Map}
\DeclareMathOperator{\RR}{\mathbf{R}}
\DeclareMathOperator{\CC}{\mathbf{C}}

\providecommand{\newoperator}[3]{\newcommand*{#1}{\mathop{#2}#3}}
\providecommand{\renewoperator}[3]{\renewcommand*{#1}{\mathop{#2}#3}}

\newcommand{\C}{C_*}
\renewcommand{\H}{H_*}

\newcommand{\ts}{\textsuperscript}
\renewcommand{\th}{\textsuperscript{th}\ }

\setcounter{tocdepth}{3}

\begin{document}
\maketitle
\tableofcontents

\section{Introduction}
These are lecture notes for the 2014 Part III Algebraic Topology course taught by Dr. Jacob Rasmussen.


The recommended books are:
\begin{itemize}
\item \href{http://www.math.cornell.edu/~hatcher/AT/ATpage.html}{Algebraic Topology} - Allen Hatcher,% check name + link
\item Homology Theory - James W. Vick,
\item Differential Forms in Algebraic Topology - Raoul Bott and Loring W. Tu.
\end{itemize}
\clearpage

\section{Homotopy}
\marginpar{Lecture 1}

\begin{defn}
Maps $f_0,f_1\colon X \to Y$ are said to be \emph{homotopic} if there is a 
continuous map $F\colon X\times I \to Y$ such that
\[
F(x,0) = f_0(x)\text{ and }F(x,1) = f_1(x)\ \forall x\in X.
\]
\end{defn}

We let $\Map(X,Y) = \{f\colon X \to Y \text{ continuous}\}$.
Then letting $f_t(x) = F(x,t)$ in the above definition we see that $f_t$ is a
path from $f_0$ to $f_1$ in $\Map(X,Y)$.

\begin{exs}
\begin{enumerate}
  \item $X = Y = \RR^n$, $f_0(\overline{x}) = \overline{0}$ and $f_1(\overline{x})
  = \overline{x}$ are homotopic via $f_t(\overline{x}) = t\overline{x}$.
  \item $S^1 = \{z\in \CC : |z| = 1\}$ then %TODO
  \item $S^n = \{ \overline{x} \in \RR^n : |\overline{x}| = 1\}$ %TODO
\end{enumerate}
\end{exs}

\begin{lem}
  Homotopy is an equivalence relation on $\Map(X,Y)$.
\end{lem}
\begin{defn}
  \[
  [X,Y] = \Map(X,Y)/\sim = \text{set of homotopy classes of maps } X \to Y.
  \]
\end{defn}

\begin{lem}
  If $f_0 \sim f_1\colon X \to Y$ and $g_0 \sim g_1\colon Y \to Z$ then 
  $g_0\circ f_0 \sim g_1\circ f_1$.
\end{lem}

\begin{cor}
  For any space $X$ the set $[X,\RR^n]$ has one element.
\end{cor}
\begin{proof}
  Define $0_X\colon X \to \RR^n$ by $0_X(x) = 0 \in \RR^n$ for any $x\in X$.
\end{proof}

\begin{defn}
  $X$ is \emph{contractible} if $1_X$ is homotopic to a constant map.
\end{defn}

\begin{prop}
  $Y$ is contractible $\iff$ $[X,Y]$ has one element for any space $X$.
\end{prop}

\begin{proof}
  ($\Rightarrow$) as in corollary.  ($\Leftarrow$) $[X,Y]$ has one element so 
  $1_Y \sim $ a constant map.
\end{proof}

Given a space $X$ how can we tell if $X$ is contractible? If $X$ is 
contractible then it must be path connected for one.

\begin{proof}
  Contractible implies that $[S^0, X]$ has one element and so $f \colon
  S^0 \to X$ extends to $D^1$, and therefore $X$ is path connected. %TODO
\end{proof}

Similarly if $[S^1, X]$ has more than one element then $X$ is not
contractible.

\begin{defns}
  We say $X$ is \emph{simply connected} if $[S^1, X]$ has only one
  element.

  We say two space $X$ and $Y$ are \emph{homotopy equivalent} if there
  exists $f\colon X \to Y$ and $g\colon Y \to X$ such that $g\circ f
  \sim 1_X$ and $f\circ g \sim 1_Y$.
\end{defns}

\begin{ex}
  $X$ is contractible if and only if $X \sim \{p\}$.
\end{ex}
\begin{proof}
  $X$ contractible $\implies 1_X \sim c$, a constant map.
  Choose $f\colon X \to \{p\}$, $f(x) = p$ and $g\colon\{p\} \to X$,
  $g(p) = c$. Then $g\circ f = c \sim 1$ and $f\circ g = 1_{\{p\}}$.
  Converse: exercise.
\end{proof}

\begin{exer}
  If $X_0 \sim X_1$ and $Y_0 \sim Y_1$ then $[X_0, Y_0]$ and $[X_1,
  Y_1]$ are in bijection.
\end{exer}

Given $X$ and $Y$ how can we determine if $X\sim Y$?
How do we determine $[X,Y]$?
For example is $S^n \sim S^m$.

\section{Homotopy groups}

\begin{defns}
  A \emph{map of pairs} $f\colon (X, A) \to (Y, B)$ is a map $f\colon X
  \to Y$ with sets $A\subset X$ and $B\subset Y$ such that $f(A)\subset
  B$.

  If we have maps of pairs $f_0, f_1\colon (X,A) \to (Y,B)$ then we
  write $f_0\sim f_1$ if there exists $F\colon(X\times I, A\times I) \to
  (Y,B)$ such that $F(x,0) = f_0(x)$ and $F(x,1) = f_1(x)$.

  If $*\in X$ then the \emph{$n$\th homology group}
  \[\pi_n(X, *) = [(D^n, S^{n-1}) \to (X, \{*\})].\]
\end{defns}

We now note several properties of this definition:
\begin{enumerate}
  \item $\pi_0(X, *) =$ set of path components of $X$.

  \item $\pi_1(X, *)$ is a group.
  
  $\pi_n(X, *)$ is an abelian group. %TODO

  \item $\pi_n$ is a functor
    \[\left\{ \begin{subarray}{l}\text{pointed spaces} \\ \text{pointed
    maps} \end{subarray}\right\} \to \left\{\begin{subarray}{l}
    \text{groups}\\ \text{group homomorphisms}\end{subarray}\right\}.\]
\end{enumerate}

\end{document}
