%%                                    %%
%% Generated from MathBook XML source %%
%%    on 2014-11-14T01:38:39Z    %%
%%                                    %%
%%   http://mathbook.pugetsound.edu   %%
%%                                    %%
\documentclass[10pt,]{article}
%% Load geometry package to allow page margin adjustments
\usepackage{geometry}
\geometry{letterpaper,total={5.0in,9.0in}}
%% Custom Preamble Entries, early (use latex.preamble.early)
%% Inline math delimiters, \(, \), made robust with next package
\usepackage{fixltx2e}
%% Page Layout Adjustments (latex.geometry)
%% For unicode character support, use the "xelatex" executable
%% If never using xelatex, the next three lines can be removed
\usepackage{ifxetex}
\ifxetex\usepackage{xltxtra}\fi
%% Symbols, align environment, bracket-matrix
\usepackage{amsmath}
\usepackage{amssymb}
%% extpfeil package for certain extensible arrows,
%% as also provided by MathJax extension of the same name
\usepackage{extpfeil}
%% allow more columns to a matrix
%% can make this even bigger by overiding with  latex.preamble.late  processing option
\setcounter{MaxMatrixCols}{30}
%% XML, MathJax Conflict Macros
%% Two nonstandard macros that MathJax supports automatically
%% so we always define them in order to allow their use and
%% maintain source level compatibility
%% This avoids using two XML entities in source mathematics
\newcommand{\lt}{<}
\newcommand{\gt}{>}
%% Semantic Macros
%% To preserve meaning in a LaTeX file
%% Only defined here if required in this document
%% Used for inline definitions of terms
\newcommand{\terminology}[1]{\textbf{#1}}
%% Subdivision Numbering, Chapters, Sections, Subsections, etc
%% Subdivision numbers may be turned off at some level ("depth")
%% A section *always* has depth 1, contrary to us counting from the document root
%% The latex default is 3.  If a larger number is present here, then
%% removing this command may make some cross-references ambiguous
%% The precursor variable $numbering-maxlevel is checked for consistency in the common XSL file
\setcounter{secnumdepth}{3}
%% Environments with amsthm package
%% Theorem-like enviroments in "plain" style, with or without proof
\usepackage{amsthm}
\theoremstyle{plain}
%% Numbering for Theorems, Conjectures, Examples, Figures, etc
%% Controlled by  numbering.theorems.level  processing parameter
%% Always need a theorem environment to set base numbering scheme
%% even if document has no theorems (but has other environments)
\newtheorem{theorem}{Theorem}[section]
\renewcommand*{\proofname}{Proof}%% Only variants actually used in document appear here
%% Numbering: all theorem-like numbered consecutively
%% i.e. Corollary 4.3 follows Theorem 4.2
\newtheorem{corollary}[theorem]{Corollary}
\newtheorem{lemma}[theorem]{Lemma}
\newtheorem{proposition}[theorem]{Proposition}
%% Definition-like environments, normal text
%% Numbering for definition, examples is in sync with theorems, etc
%% also for free-form exercises, not in exercise sections
\theoremstyle{definition}
\newtheorem{definition}[theorem]{Definition}
\newtheorem{example}[theorem]{Example}
\newtheorem{exercise}[theorem]{Exercise}
%% Figures, Tables, Floats
%% The [H]ere option of the float package fixes floats in-place,
%% in deference to web usage, where floats are totally irrelevant
%% We redefine the figure and table environments, if used
%%   1) New mbxfigure and/or mbxtable environments are defined with float package
%%   2) Standard LaTeX environments redefined to use new environments
%%   3) Standard LaTeX environments redefined to step theorem counter
%%   4) Counter for new enviroments is set to the theorem counter before caption
%% You can remove all this figure/table setup, to restore standard LaTeX behavior
%% HOWEVER, numbering of figures/tables AND theorems/examples/remarks, etc
%% WILL ALL de-synchronize with the numbering in the HTML version
%% You can remove the [H] argument of the \newfloat command, to allow flotation and 
%% preserve numbering, BUT the numbering may then appear "out-of-order"
\usepackage{float}
\newfloat{mbxtable}{H}{lot}[section]
\floatname{mbxtable}{Table}
\renewenvironment{table}%
{\begin{mbxtable}\setcounter{mbxtable}{\value{theorem}}\stepcounter{theorem}}%
{\end{mbxtable}}
%% Raster graphics inclusion, wrapped figures in paragraphs
\usepackage{graphicx}
%% Colors for Sage boxes and author tools (red hilites)
\usepackage[usenames,dvipsnames,svgnames,table]{xcolor}
%% Package for tables spanning several pages
\usepackage{longtable}
%% hyperref driver does not need to be specified
\usepackage{hyperref}
%% Hyperlinking active in PDFs, all links solid and blue
\hypersetup{colorlinks=true,linkcolor=blue,citecolor=blue,filecolor=blue,urlcolor=blue}
\hypersetup{pdftitle={Part III Algebraic Topology 2014}}
%% If you manually remove hyperref, leave in this next command
\providecommand\phantomsection{}
%% Custom Preamble Entries, late (use latex.preamble.late)
%% Convenience macros
\DeclareMathOperator{\Map}{Map}
\DeclareMathOperator{\dim}{dim}
\DeclareMathOperator{\im}{im}
\DeclareMathOperator{\CC}{\mathbf{C}}
\DeclareMathOperator{\RR}{\mathbf{R}}
\DeclareMathOperator{\ZZ}{\mathbf{Z}}
\DeclareMathOperator{\C}{C_*}
\DeclareMathOperator{\H}{H_*}
%% Title page information for article
\title{Part III Algebraic Topology 2014}
\author{Jacob Rasmussen\\
DPMMS\\
University of Cambridge
\and
Alex J. Best\\
DPMMS\\
University of Cambridge
}
\date{November 14, 2014}
\begin{document}
\maketitle
\thispagestyle{empty}
\begin{abstract}

                    These are lecture notes for the 2014 Part III Algebraic Topology course taught by Dr. Jacob Rasmussen.
                %
\end{abstract}
\setcounter{tocdepth}{1}
\renewcommand*\contentsname{Contents}
\tableofcontents
\clearpage
\typeout{************************************************}
\typeout{Section 1 Introduction}
\typeout{************************************************}
\section[Introduction]{Introduction}\label{section-1}
The recommended books are:
                \begin{itemize}
\item{}\href{http://www.math.cornell.edu/~hatcher/AT/ATpage.html}{Algebraic Topology} - Allen Hatcher,\item{}Homology Theory - James W. Vick,\item{}Differential Forms in Algebraic Topology - Raoul Bott and Loring W. Tu.\end{itemize}

            %
\par
Generated: November 14, 2014, 01:38:39 (Z)%
\typeout{************************************************}
\typeout{Section 2 Homotopy}
\typeout{************************************************}
\section[Homotopy]{Homotopy}\label{section-2}
\typeout{************************************************}
\typeout{Subsection 2.1 Homotopies}
\typeout{************************************************}
\subsection[Homotopies]{Homotopies}\label{subsection-1}
\begin{definition}[Homotopic maps]\label{definition-1}

                        Maps \(f_0,f_1\colon X \to Y\) are said to be \terminology{homotopic} if there is a
                        continuous map \(F\colon X\times I \to Y\) such that
                        \[
                            F(x,0) = f_0(x)\text{ and }F(x,1) = f_1(x)\ \forall x\in X.
                        \]\end{definition}

                    We let \(\Map(X,Y) = \{f\colon X \to Y \text{ continuous}\}\).
                    Then letting \(f_t(x) = F(x,t)\) in the above definition we see that \(f_t\) is a
                    path from \(f_0\) to \(f_1\) in \(\Map(X,Y)\).
                %
\begin{example}\label{example-1}
\begin{enumerate}
\item{}\(X = Y = \RR^n\), \(f_0(\overline{x}) = \overline{0}\) and \(f_1(\overline{x}) = \overline{x}\) are homotopic via \(f_t(\overline{x}) = t\overline{x}\).\item{}\(S^1 = \{z\in \CC : |z| = 1\}\) then \item{}\(S^n = \{ \overline{x} \in \RR^n : |\overline{x}| = 1\}\) \end{enumerate}
\end{example}
\begin{lemma}\label{lemma-1}

                        Homotopy is an equivalence relation on \(\Map(X,Y)\).
                    \end{lemma}
\label{notation-1}
\begin{lemma}\label{lemma-2}

                        If \(f_0 \sim f_1\colon X \to Y\) and \(g_0 \sim g_1\colon Y \to Z\) then
                        \(g_0\circ f_0 \sim g_1\circ f_1\).
                    \end{lemma}
\begin{corollary}\label{corollary-1}

                        For any space \(X\) the set \([X,\RR^n]\) has one element.
                    \end{corollary}
\begin{proof}

                        Define \(0_X\colon X \to \RR^n\) by \(0_X(x) = 0 \in \RR^n\) for any \(x\in X\).
                    \end{proof}
\begin{definition}[Contractible space]\label{definition-2}
\(X\) is \terminology{contractible} if \(1_X\) is homotopic to a constant map.
                    \end{definition}
\begin{proposition}\label{proposition-1}
\(Y\) is contractible \(\iff\)\([X,Y]\) has one element for any space \(X\).
                    \end{proposition}
\begin{proof}

                        (\(\Rightarrow\)) as in corollary.  (\(\Leftarrow\)) \([X,Y]\) has one element so
                        \(1_Y \sim \) a constant map.
                    \end{proof}
\par

                    Given a space \(X\) how can we tell if \(X\) is contractible? If \(X\) is
                    contractible then it must be path connected for one.

                    \begin{proof}

                        Contractible implies that \([S^0, X]\) has one element and so \(f \colon
                        S^0 \to X\) extends to \(D^1\), and therefore \(X\) is path connected. \end{proof}


                    Similarly if \([S^1, X]\) has more than one element then \(X\) is not
                    contractible.
                %
\begin{definition}[Simply connected]\label{definition-3}

                        We say \(X\) is \terminology{simply connected} if \([S^1, X]\) has only one
                        element.

                        We say two space \(X\) and \(Y\) are \emph{homotopy equivalent} if there
                        exists \(f\colon X \to Y\) and \(g\colon Y \to X\) such that \(g\circ f
                        \sim 1_X\) and \(f\circ g \sim 1_Y\).
                    \end{definition}
\begin{example}\label{example-2}

                        \(X\) is contractible if and only if \(X \sim \{p\}\).
                    %
\begin{proof}
\(X\) contractible \(\implies 1_X \sim c\), a constant map.
                        Choose \(f\colon X \to \{p\}\), \(f(x) = p\) and \(g\colon\{p\} \to X\),
                        \(g(p) = c\). Then \(g\circ f = c \sim 1\) and \(f\circ g = 1_{\{p\}}\).
                        Converse: exercise.
                    \end{proof}
\end{example}
\begin{exercise}\label{exercise-1}
\end{exercise}
\par

                    Given \(X\) and \(Y\) how can we determine if \(X\sim Y\)?
                    How do we determine \([X,Y]\)?
                    For example is \(S^n \sim S^m\).
                %
\typeout{************************************************}
\typeout{Subsection 2.2 Homotopy groups}
\typeout{************************************************}
\subsection[Homotopy groups]{Homotopy groups}\label{subsection-2}
\begin{definition}[Map of pairs]\label{definition-4}

                        A \terminology{map of pairs}\(f\colon (X, A) \to (Y, B)\) is a map \(f\colon X
                        \to Y\) with sets \(A\subset X\) and \(B\subset Y\) such that \(f(A)\subset
                        B\).

                        If we have maps of pairs \(f_0, f_1\colon (X,A) \to (Y,B)\) then we
                        write \(f_0\sim f_1\) if there exists \(F\colon(X\times I, A\times I) \to
                        (Y,B)\) such that \(F(x,0) = f_0(x)\) and \(F(x,1) = f_1(x)\).
                    \end{definition}
\begin{definition}[Homotopy groups]\label{definition-5}

                        If \(*\in X\) then the \terminology{\(n\)th homotopy group} is
                        \[\pi_n(X, *) = [(D^n, S^{n-1}) \to (X, \{*\})].\]\end{definition}

                    We now note several properties of this definition:
                    \begin{enumerate}
\item{}\(\pi_0(X, *) =\) set of path components of \(X\).\item{}\(\pi_1(X, *)\) is a group.\(\pi_n(X, *)\) is an abelian group. \item{}\(\pi_n\) is a functor
                            \[
                                \left\{ \begin{subarray}{l}\text{pointed spaces} \\ \text{pointed
                                maps} \end{subarray}\right\} \to \left\{\begin{subarray}{l}
                                \text{groups}\\ \text{group homomorphisms}\end{subarray}\right\}.
                            \]
                            So given
                            \[f\colon(X, p)\to(Y,q)\]
                            we get
                            \[f_*\colon \pi_n(X,p)\to\pi_n(y,q)\]
                            defined by
                            \[f_*(\gamma) = f\circ \gamma.\]
                            
                        \end{enumerate}

                %
\begin{example}[Homotopy groups of \(S^2\)]\label{example-3}
\begin{table}
\centering
\begin{tabular}{*{8}{c}}
\(n\)&1&2&3&4&5&6&7\\
\(\pi_n(S^2)\)&0&\(\ZZ\)&\(\ZZ\)&\(\ZZ/2\)&\(\ZZ/2\)&\(\ZZ/12\)&\(\ZZ/15\)\\
\end{tabular}
\end{table}
\end{example}
\typeout{************************************************}
\typeout{Section 3 Homology}
\typeout{************************************************}
\section[Homology]{Homology}\label{section-3}

                Our goal is to construct a functor \(\H\) from the category of topological spaces and continuous maps to the category of \(\ZZ\)-modules and \(\ZZ\)-linear maps.
                This means to each space \(X\) we associate an abelian group \(\H(X) = \bigoplus_{n\ge 0}H_n(X)\), and to each map \(f\colon X \to Y\) a function \(f_*\colon H_n(X) \to H_n(Y)\) satisfying \((1_X)_* = 1_{H_n(X)}\) and \((f\circ g)_* = f_* \circ g_*\).
            %
\par

                Some properties we would like to have for our construction are:
                \begin{enumerate}
\item{}Homotopy invariance, if \(f \sim g \colon X \to Y\) then \(f_* = g_*\).\item{}The dimension axiom, \(H_n(X) = 0\) for any \(n > \dim X\).\end{enumerate}

            %
\typeout{************************************************}
\typeout{Subsection 3.1 Chain complexes}
\typeout{************************************************}
\subsection[Chain complexes]{Chain complexes}\label{subsection-3}
\begin{definition}[Chain complex]\label{definition-6}

                        If \(R\) is a commutative ring then a \terminology{chain complex} over \(R\) is a pair \((C,d)\) satisfying:
                        \begin{enumerate}
\item{}\(C = \bigoplus_{n\in\ZZ} C_n\) for \(R\)-modules \(C_n\).\item{}\(d\colon C \to C\) where \(d = \bigoplus d_n\) for \(R\)-linear maps \(d_n\).\item{}\(d\circ d = 0\).\end{enumerate}

                        The indexing by \(n\) is called a \terminology{grading}.
                        Usually we take \(C_n = 0\) for \(n \lt 0\).
                        An element of \(\ker d_n\) is called \terminology{closed} or a \terminology{cycle}.
                        An element of \(\im d_n\) is called a \terminology{boundary}.
                        \(d\) is the \terminology{boundary map} or \terminology{differential}.
                    \end{definition}
\begin{definition}[Homology groups]\label{definition-7}

                        If \((C,d)\) is a chain complex, its \terminology{\(n\)th homology group} is
                        \[H_n(C,d) = \ker d_n/\im d_{n+1}.\]
                        If \(x \in \ker d_n\) we write \([x]\) for its image in \(H_n(C)\).
                    \end{definition}
\begin{example}\label{example-4}
\begin{enumerate}
\item{}
                            \(C_0 = C_1 = \ZZ\), \(C_i = 0\) otherwise,
                            \[0\to \ZZ \xrightarrow{\cdot 3} \ZZ \to 0.\]
                            Then \(H_1 = 0\), \(H_0 = \ZZ/3\).
                        \item{}
                            \[\ZZ\]
                        \end{enumerate}
\end{example}
\typeout{************************************************}
\typeout{Section 1 Notation}
\typeout{************************************************}
\section*{Notation}\label{section-4}
\addcontentsline{toc}{section}{Notation}
\begin{longtable}[l]{llr}
\textbf{Symbol}&\textbf{Description}&\textbf{Page}\\[1em]
\endfirsthead
\textbf{Symbol}&\textbf{Description}&\textbf{Page}\\[1em]
\endhead
\multicolumn{3}{r}{(Continued on next page)}\\
\endfoot
\endlastfoot
$$&&\pageref{notation-1}\\
\end{longtable}
\end{document}