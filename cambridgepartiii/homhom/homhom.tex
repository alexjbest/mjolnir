%%                                    %%
%% Generated from MathBook XML source %%
%%                                    %%
%%   http://mathbook.pugetsound.edu   %%
%%                                    %%
\documentclass[10pt,]{book}
%% Load geometry package to allow page margin adjustments
\usepackage{geometry}
\geometry{letterpaper,total={5.0in,9.0in}}
%% Custom Preamble Entries, early (use latex.preamble.early)
%% Inline math delimiters, \(, \), made robust with next package
\usepackage{fixltx2e}
%% Page Layout Adjustments (latex.geometry)
%% For unicode character support, use the "xelatex" executable
%% If never using xelatex, the next three lines can be removed
\usepackage{ifxetex}
\ifxetex\usepackage{xltxtra}\fi
%% Symbols, align environment, bracket-matrix
\usepackage{amsmath}
\usepackage{amssymb}
%% extpfeil package for certain extensible arrows,
%% as also provided by MathJax extension of the same name
\usepackage{extpfeil}
%% allow more columns to a matrix
%% can make this even bigger by overiding with  latex.preamble.late  processing option
\setcounter{MaxMatrixCols}{30}
%% XML, MathJax Conflict Macros
%% Two nonstandard macros that MathJax supports automatically
%% so we always define them in order to allow their use and
%% maintain source level compatibility
%% This avoids using two XML entities in source mathematics
\newcommand{\lt}{<}
\newcommand{\gt}{>}
%% Semantic Macros
%% To preserve meaning in a LaTeX file
%% Only defined here if required in this document
%% Used for inline definitions of terms
\newcommand{\terminology}[1]{\textbf{#1}}
%% Subdivision Numbering, Chapters, Sections, Subsections, etc
%% Subdivision numbers may be turned off at some level ("depth")
%% A section *always* has depth 1, contrary to us counting from the document root
%% The latex default is 3.  If a larger number is present here, then
%% removing this command may make some cross-references ambiguous
%% The precursor variable $numbering-maxlevel is checked for consistency in the common XSL file
\setcounter{secnumdepth}{3}
%% Environments with amsthm package
%% Theorem-like enviroments in "plain" style, with or without proof
\usepackage{amsthm}
\theoremstyle{plain}
%% Numbering for Theorems, Conjectures, Examples, Figures, etc
%% Controlled by  numbering.theorems.level  processing parameter
%% Always need a theorem environment to set base numbering scheme
%% even if document has no theorems (but has other environments)
\newtheorem{theorem}{Theorem}[section]
\renewcommand*{\proofname}{Proof}%% Only variants actually used in document appear here
%% Numbering: all theorem-like numbered consecutively
%% i.e. Corollary 4.3 follows Theorem 4.2
\newtheorem{corollary}[theorem]{Corollary}
%% Definition-like environments, normal text
%% Numbering for definition, examples is in sync with theorems, etc
%% also for free-form exercises, not in exercise sections
\theoremstyle{definition}
\newtheorem{definition}[theorem]{Definition}
\newtheorem{example}[theorem]{Example}
%% Equation Numbering
%% Controlled by  numbering.equations.level  processing parameter
\numberwithin{equation}{section}
%% Raster graphics inclusion, wrapped figures in paragraphs
\usepackage{graphicx}
%% Colors for Sage boxes and author tools (red hilites)
\usepackage[usenames,dvipsnames,svgnames,table]{xcolor}
%% hyperref driver does not need to be specified
\usepackage{hyperref}
%% Hyperlinking active in PDFs, all links solid and blue
\hypersetup{colorlinks=true,linkcolor=blue,citecolor=blue,filecolor=blue,urlcolor=blue}
\hypersetup{pdftitle={Part III Homological and Homotopical Algebra 2014}}
%% If you manually remove hyperref, leave in this next command
\providecommand\phantomsection{}
%% Custom Preamble Entries, late (use latex.preamble.late)
\usepackage[all]{xy}
%% Convenience macros
\DeclareMathOperator{\Hom}{Hom}
\DeclareMathOperator{\Tor}{Tor}
\DeclareMathOperator{\Ext}{Ext}
\DeclareMathOperator{\id}{id}
\DeclareMathOperator{\im}{im}
\DeclareMathOperator{\coim}{coim}
\DeclareMathOperator{\coker}{coker}
\DeclareMathOperator{\eq}{eq}
\DeclareMathOperator{\coeq}{coeq}
\DeclareMathOperator{\Ch}{Ch}
\DeclareMathOperator{\Abgp}{\text{Abgp}}
\DeclareMathOperator{\CC}{\mathbf{C}}
\DeclareMathOperator{\PP}{\mathbf{P}}
\DeclareMathOperator{\QQ}{\mathbf{Q}}
\DeclareMathOperator{\RR}{\mathbf{R}}
\DeclareMathOperator{\ZZ}{\mathbf{Z}}
\DeclareMathOperator{\Rmod}{R\text{mod}}
\DeclareMathOperator{\cA}{\mathcal{A}}
\DeclareMathOperator{\cB}{\mathcal{B}}
\DeclareMathOperator{\cO}{\mathcal{O}}
%% Title page information for book
\title{Part III Homological and Homotopical Algebra 2014}
\author{}
\date{}
\begin{document}
\frontmatter
%% half-title
\thispagestyle{empty}
\vspace*{\stretch{1}}
\begin{center}
{\Huge Part III Homological and Homotopical Algebra 2014}
\end{center}\par
\vspace*{\stretch{2}}
\clearpage
\thispagestyle{empty}
\clearpage
\maketitle
\clearpage
\thispagestyle{empty}
\vspace*{\stretch{2}}
\vspace*{\stretch{1}}
\clearpage
\setcounter{tocdepth}{1}
\renewcommand*\contentsname{Contents}
\tableofcontents
\mainmatter
\typeout{************************************************}
\typeout{Chapter 1 Elements of Homological Algebra}
\typeout{************************************************}
\chapter[Elements of Homological Algebra]{Elements of Homological Algebra}\label{chap-hom-alg}
\typeout{************************************************}
\typeout{Section 1.1 Introduction}
\typeout{************************************************}
\section[Introduction]{Introduction}\label{sec-introduction}
These are lecture notes for the 2014 Part III Homological and Homotopical Algebra course taught by Dr. Julian Holstein, these notes are part of \href{https://alexjbest.github.io/mjolnir/}{MJOLNIR}.%
\par
The recommended books are: 
          \begin{itemize}
\item{}W. G. Dwyer and J. Spalinski, Homotopy theories and model categories\item{}S. I. Gelfand and Yu. I. Manin, Methods of Homological Algebra\item{}C. Weibel, An introduction to homological algebra\end{itemize}

        %
\par

          Generated: January 20, 2015, 21:35:01 (Z)
        %
\typeout{************************************************}
\typeout{Section 1.2 Motivation}
\typeout{************************************************}
\section[Motivation]{Motivation}\label{sec-motivation}
Start with a graded ring \(\CC[x_0,\ldots,x_n]\) with \(\deg x_i = 1\).
          Consider a graded module \(M = \bigoplus_d M_d\) over \(R\).
          Hilbert looked at the map \(d\mapsto H_M (d)= \dim_{\CC} M_d\).
          For example we can take \(R\) to be the homogeneous coordinate ring of \(\PP^n\) and \(V(I)\subset \PP^n\) a subvariety where \(I\) is a homogeneous ideal.
          We then take \(M = R/I\), if \(V\) is a curve \(C\) then \(H_{R/I}(d) = \deg(V)\cdot d + (1 -g(C))\).
          Hilbert showed that the function \(H_M(d)\) is eventually polynomial.
          We can compute this function easily if \(M\) is free so we try to replace \(M\) by free modules.
          First we take
          \[K_0\to F_0 \to M\]
          where \(K_0\) is the kernel of the surjective map from \(F_0\) to \(M\).
          We can continue this getting
          \begin{gather*}
K_1\to F_1 \to K_0\\
K_2\to F_2 \to K_1\\
\vdots
\end{gather*}
          we can then write
          \[\cdots \to F_2\to F_1\to F_0 \to M \to 0,\]
          this is a free resolution of \(M\).
          We also have the following.
        %
\begin{theorem}[Hilbert]\label{theorem-1}
\(F_{n+1} = 0\).\end{theorem}
\begin{corollary}\label{corollary-1}
\(H_M(d) = \sum_i (-1)^i H_{F_i}(d)\).\end{corollary}
\typeout{************************************************}
\typeout{Section 1.3 Categorical notions}
\typeout{************************************************}
\section[Categorical notions]{Categorical notions}\label{sec-cats}
\typeout{************************************************}
\typeout{Subsection 1.3.1 Abelian Categories}
\typeout{************************************************}
\subsection[Abelian Categories]{Abelian Categories}\label{sec-ab-cats}
\begin{example}\label{example-1}
\(\Rmod\) - the category of left \(R\)-modules for \(R\) an associative ring is an abelian category.  \end{example}
\begin{example}\label{example-2}
The categories of sheaves of abelian groups on a topological space, sheaves of \(\cO\)-modules on a scheme and (quasi-)coherent sheaves on a scheme are all abelian.\end{example}
\begin{definition}[Additive categories]\label{definition-1}
An \terminology{additive category} is a category in which:
              \begin{enumerate}
\item{}Every hom-space has the structure of an abelian group.\item{}There exists a 0-object (one with exactly one map to and from every other object).\item{}Finite products exist (these are automatically equal to sums \(A\times B = A \oplus B = A \amalg B\)).\end{enumerate}

              In such a category we let \[\ker(f) = \eq(\xymatrix@+=2pc{A \ar@<0.5ex>[r]^f \ar@<-0.5ex>[r]_0 & B})\] and \[\coker(f) = \coeq(\xymatrix@+=2pc{A \ar@<0.5ex>[r]^f \ar@<-0.5ex>[r]_0 & B}).\]\end{definition}
\begin{definition}[Abelian categories]\label{definition-2}
An \terminology{abelian category}\(\cA\) is an additive category in which:
              \begin{enumerate}
\item{}Every map \(f\) has a kernel and cokernel.\item{}For all \(f\) we have \(\coker(\ker(f)) = \im(f) = \coim(f) = \ker(\coker(f))\).\end{enumerate}
\end{definition}
\begin{example}\label{example-3}
Let \(\cB\) be the category of pairs of vector spaces \(V\subset W\), with morphisms the compatible linear maps.
              Consider the natural map \(f\colon 0\subset V \to V\subset V\), we then have \(\im f \cong 0\subset V\) but \(\coim f \cong V\subset V\).
              So this category is not abelian.
            \end{example}
From now on we take \(\cA\) to be any abelian category.%
\typeout{************************************************}
\typeout{Subsection 1.3.2 Exactness}
\typeout{************************************************}
\subsection[Exactness]{Exactness}\label{sec-exactness}
\begin{definition}[Exact sequences]\label{definition-3}
A sequence of morphisms \[A\xrightarrow{f} B \xrightarrow{g}C\] in \(\cA\) is \terminology{exact at \(B\)} if \(\im f = \ker g\).
              A sequence is then exact if it is exact everywhere.
              An exact sequence of the form \[0\to A \to B \to C \to 0\] is called a \terminology{short exact sequence}.
            \end{definition}
\begin{definition}[Mono and epi morphisms]\label{definition-4}
A morphism \(f\) is a \terminology{monomorphism} if \(fg = fh \implies g=h\) and it is an \terminology{epimorphism} is \(gf = hf \implies g=h\).\end{definition}
\begin{example}\label{example-4}
In \(\Abgp\) the following are exact sequences:
              \begin{gather*}
0\to \ZZ/2 \to \ZZ/2 \oplus \ZZ/2 \to \ZZ/2 \to 0\\
0\to \ZZ/2 \to \ZZ/4\to \ZZ/2 \to 0\\
0\to \ZZ \xrightarrow{\cdot 3} \ZZ\to \ZZ/3 \to 0
\end{gather*}\end{example}
\begin{definition}[Additive functors]\label{definition-5}
A functor of additive categories is \terminology{additive} if it is a homomorphism on hom-sets.\end{definition}
\typeout{************************************************}
\typeout{Section 1.4 Chain complexes}
\typeout{************************************************}
\section[Chain complexes]{Chain complexes}\label{sec-chain-complexes}
\begin{definition}[Chain complexes]\label{definition-6}
A \terminology{chain complex}\(C_\bullet\) is a collection of objects \((C_i)_{i\in \ZZ}\) in \(\cA\) with maps \(d_i\colon C_i \to C_{i-1}\) such that \(d_{i-1}\circ d_i = 0\).\end{definition}
\begin{definition}[Cycles, boundaries, homology objects]\label{definition-7}
We define the \terminology{cycles}\(Z_i = \ker d_i\) and \terminology{boundaries}\(B_i= \im d_{i+1}\) and the \(i\)th \terminology{homology object}\(H_i(C) = \coker(B_i\to Z_i)\).
            A complex is \terminology{acyclic} if it is exact i.e. \(H_\bullet(C) = 0\).
          \end{definition}
\begin{definition}[Cochain complexes]\label{definition-8}
A \terminology{cochain complex}\(C^\bullet\) is a collection of objects \((C^i)_{i\in \ZZ}\) in \(\cA\) with maps \(d_i\colon C_i \to C_{i+1}\) such that \(d_{i+1}\circ d_i = 0\).
            We then have as above \(H^i\) the \(i\)th \terminology{cohomology object}.
          \end{definition}
We can switch between chain complexes and cochain complexes via \(C^i = C_{-i}\).%
\begin{example}\label{example-5}
We have many such complexes:
            \begin{itemize}
\item{}Singular (co-)chain complex on a top space.\item{}de Rahm complex.\item{}Cellular chain complex.\item{}Flabby resolution of a sheaf.\item{}Bar resolution of a group.\item{}Koszul complex.\end{itemize}
\end{example}
\begin{definition}[Chain maps]\label{definition-9}
Given \(B,C\) chain complexes, a \terminology{chain map}\(f\colon B\to C\) is  a collection of maps \(f_i\colon B_i \to C_i\) such that \(df=fd.\)\end{definition}
\par
We now have formed the \terminology{category of chain complexes} \(\Ch(\cA)\) using these maps.%
\typeout{************************************************}
\typeout{Chapter 2 Applications}
\typeout{************************************************}
\chapter[Applications]{Applications}\label{chap-apps}
\typeout{************************************************}
\typeout{Chapter 3 Spectral Sequences}
\typeout{************************************************}
\chapter[Spectral Sequences]{Spectral Sequences}\label{chap-spec-seq}
\typeout{************************************************}
\typeout{Chapter 4 Model Categories}
\typeout{************************************************}
\chapter[Model Categories]{Model Categories}\label{chap-model-cats}
%
\backmatter
%
\end{document}