%%                                    %%
%% Generated from MathBook XML source %%
%%    on 2015-02-28T15:28:42Z    %%
%%                                    %%
%%   http://mathbook.pugetsound.edu   %%
%%                                    %%
\documentclass[10pt,]{book}
%% Load geometry package to allow page margin adjustments
\usepackage{geometry}
\geometry{letterpaper,total={5.0in,9.0in}}
%% Custom Preamble Entries, early (use latex.preamble.early)
%% Inline math delimiters, \(, \), made robust with next package
\usepackage{fixltx2e}
%% Page Layout Adjustments (latex.geometry)
%% For unicode character support, use the "xelatex" executable
%% If never using xelatex, the next three lines can be removed
\usepackage{ifxetex}
\ifxetex\usepackage{xltxtra}\fi
%% Symbols, align environment, bracket-matrix
\usepackage{amsmath}
\usepackage{amssymb}
%% extpfeil package for certain extensible arrows,
%% as also provided by MathJax extension of the same name
\usepackage{extpfeil}
%% allow more columns to a matrix
%% can make this even bigger by overiding with  latex.preamble.late  processing option
\setcounter{MaxMatrixCols}{30}
%% XML, MathJax Conflict Macros
%% Two nonstandard macros that MathJax supports automatically
%% so we always define them in order to allow their use and
%% maintain source level compatibility
%% This avoids using two XML entities in source mathematics
\newcommand{\lt}{<}
\newcommand{\gt}{>}
%% Semantic Macros
%% To preserve meaning in a LaTeX file
%% Only defined here if required in this document
%% Used for inline definitions of terms
\newcommand{\terminology}[1]{\textbf{#1}}
%% Subdivision Numbering, Chapters, Sections, Subsections, etc
%% Subdivision numbers may be turned off at some level ("depth")
%% A section *always* has depth 1, contrary to us counting from the document root
%% The latex default is 3.  If a larger number is present here, then
%% removing this command may make some cross-references ambiguous
%% The precursor variable $numbering-maxlevel is checked for consistency in the common XSL file
\setcounter{secnumdepth}{3}
%% Environments with amsthm package
%% Theorem-like enviroments in "plain" style, with or without proof
\usepackage{amsthm}
\theoremstyle{plain}
%% Numbering for Theorems, Conjectures, Examples, Figures, etc
%% Controlled by  numbering.theorems.level  processing parameter
%% Always need a theorem environment to set base numbering scheme
%% even if document has no theorems (but has other environments)
\newtheorem{theorem}{Theorem}[section]
\renewcommand*{\proofname}{Proof}%% Only variants actually used in document appear here
%% Numbering: all theorem-like numbered consecutively
%% i.e. Corollary 4.3 follows Theorem 4.2
\newtheorem{corollary}[theorem]{Corollary}
\newtheorem{lemma}[theorem]{Lemma}
\newtheorem{proposition}[theorem]{Proposition}
%% Definition-like environments, normal text
%% Numbering for definition, examples is in sync with theorems, etc
%% also for free-form exercises, not in exercise sections
\theoremstyle{definition}
\newtheorem{definition}[theorem]{Definition}
\newtheorem{example}[theorem]{Example}
\newtheorem{remark}[theorem]{Remark}
%% Equation Numbering
%% Controlled by  numbering.equations.level  processing parameter
\numberwithin{equation}{section}
%% Figures, Tables, Floats
%% The [H]ere option of the float package fixes floats in-place,
%% in deference to web usage, where floats are totally irrelevant
%% We redefine the figure and table environments, if used
%%   1) New mbxfigure and/or mbxtable environments are defined with float package
%%   2) Standard LaTeX environments redefined to use new environments
%%   3) Standard LaTeX environments redefined to step theorem counter
%%   4) Counter for new enviroments is set to the theorem counter before caption
%% You can remove all this figure/table setup, to restore standard LaTeX behavior
%% HOWEVER, numbering of figures/tables AND theorems/examples/remarks, etc
%% WILL ALL de-synchronize with the numbering in the HTML version
%% You can remove the [H] argument of the \newfloat command, to allow flotation and 
%% preserve numbering, BUT the numbering may then appear "out-of-order"
\usepackage{float}
\newfloat{mbxfigure}{H}{lof}[section]
\floatname{mbxfigure}{Figure}
\renewenvironment{figure}%
{\begin{mbxfigure}\setcounter{mbxfigure}{\value{theorem}}\stepcounter{theorem}}%
{\end{mbxfigure}}
%% Raster graphics inclusion, wrapped figures in paragraphs
\usepackage{graphicx}
%% Colors for Sage boxes and author tools (red hilites)
\usepackage[usenames,dvipsnames,svgnames,table]{xcolor}
%% hyperref driver does not need to be specified
\usepackage{hyperref}
%% Hyperlinking active in PDFs, all links solid and blue
\hypersetup{colorlinks=true,linkcolor=blue,citecolor=blue,filecolor=blue,urlcolor=blue}
\hypersetup{pdftitle={Part III Homological and Homotopical Algebra 2014}}
%% If you manually remove hyperref, leave in this next command
\providecommand\phantomsection{}
%% Custom Preamble Entries, late (use latex.preamble.late)
\usepackage[all]{xy}
%% Convenience macros
\usepackage{mathrsfs}
\DeclareMathOperator{\id}{id}
\DeclareMathOperator{\Coh}{Coh}
\DeclareMathOperator{\Hom}{Hom}
\DeclareMathOperator{\cHom}{\underline{Hom}}
\DeclareMathOperator{\Tor}{Tor}
\DeclareMathOperator{\Ext}{Ext}
\DeclareMathOperator{\inj}{inj}
\DeclareMathOperator{\im}{im}
\DeclareMathOperator{\coim}{coim}
\DeclareMathOperator{\coker}{coker}
\DeclareMathOperator{\eq}{eq}
\DeclareMathOperator{\op}{op}
\DeclareMathOperator{\ob}{ob}
\DeclareMathOperator{\coeq}{coeq}
\DeclareMathOperator{\cone}{cone}
\DeclareMathOperator{\Sing}{Sing}

\DeclareMathOperator{\Ch}{Ch}
\DeclareMathOperator{\Ab}{\text{Ab}}
\DeclareMathOperator{\Top}{\text{Top}}
\DeclareMathOperator{\Rmod}{R\text{mod}}
\DeclareMathOperator{\SSet}{\text{SSet}}
\DeclareMathOperator{\sAb}{\text{sAb}}
\DeclareMathOperator{\cA}{\mathcal{A}}
\DeclareMathOperator{\cB}{\mathcal{B}}
\DeclareMathOperator{\cC}{\mathcal{C}}
\DeclareMathOperator{\cO}{\mathcal{O}}
\DeclareMathOperator{\cM}{\mathcal{M}}
\DeclareMathOperator{\sC}{\mathscr{C}}
\DeclareMathOperator{\sF}{\mathscr{F}}
\DeclareMathOperator{\sW}{\mathscr{W}}

\DeclareMathOperator{\dd}{\partial}
\DeclareMathOperator{\rd}{\mathrm{d}}

% Rings
\DeclareMathOperator{\Mat}{Mat}
\DeclareMathOperator{\CC}{\mathbf{C}}
\DeclareMathOperator{\PP}{\mathbf{P}}
\DeclareMathOperator{\QQ}{\mathbf{Q}}
\DeclareMathOperator{\RR}{\mathbf{R}}
\DeclareMathOperator{\ZZ}{\mathbf{Z}}
%% Title page information for book
\title{Part III Homological and Homotopical Algebra 2014}
\author{}
\date{}
\begin{document}
\frontmatter
%% half-title
\thispagestyle{empty}
\vspace*{\stretch{1}}
\begin{center}
{\Huge Part III Homological and Homotopical Algebra 2014}
\end{center}\par
\vspace*{\stretch{2}}
\clearpage
\thispagestyle{empty}
\clearpage
\maketitle
\clearpage
\thispagestyle{empty}
\vspace*{\stretch{2}}
\vspace*{\stretch{1}}
\clearpage
\setcounter{tocdepth}{1}
\renewcommand*\contentsname{Contents}
\tableofcontents
\mainmatter
\typeout{************************************************}
\typeout{Chapter 1 Elements of Homological Algebra}
\typeout{************************************************}
\chapter[Elements of Homological Algebra]{Elements of Homological Algebra}\label{chap-hom-alg}
\typeout{************************************************}
\typeout{Section 1.1 Introduction}
\typeout{************************************************}
\section[Introduction]{Introduction}\label{sec-introduction}
These are lecture notes for the 2014 Part III Homological and Homotopical Algebra course taught by Dr. Julian Holstein, these notes are part of \href{https://alexjbest.github.io/mjolnir/}{Mjolnir}.%
\par
The recommended books are: 
          \begin{itemize}
\item{}W. G. Dwyer and J. Spalinski, Homotopy theories and model categories\item{}S. I. Gelfand and Yu. I. Manin, Methods of Homological Algebra\item{}C. Weibel, An introduction to homological algebra\end{itemize}

        %
\par

          Generated: February 28, 2015, 15:28:42 (Z)
        %
\typeout{************************************************}
\typeout{Section 1.2 Motivation}
\typeout{************************************************}
\section[Motivation]{Motivation}\label{sec-motivation}
Start with a graded ring \(\CC[x_0,\ldots,x_n]\) with \(\deg x_i = 1\).
          Consider a graded module \(M = \bigoplus_d M_d\) over \(R\).
          Hilbert looked at the map \(d\mapsto H_M (d)= \dim_{\CC} M_d\).
          For example we can take \(R\) to be the homogeneous coordinate ring of \(\PP^n\) and \(V(I)\subset \PP^n\) a subvariety where \(I\) is a homogeneous ideal.
          We then take \(M = R/I\), if \(V\) is a curve \(C\) then \(H_{R/I}(d) = \deg(V)\cdot d + (1 -g(C))\).
          Hilbert showed that the function \(H_M(d)\) is eventually polynomial.
          We can compute this function easily if \(M\) is free so we try to replace \(M\) by free modules.
          First we take
          \[K_0\to F_0 \to M\]
          where \(K_0\) is the kernel of the surjective map from \(F_0\) to \(M\).
          We can continue this getting
          \begin{gather*}
K_1\to F_1 \to K_0\\
K_2\to F_2 \to K_1\\
\vdots
\end{gather*}
          we can then write
          \[\cdots \to F_2\to F_1\to F_0 \to M \to 0,\]
          this is a free resolution of \(M\).
          We also have the following.
        %
\begin{theorem}[Hilbert]\label{theorem-1}
\(F_{n+1} = 0\).\end{theorem}
\begin{corollary}\label{corollary-1}
\(H_M(d) = \sum_i (-1)^i H_{F_i}(d)\).\end{corollary}
\typeout{************************************************}
\typeout{Section 1.3 Categorical notions}
\typeout{************************************************}
\section[Categorical notions]{Categorical notions}\label{sec-cats}
\typeout{************************************************}
\typeout{Subsection 1.3.1 Abelian Categories}
\typeout{************************************************}
\subsection[Abelian Categories]{Abelian Categories}\label{sec-ab-cats}
\begin{example}\label{example-1}
\(\Rmod\) - the category of left \(R\)-modules for \(R\) an associative ring is an abelian category.  \end{example}
\begin{example}\label{example-2}
The categories of sheaves of abelian groups on a topological space, sheaves of \(\cO\)-modules on a scheme and (quasi-)coherent sheaves on a scheme are all abelian.\end{example}
\begin{definition}[Additive categories]\label{definition-1}
An \terminology{additive category} is a category in which:
              \begin{enumerate}
\item{}Every hom-space has the structure of an abelian group.\item{}There exists a 0-object (one with exactly one map to and from every other object).\item{}Finite products exist (these are automatically equal to sums \(A\times B = A \oplus B = A \amalg B\)).\end{enumerate}

              In such a category we let \[\ker(f) = \eq(\xymatrix@+=2pc{A \ar@<0.5ex>[r]^f \ar@<-0.5ex>[r]_0 & B})\] and \[\coker(f) = \coeq(\xymatrix@+=2pc{A \ar@<0.5ex>[r]^f \ar@<-0.5ex>[r]_0 & B}).\]\end{definition}
\begin{definition}[Abelian categories]\label{definition-2}
An \terminology{abelian category}\(\cA\) is an additive category in which:
              \begin{enumerate}
\item{}Every map \(f\) has a kernel and cokernel.\item{}For all \(f\) we have \(\coker(\ker(f)) = \im(f) = \coim(f) = \ker(\coker(f))\).\end{enumerate}
\end{definition}
\begin{example}\label{example-3}
Let \(\cB\) be the category of pairs of vector spaces \(V\subset W\), with morphisms the compatible linear maps.
              Consider the natural map \(f\colon 0\subset V \to V\subset V\), we then have \(\im f \cong 0\subset V\) but \(\coim f \cong V\subset V\).
              So this category is not abelian.
            \end{example}
From now on we take \(\cA\) to be any abelian category.%
\typeout{************************************************}
\typeout{Subsection 1.3.2 Exactness}
\typeout{************************************************}
\subsection[Exactness]{Exactness}\label{sec-exactness}
\begin{definition}[Exact sequences]\label{definition-3}
A sequence of morphisms \[A\xrightarrow{f} B \xrightarrow{g}C\] in \(\cA\) is \terminology{exact at \(B\)} if \(\im f = \ker g\).
              A sequence is then exact if it is exact everywhere.
              An exact sequence of the form \[0\to A \to B \to C \to 0\] is called a \terminology{short exact sequence}.
            \end{definition}
\begin{definition}[Mono and epi morphisms]\label{definition-4}
A morphism \(f\) is a \terminology{monomorphism} if \(fg = fh \implies g=h\) and it is an \terminology{epimorphism} if \(gf = hf \implies g=h\).\end{definition}
\begin{example}\label{example-4}
In \(\Ab\) the following are exact sequences:
              \begin{gather*}
0\to \ZZ/2 \to \ZZ/2 \oplus \ZZ/2 \to \ZZ/2 \to 0\\
0\to \ZZ/2 \to \ZZ/4\to \ZZ/2 \to 0\\
0\to \ZZ \xrightarrow{\cdot 3} \ZZ\to \ZZ/3 \to 0
\end{gather*}\end{example}
\begin{definition}[Additive functors]\label{definition-5}
A functor of additive categories is \terminology{additive} if it is a homomorphism on hom-sets.\end{definition}
\typeout{************************************************}
\typeout{Section 1.4 Chain complexes}
\typeout{************************************************}
\section[Chain complexes]{Chain complexes}\label{sec-chain-complexes}
\begin{definition}[Chain complexes]\label{definition-6}
A \terminology{chain complex}\(C_\bullet\) is a collection of objects \((C_i)_{i\in \ZZ}\) in \(\cA\) with maps \(d_i\colon C_i \to C_{i-1}\) such that \(d_{i-1}\circ d_i = 0\).\end{definition}
\begin{definition}[Cycles, boundaries, homology objects]\label{definition-7}
We define the \terminology{cycles}\(Z_i = \ker d_i\) and \terminology{boundaries}\(B_i= \im d_{i+1}\) and the \(i\)th \terminology{homology object}\(H_i(C) = \coker(B_i\to Z_i)\).
            A complex is \terminology{acyclic} if it is exact i.e. \(H_\bullet(C) = 0\).
          \end{definition}
\begin{definition}[Cochain complexes]\label{definition-8}
A \terminology{cochain complex}\(C^\bullet\) is a collection of objects \((C^i)_{i\in \ZZ}\) in \(\cA\) with maps \(d_i\colon C_i \to C_{i+1}\) such that \(d_{i+1}\circ d_i = 0\).
            We then have as above \(H^i\) the \(i\)th \terminology{cohomology object}.
          \end{definition}
We can switch between chain complexes and cochain complexes via \(C^i = C_{-i}\).%
\begin{example}\label{example-5}
We have many such complexes:
            \begin{itemize}
\item{}Singular (co-)chain complex on a top space.\item{}de Rahm complex.\item{}Cellular chain complex.\item{}Flabby resolution of a sheaf.\item{}Bar resolution of a group.\item{}Koszul complex.\end{itemize}
\end{example}
\begin{definition}[Chain maps]\label{definition-9}
Given \(B,C\) chain complexes, a \terminology{chain map}\(f\colon B\to C\) is  a collection of maps \(f_i\colon B_i \to C_i\) such that \(df=fd\).\end{definition}
\par
We now have formed the \terminology{category of chain complexes} \(\Ch(\cA)\) using these maps.
          We write \(\Ch(R)\) for \(\Ch(\Rmod)\).
          Note that \(\Ch(\cA)\) is an additive category moreover it is an abelian category, we can define and check everything level-wise.
          For example \(\ker(A\to B)_n = \ker(A_n\to B_n)\).
          Note that the \(H_n\) form a functor \(\Ch(\cA)\to \cA\).
          Define \(f_*\colon H_n A \to H_n B\) in the natural way and check it works.
          \(H_n\) is additive.
        %
\begin{lemma}[Snake lemma]\label{lemma-snake}
Let \(0\to A \to B \to C \to 0\) be a short exact sequence then there exist natural boundary maps \(\dd_n\) which fit into a long exact sequence of homology objects
            \begin{figure}
\centering
\[
                \xymatrix{ \cdots\ar[r] & H_n(A) \ar[r]^{f_*}&
                H_n(B) \ar[r]^{g_*} & H_n(C)\ar `r[d] `[l]
                `[llld]_{\dd_n} `[dll] [dll]\\
                & H_{n-1}(A) \ar[r] & H_{n-1}(B)
                \ar[r] & H_{n-1}(C)\ar[r] & {\dots} }
              \]\end{figure}
\end{lemma}
\begin{proof}
Exercise.\end{proof}
\par
Naturality here means given two short exact sequences and compatible chain maps the induced maps on homology are compatible with \(\dd_n\).
          (The obvious diagram commutes.)
        %
\par
Recall that \(f\) is a chain map if \(\dd f - f\dd = 0\).%
\begin{definition}[]\label{definition-10}
Let \(\cHom_n(A,B)\) consist of functions \(\{f_i \colon A_i \to B_{i+n}\} \) and define \(df = d\cdot f - (-1)^n fd\) if \(f\in\cHom_n\).
            Check that \[d^2 f = d\cdot (d\cdot f - (-1)^fd ) - (-1) (d\cdot f - (-1)^n f\cdot d)\cdot d = 0.\]\end{definition}
\par
We use the ``Sign rule'' to help with definitions, this states that if \(a\) moves past \(b\) we pick a sign \((-1)^{\deg a\deg b}\).%
\par
\(\Ch(\cA)\) can be enriched over \(\Ch(\ZZ)\).%
\begin{definition}[Shifted complexes]\label{definition-11}
The \terminology{shifted complex}\(C[n]\) for \(C\in \Ch(\cA)\) is defined by \(C[n]_i = C_{n+i}\) and \(d_i^{C[n]} = (-1)^n d_{n+i}^C\).\end{definition}
\par
Note that \(H_i(C) = H_0(C[i])\).%
\par
So a chain map \(f\colon A \to B[n]\) is exactly a cycle in \(\cHom_n(A,B)\).%
\par
Now \(\Hom(A,B) = Z_0(\cHom(A,B))\), so what is \(H_0(\cHom(A,B))\)?%
\begin{definition}[Chain homotopies]\label{definition-12}

            A \terminology{chain homotopy}\(S\) between chain maps \(f,g\colon A \to B\) is a collection \(S_i \colon A_i \to B_i\) such that \(\dd S + S\dd = f-g\).
            Equivalently we could say a map \(A \to B[1]\) such that \(dS = g -f\) (note: not a chain map).
            We write \(f \simeq g\) to denote the fact that \(f\) is chain homotopic to \(g\).
          \end{definition}
\begin{definition}[Chain homotopy equivalences]\label{definition-13}

            Two chain complexes \(A\) and \(B\) are said to be \terminology{chain homotopy equivalent} if there are some \(f\colon A\to b\), \(g\colon B \to A\) such that \(gf \simeq 1_A\) and \(fg\simeq 1_B\).
          \end{definition}
\begin{lemma}\label{lemma-2}
If \(f\simeq g\) then \(f_* = g_*\) on homology.\end{lemma}
\begin{proof}
Check.\end{proof}
\begin{definition}[Quasi-isomorphisms]\label{definition-14}

            A chain map \(f\) inducing isomorphisms on homology is called a \terminology{quasi-isomorphism}.
            Two chains \(A,B\) are quasi-isomorphic if there is a quasi-isomorphism \(A \to B\) and \(B\to A\).
          \end{definition}
\begin{example}\label{example-6}
\begin{figure}
\centering
\[
                \xymatrix{
                  \dots\ar[r] &\ZZ \ar[r]\ar[d]_n &0 \ar[r]\ar[d] &\dots \\
                  \dots\ar[r] &\ZZ \ar[r]^{\text{pr}} &\ZZ \ar[r] &\dots
                }
              \]\end{figure}

            is a quasi-isomorphism.
          \end{example}
\par
Any chain homotopy equivalence is a quasi-isomorphism, the converse is false however.%
\begin{definition}[Cones]\label{definition-15}

            Given \(f\colon A \to B\) we define a chain complex called the \terminology{cone} of \(f\) by \(\cone(f)_n = A_{n-1} \oplus B_n\) with maps \[d = \begin{pmatrix}-d_A & 0 \\ -f & d_B\end{pmatrix}.\]\end{definition}
\par

          Note that there exists a short exact sequence
          \[
            \xymatrix{B \ar[r]_{b\mapsto (0,b)}& \cone(f) \ar[r]_{(a,b)\mapsto -a}& A[-1].}
          \]
          Doing the diagram chase of the \hyperref[lemma-snake]{Snake lemma~\ref*{lemma-snake}} we see that the boundary map is induced by \(f\) on homology i.e. \[f_*\colon H_{n-1}A \to H_{n-1}B.\]
          This proves the following.
        %
\begin{lemma}\label{lemma-3}
\(f\) is a quasi-isomorphism if and only if \(\cone(f)\) is exact.\end{lemma}
\begin{proof}

            Look at the long exact sequence of \(B \to \cone(f) \to A[-1]\)\[
              H_{n}(\cone(f)) \to H_n(A) \xrightarrow{f_*} H_{n-1}(B) \to H_{n-1}(\cone(f)).
            \]\end{proof}
\typeout{************************************************}
\typeout{Section 1.5 Exact Functors}
\typeout{************************************************}
\section[Exact Functors]{Exact Functors}\label{sec-exact}
\begin{definition}[Exact functors]\label{definition-16}
An additive functor \(F\) is \terminology{exact} if it preserves short exact sequences.
            It is \terminology{left exact} if it sends a short exact sequence of the form
            \[
              0\to A \to B \to C \to 0
            \]
            to an exact sequence
            \[
              0 \to FA \to FB \to FC.
            \]
            We have a similar definition for \terminology{right exact}.
          \end{definition}
\begin{example}\label{example-7}

            The functor \(\Hom_{\cA}(M, -)\) is left exact from \(\cA\) to \(\Ab = \ZZ\text{mod}\).
            The functor \(\Hom_{\cA}( -, M)\colon \cA^{\op} \to\Ab\) is left exact.\newline{}\end{example}

          Note that left adjoint functors are right exact as they preserve colimits.
        %
\begin{example}\label{example-8}

            Let \(M\) be an \(R,S\)-bimodule (i.e. a left \(R\)-module and a right \(S\)-module).
            Then for \(A \in S\text{mod}\), \(B\in \Rmod\)\[
              \Hom_{R}(M \otimes_S A, B) \cong Hom_{S}(M, \Hom_R(A,B))
            \]\end{example}
\par

          Clearly not all functors are exact.
          However they all preserve split exact sequences, i.e. those of the form
          \[0 \to A \to A\oplus C \to C.\]
          Because they preserve finite direct sums
          \[
            \xymatrix{
              A \ar@<-0.5ex>[r]_f \ar@{=}[d]& B\ar@<-0.5ex>[l]_r \ar@<-0.5ex>[d]_{(f,g)}  \ar@<-0.5ex>[r]_g & C\ar@<-0.5ex>[l]_s \ar@{=}[d]\\
              A \ar[r]& A \oplus C \ar@<-0.5ex>[u]_{(r,s)}\ar[r] &C
            }
          \]
          \((r,g),(f,s)\) are inverse isomorphisms if and only if \(_B = fr + sg\). 
        %
\typeout{************************************************}
\typeout{Section 1.6 Derived Functors, Introduction}
\typeout{************************************************}
\section[Derived Functors, Introduction]{Derived Functors, Introduction}\label{sec-derived-intro}

          We fix \(\cA\), and \(\Ch(\cA)\).
          If we have some right exact functor \(F\) we obtain exact sequences of the form
          \[FA \to FB \to FC \to 0\]
          and the question arises, can we extend this exact sequence by placing objects to the left of it?\newline{}
          If \(F\) is exact on short exact sequences of complexes we get a long exact sequence of homology \(H_iFA\).
          \(F\) is exact on complexes if it is level wise exact, but \(F\) is exact if it is level wise exact.
          We know \(F\) is exact on split exact sequences.
          So we can try to force a short exact sequence to be exact by replacing objects by complexes.
        %
\begin{definition}[Projective and injective objects]\label{definition-17}

            An object \(M\) is \terminology{projective} if for all epimorphisms \(q\) and maps \(M \xrightarrow{f} B\) there exists a lift making
            \[
              \xymatrix{ & M \ar[d]_f \ar@{-->}[dl]&\\
              A \ar[r]^q &B \ar[r] & 0}
            \]
            commute.
            The dual notion is called \terminology{injective}\[
              \xymatrix{ & I&\\
              0 \ar[r] &B \ar[u] \ar[r] & A  \ar@{-->}[ul]}
            \]\end{definition}
\begin{example}\label{example-9}

            Free modules in \(\Rmod\) are projective.
            In \(\Mat_n(R)\text{-mod}\) the column vectors \(R^n\) form a projective object.
            \(\QQ\) is injective in \(\Ab\).
          \end{example}
\begin{lemma}\label{lemma-proj-inj-split}

            If \(C\) is projective or \(A\) is injective then
            \[
              0 \to A \to B \to C \to 0
            \]
            is split.
          \end{lemma}
\begin{proof}

            (We prove the \(C\) projective case) Consider 
            \[
              \xymatrix{ & C \ar@{=}[d]_f \ar[dl]^s&\\
              B \ar[r]^q &C \ar[r] & 0}
            \]
            then \(gs = 1\).
            Now produce \(r\) such that \(rf = 1_A\), \(fr + sg = 1_B\) and \(rs = 0\).
            Let \(h  = 1 -sg\).
            Now \(gh = 0\) giving that \(h = fr\) by the properties of the kernel.
            \[
              \xymatrix{B \ar[rr]^h\ar[rd]^{\exists ! r} &&B \ar[r]^g &C \\ & A \ar[ur] &}
            \]
            Now check \(rf = 1_A\) and \(rs = 0\).
          \end{proof}
\par

          Note that in \(\Rmod\) this shows projectives are exactly summands of free modules.
          
        %
\begin{definition}[Projective resolutions]\label{definition-18}

            A \terminology{projective resolution}\(P_\bullet \xrightarrow{\epsilon} A\) of \(A\) is a non-negative chain complex such that all \(P_i\) are projective and \(\epsilon\) is a quasi-isomorphism.
            So \(H_i P = 0\) if \(i \gt 0\) and \(H_0 P = A\).
          \end{definition}
\begin{definition}[Derived functors]\label{definition-19}

            The \(i\)th \terminology{left derived functor}\(L_i F(A)\) of a right exact functor \(F\) is defined as \(H_iF(P)\) for some projective resolution \(P\) of \(A\).\newline{}\end{definition}
\par

          Dually we may define injective resolutions \(B \xrightarrow{\sim} I^\bullet\) with \(I \in \Ch^{\ge 0}(\cA)\) and we get \terminology{right derived functors} of a left exact functor,
          \[
            R^i F(B) = H^i (FI ).
          \]
          Note \(L_{\lt 0} F(A) = 0\) and \(L_0= fP_0 / FP_1 = F(P_0/P_1) = F(A)\).
        %
\begin{example}[\(\Tor\)]\label{example-10}

            Define \(\Tor^R_i(A, B)\) to be \(L_i(- \otimes_R B)(A)\).
            Let \(\cA = \Ab\). What is \(\Tor_i(\ZZ/p , B)\)?
            \[
              \xymatrix{\ZZ\ar[d]^p &\\ \ZZ \ar[r]^{\sim} &\ZZ_p}
            \]
            is a projective resolution.
            So \(\Tor_* = H_*(B \xrightarrow{p} B)\) and we have \(\Tor^{\ZZ}_0 (\ZZ/p, B) = B/pB\) and \(\Tor^{\ZZ}_1(\ZZ/p, B) = {}_pB = \{b : pb = 0\}\).
          \end{example}
\begin{example}[\(\Ext\)]\label{example-11}

            Define \(\Ext^i_R(A, B)\) to be \(R^i\Hom_R(-, B)(A)\).
            Injective in \(\Rmod^{\op}\) correspond to projectives in \(\Rmod\).
            So \(\Ext_{\ZZ}^*i(\ZZ/p\ZZ, B) = H_*(B \xrightarrow{p} B)\) hence \(\Ext^0 (\ZZ/p, B) = {}_pB\) and \(\Ext^1(\ZZ/p, B) = B/pB\).
          \end{example}
\typeout{************************************************}
\typeout{Section 1.7 Derived Functors, Proofs}
\typeout{************************************************}
\section[Derived Functors, Proofs]{Derived Functors, Proofs}\label{sec-derived-proofs}
\begin{definition}\label{definition-20}
\(\cA\) has \terminology{enough projectives} if for all \(M\in \cA\) there exists a projective \(P\) such that \(P \to M \to 0\).\end{definition}
\begin{example}\label{example-12}
\(\Rmod\) has enough projectives.
          \end{example}

          Warning: The category of abelian sheaves on a topological space does not have enough projectives in general.
        %
\begin{lemma}\label{lemma-5}

            Projective resolutions exist in \(\cA\) if \(\cA\) has enough projectives.
          \end{lemma}
\begin{proof}

            Let \(A\in \cA\), then there exists
            \[
              0 \to K_0 \to P_0 \to A \to 0
            \]
            and inductively
            \[
              0 \to K_{n+1} \to P_{n+1} \to K_n \to 0
            \]
            with \(P_i\) projective.
            We can splice these together to get 
            \[
              \cdots \to P_2 \to P_1 \to P_0 \to A \to 0.
            \]\end{proof}
\begin{theorem}[Comparison Theorem]\label{theorem-2}

            Let \(\epsilon \colon P \to M\) and \(\eta \colon Q \to N\) be two projective resolutions and let \(f\colon M \to N\) then there exists a lift \(\tilde{f} \colon P \to Q\) (a chain map) unique up to chain homotopy.
            \[
              \xymatrix{
                P_1 \ar[d] & Q_1\ar[d]\\
                P_0 \ar[d] & Q_0\ar[d]\\
                M \ar[r]^{f} & N\\
              }
            \]\end{theorem}
\begin{proof}
Exercise.\end{proof}
\begin{corollary}\label{corollary-2}

            Projective resolutions are well defined up to chain homotopy equivalence and so derived functors are well defined.
          \end{corollary}
\begin{proof}

            Lift the identity to get chain maps in both directions.
            Uniqueness implies that they are inverse up to homotopy.
          \end{proof}
\begin{corollary}\label{corollary-3}
\(L_iF\) are functors.
          \end{corollary}
\begin{lemma}[Horseshoe Lemma]\label{lemma-horse}

            Given a short exact sequence
            \[
              A^1 \to A^2 \to A^3
            \]
            and projective resolutions \(P^1 \to A^1\) and \(P^3 \to A^3\) there exists a projective resolution \(P^2\) of \(A^2\) with \(P^2_i = P^1_i \oplus P^3_i\) and the inclusion and projection maps lift.
            So we have the following situation
            \[
              \xymatrix{
              &P_1^1 \ar[d] & P_1^2 \ar[d]&P_1^3\ar[d]&\\
              &P_0^1 \ar[d]_{\epsilon^1} & P_0^2 \ar[d]_{\epsilon^2}&P_0^3\ar[d]_{\epsilon^3}&\\
              0 \ar[r]& A^1 \ar[r]_i & A^2\ar[r]_p &A^3 \ar[r]&0
              }
            \]\end{lemma}
\begin{proof}

            By induction: To get \(\epsilon^2 \colon P_0^i \oplus P_0^3 \to A^2\) we use \(i\) and \(\epsilon^1\) and a lift of \(p\).
            Now the \hyperref[lemma-snake]{Snake lemma~\ref*{lemma-snake}} shows that \(\coker \epsilon^i\) and \(\ker \epsilon^i\) fit into a long exact sequence and hence \(\coker \epsilon^3 = 0\).
            Now apply the induction assumption to the short exact sequence of kernels 
            \[
              0 \to \ker \epsilon^1 \to \ker \epsilon^2\to \ker \epsilon^3 \to \coker \epsilon^1 = 0.
            \]\end{proof}
\begin{corollary}\label{corollary-4}

            A short exact sequence \(0 \to A \to B \to C \to 0\) in \(\cA\) gives a long exact sequence of left derived functors
            \[
              \to L_2FC \to L_1FA \to L_1 FB \to L_1FC\to FA \to FB \to FC \to 0.
            \]\end{corollary}
\begin{proof}

            Combine \hyperref[lemma-horse]{Horseshoe Lemma~\ref*{lemma-horse}}, \ref{lemma-proj-inj-split} and \hyperref[lemma-snake]{Snake lemma~\ref*{lemma-snake}}.
          \end{proof}
\begin{proposition}\label{proposition-1}

            The boundary map \(\dd\) is natural, i.e. given 
            \[
              \xymatrix{
                A^1 \ar[r]\ar[d]^{f^1} &
                A^2 \ar[r]\ar[d]^{f^2} &
                A^2 \ar[d]^{f^3} \\
                B^1 \ar[r] &
                B^2 \ar[r] &
                B^3 \\
              }
            \]
            we have lifts \(\dd\circ L_i f_3 = L_{i-1} f_1 \circ \dd\).
          \end{proposition}
\begin{proof}
See Weibel theorem 2.4.6 \end{proof}
\par

          Note that there is no extra work needed to do all of this for right derived functors.
        %
\typeout{************************************************}
\typeout{Section 1.8 The Derived Category}
\typeout{************************************************}
\section[The Derived Category]{The Derived Category}\label{sec-derived}

          Idea: We want to talk about complexes up to quasi-isomorphism.
          We will reinterpret derived functors as ways of lifting functors to derived categories.
        %
\begin{remark}\label{remark-1}

            If we add simply inverses of quasi-isomorphisms we get nasty stuff!
          \end{remark}
\begin{definition}[Homotopy categories]\label{definition-21}

            Let the homotopy category \(K(\cA)\) of \(\cA\) have objects the objects of \(\Ch(\cA)\) and morphism the chain homotopy classes of chain maps.
            We can add boundedness conditions to our categories.
            So we let \(\Ch_{+}(\cA)\) be only those chain complexes \(A\) with \(A_n = 0\) when \(n \lt \lt 0\), these are \terminology{bounded below} chain complexes.
            Similarly we define \(\Ch_-(\cA)\) and \(\Ch_b(\cA) = \Ch_-(\cA) \cap \Ch_+(\cA)\).
            We also define \(\Ch^+(\cA)\) etc. for cochain complexes.
            Finally we define \(K^+(\cA)\) etc. in the obvious way.
          \end{definition}
\begin{definition}[Localisations of categories]\label{definition-22}

            Given a category \(\cC\) and a class of morphisms \(S\) define the localisation of \(\cC\) at \(S\) to be a category \(\cC[S^{-1}]\) with a functor \(\cC \xrightarrow{Q} \cC[S^{-1}]\) such that \(Q\) sends any \(s\in S\) to an isomorphism, and also such that \(Q\) is universal with respect to having this property.
            If \(\cC \xrightarrow{R} \cB\) sends \(S\) to isomorphisms then there exists some \(P\) so that we have
            \[
              \xymatrix{
              \cC \ar[rr]^R\ar[dr]_Q &&\cB\\
              &\cC[S^{-1}] \ar[ur]_{P}&
              }
            \]\end{definition}
\begin{definition}\label{definition-23}

            Let \(D(\cA)\), the \terminology{derived category} of \(\cA\), be the localisation of \(K(\cA)\) at the quasi-isomorphisms.
            Similarly define the usual suspects \(D^b = K^b(\cA)[\text{quasi-isomorphisms}]\), etc.
          \end{definition}
\begin{theorem}\label{theorem-3}
\(D(\cA)\) exists.
          \end{theorem}
\begin{proof}
See Weibel 10.3, Gelfand-Manin III 2. \end{proof}
\par

          Although we didn't prove this we should note that we can write morphisms in \(D(\cA)\) as 
          \[
            A \xleftarrow{\sim} A' \xrightarrow{f} B
          \]
          with \(f\in \Hom_{K(\cA)}(A', B)\) and \(q\in \Hom_{K(\cA)}(A', A)\).
        %
\begin{remark}\label{remark-2}
\(D^b(\cA)\) is equivalent to the subcategory of \(D(\cA)\) with cohomology in bounded degrees.
          \end{remark}
\begin{example}\label{example-13}

            Let \(X\) be a scheme, \(\Coh(X)\) the abelian category of coherent sheaves on \(X\).
            Then the derived category of \(X\) is defined to be \(D^b(X) = D^b(\Coh(X))\).
          \end{example}
\par

          Note that \(D(\cA)\) is an additive, but not necessarily abelian category.
        %
\begin{theorem}\label{thm-inj-der-hom}

            Given a complex \(I^\bullet \in K^+(\cA)\) of injective objects and any chain complex \(A^\bullet\) then \[\Hom_{D(\cA)}(A^\bullet , I^\bullet) \cong \Hom_{K(\cA)}(A^\bullet, I^\bullet).\]\end{theorem}
\begin{proof}

            (Sketch) Crucial ingredient: \(\Hom_{K(\cA)}(-, I ^\bullet)\) sends quasi-isomorphisms to quasi-isomorphisms.
            So we can replace 
            \[
              A \xleftarrow{\sim} A' \xrightarrow{} I
            \]
            by \(A \to I\).
            By considering cones it suffices to check that \(\cHom_{K(\cA)}(-, I)\) sends acyclics to complexes homotopy equivalent to \(0\).
            One can construct the homotopy equivalence by hand, using injectivity.
          \end{proof}
\begin{corollary}\label{corollary-5}
\[
              \Hom_{D(\cA)}(A,B[i]) = \Ext^i_{\cA}(A,B).
            \]\end{corollary}
\begin{proof}

            Let \(B \to I^\bullet\) be an injective resolution.
            Then both sides are isomorphic to 
            \[
              \Hom_{K(\cA)}(A,I[i]) = H^0 \cHom_{K(\cA)}(A, I[i]).
            \]\end{proof}
\begin{corollary}\label{corollary-6}

            Assume \(\cA\) has enough injectives and write \(\inj \cA \subset \cA\) for the full subcategory of injective objects.
            Then
            \[
              K^+(\inj \cA) \cong D^+(\cA).
            \]\end{corollary}
\begin{proof}

            We have fully faithfulness by \ref{thm-inj-der-hom}.
            To see that it is essentially surjective write down injective resolutions for complexes (see later).
          \end{proof}
\typeout{************************************************}
\typeout{Section 1.9 Total Derived Functors}
\typeout{************************************************}
\section[Total Derived Functors]{Total Derived Functors}\label{sec-total-derived}
We now interpret/redefine derived functors as lifts to the derived category.%
\begin{definition}\label{definition-24}

            Let \(F\colon\)\end{definition}
\typeout{************************************************}
\typeout{Chapter 2 Applications}
\typeout{************************************************}
\chapter[Applications]{Applications}\label{chap-apps}
\typeout{************************************************}
\typeout{Chapter 3 Spectral Sequences}
\typeout{************************************************}
\chapter[Spectral Sequences]{Spectral Sequences}\label{chap-spec-seq}
\typeout{************************************************}
\typeout{Chapter 4 Homotopical Algebra}
\typeout{************************************************}
\chapter[Homotopical Algebra]{Homotopical Algebra}\label{chap-homotopical}
\typeout{************************************************}
\typeout{Section 4.1 Simplicial Sets}
\typeout{************************************************}
\section[Simplicial Sets]{Simplicial Sets}\label{sec-simplicial-sets}
\begin{definition}[Simplex category]\label{definition-25}

            The \terminology{simplex category}\(\Delta\) has objects the sets \(\{0,\ldots,n\} = [n]\) and morphisms the non-decreasing maps between such sets.
            A simplicial object in a category \(\cC\) is a functor \(\Delta^{\op} \to \cC\).
            These objects form a category, denoted \(s(\cC)\) or just \(s\cC\), where the morphisms are natural transformations.
          \end{definition}
\begin{example}\label{example-14}
\(\SSet\) is the category of simplicial sets, and \(\sAb\) is the category of simplicial groups.
          \end{example}
\begin{example}\label{example-15}

            Given a topological space \(X\), \(\Sing X\) is the singular simplicial set.
            We can then form \(\ZZ\Sing X\), the free abelian group on \((\Sing X)_n\) for each \(n\).
          \end{example}

          The \(i\)th face map \(\epsilon_i\colon[n-1] \to [n]\) is the unique injection only leaving out \(i \in [n]\).
          The \(i\)th degeneracy map \(\eta_i\colon [n+1] \to [n]\) is the unique surjective map mapping two elements to \(i\in [n]\).
        %
\begin{proposition}\label{proposition-2}

            Any \(\alpha\colon [m] \to [n]\) in \(\Delta\) can be factored uniquely as
            \[
              \alpha = \epsilon_{i_1}\cdots \epsilon_{i_k}\eta_{j_1}\cdots \eta_{j_l}.
            \]\end{proposition}
\begin{proof}
See, for example, Weibel.\end{proof}
\par

          Hence for the purposes of understanding a simplicial object it is enough to understand \(A(\epsilon_i) = \dd_i\) and \(A(\eta_j) = \sigma_j\).
          These maps satisfy (after checking!) the relations
          \begin{align*}
\dd_i\dd_j &= \dd_{j-1}\dd_i \text{ if } i \lt j.\\
\sigma_i\sigma_j &= \sigma_{j+1}\sigma_i \text{ if } i \leq j.\\
\dd_i\sigma_j &= \begin{cases}\sigma_{j-1}\dd_i & \text{ if } i \lt j, \\ 1 &\text{ if } i=j,j+1,\\\sigma_i\dd_{j+1} &\text{ if }i \gt j + 1.\end{cases}
\end{align*}
        %
\begin{example}\label{example-16}

            Define the \terminology{standard simplex}\(\Delta[n]\) to be the image of \([n]\) under contravariant Yoneda, i.e. \(\Delta[n]_i = \Hom_{\Delta}(i,[n])\).
            This is universal i n the sense that \(\Delta_n = \Hom_{\SSet}(\Delta[n], A)\).
            We call \(A_m\) the simplices of \(\Delta\) (by Yoneda).
          \end{example}
\begin{example}\label{example-17}
\(\Delta[1]_n\) is the set of maps \([n] \to [1]\), we can write these as \(0\cdots 0 1\cdots 1\) where \(0\) appears \(k\) times and \(1\) occurs \(n-k +1\) times.
            So
            \begin{align*}
\Delta[1]_0 &= \{0, 1\},\\
\Delta[1]_1 &= \{0, 01, 11\},\\
\Delta[1]_2 &= \{000, 001, 011, 111\}.
\end{align*}
            All the expressions here with repeat digits are \(\sigma_i(a)\) for some \(n\), so they all called degenerate.
            We only have 3 non-degenerate maps here.
            \end{example}
\typeout{************************************************}
\typeout{Section 4.2 Chain Complexes}
\typeout{************************************************}
\section[Chain Complexes]{Chain Complexes}\label{sec-simp-chain-complexes}
\begin{definition}[Chain complex of a simplicial set]\label{definition-26}

            Let \(A \in S(\cA)\) and define the associated chain complex \(CA\) to have \(CA_n = A_n\) with differential \(\rd_n = \sum (-1)^i \dd_i\).
          \end{definition}
\begin{example}\label{example-18}
\(C_\bullet(X; \ZZ) = C\ZZ\Sing X\).
          \end{example}
\begin{remark}\label{remark-3}

            Kozul complexes and Cech complexes can be seen as coming from \terminology{semi-simplicial sets}, i.e. those without degeneracies.
          \end{remark}
\begin{definition}[Normalised chain complexes]\label{definition-27}

            The \terminology{normalised chain complex}\(NA\) of a simplicial object is \(NA_n = \bigcap_{i=0}^{n-1} \ker(\dd_i)\) with \(\rd_n = (-1)^n \dd_n\). \end{definition}

          In fact we have \(NA \simeq CA\).
          And then we have:
        %
\begin{theorem}[Dold-Kan]\label{theorem-5}
\(N\) induces an equivalence of categories \(s(\cA) \to \Ch_{\ge 0}(\cA)\).
          \end{theorem}
\begin{proof}
Omitted, idea is to write down an explicit inverse \(\Gamma\) e.g. \(\Gamma C_n = \bigoplus_{[n]\twoheadrightarrow [k]} C_k\).\end{proof}
\typeout{************************************************}
\typeout{Section 4.3 Topological Spaces and More Examples}
\typeout{************************************************}
\section[Topological Spaces and More Examples]{Topological Spaces and More Examples}\label{sec-top-examples}
\begin{example}\label{example-19}

            Let \(\Delta^n\) be the geometric \(n\) simplex
            \[\Delta^n = \{(t_0,\ldots,t_n) \in \RR_{\ge 0}^{n+1} : \sum t_i = 1\}.\]

            Any \([m]\xrightarrow{\alpha} [n] \in \Delta\) induces a set map on the vertices which extends linearly to \(\Delta^m \xrightarrow{\alpha_*} \Delta^n\).
            This makes \(\Delta^{\bullet}\) into a cosimplicial topological space.
            Then \(\Hom_{\Top}(\Delta^{\bullet}, X)\) is naturally a simplicial set.
            It is \(\Sing X\).

            So we have a functor \(\Sing \colon \Top \to \SSet\) and we want an adjoint.
          \end{example}
\begin{definition}[Geometric realisation]\label{definition-28}

            There is a functor \(|\cdot |\colon \SSet \to \Top\) defined by
            \[
              |A_n| = \coprod_n A_n \times \Delta^n / \sim.
            \]
            Where for \(\alpha \colon [m] \to [n]\) we identify \(A_m \times \Delta^n \ni (\alpha^* x, y)\) with \((x, \alpha_* y) \in A_n \times \Delta^n\).
          \end{definition}
\begin{example}\label{example-20}
\[|\Delta[n]| = \Delta^n.\]\end{example}

          We are going to restrict these functors.
        %
\begin{example}\label{example-21}

            Let \(G\) be a group, now let \(BG_n = G^{\times n}\) and
            \[\dd_i(g_1,\ldots,g_n) = \begin{cases}(g_2,\ldots,g_n) & i = 0, \\(g_1\ldots,g_ig_{i+1},\ldots,g_n) & i = 1,\ldots,n-1,\\(g_1,\ldots,g_{n-1}) & i = n.\end{cases}\]\[\sigma_i(g_1,\ldots,g_n) = (g_1,\ldots,1,\ldots,g_n)\]
            where the \(1\) goes in the \(i\)th place.
            \(|BG|\) is called the \terminology{classifying space} of \(G\) and it is a \(K(G,1)\).
          \end{example}
\begin{example}\label{example-22}

            A group is just a category with only one object and where all arrows are isomorphisms.
            So for a small category \(\cC\) we let \(B\cC_0\) be \(\ob \cC\) and \(B\cC_{n\ge 1}\) be all compatible \(n\)-tuples of morphisms.
            We define the face and degeneracy maps by composition and identity as above.
          \end{example}
\typeout{************************************************}
\typeout{Chapter 5 Model Categories}
\typeout{************************************************}
\chapter[Model Categories]{Model Categories}\label{chap-model-cats}
\typeout{************************************************}
\typeout{Section 5.1 Model Categories}
\typeout{************************************************}
\section[Model Categories]{Model Categories}\label{sec-model-cats}
\begin{definition}[Left lifting property]\label{definition-29}

            A map \(i\) satisfies the left lifting property with respect to a map \(p\) if in any diagram
            \[
              \xymatrix{
                A \ar[r]^f \ar[d]_i & X \ar[d]^p\\B\ar@{..>}[ur]^h \ar[r]_g& Y
              }
            \]
            the map \(h\) exists making the diagram commute.
          \end{definition}

          So in a category \(\cA\) a map \(P\) is projective if and only if \(0\to P\) has the left lifting property with respect to all surjections.
          Similarly a map \(I\) is injective if and only if \(I\to 0\) has the right lifting property with respect to all injections.
        %
\begin{definition}[Retracts]\label{definition-30}

            A map \(f\colon A \to B\) is a \terminology{retract} if a map \(g\colon A' \to B'\) if there exists a diagram
            \[
              \xymatrix{
                A \ar@{=}@/^/[rr] \ar[r]\ar[d]_f &A' \ar[r] \ar[d]^g &A\ar[d]^f \\
                B \ar@{=}@/_/[rr] \ar[r] & B'\ar[r] & B
              }
            \]\end{definition}
\begin{definition}[Model categories]\label{definition-31}

            A \terminology{model category} is a category \(\cM\) with three classes of maps, \(\sW\) (\terminology{weak equivalences}), \(\sF\) (\terminology{fibrations}), \(\sC\) (\terminology{cofibrations}).
            We call maps in \(\sW \cap \sF\)\terminology{acyclic fibrations} and those in \(\sW \cap \sC\)\terminology{acyclic cofibrations}.
            We require these categories and classes to satisfy the following:
            \begin{enumerate}
\item{}\(\cM\) has all small limits and colimits.\item{}If \(f,g,gf\) are morphisms of \(\cM\) then any two lying in \(\sW\) implies the third does also.\item{}\(\sF,\sC,\sW\) are all closed under retracts.\item{}
                \begin{enumerate}
\item{}Any map in \(\sC\) has the left lifting property with respect to any map in \(\sW\cap\sF\).\item{}Any map in \(\sF\) has the left lifting property with respect to any map in \(\sW\cap\sC\).\end{enumerate}

              \item{}
                Any map \(f\) may be functorially factored as
                \begin{enumerate}
\item{}\(f = p \circ i\) for some \(i\in \sC\), \(p\in \sF \cap \sW\).\item{}\(f = q \circ j\) for some \(j\in \sC\cap \sW\), \(q\in \sF\).\end{enumerate}

              \end{enumerate}
\end{definition}
\begin{example}\label{example-23}
\(\Ch_{\ge 0} R\) forms a model category with \(\sW\) being the class of quasi-isomorphisms, \(\sF\) the class of chain maps surjective in strictly positive grading, and \(\sC\) bein chain maps \(f\) where \(f_n\) is injective with projective cokernel for all \(n\).
            This is called the projective model structure on \(\Ch_{\ge 0} R\)\end{example}
\begin{definition}[Fibrant and cofibrant objects]\label{definition-32}
\(A\in \cM\) is \terminology{cofibrant} if \(0 \to A\) is a cofibration.
            \(A\in \cM\) is \terminology{fibrant} if \(A\to 0\) is a fibration.
          \end{definition}
\begin{lemma}\label{lemma-7}

            If \(\cM\) is a model category then \(f \in \sC\) if and only if \(f\) has the left lifting property with respect to all maps in \(\sW \cap \sF\).
            The three analogous statements also hold.
          \end{lemma}
\begin{proof}
\(f\colon K \to L\)\end{proof}
%
\backmatter
%
\end{document}