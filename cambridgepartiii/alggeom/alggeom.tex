%%                                    %%
%% Generated from MathBook XML source %%
%%    on 2015-02-11T22:48:00Z    %%
%%                                    %%
%%   http://mathbook.pugetsound.edu   %%
%%                                    %%
\documentclass[10pt,]{book}
%% Load geometry package to allow page margin adjustments
\usepackage{geometry}
\geometry{letterpaper,total={5.0in,9.0in}}
%% Custom Preamble Entries, early (use latex.preamble.early)
%% Inline math delimiters, \(, \), made robust with next package
\usepackage{fixltx2e}
%% Page Layout Adjustments (latex.geometry)
%% For unicode character support, use the "xelatex" executable
%% If never using xelatex, the next three lines can be removed
\usepackage{ifxetex}
\ifxetex\usepackage{xltxtra}\fi
%% Symbols, align environment, bracket-matrix
\usepackage{amsmath}
\usepackage{amssymb}
%% extpfeil package for certain extensible arrows,
%% as also provided by MathJax extension of the same name
\usepackage{extpfeil}
%% allow more columns to a matrix
%% can make this even bigger by overiding with  latex.preamble.late  processing option
\setcounter{MaxMatrixCols}{30}
%% XML, MathJax Conflict Macros
%% Two nonstandard macros that MathJax supports automatically
%% so we always define them in order to allow their use and
%% maintain source level compatibility
%% This avoids using two XML entities in source mathematics
\newcommand{\lt}{<}
\newcommand{\gt}{>}
%% Semantic Macros
%% To preserve meaning in a LaTeX file
%% Only defined here if required in this document
%% Used for inline definitions of terms
\newcommand{\terminology}[1]{\textbf{#1}}
%% Subdivision Numbering, Chapters, Sections, Subsections, etc
%% Subdivision numbers may be turned off at some level ("depth")
%% A section *always* has depth 1, contrary to us counting from the document root
%% The latex default is 3.  If a larger number is present here, then
%% removing this command may make some cross-references ambiguous
%% The precursor variable $numbering-maxlevel is checked for consistency in the common XSL file
\setcounter{secnumdepth}{3}
%% Environments with amsthm package
%% Theorem-like enviroments in "plain" style, with or without proof
\usepackage{amsthm}
\theoremstyle{plain}
%% Numbering for Theorems, Conjectures, Examples, Figures, etc
%% Controlled by  numbering.theorems.level  processing parameter
%% Always need a theorem environment to set base numbering scheme
%% even if document has no theorems (but has other environments)
\newtheorem{theorem}{Theorem}[section]
\renewcommand*{\proofname}{Proof}%% Only variants actually used in document appear here
%% Numbering: all theorem-like numbered consecutively
%% i.e. Corollary 4.3 follows Theorem 4.2
\newtheorem{lemma}[theorem]{Lemma}
%% Definition-like environments, normal text
%% Numbering for definition, examples is in sync with theorems, etc
%% also for free-form exercises, not in exercise sections
\theoremstyle{definition}
\newtheorem{definition}[theorem]{Definition}
\newtheorem{example}[theorem]{Example}
\newtheorem{remark}[theorem]{Remark}
%% Raster graphics inclusion, wrapped figures in paragraphs
\usepackage{graphicx}
%% Colors for Sage boxes and author tools (red hilites)
\usepackage[usenames,dvipsnames,svgnames,table]{xcolor}
%% hyperref driver does not need to be specified
\usepackage{hyperref}
%% Hyperlinking active in PDFs, all links solid and blue
\hypersetup{colorlinks=true,linkcolor=blue,citecolor=blue,filecolor=blue,urlcolor=blue}
\hypersetup{pdftitle={Part III Algebraic Geometry 2014}}
%% If you manually remove hyperref, leave in this next command
\providecommand\phantomsection{}
%% Custom Preamble Entries, late (use latex.preamble.late)
\usepackage[all]{xy}
%% Convenience macros
\DeclareMathOperator{\GL}{GL}
\DeclareMathOperator{\Map}{Map}
\DeclareMathOperator{\Hom}{Hom}
\DeclareMathOperator{\Tor}{Tor}
\DeclareMathOperator{\Ext}{Ext}
\DeclareMathOperator{\im}{im}
\DeclareMathOperator{\id}{id}
\DeclareMathOperator{\coker}{coker}
\DeclareMathOperator{\bfA}{\mathbf{A}}
\DeclareMathOperator{\CC}{\mathbf{C}}
\DeclareMathOperator{\QQ}{\mathbf{Q}}
\DeclareMathOperator{\RR}{\mathbf{R}}
\DeclareMathOperator{\ZZ}{\mathbf{Z}}
\DeclareMathOperator{\RP}{\mathbf{RP}}
\DeclareMathOperator{\C}{C_*}
\DeclareMathOperator{\A}{\mathcal{A}}
\DeclareMathOperator{\B}{\mathcal{B}}
\DeclareMathOperator{\F}{\mathcal{F}}
\DeclareMathOperator{\G}{\mathcal{G}}
\DeclareMathOperator{\cI}{\mathcal{I}}
\DeclareMathOperator{\cO}{\mathcal{O}}
\DeclareMathOperator{\U}{\mathcal{U}}
%% Title page information for book
\title{Part III Algebraic Geometry 2014}
\author{}
\date{}
\begin{document}
\frontmatter
%% half-title
\thispagestyle{empty}
\vspace*{\stretch{1}}
\begin{center}
{\Huge Part III Algebraic Geometry 2014}
\end{center}\par
\vspace*{\stretch{2}}
\clearpage
\thispagestyle{empty}
\clearpage
\maketitle
\clearpage
\thispagestyle{empty}
\vspace*{\stretch{2}}
\vspace*{\stretch{1}}
\clearpage
\setcounter{tocdepth}{1}
\renewcommand*\contentsname{Contents}
\tableofcontents
\mainmatter
\typeout{************************************************}
\typeout{Chapter 1 Sheaf Theory}
\typeout{************************************************}
\chapter[Sheaf Theory]{Sheaf Theory}\label{chap-sheaves}
\typeout{************************************************}
\typeout{Section 1.1 Introduction}
\typeout{************************************************}
\section[Introduction]{Introduction}\label{sec-introduction}

          These are lecture notes for the 2014 Part III Algebraic Geometry course taught by Dr. P.M.H. Wilson.
        %
\par

          The recommended books are: 
          \begin{itemize}
\item{}Algebraic Geometry - ??\end{itemize}

        %
\par

          Generated: February 11, 2015, 22:48:00 (Z)
        %
\typeout{************************************************}
\typeout{Section 1.2 Sheaves}
\typeout{************************************************}
\section[Sheaves]{Sheaves}\label{sec-sheaves}
Let \(X\) be a topological space.%
\begin{definition}[Presheaves]\label{definition-1}
A \terminology{presheaf}\(\F\) of abelian groups (resp. rings) on \(X\) consists of data:
            \begin{enumerate}
\item{}For every open \(U \subset X\) an abelian group (resp. ring) \(\F(U)\).\item{}For every inclusion of open sets \(V\subset U\) a homomorphism called \terminology{restriction}, denoted \(\rho^U_V\colon \F(U) \to \F(V)\) such that
                \begin{enumerate}
\item{}\(\F(\emptyset) = 0.\)\item{}\(\rho^U_U\colon \F(U) \to \F(U)\) is \(\id_{\F(U)}\).\item{}If \(W \subset V \subset U\) open then \(\rho^U_W = \rho^V_W \circ \rho^U_V\).\end{enumerate}

              \end{enumerate}
\end{definition}
\begin{remark}\label{remark-1}
If \(\U\) denotes the category of open sets in \(X\) (the morphisms are inclusions) then a presheaf of abelian groups over \(X\) is a contravariant functor \(\F\colon  \U \to \text{Abgp}.\)
          i.e. an element of \(\text{Abgp}^{\U^\text{op}}\).
          \end{remark}
\par
An element \(s \in \F(U)\) is called a \terminology{section} of \(\F\) over \(U\).
          For \(s\in \F(U)\) we denote \(\rho^U_V(s)\) by \(s|_V\).
        %
\begin{definition}[Sheaves]\label{definition-2}

            A presheaf \(\F\) on \(X\) is a \terminology{sheaf} if it satisfies two further conditions:
            \begin{enumerate}
\item{}If \(U\) is open and has \(U = \bigcup_i U_i\) an open cover and if \(s \in \F(U)\) is such that \(s|_{V_i} = 0\) for all \(i\) then \(s = 0\).
                (A presheaf satisfying this condition is called a \terminology{monopresheaf}).
              \item{}If \(U = \bigcup_i V_i\) is an open cover and we have \(s_i\in \F(V_i)\) such that \(\forall i,j\ s_i|_{V_i\cap V_j} = s_j|_{V_i\cap V_j}\) then there exists \(s \in \F(U)\) with \(s|_{V_i} = s_i\) for all \(i\).\end{enumerate}
\end{definition}
\begin{example}\label{example-1}
\(X\) a topological space, \(A\) any abelian group (resp. ring).
            The constant shead \(\A\) determined by \(A\) is defined as follows: \(\A(\emptyset) = \{0\}\), for \(U \ne \emptyset\) open in \(X\)\[\A(U) = \{\text{locally constant maps }U\to A\},\] this is an abelian group (resp. ring) under pointwise operations.
            With obvious restriction maps we obtain a sheaf.
            If \(U \ne \emptyset\) is open and connected, then \(\A(U) = A\).
            If \(U\) an open set whose connected components are open (e.g. in a locally connected topological space \(X\)) then the section \(\A(U)\) is a direct product of copies of \(A\).
          \end{example}
\begin{example}\label{example-2}
If \(X\) is a differentiable (\(C^\infty\)) manifold, we can define the sheaf of \(C^\infty\)-functions (\(\RR\) or \(\CC\) valued) on \(X\).
            Which is a sheaf of rings.
            Similarly if \(X\) is a complex manifold, we can define a sheaf of holomorphic functions on \(X\).
            In both cases, the sheaf is called the \terminology{structure sheaf} of \(X\), sometimes denoted \(\cO_X\).
          \end{example}
\begin{example}\label{example-3}
For \(V\) an (irreducible) variety (affine, projective, quasi-projective).
            We can consider \(V\) as a topological space with the Zariski topology.
            For \(U\) open in \(V\) set \[\cO_V(U)=\{\text{regular functions on } U\} = \{f\in k(V) \text{ s.t. }f \text{ regular on } U\}.\]
            This is a sheaf of rings with respect to the Zariski topology, and is known as the \terminology{structure sheaf} for varieties.
            If \(V\) is affine we have that \(\cO_V(V) = k[V]\).
          \end{example}
\begin{definition}[Stalks]\label{definition-3}
If \(\F\) is a presheaf on a topological space \(X\) and \(P\in X\) we define the stalk \(\F_P\) of \(\F\) at \(P\) to be \(\varinjlim_{U\ni P} \F(U)\) i.e. an element of \(\F_P\) is represented by a pair \((U,s)\) where \(U\) is an open neighbourhood of \(P\) and \(s\in \F(U)\), where \((U,s)\) and \((V,t)\) define the same element of \(\F_P\) if there exists an open neighbourhood \(W\ni P\) with \(W\subset V\cap U\) such that \(s|_W = t|_W\) the elements \(s_P\) of \(\F_P\) are called \terminology{germs}.
            If \(\F\) is a sheaf of abelian groups or rings then \(\F_P\) is an abelian group, ring, etc.
          \end{definition}
\begin{example}\label{example-4}
For the constant sheaf \(\A\) associated to \(A\) we have \(\A_P = A\).
          \end{example}
\begin{example}\label{example-5}
For \(X\) a \(C^\infty\) manifold (resp. complex) with structure sheaf \(\cO_X\), the stalk \(\cO_{X,P}\) of \(\cO_X\) at \(P\in V\) consists of germs of \(C^\infty\) (resp. holomorphic) functions.
          \end{example}
\begin{example}\label{example-6}
For \(V\) (irreducible) affine, projective or quasi-projective variety with structure sheaf \(\cO_V\) the stalk at \(P\in V\)\(\cO_{V,P} = \text{ local ring at } P\) (defined before).
          \end{example}
\begin{definition}[Morphisms of (pre)sheaves]\label{definition-4}
If \(\F\) and \(\G\) are presheaves (resp. sheaves) on \(X\) a morphism \(\Phi\colon \F \to \G\) consists of homomorphisms \(\F(U) \xrightarrow{\Phi} \G(U)\) for all open \(U\) such that for \(V \subseteq U\) open
            \[\xymatrix@R+2em@C+2em{
              \F(U) \ar[r]^{\phi(U)} \ar[d]_{\rho^U_V} & \G(U) \ar[d]^{\rho'^U_V} \\
              \F(V) \ar[r]_{\phi(V)} & \G(V)
              }
            \]
            commutes.
          \end{definition}
\par
A morphism \(\phi\colon \F \to \G\) induces a homomorphism \(\phi_P\colon \F_P \to \G_P\) for each \(P\), namely \(\phi_P[(U,s)] = [(U,\phi(U)(s))]\), which is well defined.%
\begin{definition}[Injective and isomorphic sheaf morphisms]\label{definition-5}
A morphism \(\phi\colon \F\to\G\) of (pre)sheaves is \terminology{injective} if \(\F(U) \to \G(U)\) is injective for all open \(U\).
            e.g. sheaves of subgroups or subrings where \(\F(U) \subseteq \G(U)\) for all \(U\).
            In this case \(\F\) is called a \terminology{subsheaf} of \(\G\).\newline{}
            A morphism \(\phi\colon \F \to \G\) is called an isomorphism if there exists an inverse morphism \(\chi\colon \G\to\F\).
            This is equivalent to \(\phi(U)\colon \F(U) \to \G(U)\) being bijective for all \(U\) since we can define \(\chi(U) = \phi(U)^{-1}\) as the inverse.
          \end{definition}
\begin{lemma}\label{lemma-1}
Let \(\phi\colon \F\to\G\) be a morphism of sheaves then
            \begin{enumerate}
\item{}\(\phi\) is injective iff \(\phi_P\colon \F_P \to \G_P\) is injective for all \(P\in X\),\item{}\(\phi\) is an isomorphism iff \(\phi_P\colon \F_P \to \G_P\) is an isomorphism for all \(P\in X\).\end{enumerate}
\end{lemma}
\begin{proof}
(\(\Rightarrow\)) (true for presheaves too).
              \begin{enumerate}
\item{}Suppose there exists a germ \(s_P\in\F_P\) such that \(\phi_P(s_P) = 0\in \G_P\).
                  i.e. there exists an open neighbourhood \(W \subset U\) with \(P\in W\) such that \(\phi(U)(s|_W) = 0\).
                  So by commutativity \(\phi(W)(s|_W) = 0\) but \(\phi\) injective implies \(s|_W = 0\).
                \item{}Clear.\end{enumerate}

              (\(\Leftarrow\))
              \begin{enumerate}
\item{}Needs first sheaf condition on \(\F\).
                  If \(\phi_P\) injective for all \(P\) and \(U\) is open in \(X\) it remains to prove that \(\phi(U) \colon \F(U) \to \G(U)\) is injective.
                  Suppose not and there exists \(0 \ne s \in \F(u)\) such that \(\phi(U)(s) = 0 \in \G(U)\).
                  Let \(s_P\) denote the germ of \(s\) at \(P \in U\), then \(0 = \phi(U)(s)_P = \phi_P(s_P)\) for all \(P\in U\).
                  So \(s_P\) in \(\F_P\) for all \(P \in U\).
                  Hence for all \(P\in U\) we have an open neighbourhood \(U \supset W \ni P\) such that \(s|_W = 0\).
                  So \(U\) is covered by open sets \(U_\alpha\) such that \(s|_{U_\alpha} = 0\) for all \(\alpha\), which implies that \(s = 0\).
                \item{}\(\phi(U) \colon \F(U) \to \G(U)\) is an injection for all open \(U\) by the first part, so it remains to prove that it is surjective also.
                  Suppose \(t \in \G(U)\) and let \(t_P\in\G_P\) be its germ at \(P\in U\).
                  Since \(\phi_P\) is surjective we have some \(s_P\in \F_P\) such that \(\phi_P(s_P) = t_P\).
                  Now suppose that \(s_P\) is represented by a pair \((V,s)\) with \(P\in V \subseteq U\) and \(s\in \F(V)\).
                  We then have that \(t_P\) is represented by \(\phi(V)(s)\), i.e. \((U,t)\sim (V,\phi(V)(s))\).
                  Shrinking \(V\) we may assume that we have an ope neighbourhood \(V_P\ni P\) such that \(\phi(V)(s)|_{V_P} = t|_{V_P}\).
                  In this way we cover \(U\) by open sets giving \(U = = \bigcup_{\alpha} U_\alpha\) to obtain sections \(s_\alpha \in\F_\alpha\) such that \(\phi(U_\alpha)(s_\alpha) = t|_{U_\alpha}\).
                  On the overlaps \(U_{\alpha\beta} = U_\alpha \cap U_\beta\) we have \(\phi(U_{\alpha\beta})(s_\alpha|_{U_{\alpha\beta}}) = t|_{U_{\alpha\beta}} = \phi(U_{\alpha\beta})(s_\beta|_{U_{\alpha\beta}})\), therefore the injectivity of \(\phi(U_{\alpha\beta})\) gives that \(s_\alpha|_{U_{\alpha\beta}} = s_\beta|_{U_{\alpha\beta}}\).
                  Since \(\F\) is a sheaf the \(s_\alpha\) patch together to give a section \(s \in \F(U)\) wuch that \(s|_{U_\alpha} = s_\alpha\) (using the second sheaf condition for \(\F\)).
                  But then \(\phi(U)(s)\) and \(t\) are sections of \(\G(U)\) such that \(\phi(U)(s)|_{U_{\alpha}}  = \phi(U_\alpha)(s_\alpha) = t|_{U_\alpha}\) for all \(\alpha\).
                  The first sheaf condition for \(\G\) now gives that \(\phi(U)(s) = t\) as required.
                \end{enumerate}

            %
\end{proof}
\begin{definition}[Surjective sheaf morphisms]\label{definition-6}
A morphism of sheaves \(\phi\colon \F \to \G\) is called \terminology{surjective} if \(\phi_P\colon \F_P\to\G_P \) is surjective for all \(P \in X\).\end{definition}
\begin{definition}[Induced (pre)sheaves]\label{definition-7}
Given a (pre)sheaf \(\F\) on a space \(X\) and a continuous map \(f\colon X \to Y\) we have an \terminology{induced (pre)sheaf}, denoted \(f_*\F\) on \(Y\) defined by
            \[(f_*\F)(U) = \F(f^{-1}(U))\]
            for \(U\) open in \(Y\).
            With the restriction maps coming from \(\F\) as \(V \subset U\) implies \(f^{-1}(V)\subset f^{-1}(U)\).
            It should be checked that indeed \(\F\) being a (pre)sheaf implies \(f_*\F\) is a (pre)sheaf.
          \end{definition}
\begin{definition}[Ringed spaces]\label{definition-8}
A \terminology{ringed space} is a pair \((X,\cO_X)\) where \(X\) is a topological space and \(\cO_X\) is a sheaf of rings on \(X\).\end{definition}
\begin{definition}[Morphisms of ringed spaces]\label{definition-9}
Given two ringed spaces \((X,\cO_X)\) and \((Y, \cO_Y)\) a \terminology{morphism of ringed spaces}\((X,\cO_X)\to (Y,\cO_Y)\) is a pair \((f,f^\#)\) where \(f\colon X \to Y\) is continuous and \(f^\#\colon \cO_Y \to f_*\cO_X\) is a morphism of sheaves of rings.
            So \(f^\#\) defines homomorphisms \(\cO_Y(U) \to \cO_X(f^{-1}(U))\) for all \(U\) open in \(Y\), compatible with restrictions.
            Hence he have homomorphisms on stalks too \(\cO_{Y,f(P)} \to \cO_{X,P}\).
          \end{definition}
\begin{definition}[Ringed spaces over a ring]\label{definition-10}
If \(R\) is a commutative ring (e.g. a field), a \terminology{ringed space over \(R\)} is a ringed space with \(\cO_X\) a sheaf of \(R\)-algebras.
            (Therefore the restriction maps are homomorphisms of \(R\)-algebras.)
            A morphism of ringed spaces over \(R\) is defined in the obvious way.
          \end{definition}
\begin{definition}[Locally ringed spaces]\label{definition-11}
A ringed space \((X, \cO_X)\) is a \terminology{locally ringed space} (also known as a geometric space) if all \(\cO_{X,P}\) are local rings.\end{definition}
\begin{definition}[Morphisms of locally ringed spaces]\label{definition-12}
A \terminology{morphism of locally ringed spaces} is a morphism of ringed spaces as above where all the induced maps \(f_P^\#\colon \cO_{Y,f(P)} \to \cO_{X,P}\) are local homomorphisms of local rings.\end{definition}
\begin{example}\label{example-7}
\((X,\ZZ)\) is a ringed space but not locally ringed.\end{example}
\begin{example}\label{example-8}
If \(X\) is a \(C^\infty\) (resp. complex) manifold with structure sheaf \(\cO_X\) then \((X, \cO_X)\) is a locally ringed space over \(\RR\) (resp. over \(\CC\)).
            A smooth (resp. holomorphic) map of manifolds \(f\colon X\to Y\) yields a morphism of \(\RR\) (resp. \(\CC\))-algebras \(f^\#\colon \cO_Y\to f_* \cO_X\) namely \(f^\#\colon\cO_Y(U) \to\cO_X(f^{-1}U)\) given by \(g\to g\circ f\) (smooth (resp. holomorphic) functions on \(Y\) pullback to ones on \(X\)).
            Clearly \(g(f(P)) = 0 \iff f^\#(g)(P) = 0\) and so \(f^\#(m_{Y,f(P)} \subseteq m_{X,P}\).
            So \((f,f^\#)\) is a morphism of locally ringed spaces over \(\RR\) (resp. \(\CC\)).
          \end{example}
\begin{example}\label{example-9}
\((V,\cO_V)\) for \(V\) an irreducible affine variety \(V\) is a ringed space via its structure sheaf, it is locally ringed over the base field \(k\).
            If \(\Phi\colon V \to W\) is a morphism of affine varieties then there exists a morphism of locally ringed spaces over \(k\)\((\phi, \phi^\#)\colon (V, \cO_V) \to (W, \cO_W)\) given by \(\phi^\#(g) = g\circ \phi \in \cO_V(\phi^{-1}U)\) for \(g\in\cO_W(U)\).
          \end{example}
\begin{lemma}\label{lemma-2}
If \(V,W\) are (irreducible) affine varieties and \((f,f^\#)\colon (V,\cO_V)\to (W,\cO_W)\) is a morphism of locally ringed spaces over \(k\) then \(f\) is induced from a morphism of varieties \(\phi\colon V\to W\) with \(f^\# = \phi^\#\) defined as above.\end{lemma}
\begin{proof}
Suppose \(V\subseteq \bfA^n,\ W\subseteq\bfA^m\) let \(y_j\) be the \(j\)th coordinate function on \(W\) and define \(g_j = f^\#(y_j) \in \cO_V(V) = k[V]\).
            Let \(\phi = (g_1,\ldots,g_m)\), this is a morphism \(V \to \bfA^m\).
            Suppose that \(f(P) = (b_1,\ldots,b_m)\) for \(P\in V\) then \(y_j - b_j \in m_{W,f(P)}\) for all \(j\) which implies that \(g_j(P) = b_j\) for all \(j\).
            Since \(f^\#\) is local we have that \(\phi(P) = f(P)\) and so \(\phi \colon V \to W\) is the same map as \(f\) on topological spaces.
            Moreover \(y_1,\ldots,y_m\) generate \(k[V]\) as a \(k\)-algebra and also generate \(k(W)\) as a field over \(k\). \(f^\#(y_j) = g_j = y_j\circ\phi = \phi^\#(y_j)\) and it follows that \(f^\# = \phi^\#\) on any \(\cO_W(U)\).
            \(k[W] \subset \cO_W(U)\subset k(W)\) for \(U\) open in \(W\).
          \end{proof}
\begin{definition}[Morphisms of varieties]\label{definition-13}
Given \(V\) and \(W\) (irreducible) quasi-projective varieties, we define a \terminology{morphism} of varieties \(V \to W\) to be a morphism of the corresponding locally ringed spaces over \(k\)\((V,\cO_V)\to (W,\cO_W)\).\end{definition}
\typeout{************************************************}
\typeout{Subsection 1.2.1 \(\cO_X\)-modules}
\typeout{************************************************}
\subsection[\(\cO_X\)-modules]{\(\cO_X\)-modules}\label{subsection-1}
\begin{definition}[\(\cO_X\)-modules]\label{definition-14}
Let \(M\) be a sheaf of abelian groups on a ringed space \((X,\cO_X)\), \(M\) is said to be an \terminology{\(\cO_X\)-module} if for every open set \(U\subset X\)\(M(U)\) is an open \(\cO_X(U)\)-module and for any \(W\subseteq U\) open \(\alpha \in \cO_X(U)\), \(m\in M(U)\), we have \((\alpha m)|_W = (\alpha|_W)(m|_W)\).
              Similarly we have the obvious definition for morphisms of \(\cO_X\)-modules \(\phi\colon M \to N\) (all maps respect the \(\cO_X\)-module structure).
            \end{definition}
\begin{example}\label{example-10}
For \(V\) a (irreducible) quasi-projective variety with structure sheaf \(\cO_V\) and \(W \subset V\) a closed subvariety we have the \terminology{sheaf of ideals}\(\cI_W \subset \cO_V\) a subsheaf of \(\cO_V\) given by \[\cI_W(U) = \{f\in \cO_V(U) : f|_{W\cap U} \equiv 0\}.\]
              This is clearly an \(\cO_V\)-module.
            \end{example}
Most things go through unchanged e.g. if \(M\) is an \(\cO_X\)-module then any stalk \(M_P\) is an \(\cO_{X,P}\)-module etc.\newline{}
            However 
          %
%
\backmatter
%
\end{document}