%%                                    %%
%% Generated from MathBook XML source %%
%%    on 2015-05-21T07:04:50Z    %%
%%                                    %%
%%   http://mathbook.pugetsound.edu   %%
%%                                    %%
\documentclass[10pt,]{book}
%% Load geometry package to allow page margin adjustments
\usepackage{geometry}
\geometry{letterpaper,total={5.0in,9.0in}}
%% Custom Preamble Entries, early (use latex.preamble.early)
%% Inline math delimiters, \(, \), made robust with next package
\usepackage{fixltx2e}
%% Page Layout Adjustments (latex.geometry)
%% For unicode character support, use the "xelatex" executable
%% If never using xelatex, the next three lines can be removed
\usepackage{ifxetex}
\ifxetex\usepackage{xltxtra}\fi
%% Symbols, align environment, bracket-matrix
\usepackage{amsmath}
\usepackage{amssymb}
%% extpfeil package for certain extensible arrows,
%% as also provided by MathJax extension of the same name
\usepackage{extpfeil}
%% allow more columns to a matrix
%% can make this even bigger by overiding with  latex.preamble.late  processing option
\setcounter{MaxMatrixCols}{30}
%% XML, MathJax Conflict Macros
%% Two nonstandard macros that MathJax supports automatically
%% so we always define them in order to allow their use and
%% maintain source level compatibility
%% This avoids using two XML entities in source mathematics
\newcommand{\lt}{<}
\newcommand{\gt}{>}
%% Semantic Macros
%% To preserve meaning in a LaTeX file
%% Only defined here if required in this document
%% Used for inline definitions of terms
\newcommand{\terminology}[1]{\textbf{#1}}
%% Subdivision Numbering, Chapters, Sections, Subsections, etc
%% Subdivision numbers may be turned off at some level ("depth")
%% A section *always* has depth 1, contrary to us counting from the document root
%% The latex default is 3.  If a larger number is present here, then
%% removing this command may make some cross-references ambiguous
%% The precursor variable $numbering-maxlevel is checked for consistency in the common XSL file
\setcounter{secnumdepth}{3}
%% Environments with amsthm package
%% Theorem-like enviroments in "plain" style, with or without proof
\usepackage{amsthm}
\theoremstyle{plain}
%% Numbering for Theorems, Conjectures, Examples, Figures, etc
%% Controlled by  numbering.theorems.level  processing parameter
%% Always need a theorem environment to set base numbering scheme
%% even if document has no theorems (but has other environments)
\newtheorem{theorem}{Theorem}[section]
\renewcommand*{\proofname}{Proof}%% Only variants actually used in document appear here
%% Numbering: all theorem-like numbered consecutively
%% i.e. Corollary 4.3 follows Theorem 4.2
\newtheorem{proposition}[theorem]{Proposition}
\newtheorem{claim}[theorem]{Claim}
%% Definition-like environments, normal text
%% Numbering for definition, examples is in sync with theorems, etc
%% also for free-form exercises, not in exercise sections
\theoremstyle{definition}
\newtheorem{definition}[theorem]{Definition}
\newtheorem{exercise}[theorem]{Exercise}
%% Raster graphics inclusion, wrapped figures in paragraphs
\usepackage{graphicx}
%% Colors for Sage boxes and author tools (red hilites)
\usepackage[usenames,dvipsnames,svgnames,table]{xcolor}
%% hyperref driver does not need to be specified
\usepackage{hyperref}
%% Hyperlinking active in PDFs, all links solid and blue
\hypersetup{colorlinks=true,linkcolor=blue,citecolor=blue,filecolor=blue,urlcolor=blue}
\hypersetup{pdftitle={Part III Algebraic Number Theory 2014}}
%% If you manually remove hyperref, leave in this next command
\providecommand\phantomsection{}
%% Custom Preamble Entries, late (use latex.preamble.late)
\usepackage[all]{xy}
%% Convenience macros
\newcommand{\CC}{\mathbf{C}}
\newcommand{\QQ}{\mathbf{Q}}
\newcommand{\RR}{\mathbf{R}}
\newcommand{\ZZ}{\mathbf{Z}}
\newcommand{\cO}{\mathcal{O}}
\newcommand{\cI}{\mathcal{I}}
\newcommand{\fa}{\mathfrak{a}}
\newcommand{\fb}{\mathfrak{b}}
\newcommand{\fc}{\mathfrak{c}}
\newcommand{\fp}{\mathfrak{p}}
\newcommand{\fq}{\mathfrak{q}}
\newcommand{\fr}{\mathfrak{r}}
\DeclareMathOperator{\Frac}{Frac}
%% Title page information for book
\title{Part III Algebraic Number Theory 2014}
\author{}
\date{}
\begin{document}
\frontmatter
%% half-title
\thispagestyle{empty}
\vspace*{\stretch{1}}
\begin{center}
{\Huge Part III Algebraic Number Theory 2014}
\end{center}\par
\vspace*{\stretch{2}}
\clearpage
\thispagestyle{empty}
\clearpage
\maketitle
\clearpage
\thispagestyle{empty}
\vspace*{\stretch{2}}
\vspace*{\stretch{1}}
\clearpage
\setcounter{tocdepth}{1}
\renewcommand*\contentsname{Contents}
\tableofcontents
\mainmatter
\typeout{************************************************}
\typeout{Chapter 1 Dedekind domains}
\typeout{************************************************}
\chapter[Dedekind domains]{Dedekind domains}\label{chap-dedekind}
\typeout{************************************************}
\typeout{Section 1.1 Introduction}
\typeout{************************************************}
\section[Introduction]{Introduction}\label{sec-introduction}
These are lecture notes for the 2014 Part III Algebraic Number Theory course taught by Dr. Jack Thorne, these notes are part of \href{https://alexjbest.github.io/mjolnir/}{Mjolnir}.%
\par
The recommended books are: 
          \begin{itemize}
\item{}H. P. F. Swinnerton-Dyer - A Brief Guide to Algebraic Number Theory\item{}Serge Lang - Algebraic Number Theory\end{itemize}

        %
\par

          Generated: May 21, 2015, 07:04:50 (Z)
        %
\typeout{************************************************}
\typeout{Section 1.2 Basics}
\typeout{************************************************}
\section[Basics]{Basics}\label{sec-basics}
\begin{definition}[Number fields]\label{definition-1}
A \terminology{number field}\(K\) is a finite degree field extension of \(\QQ\).\end{definition}
\begin{definition}[Integral elements]\label{definition-2}
If \(K\) is a number field and \(\alpha\in K\) then we say \(\alpha\) is \terminology{integral} if there exists a \(f\in\ZZ[x]\) monic with \(f(\alpha) = 0\).\end{definition}
If \(\alpha\) is integral than \(\ZZ[\alpha] \subset K\) is finitely generated.
          Conversely if \(\alpha\in K\) and \(\ZZ[\alpha]\) is a finitely generated \(\ZZ\)-module then \(\alpha\) is integral over \(K\).
          (If \(\ZZ[\alpha]\) is spanned by \(f_1(\alpha),\ldots,f_k(\alpha)\ f_i\in \ZZ[x]\) then for any \(n \gt \max\deg f_i\) we can write \(\alpha^n = \sum_{i=1}^k a_i f_i(\alpha)\) for some \(\alpha_i\in \ZZ\).
          This implies that \(\alpha\) is a zero of \(x^n - \sum_{i=1}^k a_i f_i(x) \in \ZZ[x]\).
          We have shown that if \(\alpha,\beta \in K\) are integral over \(\ZZ\) then so are \(\alpha\pm\beta\) and \(\alpha\beta\) (as it is easy to see \(\ZZ[\alpha,\beta]\) is a finitely generated \(\ZZ\)-module).)
        %
\begin{definition}[Rings of integers]\label{definition-3}
If \(K\) is a number field let \(\cO_K\) be the \terminology{ring of integers}, defined by \[\cO_K = \{\alpha\in K : \alpha \text{ integral over }\ZZ\}.\]
            This is the integral closure of \(\ZZ\) in \(K\).
          \end{definition}
\typeout{************************************************}
\typeout{Section 1.3 Dedekind domains}
\typeout{************************************************}
\section[Dedekind domains]{Dedekind domains}\label{sec-dedekind}
Let \(R\) be an integral domain, \(K = \Frac(R)\).%
\begin{definition}[Dedekind domains]\label{definition-4}
We then say that \(R\) is a \terminology{dedekind domain} if it is
            \begin{enumerate}
\item{}Noetherian,\item{}integrally closed in \(K\),\item{}and in it every non-zero prime is maximal.\end{enumerate}
\end{definition}
\begin{exercise}\label{exercise-1}
Show that every PID is a dedekind domain.\end{exercise}
\begin{definition}[Fractional ideals]\label{definition-5}
If \(R\) is a dedekind domain we call every finitely generated \(R\)-submodule of \(K\) a fractional ideal.\end{definition}
\par
This definition includes ideals \(I\subset R\).%
\begin{proposition}\label{proposition-1}
Let \(R\) be a dedekind domain and let \(\cI\) be the set of non-zero fractional ideals of \(R\), then \(\cI\) is a group under multiplication.\end{proposition}
\begin{proof}
We deonte ideals of \(R\) by \(\fa,\fb,\fc\subset R\) and (non-zero) prime ideals by \(\fp,\fq,\fr\subset R\).
            Multiplication is given by \[\fa\fb = \left\{ \sum a_ib_i : a_i \in \fa,\,b_i\in \fb\right\}.\]
            Then the identity for this operation is \((1) = R\).
            The key part of this proof is the construction of inverses.
            \begin{claim}\label{claim-1}
For any non-zero ideal \(\fa \subset R\) there exist non-zero prime ideals \(\fp_1,\ldots,\fp_m\subset R\) such that \(\fp_1\cdots \fp_m\subset \fa\).\end{claim}
\begin{proof}
Suppose not, then we can find an \(\fa \subset R\) which is maximal among such ideals having this property (as \(R\) is noetherian).
                Then \(\fa\) is not prime, as otherwise the claim is clearly true, so there exists some \(\alpha,\beta\in R\) with \(\alpha\beta \in \fa\) but \(\alpha,\beta \notin \fa\).
                So we have that \(\fa \subsetneq \fa + (\alpha)\) and \(\fa \subsetneq \fa + (\beta)\).
                By the maximality of \(\fa\) we can find \(\fp_1 \cdots \fp_m \subseteq \fa + (\alpha) \) and \(\fq_1 \cdots \fq_n \subseteq \fa + (\beta)\) but now
                \[\fp_1 \cdots \fp_m\fq_1\cdots \fq_n\subseteq(\fa + (\alpha))(\fa + (\beta)) \subseteq \fa + (\alpha\beta) \subseteq \fa,\]
                contradiction.
              \end{proof}
\begin{claim}\label{claim-2}
For any non-zero prime ideal \(\fp\subset R\) there exists \(\delta \in K\smallsetminus R\) such that \(\delta \fp \subseteq R\).\end{claim}
\begin{proof}
Choose \(\beta\in \fp \smallsetminus \{0\}\) and an expression \(\fp_1\cdots \fp_m \subseteq (\beta)\) with \(\fp_i\) non-zero prime ideals and \(m\) minimal.
                Then there exists \(i\) such that \(\fp_i\subset R\) otherwise for all \(i\) there is some \(\alpha_i \in\fp_i \smallsetminus\fp\), in which case \(\alpha_1\cdots \alpha_m\in \fp_1\cdots \fp_m\subseteq (\beta) \subseteq \fp\), a contradiction 
              \end{proof}
\end{proof}
%
\backmatter
%
\end{document}