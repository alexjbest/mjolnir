%%                                    %%
%% Generated from MathBook XML source %%
%%    on 2015-05-21T06:56:48Z    %%
%%                                    %%
%%   http://mathbook.pugetsound.edu   %%
%%                                    %%
\documentclass[10pt,]{book}
%% Load geometry package to allow page margin adjustments
\usepackage{geometry}
\geometry{letterpaper,total={5.0in,9.0in}}
%% Custom Preamble Entries, early (use latex.preamble.early)
%% Inline math delimiters, \(, \), made robust with next package
\usepackage{fixltx2e}
%% Page Layout Adjustments (latex.geometry)
%% For unicode character support, use the "xelatex" executable
%% If never using xelatex, the next three lines can be removed
\usepackage{ifxetex}
\ifxetex\usepackage{xltxtra}\fi
%% Symbols, align environment, bracket-matrix
\usepackage{amsmath}
\usepackage{amssymb}
%% extpfeil package for certain extensible arrows,
%% as also provided by MathJax extension of the same name
\usepackage{extpfeil}
%% allow more columns to a matrix
%% can make this even bigger by overiding with  latex.preamble.late  processing option
\setcounter{MaxMatrixCols}{30}
%% XML, MathJax Conflict Macros
%% Two nonstandard macros that MathJax supports automatically
%% so we always define them in order to allow their use and
%% maintain source level compatibility
%% This avoids using two XML entities in source mathematics
\newcommand{\lt}{<}
\newcommand{\gt}{>}
%% Semantic Macros
%% To preserve meaning in a LaTeX file
%% Only defined here if required in this document
%% Used for inline definitions of terms
\newcommand{\terminology}[1]{\textbf{#1}}
%% Subdivision Numbering, Chapters, Sections, Subsections, etc
%% Subdivision numbers may be turned off at some level ("depth")
%% A section *always* has depth 1, contrary to us counting from the document root
%% The latex default is 3.  If a larger number is present here, then
%% removing this command may make some cross-references ambiguous
%% The precursor variable $numbering-maxlevel is checked for consistency in the common XSL file
\setcounter{secnumdepth}{3}
%% Environments with amsthm package
%% Theorem-like enviroments in "plain" style, with or without proof
\usepackage{amsthm}
\theoremstyle{plain}
%% Numbering for Theorems, Conjectures, Examples, Figures, etc
%% Controlled by  numbering.theorems.level  processing parameter
%% Always need a theorem environment to set base numbering scheme
%% even if document has no theorems (but has other environments)
\newtheorem{theorem}{Theorem}[section]
\renewcommand*{\proofname}{Proof}%% Only variants actually used in document appear here
%% Numbering: all theorem-like numbered consecutively
%% i.e. Corollary 4.3 follows Theorem 4.2
\newtheorem{corollary}[theorem]{Corollary}
\newtheorem{lemma}[theorem]{Lemma}
\newtheorem{proposition}[theorem]{Proposition}
%% Definition-like environments, normal text
%% Numbering for definition, examples is in sync with theorems, etc
%% also for free-form exercises, not in exercise sections
\theoremstyle{definition}
\newtheorem{definition}[theorem]{Definition}
\newtheorem{example}[theorem]{Example}
\newtheorem{exercise}[theorem]{Exercise}
\newtheorem{remark}[theorem]{Remark}
%% Raster graphics inclusion, wrapped figures in paragraphs
\usepackage{graphicx}
%% Colors for Sage boxes and author tools (red hilites)
\usepackage[usenames,dvipsnames,svgnames,table]{xcolor}
%% hyperref driver does not need to be specified
\usepackage{hyperref}
%% Hyperlinking active in PDFs, all links solid and blue
\hypersetup{colorlinks=true,linkcolor=blue,citecolor=blue,filecolor=blue,urlcolor=blue}
\hypersetup{pdftitle={Part III Local Fields 2014}}
%% If you manually remove hyperref, leave in this next command
\providecommand\phantomsection{}
%% Custom Preamble Entries, late (use latex.preamble.late)
\usepackage[all]{xy}
%% Convenience macros
\newcommand{\CC}{\mathbf{C}}
\newcommand{\FF}{\mathbf{F}}
\newcommand{\PP}{\mathbf{P}}
\newcommand{\QQ}{\mathbf{Q}}
\newcommand{\RR}{\mathbf{R}}
\newcommand{\ZZ}{\mathbf{Z}}
\newcommand{\cO}{\mathcal{O}}
\newcommand{\fp}{\mathfrak{p}}
\newcommand{\ab}{|\cdot|}
\DeclareMathOperator{\chara}{char}
\DeclareMathOperator{\tr}{tr}
\DeclareMathOperator{\Frac}{Frac}
%% Title page information for book
\title{Part III Local Fields 2014}
\author{}
\date{}
\begin{document}
\frontmatter
%% half-title
\thispagestyle{empty}
\vspace*{\stretch{1}}
\begin{center}
{\Huge Part III Local Fields 2014}
\end{center}\par
\vspace*{\stretch{2}}
\clearpage
\thispagestyle{empty}
\clearpage
\maketitle
\clearpage
\thispagestyle{empty}
\vspace*{\stretch{2}}
\vspace*{\stretch{1}}
\clearpage
\setcounter{tocdepth}{1}
\renewcommand*\contentsname{Contents}
\tableofcontents
\mainmatter
\typeout{************************************************}
\typeout{Chapter 1 Introduction}
\typeout{************************************************}
\chapter[Introduction]{Introduction}\label{chap-introduction}
These are lecture notes for the 2014 Part III Local Fields course taught by Dr. Tom Fisher.%
\par
The recommended books are:
        \begin{itemize}
\item{}Cassels\item{}Serre\item{}Koblitz\end{itemize}

      %
\par
Generated: May 21, 2015, 06:56:48 (Z)%
\typeout{************************************************}
\typeout{Chapter 2 \(p\)-adic numbers}
\typeout{************************************************}
\chapter[\(p\)-adic numbers]{\(p\)-adic numbers}\label{chap-padic}
\begin{definition}[Absolute value]\label{definition-1}
An \terminology{absolute value} on a field \(K\) is a function \(\ab\colon K \to \RR\) such that
          \begin{enumerate}
\item{}\(|x|\ge 0\) and \(|x| = 0 \iff x = 0\).\item{}\(|xy| = |x\|y|\).\item{}\(|x + y| \le |x| + |y|\).\end{enumerate}
\end{definition}
\begin{example}\label{example-1}
\begin{enumerate}
\item{}\(K \subset \CC\), \(|a+bi|_{\infty} = \sqrt{a^2 + b^2}\).\item{}\(K\) any field, \(|x| = 0\) if \(x = 0\) and \(|x| = 1\) otherwise, this is called the \terminology{trivial absolute value}.\end{enumerate}
\end{example}
\begin{remark}\label{remark-1}
\begin{enumerate}
\item{}If \(x^n = 1\) then \(|x|  = 1\) and hence finite fields can only be given the trivial absolute value.\item{}In particular \(| -1| = 1\) and so \(|x| = |-x|\) for all \(x \in K\).\end{enumerate}
\end{remark}
A valued field \((K, \ab)\) becomes a metric space with \(d(xy,) = |x-y|\), and hence a topological space, i.e. open sets are unions of open balls \(B(x,r) = \{y\in K : |x-y| \lt r\}\).%
\begin{exercise}\label{exercise-1}
Show that the functions \(+,\cdot \colon K\times K \to K\) and \(\ab\colon K \to \RR\) are continuous with respect to this topology.\end{exercise}
\begin{example}\label{example-2}
Let \(K = \QQ,\, p\) be a prime and \(0 \lt \alpha \lt 1\).
          For \(x\in \QQ^*\) let \(\nu_p(x) = r\) where \(x = p^r u/ v\) and \(p\nmid uv\).
          Then the \(p\)-adic absolute value if given by \[|x|_p = \begin{cases} 0 & x = 0,\\ \alpha^{\nu_p(x)} & \text{otherwise}.\end{cases}\]
          Usually we take \(\alpha = 1/p\).
          In this case we get the inequality \(|x+ y|_p \le \max\{|x|_p, |y|_p\}\), this is the \terminology{ultrametric triangle inequality}.
        \end{example}
\begin{definition}[(non-)Archimidean absolute values]\label{definition-2}
Absolute values are called \terminology{non-archimidean} if this inequality holds, otherwise they are called \terminology{archimidean}.
        \end{definition}
\par
Recall that if \(R\) is any ring then there exists a unique ring homomorphism \(\ZZ \to R\).%
\begin{lemma}\label{lemma-1}
\(\ab\) is non-archimidean if and only if \(|n|\) is bounded for all \(n \in \ZZ\).\end{lemma}
\begin{proof}

          (\(\Rightarrow\)) \(|n| \le \max\{|1|\} = 1\).\newline{}
          (\(\Leftarrow\)) Suppose \(|n| \le B\) for all \(n \in \ZZ\) then \[|x+ y|^m = \left| \sum_{r = 0}^{m} \binom{m}{r} x^{m-r}y^r\right| \le \sum_{r=0}^m\left|\binom{m}{r}\right\|x|^{m-r}|y|^r \le (m+1)B \max\{|x|^m, |y|^m\}.\]
          Now letting \(m \to \infty\) we get that \(|x+y| \le \max\{|x|, |y|\}\).
        \end{proof}
\begin{corollary}\label{corollary-1}
All absolute values on fields of characteristic \(p\) are non-archimidean.\end{corollary}
\begin{example}\label{example-3}
\begin{enumerate}
\item{}\(K = \QQ\), \(p = 5\) \(\ab = \ab_5\).
            Let \(a_1 =3,a_2=33,a_3=33\), etc.
            So \(a_n \equiv a_m \pmod{5^n}\) for all \(m \ge n\).
            Then \(|a_n -a_m|_5 \le 5^{-n}\) for all \(m \ge n\) and so \((a_n)_{n\ge1}\) is a Cauchy sequence.
            Now \(a_n = \frac{10^n - 1}{3}\) so \(|a_n - \frac13| = 5^{-n}\to 0\) as \(n \to \infty\) i.e. \( a_n\to-\frac13\) w.r.t. \(\ab_5\)
          \item{}
            We'll construct \((a_n)\) such that for all \(n \ge 1\)
            \[
              \begin{cases}a_n^2 +1  \equiv 0 \pmod{5^n}\\
              a_{n+1} \equiv a_n \pmod{5^n}
              \end{cases}
            \]
            Take \(a_1 = 2\).
            Suppose \(a_n\) is chosen and it satisfies \(a_n^2  + 1 = 5^n c\).
            \((a_n + 5^n b)^2 + 1 \equiv a_n^2 + 1 + 2\cdot 5^n a_nb \equiv 5^n(c+2ba_n) \pmod{5^{n+1}}\).
            We solve for \(b\) s.t. \(2ba_n + c \equiv 0\pmod{5}\).
            Since \((2a_n,5) = 1\) this is always possible.
            Now put \(a_{n+1} = a_n+5^nb\).
            Condition (ii) implies that \(a_n\) is Cauchy.
            Suppose it converges and \(a_n\to l\in \QQ\).
            Then \(|l^2 + 1|_5 \le |a_n^2 + 1|_5 +|a_n^2 - l^2|_5\), both of these terms tend to 0 which gives \(l^2 = -1\) a contradiction.
            This shows that \(\QQ\) is not complete under \(\ab_5\).
          \end{enumerate}
\end{example}
\begin{definition}\label{definition-3}
\(\QQ_p\) is the completion of \(\QQ\) w.r.t. \(\ab_p\).
          Note that \(\QQ_p\) has \(+,\cdot,\ab_p\) as they extend from \(\QQ\) by continuity.
          It is easy to check that \((\QQ_p,\ab_p)\) is a non-archimidean valued field.
        \end{definition}
\begin{definition}\label{definition-4}
\(\ZZ_p=\{x\in\QQ_p : |x|_p \le 1\}.\)\end{definition}
\begin{lemma}\label{lemma-2}
\(\ZZ\) is dense in \(\ZZ_p\), in particular \(\ZZ_p\) is the completion of \(\ZZ\) w.r.t. \(\ab_p\).\end{lemma}
\begin{proof}
\(\QQ\cap \ZZ_p = \{x\in\QQ: |x|_p \le 1\}= \{\frac{a}{b} \in \QQ : p \nmid b,a,b\in\ZZ\} = \ZZ_{(p)}\).
        Let \(\frac{a}{b} \in \ZZ_{(p)}\) i.e. \(a,b \in \ZZ\), \(p\nmid b\).
        For each \(n \ge 1\) we can pick \(y_n \in \ZZ\) s.t. \(by_n \equiv 1 \pmod{p^n}\) implying \(by_n \to_p 1\) as \(n \to \infty\).
        This implies that \(ay_n \to \frac{a}{b}\) as \(n\to \infty\).
        Hence \(\ZZ\) is dense in \(\ZZ_{(p)}\).

        Now for\(\QQ\) is dense in \(\QQ_p\) and \(\ZZ_p \subset \QQ_p\) being open give that \(\QQ\cap \ZZ_p\) is dense in \(\ZZ_p\).
        \end{proof}
\par

        The global situation is as follows.
        \([K: \QQ] \lt \infty  \) \(\cO_K = \) integral closure of \(\ZZ\) in \(K\).
        \(\cO_K\) need not be a UFD.
      %
\par

        The local situation is as follows.
        \([K: \QQ] \lt \infty  \) \(\cO_K = \) integral closure of \(\ZZ_p\) in \(K\).
        \(\cO_K\) is always a UFD! In fact it is a DVR.
        \(K\) number field \(\fp \subset\cO_K\) a prime ideal \(0\lt \alpha\lt 1\).
        For \(x\in K^*\) \(v_{\fp}(x) = \) power of \(\fp\) in the factorisation of \(x\cO_K\).
        Define \[|x|_p = \begin{cases} \alpha^{v_p(x)} \text{if } x\ne 0\\0 \text{otw}\end{cases}\]
        \(\ab_{\fp}\) is an absolute value on \(K\).
        \(K_{\fp}\) is the completion of \(K\) w.r.t. \(\ab_{\fp}\).
        Note that for a suitable choice of \(\alpha\) \(\ab_{\fp}\) extends \(\ab_p\) on \(\QQ\) (where \(\fp \cap \ZZ = p\ZZ\)).
      %
\begin{remark}\label{remark-2}
\([K_{\fp} : \QQ_p] \le [K: \QQ]\).
          Every finite extension of\(\QQ_p\) arises as the completion of some number field.
          Proofs later.
        \end{remark}
\begin{lemma}\label{lemma-3}

          Let \(\ab_1\) and \(\ab_2\) be non-trivial absolute values on a field \(K\) then TFAE:
          \begin{enumerate}
\item{}\(\ab_1\) and \(\ab_2\) define the same topology on \(K\).\item{}\(|x|_1\lt 1 \iff |x|_2 \lt 1\).\item{}\(|x|_2 = |x|_1^c\) for some \(c \gt 0\).\end{enumerate}

          If these conditions hold we say that \(\ab_1\) and \(\ab_2\) are equivalent.
        \end{lemma}
\begin{proof}

          1) \(\implies\) 2) \(|x|_1 \lt 1 \iff x^n \to 0\) as \(n \to \infty\) w.r.t. \(\ab_1 \iff x^n \to 0\) w.r.t. \(\ab_2\) iff \(|x|_2 \lt 1\).\newline{}
          2) \(\implies\) 3) Pick \(a \in K^*\) with \(|a|_1 \lt 1\).
          Let \(x\in K^*\)\(m,n\in \ZZ\)\(n \gt 0\).
          \end{proof}
\begin{definition}\label{definition-5}

          A \terminology{place} of \(K\) is an equivalence class of absolute values on \(K\).
        \end{definition}
\begin{theorem}\label{theorem-1}

          A non-trivial absolute value on \(\QQ\) is equivalent to either \(\ab_\infty\) or \(\ab_p\) for some prime \(p\).
        \end{theorem}
\begin{proof}

          First let \(\ab\) be archimidean.
          Let \(a,b\gt 1\) be integer.
          Write \(b^n\) in base \(a\)\[b^n = c_m a^m + \cdots + c_1a + c_0\]
          where \(0\le c_i\lt a\) and \(m\le n\log_a b\).
          Let \(B= \max\{c : 0 \le c \lt a\}\) then \(|b^n| \le (m+1)B \max\{|a|^m,1\}\).
          This implies \(|b| \le ((n\log_a b + 1)B)^{\frac{1}{n}}\max(|a|^{\frac{m}{n}}, 1)\) taking the limit as\(n \to \infty\) gives that \(|b| \le \max(|a|^{\log_a b}, 1)\).
          Since \(\ab\) is archimidean we may pick an integer \(b\gt 1\) s.t. \(|b| \gt 1\).
          Applying the above inequality for any integer \(a \gt 1\) we get \(|b| \le |a|6{\log_a b}\).
          So \(|a| \gt 1\).
          Swapping \(a\) and \(b\) in the inequality we get \(|a| \le |b|^{\log_b a}\).
          So \[\frac{\log|a|}{\log a} = \frac{\log|b|}{\log b} = \lambda.\]
          Then \(|a| = a^{\lambda}\) for all \(a\in \ZZ_{\ge 1}\) implying \(\ab \sim \ab_\infty\).
          \newline{}
          Now for non-archimidean \(\ab\).
          The ultrametric law implies that \(|n| \le 1\) for all \(n\in \ZZ\),
          \(\ab\) being non-trivial implies that \(|u| \lt 1\) for some \(n\in \ZZ_{\gt 1}\).
          Writing \(n = p_1^{e_1}\cdots p_k^{e_k}\) we get \(|p| \lt 1\) for some \(p\).
          Suppose that \(|p| \lt 1\) and \(|q| \lt 1\) for \(p\ne q\).
          Write \(1 = rp + sq\) so that \(1 = |rp + sq| \le \max \lt 1\) a contradiction.
          So \(|p| =\alpha \) for some \(p\) and \(\ab\) is 1 for all other primes.
          Hence \(\ab \sim\ab_p\).
        \end{proof}
\begin{remark}\label{remark-3}

          If \((K,\ab)\) is archimidean then \(\chara(K) = 0\) and so \(\QQ \subset K\).
          Ostrowski  then impliees that restriction of \(\ab\) to \(\QQ\) is equivalent to \(\ab_\infty\).
          so if \(K\) is complete then it contains a copy of \(\RR\).
        \end{remark}
\par

        Fact: If \((K,\ab)\) is complete and archimidean then \(K= \RR\) or \(K= \CC\) and \(\ab \sim \ab_\infty\) (see Cassels).
      %
\par
From now on we take \(K\) non-archimidean.%
\begin{lemma}\label{lemma-4}

          Let \((K,\ab)\) be non-archimidean.
          Then
          \begin{enumerate}
\item{}\(|x| \gt|y|\) implies \(|x+y | = |x|\).\item{}\(|x_1 +\cdots + x_n| \le \max\{|x_i|\}\) with equality only .\item{}If \((K,\ab)\) is complete then \(\sum_{n=1}^\infty a_n\) converges iff \(a_n\to 0\).\end{enumerate}
\end{lemma}
\begin{proof}
\begin{enumerate}
\item{}\(|x+y| \le \max(|x|,|y|) = |x| \le \max(|x+y|, |y|) = |x + y|\).\item{}Ultrametric + induction. Apply (i) with \(x = x_1\) and \(y = x_1 + \cdots + x_n\).\item{}Let \(s_n = \sum_{i=0}^{n} a_i\), if \(s_n \to l\) then \(a_n = s_n - s_{n-1} \to 0\) as \(n \to \infty\), conversely for \(m \ge n\) we have \(s_m - s_n = |a_{n+1} + \cdots + a_m| \le \max_{i = n+1,\ldots,m}(|a_i|) \lt \max_{i \gt n} |a_i| \to 0\) as \(n \to \infty\) so \(s_n\) is Cauchy, hence convergent.\end{enumerate}
\end{proof}
\par
For \(x\in L\) \(r \gt 0\) we let \(B(x,r) = \{y\in K : |x - y| \lt r\}\) and \(\overline{B}(x,r) = \{y\in K : |x - y| \le r\}\).%
\begin{lemma}\label{lemma-5}
\begin{enumerate}
\item{}If \(y\in B(x,r)\) then \(B (x,r ) = B(y,r)\).\item{}If \(y\in \bar{B}(x,r)\) then \(\bar{B}(x,r) = \bar{B}(y,r)\).\item{}\(B(x,r)\) is both open and closed.\item{}\(\bar{B}(x,r)\) is both open and closed.\item{}\(K\) is totally disconnected (i.e. the only connected subsets are singletons).\end{enumerate}
\end{lemma}
\begin{proof}
\begin{enumerate}
\item{}Ultrametric.\item{}Ultrametric.\item{}\(B(x,r)\) is open, it is closed since if \(y \not\in B(x,r)\) then \(B(x,r)\cap B(y,r) = \emptyset\) as they are not the same ball.\item{}\(\bar{B}(x,r)\) is closed since if \(y\in \bar{B}(x,r)\) then \(B(y,r/2)\subset\bar{B}(y,) = \bar{B}(x,r)\).\item{}Given any \(x,y\in K\) distinct, let \(r = |x-y|/2 \gt 0\) then \(B (x,r)\) and its complement are open sets, one containing \(x\), the other \(y\).\end{enumerate}
\end{proof}
\typeout{************************************************}
\typeout{Chapter 3 Valuations}
\typeout{************************************************}
\chapter[Valuations]{Valuations}\label{chap-valuations}
\typeout{************************************************}
\typeout{Section 3.1 Valuations}
\typeout{************************************************}
\section[Valuations]{Valuations}\label{sec-valuations}
Let \(K\) be a field.%
\begin{definition}[Valuations, discrete, normalised, valuation ring, units maximal ideal]\label{definition-6}
\(v\colon K^* \to \RR\) is called a valuation if
            \begin{enumerate}
\item{}\(v(xy) = v(x)+v(y)\)\item{}\(v(x+y) \ge \min(v(x),v(y))\).\end{enumerate}

            Fix some \(0\lt \alpha\lt 1\), then a valuation \(v\) determines a non-archimidean absolute value, via \[|x| = \begin{cases} \alpha^{v(x)} \text{ if } x \ne 0,\\ 0 \text{ if } x =0.\end{cases}\]
            Conversely given some \(\ab\) we can put \(v(x) = \log|x| / \log\alpha\).
            We ignore the trivial valuation.
            We say two valuations \(v_1,v_2\) are equivalent if for some \(c\in \RR_{\gt 0}\)\( v_1(x )= cv_2(x)\) for all \(x\in K^*\).
            The image \(v(K^*)\) is a subgroup of \(\RR\).
            If it is discrete (i.e. isomorphic to \(\ZZ\)) we say that \(v\) is a discrete valuation, we say it is normalised if \(v(K^*) = \ZZ\).
            We let \(\cO_v = \{x \in K : v(x) \le 1\}\) be the valuation ring.
            \(\cO_v^* = \{x \in K : v(x) = 1\}\) is its unit group.
            \(m = \{x\in K: |x| \lt 1\}\) is a maximal ideal.
            \(k =\cO_v/m\) is the residue field.
          \end{definition}
\begin{remark}\label{remark-4}
\begin{enumerate}
\item{}\(m = \cO_v \smallsetminus \cO_v^*\) so \(m\) is the unique maximal ideal, hence \(\cO_v\) is a local ring.\item{}Let \(x,y \in K^*\), then \(x\cO_v \subset y\cO_v \iff x/y \in \cO_v\iff |x/y| \le 1\iff |x| \le |y|\).\item{}If \(0 \ne x \in m\) then \(K = \cO_v[1/x] = \Frac(\cO_v)\).\item{}\(\cO_v\) is integrally closed in \(K\).
              This is as if \(x\in K\) satisfies \(x^n + \cdots + a_0 = 0\) with \(a_i \in \cO_v\) then \(|x^n| \le \max_{i = 0,\ldots,n-1} |a_ix^i| \le \max(1,|x|^{n-1})\) which implies \(|x| \le 1\) i.e. \(x\in \cO_v\).\end{enumerate}
\end{remark}
\begin{lemma}\label{lemma-6}

            TFAE:
            \begin{enumerate}
\item{}\(v\) is discrete.\item{}\(\cO_v\) is a PID.\item{}\(\cO_v\) is a Noetherian.\item{}\(m\) is principal.\end{enumerate}
\end{lemma}
\begin{proof}

            Note that \(x\cO_v \subset u \cO_v \iff |x| \le |y|\).
            For i) \(\implies\) ii) Take \(I \subset \cO_v\) and pick \(a\in I\) with \(|a| = \max\{|x| : x\in I\}\) equivalently \(v(a) = \min\{v(x) : x\in I\}\).
            This minimum exists as \(v\) is discrete.
            Then \(I = a\cO_v\).
            ii) \(\implies\) iii) is clear.
            For iii) \(\implies\) iv) assume \(m = x_1 \cO_v + \cdots + x_n \cO_v\) wlog \(|x_1|\ge \cdots \ge |x_n|\) then \(m = x_1 \cO_v\).
            For iv) \(\implies\) i) Write \(m = \pi \cO_v\) and let \(c = v(\pi) \gt 0\). If \(x\in K^*\) with \(v(x) \gt 0\) then \(x\in m\) and so \(v(X) \ge c\) and therefore \(v\) is discrete.
          \end{proof}
\begin{definition}[DVR]\label{definition-7}

            A DVR is a PID with exactly one non-zero prime ideal.
          \end{definition}
\begin{lemma}\label{lemma-7}
\begin{enumerate}
\item{}
                If \(v\) is discrete then \(\cO_v\) is a DVR.
              \item{}Let \(R\) be a DVR, then there exists a discrete valuation on \(K = \Frac(R)\) s.t. \(R = \cO_v\) with \(v\) unique up to normalisation.\end{enumerate}
\end{lemma}
\begin{proof}
\begin{enumerate}
\item{}\(\cO_v\) is a local ring, lemma 2.1  implies that \(\cO_c\) is a PID, this gives that \(\cO_v\) is a DVR\item{}Let \(R\) be a DVR with prime element \(\pi\).
              Every \(x\in R\smallsetminus \{0\}\) can be written as \(u\pi^m\) with \(u\in R^*\), \(m\ge 0\).
              Every \(x\in K^*\) can be written as \(u\pi^m\) with \(u\in R^*\), \(m\in \ZZ\).
              We define \(v \colon K^* \to \ZZ\) by \(u\pi^m \mapsto m\), this is a discrete valuation with \(\cO_v = R\).\end{enumerate}
\end{proof}
\begin{example}\label{example-4}
\(\ZZ_{(p)} = \{x\in \QQ : |x|_p \le 1\}\) is a DVR with field of fractions \(\QQ\).
            \(\ZZ_{(p)} = \{x\in \QQ_p : |x|_p \le 1\}\) is a DVR with field of fraction \(\QQ_p\).
            In both of these examples the residue field is \(\FF_p\).
          \end{example}
\par

          For the rest of this section let \(v\colon K^* \to \ZZ\) be a normalised discrete valuation.
          Additionally pick \(\pi \in K^*\) with \(v(\pi) = 1\) so that \(m = \pi\cO_v\) (\(\pi\) is called the uniformiser).
        %
\begin{lemma}[Hensel's lemma (version 1)]\label{lemma-8}

            Assume \(K\) is complete w.r.t. \(v\) and let \(f \in \cO[x]\).
            Suppose that the reduction \(\bar{f} \in k[x]\) has a simple root in \(k\) i.e. there exists \(a\in \cO\) s.t. \(f(a) \equiv 0 \pmod{\pi}\) (i.e \(|f(a)| \lt 1\)) and  \(f'(a) \not\equiv 0 \pmod{\pi}\) (i.e \(|f'(a)| = 1\)).
            Then there exists a unique \(x\in \cO\) s.t.
            \(f(x) = 0\) and \(x\equiv a\pmod{\pi}\).
          \end{lemma}
\begin{proof}
Follows from version 2 of Hensel's Lemma .\end{proof}
\begin{lemma}[Hensel's lemma (version 2)]\label{lemma-9}

            Assume \(K\) is complete w.r.t. \(v\) and let \(f \in \cO[x]\).
            Suppose that there exists \(a\in \cO\) s.t. \(|f(a)| \lt |f'(a)|^2\)).
            Then there exists a unique \(x\in \cO\) s.t.
            \(f(x) = 0\) and \(|x-a| \lt |f'(a)|\).
          \end{lemma}
\begin{proof}

            Let \(r = v(f'(a))\).
            We construct a sequence \((x_n)\in \cO\) s.t.
            \begin{enumerate}
\item{}\(f(x_n) \equiv 0 \pmod{\pi^{n + 2r}}\),\item{}\(x_{n+1} \equiv x_n \pmod{\pi^{n + r}}\).\end{enumerate}

            We put \(x_1 = a\).
            Let \(n\ge 1\) and suppose \(x_n\) satisfies 1, i.e. \(f(x_n) = c\pi^{n + 2r}\) for some \(c\in \cO\).
            We'll put \(x_{n+1} = x_n + b\pi^{n+r}\) for some \(b \in \cO\).
            We have that
            \[f(X+Y) = f_0(X) + f_1(X) Y + f_2(X) Y^2 + \cdots\]
            for some \(f_i\in \cO[X]\).
            We have \(f_0 = f\), \(f_1  =f'\).
            \(f(x_{n+1}) = f(x_n + b\pi^{n +r}) \equiv f(x_n) + f'(x_n)b\pi^{n+r} \pmod{\pi^{n + 2r + 1}}\)
            But \(x_n \equiv a \pmod{\pi^{r+1}}\) which implies that \(f'(x_n) \equiv f'(a) \pmod{\pi^{r+1}}\), implying \(f'(x_n) = u \pi^r\) for some \(u \in \cO^*\).
            Note that this argument shows that if \(x \in \cO\) and \(|x-a| \lt |f'(a)|\) then \(|f'(a)| = |f'(x)|\).
            Therefore \(f(x_{n+1}) \equiv c\pi^{n+2r} + u \pi^r b\pi^n \pmod{\pi^{n+2r + 1}} \equiv (c + ub) \pi ^{n + 2r}\pmod{\pi^{n+2r+1}}\).
            Taking \(b = -c/u\) (note \(b \in \cO\)) gives that \(f(x_{n+1}) \equiv 0 \pmod{\pi^{n + 2r + 1}}\), so the first property holds for \(n+1\).
            The second property implies that \((x_n)\) is Cauchy and the first gives that \(x = \lim_{n\to \infty}x_n\) is a root of \(f\).
            Note that \(x_n \equiv a \pmod{\pi^{n + 1}}\) for all \(n\) implies that \(x \equiv a \pmod{\pi^{r+1}}\) and so \(|x-a| \lt f'(a)|\).
            \newline{}
            To see uniqueness suppose \(x\) and \(y\) both satisfy the above note .
            Assume moreover that \(s = x-y \ne 0\).
            The inequalities \(|x-a| \lt |f'(a)|\) and \(|y-a| \lt |f'(a)|\) together imply that \(|s| \lt |f'(a)|\).
            Now \(0 = f(y) = f(x + s) = f(x) + f'(x) s + \cdots\) and therefore \(|f'(x)s |\lt |s|^2\) and so \(|f'(a)| = |f'(x)| \le |s|\) which contradicts the above.
          \end{proof}
\begin{remark}\label{remark-5}

            In the proof of Hensel's lemma \(x_{n+1} = x_n - f(x_n)/f'(x_n)\) which is as in the Newton-Raphson method.
          \end{remark}
\begin{remark}\label{remark-6}

            Uniqueness in Hensel's .
            We saw that every element of \(K^*\) is of the form \( u \pi^r\) for some \(u \in R^*\) and \(m \in \ZZ\).
            If \(R= \cO_v\) for some valuation \(v \colon K^* \to \RR\) then \(v(x) \ge 0\) for all \(x \in R\).
            So if \(u \in R^*\) then \(v(u) = 0\).
            In particular \(v(u\pi^m) = m v(\pi)\), i.e. \(v\) is uniquely determined by \(v(\pi)\) so it is unique if we normalise it.
          \end{remark}
\begin{lemma}\label{lemma-10}

            Let \(A \subset \cO_v\) be a set of coset representatives for \(k = \cO_v/m\).
            Then every \(x \in \cO_v\) can be written uniquely as \[x = \sum_{n=0}^\infty a_n \pi ^n\] with \(a_n \in A\).
          \end{lemma}
\begin{proof}

            There exists a unique \(a_0 \in A\) s.t. \(x \equiv a_0 \pmod{\pi}\).
            So \(x = a_0 + \pi x_1\) for some \(x_1 \in \cO_v\).
            There now exists a unique \(a_1 \in A\) s.t. \(x_1 \equiv a_1 \pmod{\pi}\).
            So \(x = a_0 + \pi a_1 + \pi^2 x_2\) for some \(x_2 \in \cO_v\).
            We may continue this process.
            Letting \(s_N = \sum_{n=0}^\infty a_n \pi ^n\) we get \(v(x - s_N) \gt N\).
            This gives that \(s_N \to x\) as \(N \to \infty\).
            So \(x = \lim_{N \to \infty} s_N = \sum_{n=0}^\infty a_n \pi ^n\).
            Uniqueness is clear.
          \end{proof}
\begin{remark}\label{remark-7}
\(K\) being complete is equivalent to every sequence \(\sum_{n=0}^\infty a_n \pi^n\) converging.
            One direction is trivial the other is an exercise.
          \end{remark}
\begin{proposition}\label{proposition-1}
\[\QQ_p^* /(\QQ_p^*)^2\cong \begin{cases} (\ZZ/2\ZZ)^2& \text{ if } p \ne 2,\\ (\ZZ/2\ZZ)^3& \text{ if } p= 2.\end{cases}\]\end{proposition}
\begin{proof}
\begin{enumerate}
\item{}
                Assume \(p \ne 2\).
                For \(b \in \ZZ_p^*\) we have \(b \in (\ZZ_p^*)^2 \iff \bar b \in (\FF_p^*)^2\) (applying Hensel's lemma  to \(f(x) = x^2 -b\)).
                Therefore \(\ZZ_p^*/(\ZZ_p^*)^2 \xrightarrow{\sim} \FF_p^*/(\FF_p^*)^2 \cong \ZZ/2\ZZ\).
                But \(\QQ_p^* \cong \ZZ_p^* \times \ZZ\) via the map \(u p^r \mapsto (u,r)\).
                Therefore \(\QQ_p^*/(\QQ_p^*)^2\cong \ZZ_p^*/(\ZZ_p^*)^2 \times \ZZ/2\ZZ\cong (\ZZ/2\ZZ)^2\).
                We have coset reps \(1,p,u,pu\) where \(u\) is a non-square mod \(p\).
              \item{}
                Let \(p = 2\).
                Take \(b \in \ZZ_2^*\) with \(b \equiv 1 \pmod{8}\), let \(f(x) = x^2 -b\).
                Then \(|f(1) | \le 2^{-3}\lt 2^{-2} = |f'(1)|^2\).
                Hensel's lemma  now gives us that \(f\) has a root in \(\ZZ_2\).
                So \(\ZZ_2 ^* \to (\ZZ/8\ZZ)^* \cong (\ZZ/2\ZZ)^2\).
                Clearly \((\ZZ_2^*) \subset \ker\).
                We just checked that \(\ker \subset (\ZZ_2^*)^2\).
                Therefore \(\ZZ_2^*/(\ZZ_2^*)^2  \cong (\ZZ/2\ZZ)^2\).
                Hence \(\QQ_2^*/(\QQ_2^*)^2\cong (\ZZ/2\ZZ)^3\).
                We have coset reps \(2^a(-1)^b5^c\) where \(a,b,c\in\{0,1\}\).
              \end{enumerate}
\end{proof}
\begin{corollary}\label{corollary-3}
\(\QQ_p\) (for \(p \ne 2\)) has exactly 3 quadratic extensions.
            \(\QQ_2\) has exactly 7.
          \end{corollary}
\typeout{************************************************}
\typeout{Section 3.2 Examples of DVR's continued}
\typeout{************************************************}
\section[Examples of DVR's continued]{Examples of DVR's continued}\label{sec-dvrs}
\begin{example}\label{example-5}

            Let \(k\) be any field, \(K = k(t)\), \(v_0(t^n p(t)/q(t)) = n\), where \(p,q\in k[t]\) and \(p(0),q(0) \ne 0\).
            Then \(\cO = \{f(t) = p(t)/q(t) : f(0)\text{ is defined i.e. } q(0) \ne 0\}\).
            And \(\cO^* = \{f(t) = p(t)/q(t) : f(0)\text{ is defined and non-zero}\}\).
            \(m = \{f(t) : f(0) = 0\}\).
            \(\cO/m \cong k\) via the map \(f\mapsto f(0)\).
            Likewise for other \(a \in k\), \(v_a((t-a)^n p(t)/q(t)) = n\), where \(p,q\in k[t]\) and \(p(a),q(a) \ne 0\), this is the order of zero/pole at \(t = a\).
            We also have \(v_\infty(p(t) / q(t)) = v_0 (p (1/\epsilon)/q(1/\epsilon)) = \deg (q) - \deg (p)\).
          \end{example}
\begin{remark}\label{remark-8}
\begin{enumerate}
\item{}
                If \(k = \bar k\) then these are the only valuations on \(K = k(t)\) with \(v (k^* ) = 0\).
              \item{}
                \(K = k(t)\) is the function field of \(\PP^1\).
                Similar examples arise for any smooth point of an algebraic curve/Riemann surface.
              \end{enumerate}
\end{remark}
\begin{example}\label{example-6}
\(K = k((t)) = \) the field of Laurent power series \[ = \left\{ \sum_{n \ge n_0} a_n t^n : a_n \in k\right\}.\]
            We have \(v ( \sum a_n t^n ) = \min \{n : a_n  \ne 0\}\).
            Then \(\cO = k[[t]] =\) ring of power series in \(t\).
            We get \(m = \{ f \in k[[t]]: f(0) = 0\}\) and we have \(\cO/m = k\).
          \end{example}
\begin{lemma}\label{lemma-11}
\begin{enumerate}
\item{}\(k[[t]]^* = \{\sum _{n=0}^\infty a_n t^n : a_0 \ne 0\}\).\item{}\(k((t))\) is a field and \(v\) extends to \(v_0\) on \(k(t)\).\item{}\(k[[t]]\) is the completion of \(k[t]\) w.r.t. \(v_0\).\item{}\(k((t))\) is the completion of \(k(t)\) w.r.t. \(v_0\).\end{enumerate}
\end{lemma}
\begin{proof}
\begin{enumerate}
\item{}
                Let \(\sum_{n=0}^\infty a_n t^n\in k[[t]]\) with \(a_0 \ne 0\).
                We solve for \(b_n\) such that \[\left(\sum_{n=0}^\infty a_n t^n\right)\left(\sum_{n=0}^\infty b_n t^n\right) = 1.\]
              \item{}
                By (i) we have \(k((t)) = \Frac k[[t]]\).
                In particular \(k((t))\) is a field containing \(k[t]\) so \(k(t) \subset k((t))\).
                If \(f(t) = t^n p(t)/q(t)\) with \(p,q\in k[t]\) and \(p(o),q(o) \ne 0\) then by (i) \(p,,q \in k[[t]]^*\) so \(v(f) = n  = v_0 (f)\).
              \item{}
                Let \(f_1,f_2,\ldots\) be a Cauchy sequence in \(k[[t]]\).
                Then given \(r\) there exists \(N\) s.t. for all \(m,n \ge N\) we have \(f_m \equiv f_n \pmod{t^{n+1}}\).
                Let \(c_r  = \) coefficient of \(t^r\) in \(f^N\).
                Then \(f_n \to g\) where \(g = \sum_{r=0}^\infty c_r t^r\) and therefore \(k[[t]]\) is complete.
                But \(k[t] \subset k[[t]]\) is a dense subset, therefore \(k[[t]]\) is the completion of \(k[t]\).
              \item{}
                Likewise.
              \end{enumerate}
\end{proof}
\typeout{************************************************}
\typeout{Section 3.3 The Teichmuller map}
\typeout{************************************************}
\section[The Teichmuller map]{The Teichmuller map}\label{sec-teichmuller}
\begin{definition}[Teichmuller representatives]\label{definition-8}

            Let \(k\) be complete w.r.t. a discrete valuation \(v\).
            Suppose that the residue field \(k\) is finite, say \(|k| = q\).
            Let \(f(x) = x^q - x \in \cO[X]\).
            Each \(\alpha \in k\) is a simple root of \(\bar f \in k[x]\).
            Hensel's lemma implies that there is a unique \(a \in\cO\) s.t. \(a^q = a\) and \(a \equiv \alpha \pmod{\pi}\).
            The \(a \in \cO\) constructed here is the \terminology{Teichmuller representative} for \(\alpha \in k\).
          \end{definition}
\begin{lemma}\label{lemma-12}

            The map \([\cdot]\colon k \to \cO\) given by \(\alpha \mapsto a\) is multiplicative.
          \end{lemma}
\begin{proof}

            Let \(\alpha,\beta \in k\) we have \(([\alpha][\beta])^q = [\alpha]^q[\beta]^q = [\alpha][\beta]\) and \([\alpha][\beta] \equiv \alpha\beta \pmod{\pi}\) giving \([\alpha\beta]  =[\alpha][\beta]\).
          \end{proof}
\begin{example}\label{example-7}
\(\mu_{p-1} \subset \ZZ_p^*\).
          \end{example}
\begin{theorem}\label{theorem-2}

            Let \(K\) be field complete w.r.t. a discrete valuation \(v\).
            If \(\chara K \gt 0\) and \(k\) is finite then \(K \cong k((t))\).
          \end{theorem}
\begin{proof}
\(\chara(K) = \chara(k) = p\) and \(|k| = q = p^l\).
            Let \(\alpha,\beta \in k\) so we have \(([\alpha] + [\beta])^q = [\alpha]^q + [\beta]^q = [\alpha] + [\beta]\) and hence \([\alpha + \beta] = [\alpha] + [\beta]\).
            Therefore the Teichmuller map \(k \hookrightarrow K\) is a field embedding.
            By lemma 2.3 \[K = \left\{ \sum_{n \ge n_0}^\infty a_n \pi^n : a_n \in k\right\}\xrightarrow{\sim} k((t))\]
            via the map \(\pi \mapsto t\).
          \end{proof}
\typeout{************************************************}
\typeout{Chapter 4 Dedekind domains}
\typeout{************************************************}
\chapter[Dedekind domains]{Dedekind domains}\label{chap-dedekind}
\typeout{************************************************}
\typeout{Section 4.1 Dedekind domains}
\typeout{************************************************}
\section[Dedekind domains]{Dedekind domains}\label{sec-dedekind}
\begin{definition}[Dedekind domains]\label{definition-9}

            A \terminology{Dedekind domain} is a ring \(R\) that is
            \begin{enumerate}
\item{}an integral domain,\item{}Noetherian,\item{}integrally closed,\item{}has all non-zero prime ideals maximal (Krull dimension \(\le 1\)).\end{enumerate}
\end{definition}
\begin{example}\label{example-8}

            Any PID is a Dedekind domain.\newline{}
            The ring of integers of a number field is a Dedekind domain.
          \end{example}
\begin{theorem}\label{theorem-3}

            Let \(R\) be a Dedekind domain.
            Then every non-zero ideal \(I \subset R\) can be written uniquely as a product of prime ideals \(I = p_1^{\alpha_1} \cdots p_r^{\alpha_r}\).
          \end{theorem}
\begin{proof}

            Omitted.
          \end{proof}
\begin{remark}\label{remark-9}

            If \(R\) is a PID then the above theorem  follows from PID implying UFD.
          \end{remark}
\begin{theorem}\label{theorem-4}
\(R\) is a DVR if and only if it is a Dedekind domain with exactly one non-zero prime.
          \end{theorem}
\begin{proof}
\(\implies\) is clear since being a PID implies being Dedekind.\newline{}\end{proof}
\typeout{************************************************}
\typeout{Section 4.2 Localisation}
\typeout{************************************************}
\section[Localisation]{Localisation}\label{sec-localisation}
Let \(R\) be an integral domain and \(p \subset R\) a prime ideal.
          Let \(S = R\smallsetminus p\) and \(S^{-1}R = \{\frac{r}{s} : r\in R,\, s\in S\}\subset\Frac R\).
          This is a local ring with maximal ideal \(S^{-1}p\).
          \(R\) being Dedekind implies \(S^{-1}R\) is, and hence \(S^{-1}R\) is a DVR by the above theorem 
        %
\begin{theorem}\label{theorem-5}

            Let \(\cO_K\) be a Dedekind domain, \(K = \Frac \cO_K\) and \(L/K\) a finite field extension.
            Then take \(\cO_L\) to be the integral closure of \(\cO_K\) in \(L\), then \(\cO_L\) is a Dedekind domain.
          \end{theorem}
\begin{proof}

            That \(\cO_L\) is a domain is clear.\newline{}
            It is also clear that it is integrally closed.\newline{}
            To see it is Noetherian we suppose that \(L/K\) is a separable and write \(n = [L:K]\).
            There are \(n\) distinct embeddings \(\sigma_1,\ldots,\sigma_n\colon L \to \bar K\).
            The trace form \(L\times L \to K\) given by \((x,y)\mapsto \tr xy\) is a non-degenerate \(K\)-bilinear form.
            (Using primitive element and expressing as a Vandermonde determinant)
            Let \(x_1,\ldots,x_n\) be a basis for \(L\) as  \(K\)-vector space.
            Clearing denominators allows us to take \(x_i \in \cO_L\).
            Let \(y_1,\ldots,y_n\) be the dual basis w.r.t. the trace form.
            Let \(z\in \cO_L\) then \(z = \sum_{j =1}^{n}\lambda_j y_j\) for some \(\lambda_i \in K\), giving that \(\lambda_i = \tr_{L/K}(x_i z)\in \cO_K\).
            So \(\cO_L \subset y_1\cO_K + \cdots + y_n \cO_K\).
            \(\cO_K\) being noetherian gives that \(\cO_L\) is a f.g. \(\cO_K\)-module and so \(a\cO_L\) is noetherian too.\newline{}
            Now to see that all non-zero primes are maximal we fix \(P\) a non-zero prime of \(\cO_L\).
            Then taking \(p = P \cap \cO_K\) we have a prime of \(\cO_K\).
            We can find \(x \in p \smallsetminus 0\) so that \(0 \ne N_{L/K} (x) \in P \cap \cO_K = p\).
            \(\cO_K\) being Dedekind gives \(p\) maximal and so \(k = \cO_K/p \hookrightarrow \cO_L/P\) gives that \(\cO_L/P\) is a f.d. \(k\)-algebra.
            But \(\cO_L/P\) is an integral domain and so applying rank-nullity to \(\cdot x\) gives \(\cO_L/P\) is a field.
          \end{proof}
\begin{lemma}\label{lemma-13}

            If \(R\) is a Dedekind domain and \(K = \Frac R\) with an absolute value \(\ab\) on \(K\) with \(|x| \le 1 \) for all \(x\in R\) then \(\ab \sim \ab_p\) for some prime \(p\) of \(R\).
          \end{lemma}
\begin{proof}

            Lemma 1.1  implies \(\ab\) is non-archimidean.
            Let \(p = \{x\in R : |x| \lt 1\}\), this is a prime ideal.
            Localising at \(R\smallsetminus p\) gives a DVR by Theorem 3.2  this has valuation \(v_p\).
            Take \(\pi \in p \smallsetminus p^2 \) so we can write \(x \in K^*\) as \(x= u\pi^r\) where \(|u|_p = 1\) and \(r\in \ZZ\).
            To prove \(\ab\sim\ab_p\) we need to show that \(|\pi| \lt 1\) (which is true as \(\pi \in p\)) and also that \(|u| = 1\).
            We have \(|u|_p \le 1\) which gives \(u \in S^{-1}R\) so \(u = r/s\) for \(r\in R, s\in R\) we know \(|r| \le 1\) and \(|s| = 1\) so \(|u| \le 1\).
            We can do the same for \(u^{-1}\) to get \(|u| = 1\).
          \end{proof}
\begin{theorem}\label{theorem-6}
\(\cO_K\) is a Dedekind domain, \(K = \Frac\cO_K\), \(L/K\) finite field extension, \(\cO_L\) integral closure of \(\cO_K\) in \(L\).
            Let \(p \subset \cO_K\) be a prime ideal, then \(p \cO_L = P_1^{e_1} \cdots P_r^{e_r}\) for distinct primes \(P_i\).
            Then the absolute values on \(L\) extending \(\ab_p\) on \(K\) are (up to equivalence) \(\ab_{P_1},\ldots,\ab_{P_r}\).
          \end{theorem}
\begin{proof}
\(\cO_L\) is Dedekind .
            Take \(x \in K^*\) we have \(v_{P_i}(x) = e_i v_p(x)\) so \(\ab_{P_i}\) is equivalent to an absolute value extending \(\ab_p\).
            Now let \(\ab\) be any such absolute value on \(L\), it will be non-archimidean by .
            \(\cO_K \subset \{x\in L : |x\ \le 1\}\) so taking integral closures in \(L\) we get \(\cO_L \subset \{x\in L : |x| \le 1\}\).
            Lemma 3.4  gives that \(\ab\sim \ab_P\) for some prime \(P\subset \cO_L\).
            But \(\ab\) extends \(\ab_p\) so \(P \cap \cO_K = p\) giving \(P = P_i\) for some  \(i\).
          \end{proof}
\begin{corollary}\label{corollary-4}

            The non-archimidean places of a number field \(K\) are \(\ab_p\) for \(p\) a prime of \(\cO_K\).
          \end{corollary}
\typeout{************************************************}
\typeout{Chapter 5 Relative extensions}
\typeout{************************************************}
\chapter[Relative extensions]{Relative extensions}\label{chap-relative}
\begin{definition}[Norms]\label{definition-10}

          Let \(V\) be a vector space over\(K\).
          A \terminology{norm} on \(v\) is a map \(\|\cdot\|\colon V \to \RR\) s.t.
          \begin{enumerate}
\item{}\(\|v\| \ge 0\) with equality iff \(v = 0\).\item{}\(\|\lambda v\| = |\lambda| \|v\|\)\item{} \(\|v+w\| \le \|v\| + \|w\|\) .\end{enumerate}
\end{definition}
\begin{theorem}\label{theorem-7}

          Let \(K\) be complete \(\ab\) on \(K\). If \(\dim_K V \lt \infty\) then any two norms on \(V\) are equivalent and \(V\) is complete (w.r.t. any one of them).
        \end{theorem}
\begin{proof}

          WLOG \(V = K^d\) and we will show every norm is equivalent to \(\|\cdot\|_{\sup}\), the proof is via induction on \(d\).
          For \(d = 1\)\(\|v\| = c\|v\|_{\sup}\) for some \(c \gt 0\) and the result is clear.\newline{}
          For generl \(d\) we let \(e_1,\ldots,e_d\) be the standard basis, so
          \[
            \|x\| = \left\| \sum_{i=1}^d x_ie_i\right\| \le \left(\sum_{i=1}^d \|e_i\|\right)\max_{1\le i\le d}|x_i|.
          \]
          Let \(S = \{v\in V | \|v\|_{\sup} = 1\}\) (the equation  implies that \(\|\cdot\| \colon S\to \RR_{\ge 0}\) is continuous w.r.t. \(\|\cdot\|_{\sup}\) but we don't know \(S\) is compact).\newline{}
          We now claim that there exists \(\epsilon \gt 0\) s.t. \(\|x\| \gt \epsilon \) for all \(x\in S\).
          To see this suppose otherwise, i.e. that there exists a sequence \((x^{(n)})\) in\(S\) with \(\|x^(n) \|\to 0\) as \(n \to \infty\).
          For at least one \(1\le i \le d\)\(\|x^{(n)}\|_{\sup} = |x_i^{(n}|\) for infinitely many \(n\).
          WLOG this is \(i=d\) and we may pass to a subsequence and multiply through by \(\lambda\in K\) with \(|\lambda| = 1\) to ensure \(x_d^{(n)} = 1\) i.e. \(x^{(n)} = y^{(n)} + e_d \) for some \(y^{(n)} \in \langle e_1,\ldots,e_{d-1}\rangle\).
          But as \( x^{(n)} \to 0\) w.r.t. \(\|\cdot\|\) we have that \(x^{(n)}\) is Cauchy and hence so is \(y^{(n)}\) w.r.t. \(\|\cdot\|\).
          This implies that \(y^{(n)} \to y\) w.r.t. \(\|\cdot\|\) (since \(K^{d-1}\) is complete by the induction hypothesis) for some \(y \in \langle e_1,\ldots,e_{d-1}\rangle\) but \(y^{(n)} = x^{(n)} -e_d \to -e_d\) w.r.t. \(\|\cdot\|\) therefore \(y = -e_d \not\in \langle e_1,\ldots,e_{d-1}\rangle\) a  contradiction, proving the claim. \newline{}
          Now let \(x\in V\)\(x\ne 0\) and \(\|\cdot\|_{\sup}=|x_i|\) for some \(1\le i\le d\). 
          \(x/x_i\in S\) son \(\|x/x_i\|\gt \epsilon\) implying \(\|x\|\gt \epsilon |x_i | = \epsilon \|x\|_{\sup}\).
          This together with the above equation give that \(\|\cdot\|\) and \(\|\cdot\|_{\sup}\) are equivalent, \(K\) complete implies that \(V\) is complete w.r.t. \(\|\cdot\|_{\sup}\).
        \end{proof}
\begin{theorem}\label{theorem-8}
\((K,\ab)\) complete \(L/K\) finite extension.
          If \(\ab_1,\ab_2\) absolute values on \(L\) extending \(ab\) on \(K\) then \(\ab_1=\ab_2\) and \(L\) is complete w.r.t. \(\ab_1\).
        \end{theorem}
%
\backmatter
%
\end{document}