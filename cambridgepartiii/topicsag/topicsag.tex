%%                                    %%
%% Generated from MathBook XML source %%
%%    on 2015-03-03T12:20:37Z    %%
%%                                    %%
%%   http://mathbook.pugetsound.edu   %%
%%                                    %%
\documentclass[10pt,]{book}
%% Load geometry package to allow page margin adjustments
\usepackage{geometry}
\geometry{letterpaper,total={5.0in,9.0in}}
%% Custom Preamble Entries, early (use latex.preamble.early)
%% Inline math delimiters, \(, \), made robust with next package
\usepackage{fixltx2e}
%% Page Layout Adjustments (latex.geometry)
%% For unicode character support, use the "xelatex" executable
%% If never using xelatex, the next three lines can be removed
\usepackage{ifxetex}
\ifxetex\usepackage{xltxtra}\fi
%% Symbols, align environment, bracket-matrix
\usepackage{amsmath}
\usepackage{amssymb}
%% extpfeil package for certain extensible arrows,
%% as also provided by MathJax extension of the same name
\usepackage{extpfeil}
%% allow more columns to a matrix
%% can make this even bigger by overiding with  latex.preamble.late  processing option
\setcounter{MaxMatrixCols}{30}
%% XML, MathJax Conflict Macros
%% Two nonstandard macros that MathJax supports automatically
%% so we always define them in order to allow their use and
%% maintain source level compatibility
%% This avoids using two XML entities in source mathematics
\newcommand{\lt}{<}
\newcommand{\gt}{>}
%% Semantic Macros
%% To preserve meaning in a LaTeX file
%% Only defined here if required in this document
%% Used for inline definitions of terms
\newcommand{\terminology}[1]{\textbf{#1}}
%% Subdivision Numbering, Chapters, Sections, Subsections, etc
%% Subdivision numbers may be turned off at some level ("depth")
%% A section *always* has depth 1, contrary to us counting from the document root
%% The latex default is 3.  If a larger number is present here, then
%% removing this command may make some cross-references ambiguous
%% The precursor variable $numbering-maxlevel is checked for consistency in the common XSL file
\setcounter{secnumdepth}{3}
%% Environments with amsthm package
%% Theorem-like enviroments in "plain" style, with or without proof
\usepackage{amsthm}
\theoremstyle{plain}
%% Numbering for Theorems, Conjectures, Examples, Figures, etc
%% Controlled by  numbering.theorems.level  processing parameter
%% Always need a theorem environment to set base numbering scheme
%% even if document has no theorems (but has other environments)
\newtheorem{theorem}{Theorem}[section]
\renewcommand*{\proofname}{Proof}%% Only variants actually used in document appear here
%% Numbering: all theorem-like numbered consecutively
%% i.e. Corollary 4.3 follows Theorem 4.2
\newtheorem{proposition}[theorem]{Proposition}
%% Definition-like environments, normal text
%% Numbering for definition, examples is in sync with theorems, etc
%% also for free-form exercises, not in exercise sections
\theoremstyle{definition}
\newtheorem{definition}[theorem]{Definition}
\newtheorem{example}[theorem]{Example}
\newtheorem{exercise}[theorem]{Exercise}
%% Equation Numbering
%% Controlled by  numbering.equations.level  processing parameter
\numberwithin{equation}{section}
%% Raster graphics inclusion, wrapped figures in paragraphs
\usepackage{graphicx}
%% Colors for Sage boxes and author tools (red hilites)
\usepackage[usenames,dvipsnames,svgnames,table]{xcolor}
%% hyperref driver does not need to be specified
\usepackage{hyperref}
%% Hyperlinking active in PDFs, all links solid and blue
\hypersetup{colorlinks=true,linkcolor=blue,citecolor=blue,filecolor=blue,urlcolor=blue}
\hypersetup{pdftitle={Part III Topics in Algebraic Geometry 2014}}
%% If you manually remove hyperref, leave in this next command
\providecommand\phantomsection{}
%% Custom Preamble Entries, late (use latex.preamble.late)
\usepackage[all]{xy}
%% Convenience macros
\DeclareMathOperator{\CC}{\mathbf{C}}
\DeclareMathOperator{\QQ}{\mathbf{Q}}
\DeclareMathOperator{\RR}{\mathbf{R}}
\DeclareMathOperator{\ZZ}{\mathbf{Z}}
\DeclareMathOperator{\aff}{\mathbf{A}}
\DeclareMathOperator{\sO}{\mathcal{O}}
\DeclareMathOperator{\fm}{\mathfrak{m}}
\DeclareMathOperator{\Spec}{Spec}
\DeclareMathOperator{\Gal}{Gal}
%% Title page information for book
\title{Part III Topics in Algebraic Geometry 2014}
\author{}
\date{}
\begin{document}
\frontmatter
%% half-title
\thispagestyle{empty}
\vspace*{\stretch{1}}
\begin{center}
{\Huge Part III Topics in Algebraic Geometry 2014}
\end{center}\par
\vspace*{\stretch{2}}
\clearpage
\thispagestyle{empty}
\clearpage
\maketitle
\clearpage
\thispagestyle{empty}
\vspace*{\stretch{2}}
\vspace*{\stretch{1}}
\clearpage
\setcounter{tocdepth}{1}
\renewcommand*\contentsname{Contents}
\tableofcontents
\mainmatter
\typeout{************************************************}
\typeout{Chapter 1 Schemes}
\typeout{************************************************}
\chapter[Schemes]{Schemes}\label{chap-schemes}
\typeout{************************************************}
\typeout{Section 1.1 Introduction}
\typeout{************************************************}
\section[Introduction]{Introduction}\label{sec-introduction}
These are lecture notes for the 2014 Part III Topics in Algebraic Geometry course taught by Dr. Mark Gross, these notes are part of \href{https://alexjbest.github.io/mjolnir/}{Mjolnir}.%
\par
Some reference for commutative algebra:
          \begin{itemize}
\item{}Atiyah-Macdonald\item{}Matsumura - Commutative Algebra \item{}Matsumura - Commutative Ring Theory\end{itemize}

          The general course reference is Hartshorne.
        %
\par

          Generated: March 3, 2015, 12:20:37 (Z)
        %
\typeout{************************************************}
\typeout{Section 1.2 Motivation}
\typeout{************************************************}
\section[Motivation]{Motivation}\label{sec-motivation}
Take \(I \subset k[x_1,\ldots,x_n]\) then \(X = V(I) \subset \aff^n\) and \((X,\sO_X)\) is a ringed space.
          \(A(X) = k[x_1,\ldots,x_n]/\sqrt{I}\).
          A point of \(X\) is given by a map of \(k\)-algebras \(\phi\colon A(X) \to k\) \(x_i\mapsto a_i\), giving the point \((a_1,\ldots,a_n)\in X\).
          The kernel of \(\phi\) is a maximal ideal and conversely given a maximal ideal \(m\subset A(C)\) we get a map \(A(X) \to A(X)/m  = k\) if \(k = \bar{k}\) by the nullstellensatz.
          Similarly if \(l\) is a field extension of \(k\) then a \(k\)-algebra homomorphism \(A(X) \to l\) can be viewed as giving a solution to the system of equations with values in \(l\).
          Note that the group \(\Gal(l|k)\) acts on the set of solutions over \(l\) by postcomposition.
          We might as well consider all possible field extensions \(l|k\), then \(\ker(A(X) \to l)\) might not be maximal.
        %
\begin{example}\label{example-1}
\(X = \aff^1\)\(k[X] \hookrightarrow k(X)\).
            This is a \(k(X)\)-valued point on \(\aff^1\).
          \end{example}
\par
More generally if \(R\) is a \(k\)-algebra an \(R\)-valued point of \(X\) is given by a \(k\)-algebra homomorphism \(\phi\colon A(X) \to R\).%
\begin{example}\label{example-2}
\(R =k[t]/(t^2)\)  and the \(k\)-algebra homomorphism \(\phi\colon A(X) \to R\) induces by composition with \(t\mapsto 0\) a \(k\)-valued point \(x\in X\). (Assuming now \(k = \bar{k}\)) \(x\) corresponds to a maximal ideal \(m_x = \ker(A(X) \to k)\) and \(\phi(m_x) \subseteq (t)\).
            So \(\phi(m_x^2) = 0\). 
            Thus we get a map \(\phi \colon m_x/m_x^2 \to (t) = k\).

            \begin{exercise}\label{exercise-1}
Check that giving \(x\in X\) and a map \(m_x/m_x^2 \to k\) is equivalent to giving a map \(\phi\colon A(X) \to R\).\end{exercise}


            A map \(m_x/m_x^2 \to k\) is an element of \((m_x/m_x^2)^*\).
          \end{example}
\par
Recall that if \(X,Y\) are affine varieties, giving a morphism \(X\to Y\) is equivalent to giving a \(k\)-algebra homomorphism \(A(Y) \to A(X)\).
          This suggests that we should allow any \(k\)-algebra to be a coordinate ring and if \(A,B\) are \(k\)-algebras then a map of \(k\)-algebras \(A\to B\) should be equivalent to giving a morphism of the corresponding ``varieties'' .
          More generally we could work over a ring \(R\), rather than a field \(k\).
          \(A\) and \(B\) could then be \(R\)-algebras.
          This includes the case where \(R= \ZZ\) and \(A\) and \(B\) are just rings.
        %
\begin{definition}[]\label{definition-1}
The category of affine schemes is the opposite category of the category of commutative rings.\end{definition}
\begin{definition}[]\label{definition-2}
A \terminology{scheme} is a geometric object covered by affine schemes.\end{definition}
\par
In general we tend to work with schemes over a base scheme \(S\) (e.g. \(S\leftrightarrow k\)) and consider morphisms defined over \(S\) i.e. diagrams .
        
        For \(T,X\) schemes over \(S\) a \(T\)-valued point of \(X\) is a diagram .%
\begin{definition}[]\label{definition-3}
If \(A\) is a commutative ring \[\Spec A = \{p\subset A : p \text{ is a prime ideal}\}.\]
            If \(I\subset A\) is an ideal (or set) we define \(V(I) = \{p\in \Spec A : p\supseteq I\}\).
          \end{definition}
\par
Note that if \(k = \bar{k}\) and \(A = k[x_1,\ldots,x_n]\), \(m = (x_1 - a_1,\ldots,x_n - a_n)\) contains \(I\) if and only if \((a_1,\ldots,a_n) \in V(I)\).%
\begin{exercise}\label{exercise-2}
Show that the sets \(V(I)\) form the closed sets of a topology on \(\Spec A\), the \terminology{Zariski topology}.\end{exercise}
\par

          \(\Gamma(X,\sO_X) = A(X)\).
          Our goal is to construct a sheaf \(\sO_{\Spec A}\) with stalks \(\sO_{\Spec A, p} = A_p\) and \(\Gamma(\Spec A, \sO_{\Spec A}) = A\).
        %
\begin{definition}[]\label{definition-4}
Let \(\sO\) be the sheaf on \(\Spec A\) whose sections over \(U\) open are functions \[s\colon U \to \coprod_{p\in U} A_p\]
            such that 
            \begin{enumerate}
\item{}\(s(p)\in A_p\)\item{}for \(p\in U\) there exists some open \(V \subset U\) with \(p\in V\) and \(f,g\in A\) such that for all \(q\in V\) we have \(g\in q\) and \(s(q) = f/g \in A_q\)\end{enumerate}
\end{definition}
\begin{definition}[Spectrums of rings]\label{definition-5}
The spectrum of \(A\) is the locally ringed space \((\Spec A,\sO_{\Spec A})\).\end{definition}
\par
Here locally ringed space means \(\sO_{\Spec A,p}\) is local for all \(p\in \Spec A\).%
\begin{proposition}\label{proposition-1}
\begin{enumerate}
\item{}For any \(p\in \Spec A\) \(\sO_p = A_p\).\item{}For any \(f\in A\) let \(D(f) = \{p\in \Spec A : f\notin p\} = \Spec A \smallsetminus V(f)\).\item{}\(\Gamma(\Spec A,\sO) = A\).\end{enumerate}
\begin{exercise}\label{exercise-3}
Show that sets of the form \(D(f)\) form a basis of the topology on \(\Spec A\).\end{exercise}
\end{proposition}
\begin{proof}
\begin{enumerate}
\item{}Define a map \(\sO_p\to A_p\) by \((U,s)\mapsto s(p)\).
                To see this is surjective we take \(f/g\in A_p\) (\(g\in p\)) then \((D(g),f/g) \in \sO_p\) which maps to \(f/g\).
                To see injectivity we let \((U,s),\,(V,t) \in \sO_p\) with \(s(p) = t(p)\).
                By shrinking \(U\) and \(V\) we can assume \(U = V\) and \(s\) is given by \(f/g\) with \(t\) given by \(f'/g'\) where \(g,g'\notin p\).
                Thus \(f/g = f'/g' \in A_p\) and hence there exists some \(h\notin p\) with \((f'g - g'f)h = 0\).
                Then for and
                
              \end{enumerate}
\end{proof}
\begin{definition}[Morphisms of locally ringed spaces]\label{definition-6}

            A morphism \((X,\sO_x) \to (Y,\sO_Y)\) of locally ringed spaces is a continuous map \(f\colon X \to Y\) along with a morphism of sheaves of rings 
            \[
              f^\# \colon \sO_Y \to \sO_X
            \]
            such that the induced maps \(f_p^\# \colon \sO_{Y,f(p)} \to \sO_{X,p}\) are local homomorphisms for all \(p\).
            The maps \(f_p^\#\) are induced by \((U,s) \mapsto (U,f^\#(s))\). \end{definition}
\begin{proposition}\label{proposition-2}
\begin{enumerate}
\item{}If \(\phi\colon A \to B\) is a ring homomorphism then \(\phi\) induces a morphism of locally ringed spaces \[(f,f^\#)\colon (\Spec B, \sO_{\Spec B}) \to (\Spec A, \sO_{\Spec A}).\]\item{}Any morphism of locally ringed spaces is induced in this way.\end{enumerate}
\end{proposition}
\begin{proof}
\begin{enumerate}
\item{}
                Define \(f\colon \Spec B \to \Spec A\) by \(f(p) = \phi^{-1}(p) \in \Spec A\).
                This is continuous as
                \begin{align*}
f^{-1}(V(I)) &= \{p\in \Spec B : \phi^{-1}(p) \supseteq I\}\\
             &= \{p\in \Spec B : p \supseteq \phi(I)\}\\
             &= V(\phi(I)).
\end{align*}
                For any \(p\in \Spec B\) we have a ring map 
                \begin{align*}
\phi_p \colon A_{\phi^{-1} (p)} &\to B_p\\
\frac as &\mapsto \frac{\phi(a)}{\phi(s)}.
\end{align*}
                Define
                \begin{align*}
f^\# \colon \sO_{\Spec A}(V) &\to \sO_{\Spec B}(f^{-1}(V) = (f_*\sO_{\Spec B})(V)\\
\left(s\colon V \to \coprod_{p\in V}A_p\right)&\mapsto \left(f^\#(s)\colon f^{-1}(V) \to \coprod_{q\in f^{-1} (V)}B_q\right)
\end{align*}
                with \((f^\#(s))(q) = \phi_{f(q)}(s(f(q)))\), \(s(f(q)) \in A_{f(q)} = A_{\phi^{-1}(q)}\).
                Note that \(f^\#\) induces the map \(\phi_q\) on stalks and \(\phi_q\) is a local homomorphism.
              \item{}
                Suppose we are given \((f, f^\#)\colon (\Spec B, \sO_{\Spec B}) \to (\Spec A, \sO_{\Spec A})\), then we can get \(\phi\colon \Gamma(\Spec A, \sO_{\Spec A}) = A \xrightarrow{f^\#} \Gamma(\Spec B, \sO_{\Spec B}) = B\).
                For \(p\in \Spec B\) we get 
                \begin{align*}
f^\#_p \colon \sO_{\Spec A, f(p)} & \to \sO_{\Spec B, p}\\
A_{f(p)} &\to B_p
\end{align*}
                a local homomorphism.
                We also have a commutative diagram
                \[
                  \xymatrix{
                  A \ar[r]^{\phi} \ar[d]_{\substack{a\\\downarrow\\a/1}}& B \ar[d]^{\substack{b\\ b/1}}\\
                  A_{f(p)} \ar[r]_{f_p^\#} & B_p
                  }
                \]
              \end{enumerate}
\end{proof}
\typeout{************************************************}
\typeout{Section 1.3 Properties of Schemes}
\typeout{************************************************}
\section[Properties of Schemes]{Properties of Schemes}\label{sec-props-schemes}
\begin{definition}[Irreducible schemes]\label{definition-7}

            A scheme \((X, \sO_X)\) is said to be \terminology{irreducible} if \(X\) is irreducible as a topological space.
          \end{definition}
\begin{definition}[Reduced schemes]\label{definition-8}

            A scheme is \terminology{reduced} if \(\sO_X(U)\) is an integral domain for any \(U \subset X\) open.
          \end{definition}
\begin{proposition}\label{proposition-3}

            A scheme is integral if and only if it is reduced and irreducible.
          \end{proposition}
\begin{proof}

            Integral implies reduced is clear. \newline{}
            Suppose \(X  = Y_1 \cup Y_2\) with \(Y_1,Y_2 \subset X\) closed \(Y_1,Y_2 \ne X\).
            So we find that \(U_1 = Y_1 \smallsetminus Y_2 = X_1 \smallsetminus Y_2\) and \(U_2 = Y_2 \smallsetminus Y_1 = X_2 \smallsetminus Y_1\) are open disjoint sets.

            Then \(\sO_X(U_1\cup U_2) = \sO_X(U_1) \times \sO_X(U_2)\) by the sheaf axioms {$\langle\langle$Unresolved xref, ref="defn-sheaf"; check spelling or use "provisional" attribute$\rangle\rangle$}.
            But this is not an integral domain.
            Conversely suppose \(X\) is reduced and irreducible.
            If \(U \subset X\) open, \(f,g\in \sO_X(U)\) with \(fg = 0\) we want to show that either \(f=0\) or \(g= 0\).
            Let \(Y  =\{x\in U : f_x \in \fm_x\}\) where \(f_x\) is the germ of \(f\) in \(\sO_{X,x}\) and \(\fm_x\subset \sO_{X,x}\) is the maximal ideal.
            Let \(Z = \{x\in  : g_x \in \fm_x\}\) then \(Y\) and \(Z\) are closed subsets of \(U\) (its enough to check this on an open cover of \(U\) which we can assume to be affine, but if \(U = \Spec A\) is affine \(Y = V(f)\), which is closed).
            Since \(fg = 0\) we have \(f_x g_x = 0\) for all \(x\) and so \(U = Z\cup Y\).
            \(U\) is an open subset of \(X\) which is irreducible so \(U\) is irreducible (exercise!).
            So \(U = Y\) or \(U = Z\).
            Assume \(U = Y\), we will show \(f = 0\).
            We can restrict to affine open subsets of \(U\) and hence \(U =\Spec A\) is affine. Thus \(\emptyset = U \smallsetminus Y = D(f)\).
            Thus \(f = \bigcap_{p\in\Spec A} p = \sqrt{(0)}\).
            Thus \(f\) is nilpotent so \(f = 0\) since \(X\) is reduced, therefore \(\sO_X(U)\) is an integral domain.
          \end{proof}
\begin{example}\label{example-3}
\(\Spec k[x,y]/(xy)\) is the two axes, and not irreducible.\newline{}\(\Spec k[t]/(t^2)\) is a point, with global sections \(k[t]/(t^2)\).\newline{}\end{example}
\begin{definition}[(Locally) Noetherian schemes]\label{definition-9}

            A scheme \(X\) is said to be \terminology{locally Noetherian} if it can be covered by opne affines of the form \(\Spec A\) with \(A\) a Noetherian ring.
            It is said to be \terminology{Noetherian} if there is a finite such cover.
          \end{definition}
\begin{proposition}\label{proposition-4}
\(X\) is locally Noetherian is and only if for every affine open subset \(\Spec A \subset X\) we have \(A\) Noetherian.
            In particular \(\Spec A\) is noetherian if and only if \(A\) is Noetherian.
          \end{proposition}
\begin{proof}
\(\Leftarrow\) clear, \(\Rightarrow\) let \(U \subset X\) be open affine, \(U = \Spec A\).
            In general if \(B\) is Notherian and \(f \in B\) then \(B_f\) is Noetherian and \(D(f) \cong \Spec B_p\) as schemes.
            Also the \(D(f)\)s form a basis for the topology on \(\Spec B\).
            Thus any open set of \(\Spec B\) can be covered by open affines of the form \(\Spec B_f \) with \(B_f\) Noetherian.
            \(U \cap \Spec B\) can be covered by affine schemes of the form \(\Spec B_f\) with \(B_f \) Noetherian.
            We need to show that if \(\Spec A\) can be covered by sets of the form \(\Spec B\) with \(B\) Noetherian, then \(A\) is Noetherian.
          \end{proof}
%
\backmatter
%
\end{document}