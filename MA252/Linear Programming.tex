%% LyX 2.0.3 created this file.  For more info, see http://www.lyx.org/.
%% Do not edit unless you really know what you are doing.
\documentclass[english]{article}
\usepackage[T1]{fontenc}
\usepackage[latin9]{inputenc}
\usepackage{geometry}
\geometry{verbose,tmargin=2.5cm,bmargin=2.5cm,lmargin=2.5cm,rmargin=2.5cm}
\setlength{\parindent}{0bp}
\usepackage{amsbsy}
\usepackage{amstext}
\usepackage{amssymb}

\makeatletter
%%%%%%%%%%%%%%%%%%%%%%%%%%%%%% Textclass specific LaTeX commands.
\newenvironment{lyxlist}[1]
{\begin{list}{}
{\settowidth{\labelwidth}{#1}
 \setlength{\leftmargin}{\labelwidth}
 \addtolength{\leftmargin}{\labelsep}
 \renewcommand{\makelabel}[1]{##1\hfil}}}
{\end{list}}

%%%%%%%%%%%%%%%%%%%%%%%%%%%%%% User specified LaTeX commands.
\date{}

\makeatother

\usepackage{babel}
\begin{document}

\title{MA252 Combinatorial Optimisation - Linear Programming}

\maketitle
These notes are for the second half of MA252, as taught by Dr. Diane
Maclagan in 2012. They are not exhaustive, but instead are designed
to provide an overview and a quick reference guide. Please send corrections
to Julian Bhardwaj <j.bhardwaj@warwick.ac.uk>.


\section{Definitions}

A \textbf{linear programme} (LP) is a problem of the form:

\[
\min\boldsymbol{c}.\boldsymbol{x}\text{ for }\boldsymbol{x}\in\mathbb{R}^{n}\text{ subject to }\boldsymbol{Ax}\leq\boldsymbol{b},\,\boldsymbol{b}\in\mathbb{R}^{d}
\]


Note, the problem can be to maximise instead of minimise, and the
conditions can have varying inequalities ($\leq,\,\geq,\,<,\,>$)\\


In the above formulation, $\boldsymbol{c}$ is called the \textbf{cost
function} and the \textbf{feasible region} is defined as:

\[
\{\boldsymbol{x}\in\mathbb{R}^{n}\::\:\boldsymbol{Ax\leq b}\}
\]
\\


A \textbf{polyhedron} is a set in $\mathbb{R}^{n}$ of the form:

\[
\{\boldsymbol{x}\in\mathbb{R}^{n}\::\:\boldsymbol{Ax\leq b}\}
\]
\\


A \textbf{polytope} is a bounded polyhedron.\\


The \textbf{face} of a polyhedron $P=\{\boldsymbol{x}\in\mathbb{R}^{n}\::\:\boldsymbol{Ab}\leq\boldsymbol{b}$\}
minimising $\boldsymbol{c}$ is:

\[
\text{face}_{\boldsymbol{c}}(P)=\{\boldsymbol{x}\in P\::\:\boldsymbol{c.x\leq c.y}\:\forall\boldsymbol{y}\in P\}
\]
\\


The \textbf{linear span} of $P$ is the subspace span $(\boldsymbol{x-y}\,;\,\boldsymbol{x},\,\boldsymbol{y}\in P)$
of $\mathbb{R}^{n}$. The \textbf{dimension} of $P$ is the dimension
of its linear span.\\


A linear program is in \textbf{standard form }if it is:

\[
\min\boldsymbol{c.x}\text{ for }\boldsymbol{x\in}\mathbb{R}^{n}\text{ satisfying }\boldsymbol{Ax=b},\,\boldsymbol{x\geq0}
\]
\\
A vector $\boldsymbol{y}\in\mathbb{R}^{n}$ is a \textbf{basic solution}
if $\boldsymbol{Ay=b}$ and $\{A_{i}\::\: y_{i}\neq0\}\subseteq\mathbb{R}^{d}$
is linearly independent.\\


A vector $\boldsymbol{y}\in\mathbb{R}^{n}$ is a \textbf{basic feasible
solution }if $\boldsymbol{y}$ is a basic solution with $y_{i}\geq0,\,\forall i$.\\


Let $\boldsymbol{x}\in P$. A vector $\boldsymbol{d}$ is a \textbf{feasible
direction} at $\boldsymbol{x}$ if $\exists\theta>0\: s.t.\:\boldsymbol{x}+\theta\boldsymbol{d}\in P$\\


A basic feasible solution $\boldsymbol{y}$ is\textbf{ degenerate
}if $|\{i\::\: y_{i}\neq0\}|<d$. It is \textbf{nondegenerate} otherwise.\\


The \textbf{reduced cost vector }in direction $i$ w.r.t. a basic
feasible solution $\boldsymbol{x}$ corresponding to $I$ is defined
by:

\[
\bar{c_{i}}=c_{i}-\boldsymbol{c_{I}^{T}}\boldsymbol{B^{-1}A_{i}}
\]
\\
The full \textbf{Simplex Tableau }for the basic feasible solution
corresponding to $I$ is the $(d+1)\times(n+1)$ matrix:

\[
\left[\begin{array}{c|c}
-\boldsymbol{c}_{I}\boldsymbol{B}^{-1}\boldsymbol{b} & \boldsymbol{c}-\boldsymbol{c}_{I}^{T}\boldsymbol{B}^{-1}\boldsymbol{A}\\
\boldsymbol{B}^{-1}\boldsymbol{B} & \boldsymbol{B}_{I}^{-1}\boldsymbol{A}
\end{array}\right]
\]
\\
For $\boldsymbol{u,\, v}\in R^{n}$ we write $\boldsymbol{u}>_{lex}\boldsymbol{v}$
if the first nonzero element of $\boldsymbol{u-v}$ is positive.\\


Let $\boldsymbol{T}$ be a Simplex tabeau and choose any $i$ with
$\bar{c_{i}}<0$. The \textbf{lexicographic pivot rule }is to choose
$s$ amongst those with $\boldsymbol{T}_{s,1}>0$ and $\frac{\boldsymbol{T}_{s,0}}{\boldsymbol{T}_{s,i}}$
minimum such that $\frac{1}{\boldsymbol{T}_{s,i}}(rows)$ is smallest
w.r.t. lex order.


\section{Formulation}


\subsection{Conversion to Standard Form}
\begin{enumerate}
\item Make $\boldsymbol{x}\geq0$. 

\begin{enumerate}
\item If there is no inequality $x_{i}\geq0$ or $x_{i}\leq0$ for some
$i$, write $x_{i}=x_{i}^{+}-x_{i}^{-}$ with $x_{i}^{+},\, x_{i}^{-}\geq0$. 
\item If there is $x_{i}\leq0$, replace $x_{i}$ by $-x_{i}$.
\end{enumerate}
\item Add slack variables.

\begin{enumerate}
\item $\boldsymbol{a.x\leq b}\rightarrow\boldsymbol{a.x+s=b},\,\boldsymbol{s\geq0}$
\end{enumerate}
\item Update $\boldsymbol{c}$.

\begin{enumerate}
\item $c_{i}\rightarrow(c_{i}^{+},\,-c_{i}^{-})$
\item $c_{s}=0$
\end{enumerate}
\end{enumerate}

\section{Simplex Algorithm}


\subsection{Finding basic solutions}

(Not actually part of the Simplex Algorithm)
\begin{enumerate}
\item Choose $d$ linearly independant columns of $\boldsymbol{A}\::\:\{A_{i}\::\: i\in I_{d}\}$
\item Let $\boldsymbol{B}$ be the submatrix of $\boldsymbol{A}$ with columns
indexed by $I_{d}$.
\item Solve $\boldsymbol{By_{B}=b}$ for $\boldsymbol{y_{B}}\in\mathbb{R}^{d}$
\item Set $\boldsymbol{y}\in\mathbb{R}^{n}$ to be the vector with $\boldsymbol{y_{B}}$
in the coordinates indexed by $I$, and $0$ otherwise.
\end{enumerate}
Then $\boldsymbol{Ay=b}$ and $\{A_{i}\::\: y_{i}\neq0\}$ is linearly
indepedent.


\subsection{Finding an intial basic feasible solution}
\begin{description}
\item [{Easy-Case}] $\boldsymbol{A}$ contains an identity matrix and $\boldsymbol{b\geq0}$.

\begin{lyxlist}{00.00.0000}
\item [{i)}] Choose $I$ to be the indices of the identity submatrix of
$\boldsymbol{A}$.
\item [{ii)}] Set $\boldsymbol{x}$ accordingly: $\boldsymbol{Ix=b}$
\end{lyxlist}
\item [{Hard-Case}] 'Clever Trick'

\begin{lyxlist}{00.00.0000}
\item [{i)}] Multiply rows by $-1$ to make $\boldsymbol{b\geq0}$.
\item [{ii)}] Set up new LP: $\min\, y_{1}+\dots+y_{d}\text{ for }(\boldsymbol{x,\, y)}\in\mathbb{R}^{n+d}\: s.t.\:\boldsymbol{Ax+y=b};\,\boldsymbol{x,\, y\geq0}$
\item [{iii)}] Rewrite the objective function in terms of the non-basic
variables.
\item [{iv)}] Apply 'Easy-Case' to get a BFS for this new LP and solve.
\item [{v)}] If the optimal value for the new LP is $0$, then the solution
gives a BFS for the original LP.
\end{lyxlist}
\end{description}

\subsection{The Algorithm}
\begin{enumerate}
\item Find a vertex of $P$

\begin{enumerate}
\item Find a basic feasible solution $\boldsymbol{x}$ corresponding to
$I$.
\end{enumerate}
\item If there is no edge leading to a vertex of lower cost, stop.

\begin{enumerate}
\item Compute the reduced cost vector $\bar{\boldsymbol{c}}$.
\item Stop when there is no such $\bar{c_{i}}<0$.
\end{enumerate}
\item Walk along an edge of $P$ moving to a vertex of lower cost.

\begin{enumerate}
\item Choose $i\notin I$ with $\bar{c_{i}}<0$ .
\item Move in the $i$th basic feasible direction to $\boldsymbol{x}^{\prime}=\boldsymbol{x}+\theta^{*}\boldsymbol{d}$

\begin{enumerate}
\item Compute $\boldsymbol{d}$ with $d_{i}=1,\;\boldsymbol{d}_{I}=-\boldsymbol{B}_{I}^{-1}\boldsymbol{A_{i}},\; d_{j}=0\,\forall j\notin I\cup\{i\}$
\item If $\boldsymbol{d\geq0}$ then the optimal cost is $-\infty$, stop.
\item Else, let $\theta^{*}=\min_{d_{j}<0}\frac{x_{j}}{-d_{j}}$ and let
$l=\text{argmin}_{d_{j}<0}\frac{x_{j}}{-d_{j}}$.
\item Set $\boldsymbol{x}^{\prime}=\boldsymbol{x}+\theta^{*}\boldsymbol{d}$
\end{enumerate}
\end{enumerate}
\item Goto 2.
\end{enumerate}

\section{Results}


\paragraph*{Prop.}

If the feasible region of the LP is a polytope, then the LP has a
solution.


\paragraph*{Thm.}
\begin{lyxlist}{00.00.0000}
\item [{a)}] A vector $\boldsymbol{y}$ is a basic feasible solution iff
$\boldsymbol{y}$ is a vertex of $P=\{\boldsymbol{x}\in\mathbb{R}\::\:\boldsymbol{Ax=b},\,\boldsymbol{x\geq0}\}$
\item [{b)}] Either $\min_{\boldsymbol{x}\in P}\boldsymbol{c.x}=-\infty$
or there exists a basic feasible solution $\boldsymbol{y}$ with $\boldsymbol{c.y\leq c.x}\:\forall\boldsymbol{x}\in P$
\end{lyxlist}

\paragraph*{Prop.}

If $\boldsymbol{x}$ is a nondegenerate basic feasible solution then
the ith basic direction is feasible.


\paragraph*{Prop.}

Let $\boldsymbol{x}$ be a basic feasible solution corresponding to
$I\subseteq\{1,\,\dots,\, n\}$ and let $\boldsymbol{\bar{c}}$ be
the corresponding reduced cost vector.
\begin{lyxlist}{00.00.0000}
\item [{a)}] If $\bar{\boldsymbol{c}}\geq0$ then $\boldsymbol{x}$ is
optimal.
\item [{b)}] If $\boldsymbol{x}$ is optimal and nondegenerate then $\bar{\boldsymbol{c}}\geq0$.
\end{lyxlist}

\paragraph*{Thm.}

With the lexicographic pivot rule, the Simplex algorithm terminates
correctly.


\paragraph*{Thm.}

(Linear Programming Duality) If the primal LP $\min\boldsymbol{c.x}$
for $\boldsymbol{x}\in\mathbb{R}^{n}$ with $\boldsymbol{Ax=b,\, x\geq0}$
has an optimal solution then so does the dual programme $\max\boldsymbol{b.p}$
for $\boldsymbol{p}\in\mathbb{R}^{d}$ with $\boldsymbol{A}^{T}\boldsymbol{p}\leq\boldsymbol{c}$
and the respective costs are the same:

\[
\min_{\boldsymbol{Ax=b},\,\boldsymbol{x\geq0}}\boldsymbol{c.x}=\max_{\boldsymbol{A}^{T}\boldsymbol{p}}\boldsymbol{b.p}
\]


Moreover,
\begin{lyxlist}{00.00.0000}
\item [{a)}] If the primal is feasible and bounded then so is the dual.
\item [{b)}] If the primal is unbounded then the dual is not feasible.
\item [{c)}] If the primal is not feasible then the dual is unbounded.\end{lyxlist}

\end{document}
