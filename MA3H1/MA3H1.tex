\title{MA3H1 Topics in Number Theory - Lecture Notes}
\author{Based on lectures by Alex Bartel \\ Typeset by Alex J. Best}
\date{\today}

\documentclass{book}
\usepackage{amsfonts}
\usepackage{amsmath}
\usepackage{amssymb}
\usepackage{amsthm}

\newtheorem*{thm}{Theorem}
\newtheorem*{lem}{Lemma}
\newtheorem*{cor}{Corollary}
\theoremstyle{definition}
\newtheorem*{defn}{Definition}
\newtheorem*{ex}{Example}
\newtheorem*{rem}{Remark}
\newtheorem*{nota}{Notation}

\newcommand{\ZZ}{\mathbb{Z}}
\newcommand{\QQ}{\mathbb{Q}}
\newcommand{\NN}{\mathbb{N}}

\setcounter{chapter}{0}

\begin{document}
\maketitle

\chapter{Revision}
\section{Divisibility}
\begin{defn}
Given $a,b \in \ZZ$ we say $a$ divides $b$, written $a|b \iff b=an$ for some $n \in \ZZ$. 
\end{defn}
Some immediate consequences are:
\begin{itemize}
\item $a|0\ \forall a \in \ZZ$
\item $a|b$ and $b|c \implies a|c$
\item $a|b$ and $b|a \implies a = \pm b$
\item $a|b \implies a|bc\ \forall c \in \ZZ$
\item $a|b$ and $a|c \implies a = b \pm c$
\item $(\pm 1)|a\  \forall a \in \ZZ$
\item $a|b,\ b\neq 0 \implies |a| \leq |b|$
\item $a|\pm1 \implies a = \pm 1$
\end{itemize}
\begin{ex}
$$55| 11^n - 11\ \forall n \geq 1$$
\end{ex}
\begin{proof}
Let $n=1$ then $55|0$ \\
Assume holds for $n$,
    $$11^{n+1}-11=11(11^n)-11=11(11^n-1) = 11(11^n-11)+110$$
So $55|11^n-11\ \forall n$ by induction.
\end{proof}
\begin{thm}
If $a,b \in \ZZ,\ b\neq 0\text{ then }\exists!\ q,r \text{ s.t. } a = bq +r\text{ with }0\leq r \leq |b|-1$
\end{thm}
\begin{proof}(Uniqueness)
Suppose $qb+r=q'b+r'$, $0\leq r,r' \leq|b|-1$ then,
	$$(q-q')b=r'-r$$
	$$\implies 0 \leq |(q-q')b|<|b|$$
	$$\implies 0 \leq |q-q'| < 1$$
	$$\implies q=q'$$
(Existance) Suppose $a \geq 0,\ b\geq 0$. Consider the set $\{mb|mb \leq a,\ m\in\ZZ\}$\\
	Let $q=\max\{m:mb\leq a\}$ \\
	This is our $q$ and $r = a-qb$
\end{proof}
\section{Ideals}
\begin{defn}
A ring $R$ is a set together with two operations $+,\cdot$ such that $(R,+)$ is a commutative group and:
\begin{itemize}
\item $(a\cdot b)\cdot c = a\cdot (b\cdot c)$
\item $(a+b)\cdot c = a\cdot c+b\cdot c$
\item $\exists 1$ s.t. $a\cdot 1 = 1\cdot a = a$
\end{itemize}
\end{defn}
We will often omit the $\cdot$ when talking about rings, and write $ab$ for $a\cdot b$.
\begin{ex}
$\ZZ$, $\QQ$ (actually a field), $M_n(\QQ) = \{n\times n \text{ matrices over } \QQ\}$, $\ZZ[i] = \{a+bi|a,b \in\ZZ\}$, $\ZZ[\omega] = \{a+b\omega |a,b \in\ZZ\}$ where $\omega = e^{2\pi i /3}$.
\end{ex}
\begin{defn} A subring of $R$ is a subset of $R$ that is also a ring under the same operations\end{defn}
\begin{ex} .... \end{ex}
\begin{defn} An ideal $I \subset R$ is a subset of $R$ s.t. when you take $(I,+)$ it is a subgroup and $\forall r\in R, i\in I, r\cdot i \in I$
\end{defn}
\begin{ex}
$2\ZZ = \{2a|a\in\ZZ\}$ is an ideal in $\ZZ$.
\end{ex}
For $S$ is a subring of $R$ we write $S\leq R$. \\
For an ideal we write $S \trianglelefteq R$. \\
If $r_1,...,r_n \in R$ then the ideal generated by them is
$$(r_1,...,r_n)_R = <r_1,...,r_n>_R = \left\{ \sum_{i=1}^n \alpha_i r_i | \alpha_i \in R \right\}$$
\begin{ex}
If we take $R = \ZZ$
$$\left. \begin{matrix}
(2,3)_R = \ZZ = (1)_R \\
(10,15)_R = 5\ZZ = (5)_R
\end{matrix} \right\} \text{Ideals generated by only one element}$$
\end{ex}
\begin{defn}
We call an ideal prinicpal if it is of the form $(r) = \{\alpha r|\alpha\in R\}$
\end{defn}
\begin{defn}
A ring $R$ where every ideal of $R$ is principal is known as a principal ideal domain (PID).
\end{defn}
\begin{thm}
$\ZZ$ is a PID
\end{thm}
\begin{proof}
Let $I \triangleleft \ZZ$ if $I=\{0\}$ then $I = (0)$ and we are done. \\
Otherwise take $b \in I\setminus \{0\}$ s.t. $|b|$ is minimal. After multiplying by -1 if necessary assume $b > 0$. \\
Take any $a \in I$ and write $a = bq + r$ with $0 \leq r \leq b-1$. \\
$a \in I, b \in I \implies bq \in I \implies a-bq =r \in I$ Since among the non-zero elements $b$ has the smallest absolute value. So $r=0$, which gives us that $a$ is a multiple of $b$. \\
We can do this for an arbitary $a \in I$ so $I \subseteq (b)$ but $b \in I \implies (b) \subseteq I \implies (b) = I$.
\end{proof}
\begin{rem}
What made this proof work is that fact that there is a function $|\cdot|:\ZZ \rightarrow \NN \setminus\{0\}$ such that $\forall a,b \in \ZZ$ with $b\neq 0\ \exists q,r \in \ZZ: a=bq+r$ and $0 \leq r \leq b-1$. \\
Such a function is called a euclidean function and exists in other circumstances. Other rings may also have such a function.
\end{rem}
\begin{ex}
\begin{enumerate}
\item $\ZZ[i] = \{a+bi|a,b \in \ZZ\}$ has the euclidean function $a^2+b^2$.
\item Write $\omega = e^{2\pi i /3}$, and consider $R = \ZZ[\omega] = \{a+b\omega|a,b\in\ZZ\}$. This ring has the euclidean function $f(a+b\omega) = a^2-ab+b^2$.
\item $\QQ[x]$ the ring of polynomials over $\QQ$ has a euclidean function that takes a polynomial to it's degree.
\end{enumerate}
\end{ex}
\section{GCD}
\begin{thm}
Let $a_1,...,a_n \in \ZZ$ and $c$ be any element s.t. $c|a_i\ \forall i$ then
$$\exists! d \in \NN\setminus\{0\} \text{ s.t. } d|a_i\ \forall i \text{ and also } c|d$$
$$\text{Moreover }d = \sum_{i=1}^n x_i a_i, x_i \in\ZZ$$
\end{thm}
\begin{ex}
$a_1=6,a_2=10,a_3=15 \implies d=1$ No $x_i$ can be 0 since any of these 2 have a common divisor $> 1$.
\end{ex}
\begin{lem}

\end{lem}
\begin{proof}[Theorem]
Let $I=(a_1,...,a_n)_{\ZZ} \trianglelefteq \ZZ$. \\
$\ZZ$ is a PID $\implies \exists d $ s.t. $(d)=(a_1,...,a_n)_{\ZZ}$, if $d=0$ this is boring.\\
So assume $d\neq 0$,\\
Since $a_i \in(d) \implies dx_i = a_i$ for some $x_i \in \ZZ$
\end{proof}
\begin{cor}[Euler's Lemma]
If $x,a,b \in\ZZ$ are s.t. $x|ab$ and $\gcd(a,a)=1$ then $x|b$.
\end{cor}
\begin{proof}
Since $\gcd(x,a)=1$ we can write $1=xu+av$ for some $u,v\in\ZZ$.\\
So $b=xub+avb$, and as $x|ab$ and $x|xub$ we have that $x|xub+avb$, so $x|b$.
\end{proof}
In a ring $R$ in which $ab=0 \iff a=b$ or $b=0$. \\
$a|b,\ b|a \implies a=bu$ for a unit $u$ of $R$.
\begin{defn}
A unit $u\in R$ is an element for which $\exists v$ s.t. $uv=vu=i$.
\end{defn}
So in general if $(c)=(d) \trianglelefteq R$ then $c=ud$ for a unit $u\in R$.
\begin{nota}
The set of units in $R$ is denoted $R^\times$, this forms a group under the ring operation. \\
We write $[a]_n$ for the set of integers congruent to $a$ mod $n$, that is $\{x \in\ZZ | x \equiv a \pmod{n}\}$. \ \
We also sometimes write $(a,b)$ for $\gcd(a,b)$.
\end{nota}
\begin{lem}
$[a]_n\in (\ZZ /n\ZZ)^\times \iff \gcd(a,n)=1$.
\end{lem}
\begin{proof}
$[a]_n\in (\ZZ /n\ZZ)^\times \iff \exists b \in \ZZ$ s.t. $[a]_n\cdot [b]_n=[1]_n=[ab]_n \iff ab-tn=1$ for some $b,t\in\ZZ$ $\iff \gcd(a,n)=1$
\end{proof}
\begin{lem}
Let $n\in\NN ,\ a,x \in\ZZ$.
\begin{enumerate}
\item $\exists b \in\ZZ$ s.t. $ab \equiv x\pmod{n} \iff \gcd(a,n)|x$
\item If $\gcd(a,n)|x$ then $\#\{[b]_n\in\ZZ/n\ZZ|ab\equiv x\pmod{n}\} =\gcd(a,n)$
\end{enumerate}
\end{lem}
\begin{proof}
\begin{enumerate}
\item $\exists b$ with $ab \equiv x\pmod{n} \iff \exists b,t \in\ZZ$ s.t. $ab-tn=x \iff \gcd(a,n)|x$
\item Suppose $\gcd(a,n)|x$ let $b_0, t_0\in\ZZ$s.t. $ab_0-t_0n=x$ \\
For any $b,t\in\ZZ$ that also satisfy $ab-tn=x$ we have $a(b-b_0)=n(t-t_0)$ \\
Let $d=\gcd(a,n)|x$ \\
then $d|a$ and $d|n$ so $\frac a d (b-b_0)=\frac m d (t-t_0) \implies \exists r \in\ZZ$ s.t. $\frac a d r = t-t_0$ and $\frac n d r = b-b_0$ \\
$\#\{[b]_n|ab\equiv x \pmod{n}\} = \#\left\{[b]_n - \left[ \frac{nr}{d}\right]_n : r\in\ZZ\right\} = \#\{r\in\ZZ|0\leq \frac{nr}{d} \leq n-1\} = d = \gcd(a,n)$
\end{enumerate}
\end{proof}
\chapter{Chinese Remainder theorem \& Fermat's Little theorem}

\chapter{Revision}

\chapter{Revision}

\chapter{Revision}

\chapter{Revision}

\end{document}
