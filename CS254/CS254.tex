\title{CS254 Algorithmic Graph Theory - Lecture Notes}
\author{Based on lectures by Prof. Maxim Sviridenko \\ Typeset by Alex J. Best}
\date{\today}

\documentclass{book}
\usepackage{amsfonts}
\usepackage{amsmath}
\usepackage{amssymb}
\usepackage{amsthm}

\newtheorem*{thm}{Theorem}
\newtheorem*{lem}{Lemma}
\newtheorem*{cor}{Corollary}
\newtheorem*{prop}{Proposition}
\theoremstyle{definition}
\newtheorem*{defn}{Definition}
\newtheorem*{defns}{Definitions}
\newtheorem*{ex}{Example}
\newtheorem*{rem}{Remark}
\newtheorem*{nota}{Notation}
\newtheorem*{alg}{Algorithm}

\newcommand{\ZZ}{\mathbb{Z}}
\newcommand{\QQ}{\mathbb{Q}}
\newcommand{\NN}{\mathbb{N}}
\newcommand{\RR}{\mathbb{R}}
\DeclareMathOperator{\wt}{wt}
\DeclareMathOperator{\depth}{depth}


\setcounter{chapter}{0}

\begin{document}
\maketitle
\chapter{Basics}
\section{Graphs}
\marginpar{Lecture 1}
\begin{defn}
A graph $G$ is an ordered pair $V(G),E(G))$ consisting of a set $V(G)$ of vertices and a set $E(g)$ of edges.
Each edge $e\in E(G)$ is associated with a pair of vertices $\{u,v\}$.
We will say that $u,v$ are endpoints of $e$.
We also say $u,v$ are incident to $e$ and vice versa.
\end{defn}

\begin{defn}
Two vertices which are incident with a common edge are adjacent (or neighbours).
Two or more edges with the same pair of endpoints are said to be parallel.
\end{defn}

\begin{defn}
A simple graph is one without parallel edges.
\end{defn}

\begin{nota}
By convention, $G$ will denote a graph, $n$ and $m$ will be the number of vertices $|V(G)|$ and the number of edges $|E(G)|$ respectively.
\end{nota}

\section{Special graphs}
\begin{defn}
A complete graph is a simple graph in which every two vertices are adjacent. \\
Such a graph on $n$ vertices is denoted $K_n$.
\end{defn}
%TODO: pictures

We can see that $|E(K_n)| = \frac{n(n-1)}{2}$.
\begin{defn}
A graph $G=(V,E)$ is called empty if $E=\emptyset$.
\end{defn}

\begin{defn}
A graph is called bipartite is its vertex set can be partitioned into two subsets $X$ and $Y$ so that every edge has one endpoint in $X$ and one in $Y$.
\end{defn}

\begin{ex}
The set of all (monogamous) people can be represented by a graph with edges joining married couples, this graph is bipartite.
\end{ex}

We will call $X,Y$ the parts of the graph. So a bipartite graph is often denoted as $G = (X,Y,E)$.
\begin{nota}
We write $K_{m,n}$ for the complete bipartite graph with $|X| = m,\ |Y| = n$ and $E$ containing all edges between $X$ and $Y$.
\end{nota}

\begin{defn}
The graph $K_{1,n}$ is often called a star.
\end{defn}
%TODO: pictures

\begin{defn}
A path is a simple graph whose vertices can be rearranges into a linear sequence in such a way that two adjacent vertices are always consecutive in the sequence and vice versa, non-adjacent vertices are non-consecutive, such a graph on $n$ vertices is often denoted $P_n$.
\end{defn}
%TODO: pictures

\begin{defn}
A cycle on $n$ vertices is a graph defined in a similar way but a cyclic sequence of vertices.
\end{defn}
%TODO: pictures

\begin{nota}
The length of a path or cycle is the number of its edges. A path or cycle of length $k$ is often called a $k$-path of $k$-cycle.
\end{nota}
%TODO: pictures

\subsection{Adjacency and incidence matrices}
\begin{defn}
The incidence matrix of $G = (V,E)$ when $|V| = m$ and $|E| = n$ is the $m\times n$ matrix $M_G = (m_{ve})$ where $m_{ve}$ is the number of times that vertex $v$ and $e$ are incident ($m_{ve} \in \{0, 1\}$).
\end{defn}
\begin{ex}
%TODO: pics + matrices
\end{ex}
\begin{defn}
The adjacency matrix of $G = (V,E)$ when $|V| = m$ is the $m\times m$ matrix $A_G = (a_{uv})$ where $m_{uv}$ is the number of edges joining $u$ and $v$.
\end{defn}
\begin{ex}
$$ = A_{K_{3}}$$
%TODO: pics + matrices
\end{ex}
\begin{defn}
An adjacency list is a list $(N(v),v\in V)$ used to represent matrices, $N(v)$ is the set of neighbours of $v$ in arbitrary order.
\end{defn}

\subsection{Adjacency and incidence matrices}
\begin{defn}
The degree of a vertex $v$ in a graph $G$ is denoted $d_G(v)$ or $d_v$ and is the number of edges of $G$ incident with $v$.
\end{defn}
\begin{thm}[Handshake lemma]
For any graph
\begin{equation}
\label{eq:hand}
\sum_{v\in V}d_v = 2|E|
\end{equation}
\end{thm}
\begin{thm}
In any graph, the number of vertices of odd degree is even.
\end{thm}
\begin{proof}
Consider the equation \ref{eq:hand} module 2.
We have degree of each vertex $d_v \equiv$ 1 if $d_v$ is odd, or 0 is $d_v$ is even.
Therefore the left hand side of \ref{eq:hand} is congruent to the number of vertices of odd degree and the RHS is 0.
Therefore the number of vertices of odd degree is congruent to 0 mod 2.
\end{proof}
\begin{defn}
A graph $G$ is called $k$=regular is $d_G(v) = k$ for all $v\in V(G)$.
A graph is regular if it is $k$-regular for some $k$.
\end{defn}
\begin{ex}
$K_n$ is $(n-1)$-regular. \\
$C_n$ is $2$-regular. \\
$P_n$ is not regular. \\
\end{ex}
\marginpar{Lecture 2}
\begin{thm}
Let $G=(X,Y,E)$ be a bipartite graph without isolated vertices s.t. $d_x\geq d_y$ for all $x\in X$ and $y\in Y$ s.t. $(x,y)\in E$.
Then $|X| \leq |Y|$ with equality iff $d_x = d_y \forall (x,y)\in E$ .
\end{thm}
\begin{proof}
Take the adjacency matrix $B$.
Define the matrix $B'$ by dividing each row of $B$ corresponding to vertex $u$ by $d_u$.
consider one column sum corresponding to $y\in Y$.
$$\sum_{x\in X} b'_{xy} = \sum_{(x,y)\in E} \frac{1}{d_x} = \frac{d_y}{d_x} \leq 1$$
Therefore $|X|=\sum_{x\in X}\sum_{y\in Y,(x,y)\in E} \frac{1}{d_x}$
$$=\sum_{x\in X, y\in Y}\sum_{(x,y)\in E} \frac{1}{d_x} \leq \sum_{x\in X, y\in Y}^{(x,y)\in E} \frac{1}{d_y}$$
$$=\sum_{y\in Y} \sum_{x\in X,(x,y)\in E} \frac{1}{d_y} = |Y|$$
If $|X|=|Y|$, then the middle inequality must be an inequality implying that $d_x=d_y$ for all $(x,y)\in E$.
\end{proof}
\subsection{Isomorphisms}
\begin{defn}
Two graphs $G$ and $H$ are isomorphic, written $G\cong H$, if there is a bijection $\Theta\colon V(G) \to V(H)$ such that $(u,v)\in E$ iff $(\Theta(u),\Theta(v))\in E(H)$.
\end{defn}
\begin{ex}
%TODO: Examples
\end{ex}
\begin{rem}
\begin{itemize}
\item The mapping $\Theta$ is not unique.
\item Isomorphic graphs have the same number of vertices and edges.
\item Up to isomorphism there is only one complete graph on $n$ vertices denoted $K_n$.
So $K_n$ is a class of graphs that are all isomorphic to each other.
\item The same as above goes for each of $K_{m,n},\ P_n,\ C_n$.
\end{itemize}
\end{rem}
\subsection{Directed graphs}
\begin{defns}
In a graph $G= (V,E)$ a \emph{walk} from a starting vertex $v_0$ to a destination vertex $v_n$ is a sequence
$$W=\langle v_0,e_1,v_1,e_2,\ldots,e_n,v_n\rangle\text{ s.t. }e_i=(v_{i-1},v_i)$$
The \emph{length} of the walk is the number of edges in the sequence $W$.
A \emph{closed walk} has the same starting and ending vertices.
A \emph{path} is a walk where every vertex appears at most once.\\
The \emph{distance} $d(s,t)$ between a vertex $s$ and a vertex $t$ in a graph $G$ is the length of the shortest path between $s$ and $t$ (if one exists) and $d(s,t)=+\infty$ otherwise.\\
A \emph{weighted graph} $G= (V,E,\omega)$ is a graph and a function $\omega\colon E\to \RR$ assigning weights to edges.
\end{defns}

We now define some problems that drive our study of algorithms in graph theory.
\paragraph{Problem 1}
Given an edge weighted graph $G=(V,E,\omega)$.
Find a shortest path between $s$ and $t$ (length is the sum of the edge weights).

\paragraph{Problem 2}
Minimum-weight spanning tree:
Given an undirected graph $G=(V,E,\omega)$ find an acyclic connected subgraph of $G$ of smallest weight.

\paragraph{Problem 3}
Travelling salesman problem:
We are given a complete weighted graph $G$.
Find a cycle that visits every vertex exactly once and minimises its length.

\paragraph{Problem 4}
Maximum flow problem:
You have a directed graph $G$ representing a system of pipes, each pipe has a capacity.
We are given also the source $S$ and the sink $T$, $S,T\in V(G)$.
How much water can we pump through the network per time unit?

\begin{defn}
A \emph{leaf} is a vertex of degree one.
\end{defn}

\begin{prop}
Every tree $T$ has at least two leaves.
\end{prop}
\begin{proof}
Consider $P=\langle v_1,\ldots,v_k\rangle$ a path of maximum length in $T$.
Consider $v_1$ if $v_1$ has degree more than one then we can either extend the path or $T$ is not acyclic.
\end{proof}
\marginpar{Lecture 3}
\begin{cor}
If the degree of each node in $G$ is $\ge 2$ then $G$ has a cycle.
\end{cor}
\begin{prop}
Each tree has exactly $|V| -1$ edges.
\end{prop}
\begin{proof}
\lbrack By induction\rbrack
\end{proof}
\begin{defns}
An acyclic graph is called a \emph{forest}.\\
A \emph{connected component} of $G$ is a maximal connected subgraph of $G$.
Let $c(G)$ be the number of connected components of $G$. %0th betti number
\end{defns}
\begin{cor}
Any graph on $n$ vertices has at least $n-c(G)$ edges.
\end{cor}
\begin{prop}
If $G$ is a simple graph with $n$ vertices then
$$|E(G)| \le \frac{(n-k)(n-k+1)}{2} \text{ where }k=c(G)$$
\end{prop}
\begin{proof}
Let $c_1,\ldots,c_k$ be the connected components of $G$, i.e. $c_i=(V_i,E_i)$ and $E(G) = \bigcup_{i=1}^k E_i,\ V=\bigcup_{i=1}^k V_i,\ n=|V|$.
Then $n=\sum_{i=1}^k n_i$ and $n_i \ge 1$.
Obviously $|E(G_i)| \le \frac{n_i^2 -n_i}{2}$.
Thus $|E(G)| \le \sum_{i=1}^k \frac{n_i^2-n_i}{2}$.
The maximum of RHS occurs when $|C_1| = |C_2| = \ldots = |C_{k-1}| = 1$ and $|C_k| = n-k +1$. \\
Assume the contrary that the maximum occurs when $c_i = k$, and $c_j = k_s$ where $r\ge s \ge 2$.
Then $|E(c_i)| + |E(c_j)| = \frac{r^2 + s^2 - r -s}{2}$.
But is $c_i = k_{r+1}$ and $c_j = k_{s-1}$, then we have $\frac{(r+1)^2 + (s-1)^2 - (r+1) -(s-1)}{2}$ edges.
We get a contradiction since $(r+1)^2 + (s-1)^2 > r^2 + s^2$.
\end{proof}
\begin{defn}
A \emph{cut-edge} is an edge whose removal increases the number of connected components.
\end{defn}
\begin{thm}
Let $T$ be a graph with $n$ vertices, then the following statements are equivalent:
\begin{enumerate}
\item $T$ is a tree.
\item $T$ contains no cycles and has $(n-1)$ edges.
\item $T$ is connected and has $(n-1)$ edges.
\item $T$ is connected and every edge is a cut-edge.
\item Any two vertices of $T$ are connected by exactly one path.
\item $T$ contains no cycles and for any new edges the graph $T + e$ has exactly one cycle.
\end{enumerate}
\end{thm}
\begin{proof}
$(1\implies 2)$ \lbrack Previous proposition\rbrack and definition of tree. \\
$(2\implies 3)$ Suppose $T$ has $k$ connected components.
Then each component must be a tree (because they have no cycles) therefore $T$ must contain $n-k$ edges, exactly $\implies k=1$. \\
$(3\implies 4)$ Let $e$ be an edges of $T$.
Since $T\setminus e$ has $n-2$ edges by one of the corollaries, we must have $c(T)\ge 2$. \\
$(4\implies 5)$ (By contradiction) Suppose $x$ and $y$ are two vertices of tress $T$ that have two different paths, say $p_1$ and $p_2$ connecting them.
Let $u$ be the the first vertex where the paths diverge and let $v$ be the first vertex where the paths meet again.
Any edge of the cycle containing $u$ and $v$ is not a cut edge which contradicts (4). \\
$(5\implies 6)$ $T$ contains no cycles otherwise we would have two paths between $x,y\in$ cycle.
If we add an edge $e= (u,v)$ we will create a cycle since $T$ contains a path between $u$ and $v$. If $T+e$ contains two cycles then both cycles must contain $e$.
$u,v$ have two different paths containing them in $T$ is a contradiction. \\
$(6\implies 1)$ \lbrack By definition \rbrack \\
\end{proof}
\subsection{Huffman Trees and Optimal Prefix Codes}
\begin{defns}
A \emph{binary code} is an assignment of symbols or other meanings to a set of bitstrings.\\
A \emph{prefix code} is a binary code with the property that no codeword is an initial substring of any other codeword.
\end{defns}
This allows for transmission without confusion.
\begin{ex}
A prefix code for an alphabet: construct a binary tree with a leaf corresponding to a letter of your alphabet, every left-hand edge is labelled by 0 and right-hand edge is labelled by 1.
%TODO: Pictures!
\end{ex}
\marginpar{Lecture 4}
\begin{defns}
Assume we have a code.
Each codeword has a length $l_c$ and each symbol has its frequency $p_c$.
The \emph{code efficiency} is then$\sum p_cl_c$. \\
Let $T$ be a binary tree with leaves $s_1,\ldots,s_l$ such that each leaf $s_i$ is assigned weight $w_i$.\\
The \emph{average weighted depth} of the binary tree $T$ is $\wt(T) = \sum_{i=1}^l\depth(s_i)w_i$ where depth is the number of edges to the root.
\end{defns}
\begin{alg}[Constructing Huffman Prefix Codes]~\\
\textbf{Input} $s_1,\ldots,s_l$ of symbols and a list of $w_1,\ldots,w_l$ of weights $w_i$ corresponding to $s_i$.\\
\textbf{Output} A binary tree representing a prefix code for a set of symbols whose codewords have minimum average weighted length.\\
Initialise $F$ to be a forest of isolated vertices, labelled $s_1,\ldots,s_l$ with respective weights $w_1,\ldots, w_l$. \\
For $i=1$ to $l-1$ Choose two trees $T$ and $T'$ from the forest of smallest weights.\\
Create a binary tree by adding a new vertex with $T$ as a child on the left and $T'$ a child on the right. Label the edges to these new child trees 0 and 1 respectively. \\
Assign to the new tree the weight $\wt(T)+\wt(T')$. \\
Replace $T$ and $T'$ in $F$ by the new tree. \\
When finished return F.
\end{alg}
\begin{ex}
%TODO:examples
\end{ex}
\begin{lem}
If the leaves of a binary tree are assigned weights and if each internal vertex (node) is assigned a weight equal to the sum of its children's weights, then the tree's average weighted depth equals the sum of the weights of its internal (non-leaf) vertices.
\end{lem}
\begin{proof}
In assignments.
\end{proof}
\begin{thm}
For a given list of weights $w_1,\ldots,w_l$ a Huffman tree has the smallest average weighted depth among all binary trees whose leaves are assigned these weights.
\end{thm}
\begin{proof}
Induction on $l$.
Let $T$ be the Huffman tree.
\textbf{Base case} $l=2$. Then $\wt(T) = w_1 + w_2$, because $T$ is the only tree on 3 vertices.
\textbf{Inductive step} Assume that for some $l\ge 2$ the Huffman algorithm produces a Huffman tree of minimum average weighted depth for any list of $l$ weights. \\
We are given $w_1\le w_2\le \ldots\le w_{l+1}$. The Huffman tree $H$ that is constructed by the algorithm consists of the Huffman tree $\bar{H}$ for the weights $w_1+w_2,w_3,\ldots,w_{l+1}$ where the leaf with weight $w_1+w_2$ is replaced by a tree with two children of weights $w_1$ and $w_2$.
By the previous lemma, we know $\wt(H) = \wt(\bar{H}) + w_1 + w_2$.
By the inductive hypothesis, $\bar{H}$ is optimal among all binary trees whose leaves are assigned weights  $w_1+w_2,w_3,\ldots,w_{l+1}$. \\
Suppose $T^*$ is an optimal binary tree for the weights $w_1,\ldots,w_l$.
Let $x$ be an internal vertex of $T^*$ of greatest depth and suppose $y$ and $z$ are its left child and right child respectively. Without loss of generality, $y$ and $z$ have weights $w_1$ and $w_2$ otherwise we can swap their weights with $w_1$ and $w_2$ and produce a tree with the same or better weights.
Delete $y$ and $z$ from $T^*$ and call this tree $\bar{T}$, we have $\wt(T^*) = \wt(\bar{T}) + w_1 + w_2$. We also know that $\wt(\bar{T}) \ge \wt(\bar{H})\implies \wt(T^*)\ge \wt(H)$.
\end{proof}
\marginpar{Lecture 5}
\begin{defn}
Given trees $T=(V,E)$ and $T' = (V,E')$ we say $T=T'$ if $E = E'$.
\end{defn}
\begin{ex}
%TODO:examples of unequal but isomorphic trees.
\end{ex}
\begin{thm}[Cayley's Formula]
The number of $n$ vertex labelled trees is $n^{n-2}$.
\end{thm}
\begin{defn}
A \emph{Pr\"ufer sequence} of length $n-2$ for $n\ge 2$ is any sequence of integers between 1 and $n$ with repetitions allowed.
There are $n^{n-2}$ of these.
\end{defn}
\begin{alg}~\\
\textbf{Input} An $n$-vertex tree.\\
\textbf{Output} A Pr\"ufer sequence of length $n-2$.\\
Initialise $T$ to be a given tree \\
For $i=1$ to $n-2$ \\
\indent Let $v$ be the vertex of degree 1 with smallest label.\\
\indent Let $s_i$ be the only neighbour of $v$.\\
\indent Define $T = T\setminus v$\\
Return the sequence $\langle s_1,\ldots,s_{n-2}\rangle$
\end{alg}
\begin{ex}
%TODO:yet more examples
\end{ex}
\begin{prop}
Let $d_k$ be the number of occurrences of the number $k$ in a Pr\"ufer encoding sequence for a labelled tree $T$.
Then the degree of $k$ in $T$ is $d_k+1$.
\end{prop}
\begin{proof}
By induction: \\
The assertion is true for any tree on 3 vertices since the Pr\"ufer sequence consists of a single label of the vertex of degree 2. \\
Assume the assertion is true for all labelled trees on $n$ vertices for some $n\ge 3$.\\
Let $T$ be a labelled tree on $n+1$ vertices.
Let $v$ be a leaf with smallest label $l(v)$.
Let $w$ be a neighbour of $v$.
Then the Pr\"ufer sequence $S$ for $T$ consists of $l(w)$ followed by $S^*$, the Pr\"ufer sequence of $T^* = T\setminus v$. \\
Now by the inductive hypothesis, for each $u\in T^*$ the number of occurrences of $l(W)$ in $S^*$ is $d_{T^*}(u)-1$.
But for all $u\ne w$ the number of occurrences of the label $l(u)$ in $S^*$ is the same as in $S$, and $d_T(u)=d_{T^*}(u)$.
Moreover, $d_T(w) = d_{T^*}(w) + 1$ and $l(w)$ has one more occurrence in $S$ than in $S^*$.
\end{proof}
\begin{alg}[for Pr\"ufer decoding]~\\
\textbf{Input} A Pr\"ufer sequence of length $n-2$.\\
\textbf{Output} An $n$-vertex labelled tree.\\
%TODO:Finish
\end{alg}

\end{document}
