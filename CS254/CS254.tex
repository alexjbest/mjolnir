\title{CS254 Algorithmic Graph Theory - Lecture Notes}
\author{Based on lectures by Prof. Maxim Sviridenko \\ Typeset by WULNR contributors}
\date{\today}

\documentclass{book}
\usepackage{amsfonts}
\usepackage{amsmath}
\usepackage{amssymb}
\usepackage{amsthm}

\newtheorem*{thm}{Theorem}
\newtheorem*{lem}{Lemma}
\newtheorem*{cor}{Corollary}
\theoremstyle{definition}
\newtheorem*{defn}{Definition}
\newtheorem*{ex}{Example}
\newtheorem*{rem}{Remark}
\newtheorem*{nota}{Notation}

\newcommand{\ZZ}{\mathbb{Z}}
\newcommand{\QQ}{\mathbb{Q}}
\newcommand{\NN}{\mathbb{N}}

\setcounter{chapter}{0}

\begin{document}
\maketitle

\chapter{Basics}
\section{Graphs}
\begin{defn}
A graph $G$ is an ordered pair $V(G),E(G))$ consisting of a set $V(G)$ of vertices and a set $E(g)$ of edges.
Each edge $e\in E(G)$ is associated with a pair of vertices $\{u,v\}$.
We will say that $u,v$ are endpoints of $e$.
We also say $u,v$ are incident to $e$ and vice versa.
\end{defn}

\begin{defn}
Two vertices which are incident with a common edge are adjacent (or neighbours).
Two or more edges with the same pair of endpoints are said to be parallel.
\end{defn}

\begin{defn}
A simple graph is one without parallel edges.
\end{defn}

\begin{nota}
By convention, $G$ will denote a graph, $n$ and $m$ will be the number of vertices $|V(G)|$ and the number of edges $|E(G)|$ respectively.
\end{nota}

\section{Special graphs}
\begin{defn}
A complete graph is a simple graph in which every two vertices are adjacent. \\
Such a graph on $n$ vertices is denoted $K_n$.
\end{defn}
%TODO: pictures

We can see that $|E(K_n)| = \frac{n(n-1)}{2}$.
\begin{defn}
A graph $G=(V,E)$ is called empty if $E=\emptyset$.
\end{defn}

\begin{defn}
A graph is called bipartite is it's vertex set can be partitioned into two subsets $X$ and $Y$ so that every edge has one endpoint in $X$ and one in $Y$.
\end{defn}

\begin{ex}
The set of all (monogamous) people can be represented by a graph with edges joining married couples, this graph is bipartite.
\end{ex}

We will call $X,Y$ the parts of the graph. So a bipartite graph is often denoted as $G = (X,Y,E)$.
\begin{nota}
We write $K_{m,n}$ for the complete bipartite graph with $|X| = m,\ |Y| = n$ and $E$ containing all edges between $X$ and $Y$.
\end{nota}

\begin{defn}
The graph $K_{1,n}$ is often called a star.
\end{defn}
%TODO: pictures

\begin{defn}
A path is a simple graph whose vertices can be rearranges into a linear sequence in such a wayy that two adjacent vertices are always consecutive in the sequence and vice versa, non-adjacent vertices are non-consecutive, such a graph on $n$ vertices is often denoted $P_n$.
\end{defn}
%TODO: pictures

\begin{defn}
A cycle on $n$ vertices is a graph defined in a similar way but a cyclic sequence of vertices.
\end{defn}
%TODO: pictures

\begin{nota}
The length of a path or cycle is the number of it's edges. A path or cycle of length $k$ is often called a $k$-path of $k$-cycle.
\end{nota}
%TODO: pictures

\subsection{Adjacency and incidence matrices}
\begin{defn}
The incidence matrix of $G = (V,E)$ when $|V| = m$ and $|E| = n$ is the $m\times n$ matrix $M_G = (m_{ve})$ where $m_{ve}$ is the number of times that vertex $v$ and $e$ are incident ($m_{ve} \in \{0, 1\}$).
\end{defn}
\begin{ex}
%TODO: pics + matrices
\end{ex}
\begin{defn}
The adjacency matrix of $G = (V,E)$ when $|V| = m$ is the $m\times m$ matrix $A_G = (a_{uv})$ where $m_{uv}$ is the number of edges joining $u$ and $v$.
\end{defn}
\begin{ex}
$$ = A_{K_{3}}$$
%TODO: pics + matrices
\end{ex}
\begin{defn}
An adjacency list is a list $(N(v),v\in V)$ used to represent matrices, $N(v)$ is the set of neighbours of $v$ in arbitary order.
\end{defn}

\subsection{Adjacency and incidence matrices}
\begin{defn}
The degree of a vertex $v$ in a graph $G$ is denoted $d_G(v)$ or $d_v$ and is the number of edges of $G$ incident with $v$.
\end{defn}
\begin{thm}[Handshake lemma]
For any graph
\begin{equation}
\label{eq:hand}
\sum_{v\in V}d_v = 2|E|
\end{equation}
\end{thm}
\begin{thm}
In any graph, the number of vertices of odd degree is even.
\end{thm}
\begin{proof}
Consider the equation \ref{eq:hand} module 2.
We have degree of each vertex $d_v \equiv$ 1 if $d_v$ is odd, or 0 is $d_v$ is even.
Therefore the left hand side of \ref{eq:hand} is congruent to the number of vertices of odd degree and the RHS is 0.
Therefore the number of vertices of odd degree is congruent to 0 mod 2.
\end{proof}
\begin{defn}
A graph $G$ is called $k$=regular is $d_G(v) = k$ for all $v\in V(G)$.
A graph is regular if it is $k$-regular for some $k$.
\end{defn}
\begin{ex}
$K_n$ is $(n-1)$-regular. \\
$C_n$ is $2$-regular. \\
$P_n$ is not regular. \\
\end{ex}
\end{document}
