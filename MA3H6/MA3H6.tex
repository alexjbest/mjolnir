\title{MA3H6 Algebraic Topology - Lecture Notes}
\author{Based on lectures by Dr. Saul Schleimer \\ Typeset by Alex J. Best}
\date{\today}

\documentclass{article}
\usepackage{amsfonts}
\usepackage{amsmath}
\usepackage{amssymb}
\usepackage{amsthm}
\usepackage{hyperref}
\usepackage{enumerate}


\newtheorem*{thm}{Theorem}
\newtheorem*{lem}{Lemma}
\newtheorem*{cor}{Corollary}
\newtheorem*{prop}{Proposition}
\theoremstyle{definition}
\newtheorem*{defn}{Definition}
\newtheorem*{defns}{Definitions}
\newtheorem*{ex}{Example}
\newtheorem*{nex}{Non Example}
\newtheorem*{exer}{Exercise}
\newtheorem*{rem}{Remark}
\newtheorem*{nota}{Notation}
\newtheorem*{alg}{Algorithm}
\newtheorem*{hint}{Hint}

\newcommand{\ZZ}{\mathbb{Z}}
\newcommand{\QQ}{\mathbb{Q}}
\newcommand{\NN}{\mathbb{N}}
\newcommand{\RR}{\mathbb{R}}

\DeclareMathOperator{\im}{im}


%TODO List:
%Fix/complete lecture numbering

\setcounter{tocdepth}{3}

\begin{document}
\maketitle
\tableofcontents

\section{Introduction}
These are lecture notes that I typeset for MA3H6 Algebraic Topology in 2014, they are currently full of gaps, mistakes, wrong statements, notation abuse and lots of other badness.
However they might be useful to someone, despite the fact they lack very many pictures at present.
If you find anything else that can be improved send me an email at a.j.best@warwick.ac.uk, thanks.
\clearpage

\section{Basics}
\subsection{Topological review}

\begin{nota}
\marginpar{Lecture 1}
\begin{align*}
\RR^n &= \{(x_1, \ldots,x_n)\mid x_j\in\RR\}\text{ with the product topology of open intervals}.\\
\|x\| &= \sqrt{\sum x_i^2}.\\
B^n &= \{x\in \RR^n \mid \|x\| \le 1\}\text{ the $n-1$ sphere}.\\
S^{n-1} &= \{x\in\RR^n\mid\|x\| = 1\}.\\
\RR^0 &= \{()\}.
\end{align*}
\end{nota}

%TODO pictures

\begin{exer}
\[B^n \times B^m \cong B^{n+m}.\]
\end{exer}

\begin{exer}
\[S^n \times S^m\not\cong S^{n+m}.\]
\end{exer}

\begin{hint}
Find an invariant of topological spaces that distinguishes them.
\end{hint}

\paragraph{Invariants}
Connectedness, Hausdorffness, $\pi_1$, compactness, Euler characteristic.
But none of these work.

\paragraph{Quotients}
We recall that the quotient topology is defined by $a \subseteq X/\sim$ is open iff its preimage under the map $f\colon X \to X/\sim$ is open.
This topology makes as many of the sets of the quotient as possible open while keeping the quotient map continuous.

There are more ways to produce $S^1$, for example
\[S^1 \cong [0,1]/0\sim1\]
when equipped with the quotient topology.

Another way is to consider $\RR/\ZZ = \RR/\{x\sim y \iff x-y\in \ZZ\}$.
So there is a map $\RR \to \RR/\ZZ$ which is the covering map of $\RR/\ZZ$ by its universal cover.

\section{Simplicial homology}
\subsection{Simplices}
\begin{defn}
We define the $n$-simplex to be
\[\Delta^n = \left\{x\in \RR^{n+1} \mid x_i \ge 0\ \forall i,\ \sum x_i = 1\right\}.\]
\end{defn}

%TODO pictures

\begin{exer}
What is a ``hedron''?
\end{exer}

In general if $v_i\in \RR^m$ are a collection of $n+1$ affinely independent points (do not lie in an $n-1$ dimensional subspace) then we define
\[[v] = [v_0,v_1,\ldots,v_n] = \left\{\sum x_i v_i \mid x_i\in \Delta^n\right\}.\]

%TODO picture

If we omit some of the $v_i$ we obtain a facet of $[v]$.
If we only omit one of them we get a face.
This is denoted by
\[[v_0,v_1, \ldots, \hat v_i, \ldots , v_n]\]
where the $v_i$ is read to be omitted.

The vertices are ordered and if $[v]$, $[w]$ are simplices of the same dimension then there exists a unique affine map extending the ordering of the vertices.
The standard map $f\colon [v]\to [w]$ sends $v_i$ to $w_i$ and respects barycentric coordinates. 


\begin{defn}
\marginpar{Lecture 2}
A facet of $\Delta$ is a subsimplex (i.e. pick some $x_i$ and set them to zero).
\end{defn}

\begin{defn}
A face is a codimension one facet.
\end{defn}

\begin{defn}
The boundary of $\Delta^n$ is denoted by $\partial \Delta^n$ and consists of the union of its faces.
\end{defn}

We have that $\mathring{\Delta} =\Delta - \partial \Delta$.

\begin{ex}
%TODO pictures
\end{ex}

\begin{exer}
Count the $k$-dimensional faces of $\Delta^n$.
\end{exer}

\subsection{$\Delta$-complexes}
\begin{defn}
Fix $X$ a topological space and a collection of maps
\[\{\sigma_\alpha \colon \Delta_\alpha \to X \mid \alpha \in A\}.\]
This is known as a \emph{$\Delta$-complex structure} on $X$ if:
\begin{enumerate}[(i)]
\item (Partition) for all $\alpha$ $\sigma_\alpha|\mathring{\Delta}_\alpha$ is injective and for $x\in X$ there is a \emph{unique} $\alpha\in A$ s.t. $x\in\sigma_\alpha(\mathring{\Delta}_\alpha)$.
\item (Tiling) If $\Delta\subset \Delta_\alpha$ is a face then there is a unique $\beta\in A$ s.t. $\sigma_\setminus|\Delta = \sigma_\beta\circ f$ where $f \colon \Delta \to \Delta_\beta$ is the canonical map.
\item (Topology) $U\subset X$ is open iff $\forall \alpha\ \sigma^{-1}_\alpha (U) \subset \Delta_\alpha$ is open.
\end{enumerate}
\end{defn}

We can state this equivalently as: $X$ must be homeomorphic to the quotient space
\[\bigsqcup_{\alpha\in A} \Delta_\alpha / \text{face gluings}.\]

\begin{ex}
%TODO
\end{ex}

\begin{ex}
$\partial \Delta^n$ gives a $\Delta$-complex structure on $S^{n-1}$.
\end{ex}

\begin{ex}
If we double $\Delta^n$ across $\partial \Delta$ we get a $\Delta$-complex structure on $S^n$.
%TODO picture
\end{ex}

\begin{ex}
Check these are homeomorphic to $S^n$.
\end{ex}

\begin{nex}
%TODO picture
Violates tiling on the edge marked $[0,2]$ and so is not a $\Delta$-complex structure.
\end{nex}

\begin{exer}
\begin{enumerate}
\item Find a $\Delta$-complex structure on the space in the non-example above.
\item Show that every graph admits a $\Delta$-complex structure.
\end{enumerate}
\end{exer}

\begin{ex}
%TODO picture 
Here the indexing set $A = \RR$ (very big!).
\end{ex}

\begin{defn}
A $\Delta$-complex is \emph{finite dimensional} if there exists $n$ s.t. for all $\alpha$ $\dim(\Delta_\alpha) \le n$.
\end{defn}

\begin{defn}
A $\Delta$-complex structure is \emph{finite} if $|A| < \infty$ (where as above $A$ is the index set).
\end{defn}

\begin{exer}
Show that if $X$ admits a $\Delta$-complex structure then $X$ is Hausdorff.
\end{exer}

\begin{exer}
Show that if $\{\sigma_\alpha\}$ is a $\Delta$-complex structure on $X$ and $K\subset X$ is compact then $K$ meets the interiors of only finitely many of the $\sigma_\alpha$'s.
\end{exer}

\begin{exer}
If $X,Y$ admit $\Delta$-complex structures then so does $X\times Y$.
\end{exer}

\subsection{Abelian groups}
Fix $A$ a set.
Define $\ZZ[A]$ to be the \emph{free abelian group} on $A$ given by
\begin{align*}
\ZZ[A] &= \left\{ \sum_{\alpha\in A} n_\alpha \cdot \alpha \middle| n_\alpha\in\ZZ\text{ and all but finitely many are non-zero}\right\} \\
&= \text{all finite $\ZZ$-linear sums}.
\end{align*}

\begin{ex}
\[\ZZ[\{\alpha,\beta\}] \cong \ZZ^2 = \{n\alpha + m\beta \mid m,n\in \ZZ\}.\]
If $A$ is finite then $\ZZ[A] \cong \ZZ^A$.
But if $|A| = \infty$ then this is false.
\end{ex}

\begin{exer}
$\QQ$ is \emph{not} a free abelian group.
\end{exer}

\subsection{Chains}
Suppose $(X,\{\sigma\})$ is a space equipped with a $\Delta$-complex structure.

\begin{defn}
We define the set of \emph{$n$-chains} to be
\[C_n^\Delta = \ZZ[\{\sigma_\alpha \mid \dim(\Delta_\alpha) = n\}].\]
\end{defn}

\begin{ex}
%TODO
\end{ex}

\subsection{Boundary operators}
\marginpar{Lecture 3}
Recall $\Delta_v = [v_0, v_1,\ldots,v_n]$ is an $n$-simplex.

The $i$th face of $\Delta$ is $[v_0, v_1, \ldots , \hat{v}_i, \ldots ,v_n]$. %TODO th

\begin{defn}
We define the \emph{boundary operator} as follows.
First suppose $\sigma \colon \Delta \to X$ is a map.

We then define
\[\partial \sigma = \sum_{i=0}^{n} (-1)^i \sigma \mid_{[v_0,\ldots,\hat{v}_i, \ldots, v_n]}.\]
Which is an $(n-1)$-chain.

So we extend linearly to define 
\[\partial \colon C_n^\Delta(X) \to C_{n-1}^\Delta (X)\]
given by
\[\sum n_\alpha \sigma_\alpha \mapsto \sum n_\alpha \partial \sigma_\alpha.\]
\end{defn}

\begin{ex}
%TODO
\end{ex}

\begin{lem}
\[\partial_{n-1} \circ \partial_n = 0.\]
``The extremes of the extremes are empty''.
\end{lem}
\begin{proof}
It suffices to check this on a basis element
\[\sigma \colon \Delta^n \to X\]
so
\[\partial_n \sigma = \sum_{i=0}^{n} (-1)^i \sigma \mid_{[v_0, \ldots, \hat{v}_i,\ldots,v_n]}\]
now we apply $\partial_{n-1}$:
\begin{align*}
\partial_{n-1} \partial_n \sigma &= \partial_{n-1} \left(\sum_{i=0}^{n} (-1)^i \sigma \mid_{[v_0, \ldots, \hat{v}_i,\ldots,v_n]} \right) = \sum_{i=0}^{n} (-1)^i \partial_{n-1} \left(\sigma \mid_{[v_0, \ldots, \hat{v}_i,\ldots,v_n]}\right)\\
&= \sum_{i=0}^{n} (-1)^i \sum_{j= 0}^{n-1} (-1)^j\left(\sigma \mid_{[v_0, \ldots, \hat{v}_i,\ldots,v_n]}\right)\mid_{[w_0, \ldots, \hat{w}_j,\ldots,w_{n-1}]}\\
&= \sum_{i=0}^{n} (-1)^i\left( \sum_{j< i} (-1)^j \sigma \mid_{[v_0, \ldots,\hat{v}_j, \ldots, \hat{v}_i,\ldots,v_n]} + \sum_{j> i} (-1)^{j+1} \sigma \mid_{[v_0, \ldots, \hat{v}_i,\ldots,\hat{v}_j,\ldots,v_n]}\right) \\
&= \sum_{j< i} (-1)^{j+i} \sigma \mid_{[v_0, \ldots,\hat{v}_j, \ldots, \hat{v}_i,\ldots,v_n]} - \sum_{j> i} (-1)^{j+i} \sigma \mid_{[v_0, \ldots, \hat{v}_i,\ldots,\hat{v}_j,\ldots,v_n]} \\
&= 0
\end{align*}
\end{proof}

\subsection{Chain complexes}
\begin{defn}
A sequence $\{C_n\}_{n=0}^{\infty}$ of abelian groups with homomorphisms
\[\partial_n\colon C_n\to C_{n-1}\]
 s.t. $\partial^2 = 0$ is called a \emph{chain complex}.
\end{defn}

By convention we take $C_{-1}$ to be 0.

\begin{ex}
\[0\to\ZZ\xrightarrow{\times 2} \ZZ \to 0.\]
\end{ex}

\paragraph{Terminology}
If $c\in C_n$ we call $c$ an \emph{$n$-chain}.

If $z\in Z_n = \ker(\partial_n)$ we call $z$ an \emph{$n$-cycle}.

If $b\in B_n = \im(\partial_{n-1})$ we call $b$ an \emph{$n$-boundary}.

If $h\in Z_n/B_n = H_n$ we call $h$ a \emph{homology class}.

%TODO diagram

Since $\partial^2 = 0$ we deduce that $B_n \le Z_n$ and $H_n = Z_n/B_n$ makes sense.

\begin{ex}
For
\[0\to\ZZ\xrightarrow{\times 2} \ZZ \to 0\]
we have $H_1 = 0$, $H_0 = \ZZ/2\ZZ$ and $H_k = 0$ for all $k \ge 1$.
\end{ex}

\begin{defn}
If $(X,\sigma)$ is a $\Delta$-complex then set $C_n^\Delta(X) = \ZZ[\{\sigma_\alpha \mid \dim(\Delta_\alpha) = n\}]$ and $\partial_n\colon C_n^\Delta(X) \to C_{n-1}^\Delta(X)$ is the boundary operator.

Then $H_n^\Delta(X)$ are called the \emph{simplicial homology groups} of $X$.
\end{defn}

\begin{thm}
This is independent of the choice of $\Delta$-complex structure on $X$.
\end{thm}

\subsection{Computations}
\begin{enumerate}
\item $X = \{\text{pt}\}$. $C_0^\Delta (X) \cong \ZZ$ and all others are 0, so we have the chain complex:
\[
\cdots \rightarrow 0 \rightarrow 0 \rightarrow \ZZ \rightarrow 0.
\]
So $H_0^\Delta(\text{pt}) \cong \ZZ$ and $H_k^\Delta(\text{pt}) \cong 0$ if $k\ge 1$.

\item $X = S^1$.
$C_0^\Delta (X) \cong \ZZ$, $C_1^\Delta (X) \cong \ZZ$ and all others are 0, so we have the chain complex:
\[
\cdots \rightarrow 0 \rightarrow \ZZ \xrightarrow{\partial} \ZZ \rightarrow 0.
\]
We see that $\partial e = \sum_{i=0}^{1} (-1)^i e|_{[v_0,\ldots,\hat{v}_i,\ldots,v_1]} = e|_{[v_1]} - e|_{[v_0]} = v-v = 0$.
So 
\[H_k^\Delta(S^1) \cong\begin{cases}
\ZZ &\text{ if } k = 0 \text{ or }1, \\
0 &\text{ otherwise}.
\end{cases} \]

%TODO more here?




\end{enumerate}

\begin{exer}
Compute $H_\bullet^\Delta (S^1)$ for the $\Delta$-complex structure on $S^1$ with $k$ vertices and $k$ edges.
\end{exer}

\begin{exer}
Compute $H_\bullet^\Delta (X)$ for the $X = B^2,\ S^1$ and $K^2$ (the Klein bottle).
\end{exer}

\begin{exer}
Using the fact that $\Delta^n$ is a $\Delta$-complex structure on $B^n$ compute $H_\bullet^\Delta (B^n)$.

In general you'll want to make use of the Smith normal form.
\end{exer}

\section{Singular homology}
\begin{defn}
A \emph{singular $n$-simplex} in $X$ is a map $\sigma \colon \Delta^n \to X$.
\end{defn}

\begin{defn}
\[
C_n^{\text{sing}}(X) = \ZZ[\{\sigma\colon \Delta^n \to X\}].\]
\end{defn}

We call $c \in C_n^\text{sing}(X)$ a singular $n$-chain.

\begin{defn}
We define $\partial\colon C_n^\text{sing}(X) \to C_{n-1}^\text{sing}(X)$ exactly as before by
\[\partial \sigma = \sum_{i=0}^{n} (-1)^i \sigma|_{[v_0,\ldots,\hat{v}_i, \ldots, v_n]}.\]
\end{defn}

And again we define $Z_n^\text{sing}(X)$ (resp. $B_n^\text{sing}(X)$) exactly as above and call it the group of singular $n$-cycles (resp. $n$-boundaries).

\begin{defn}
Now $H_n^\text{sing}(X) = Z_n^\text{sing}(X)/B_n^\text{sing}(X)$ is the $n$-th \emph{singular homology group}.
\end{defn}

\begin{rem}
We have that $\partial_{n-1}\circ \partial_n = 0$ exactly as before.
\end{rem}

\end{document}
