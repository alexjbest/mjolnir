\title{MA3H6 Algebraic Topology - Lecture Notes}
\author{Based on lectures by Dr. Saul Schleimer \\ {\normalsize Typeset by Alex J. Best}}
\date{\today}

\documentclass[12pt]{article}
\usepackage{amsfonts, amsmath, amssymb, amsthm, hyperref, enumerate, fullpage}
\usepackage[all]{xy}

\theoremstyle{definition}
\newtheorem*{thm}{Theorem}
\newtheorem*{lem}{Lemma}
\newtheorem*{cor}{Corollary}
\newtheorem*{prop}{Proposition}
\theoremstyle{definition}
\newtheorem*{defn}{Definition}
\newtheorem*{defns}{Definitions}
\newtheorem*{ex}{Example}
\newtheorem*{nex}{Non Example}
\newtheorem*{exer}{Exercise}
\newtheorem*{rmk}{Remark}
\newtheorem*{nota}{Notation}
\newtheorem*{alg}{Algorithm}
\newtheorem*{hint}{Hint}

\newcommand{\ZZ}{\mathbb{Z}}
\newcommand{\QQ}{\mathbb{Q}}
\newcommand{\NN}{\mathbb{N}}
\newcommand{\RR}{\mathbb{R}}
\newcommand{\C}{C_*}
\newcommand{\U}{\mathcal{U}}
\renewcommand{\H}{H_*}

\DeclareMathOperator{\im}{im}
\DeclareMathOperator{\diam}{diam}

\allowdisplaybreaks

\setcounter{tocdepth}{3}

\begin{document}
\maketitle
\tableofcontents

\section{Introduction}
These are lecture notes that I typeset for MA3H6 Algebraic Topology in 2014, they are currently full of gaps, mistakes, wrong statements, notation abuse and lots of other badness.
However they might be useful to someone, despite the fact they lack very many pictures at present.
If you find anything else that can be improved send me an email at a.j.best@warwick.ac.uk, thanks.

\section{Basics}
\subsection{Topological review}

\begin{nota}
\begin{align*}
\RR^n &= \{(x_1, \ldots,x_n)\mid x_j\in\RR\}\text{ with the product topology of open intervals}.\\
\|x\| &= \sqrt{\sum x_i^2}.\\
B^n &= \{x\in \RR^n \mid \|x\| \le 1\}\text{ the }n-1\text{ sphere}.\\
S^{n-1} &= \{x\in\RR^n\mid\|x\| = 1\}.\\
\RR^0 &= \{()\}.
\end{align*}
\end{nota}

%TODO pictures

\begin{exer}
\[B^n \times B^m \cong B^{n+m}.\]
\end{exer}

\begin{exer}
\[S^n \times S^m\not\cong S^{n+m}.\]
\end{exer}

\begin{hint}
Find an invariant of topological spaces that distinguishes them.
\end{hint}

\paragraph{Invariants}
Connectedness, Hausdorffness, $\pi_1$, compactness, Euler characteristic.
But none of these work.

\paragraph{Quotients}
We recall that the quotient topology is defined by $a \subseteq X/\sim$ is open iff its preimage under the map $f\colon X \to X/\sim$ is open.
This topology makes as many of the sets of the quotient as possible open while keeping the quotient map continuous.

There are more ways to produce $S^1$, for example
\[S^1 \cong [0,1]/0\sim1\]
when equipped with the quotient topology.

Another way is to consider $\RR/\ZZ = \RR/\{x\sim y \iff x-y\in \ZZ\}$.
So there is a map $\RR \to \RR/\ZZ$ which is the covering map of $\RR/\ZZ$ by its universal cover.

\section{Simplicial homology}
\subsection{Simplices}
\begin{defn}
We define the $n$-simplex to be
\[\Delta^n = \left\{x\in \RR^{n+1} \mid x_i \ge 0\ \forall i,\ \sum x_i = 1\right\}.\]
\end{defn}

%TODO pictures

In general if $v_i\in \RR^m$ are a collection of $n+1$ affinely independent points (do not lie in an $n-1$ dimensional subspace) then we define
\[[v] = [v_0,v_1,\ldots,v_n] = \left\{\sum x_i v_i \mid x_i\in \Delta^n\right\}.\]

%TODO picture

If we omit some of the $v_i$ we obtain a facet of $[v]$.
If we only omit one of them we get a face.
This is denoted by
\[[v_0,v_1, \ldots, \hat v_i, \ldots , v_n]\]
where the $v_i$ is read to be omitted.

The vertices are ordered and if $[v]$, $[w]$ are simplices of the same dimension then there exists a unique affine map extending the ordering of the vertices.
The standard map $f\colon [v]\to [w]$ sends $v_i$ to $w_i$ and respects barycentric coordinates. 


\begin{defn}
A facet of $\Delta$ is a subsimplex (i.e. pick some $x_i$ and set them to zero).
\end{defn}

\begin{defn}
A face is a codimension one facet.
\end{defn}

\begin{defn}
The boundary of $\Delta^n$ is denoted by $\partial \Delta^n$ and consists of the union of its faces.
\end{defn}

We have that $\mathring{\Delta} =\Delta - \partial \Delta$.

\begin{ex}
%TODO pictures
\end{ex}

\begin{exer}
Count the $k$-dimensional faces of $\Delta^n$.
\end{exer}

\subsection{$\Delta$-complexes}
\begin{defn}
Fix $X$ a topological space and a collection of maps
\[\{\sigma_\alpha \colon \Delta_\alpha \to X \mid \alpha \in A\}.\]
This is known as a \emph{$\Delta$-complex structure} on $X$ if:
\begin{enumerate}[(i)]
\item (Partition) for all $\alpha$ $\sigma_\alpha|\mathring{\Delta}_\alpha$ is injective and for $x\in X$ there is a \emph{unique} $\alpha\in A$ s.t. $x\in\sigma_\alpha(\mathring{\Delta}_\alpha)$.
\item (Tiling) If $\Delta\subset \Delta_\alpha$ is a face then there is a unique $\beta\in A$ s.t. $\sigma_\alpha|\Delta = \sigma_\beta\circ f$ where $f \colon \Delta \to \Delta_\beta$ is the canonical map.
\item (Topology) $U\subset X$ is open iff $\forall \alpha\ \sigma^{-1}_\alpha (U) \subset \Delta_\alpha$ is open.
\end{enumerate}
\end{defn}

We can state this equivalently as: $X$ must be homeomorphic to the quotient space
\[\bigsqcup_{\alpha\in A} \Delta_\alpha / \text{face gluings}.\]

\begin{ex}
%TODO
\end{ex}

\begin{ex}
$\partial \Delta^n$ gives a $\Delta$-complex structure on $S^{n-1}$.
\end{ex}

\begin{ex}
If we double $\Delta^n$ across $\partial \Delta$ we get a $\Delta$-complex structure on $S^n$.
%TODO picture
\end{ex}

\begin{ex}
Check these are homeomorphic to $S^n$.
\end{ex}

\begin{nex}
%TODO picture
Violates tiling on the edge marked $[0,2]$ and so is not a $\Delta$-complex structure.
\end{nex}

\begin{exer}
\begin{enumerate}
\item Find a $\Delta$-complex structure on the space in the non-example above.
\item Show that every graph admits a $\Delta$-complex structure.
\end{enumerate}
\end{exer}

\begin{ex}
%TODO picture 
Here the indexing set $A = \RR$ (very big!).
\end{ex}

\begin{defn}
A $\Delta$-complex is \emph{finite dimensional} if there exists $n$ s.t. for all $\alpha$ $\dim(\Delta_\alpha) \le n$.
\end{defn}

\begin{defn}
A $\Delta$-complex structure is \emph{finite} if $|A| < \infty$ (where as above $A$ is the index set).
\end{defn}

\begin{exer}
Show that if $X$ admits a $\Delta$-complex structure then $X$ is Hausdorff.
\end{exer}

\begin{exer}
Show that if $\{\sigma_\alpha\}$ is a $\Delta$-complex structure on $X$ and $K\subset X$ is compact then $K$ meets the interiors of only finitely many of the $\sigma_\alpha$'s.
\end{exer}

\begin{exer}
If $X,Y$ admit $\Delta$-complex structures then so does $X\times Y$.
\end{exer}

\subsection{Abelian groups}
Fix $A$ a set.
Define $\ZZ[A]$ to be the \emph{free abelian group} on $A$ given by
\begin{align*}
\ZZ[A] &= \left\{ \sum_{\alpha\in A} n_\alpha \cdot \alpha \middle| n_\alpha\in\ZZ\text{ and all but finitely many are non-zero}\right\} \\
&= \text{all finite }\ZZ\text{-linear sums of elements of }A.
\end{align*}

\begin{ex}
\[\ZZ[\{\alpha,\beta\}] \cong \ZZ^2 = \{n\alpha + m\beta \mid m,n\in \ZZ\}.\]
If $A$ is finite then $\ZZ[A] \cong \ZZ^A$.
But if $|A| = \infty$ then this is false.
\end{ex}

\begin{exer}
$\QQ$ is \emph{not} a free abelian group.
\end{exer}

\subsection{Chains}
Suppose $(X,\{\sigma\})$ is a space equipped with a $\Delta$-complex structure.

\begin{defn}
We define the set of \emph{$n$-chains} to be
\[C_n^\Delta = \ZZ[\{\sigma_\alpha \mid \dim(\Delta_\alpha) = n\}].\]
\end{defn}

\begin{ex}
%TODO
\end{ex}

\subsection{Boundary operators}
Recall $\Delta_v = [v_0, v_1,\ldots,v_n]$ is an $n$-simplex.

The $i$th face of $\Delta$ is $[v_0, v_1, \ldots , \hat{v}_i, \ldots ,v_n]$. %TODO th

\begin{defn}
We define the \emph{boundary operator} as follows.
First suppose $\sigma \colon \Delta \to X$ is a map.

We then define
\[\partial \sigma = \sum_{i=0}^{n} (-1)^i \sigma \mid[v_0,\ldots,\hat{v}_i, \ldots, v_n].\]
Which is an $(n-1)$-chain.

So we extend linearly to define 
\[\partial \colon C_n^\Delta(X) \to C_{n-1}^\Delta (X)\]
given by
\[\sum n_\alpha \sigma_\alpha \mapsto \sum n_\alpha \partial \sigma_\alpha.\]
\end{defn}

\begin{ex}
%TODO
\end{ex}

\begin{lem}
\[\partial_{n-1} \circ \partial_n = 0.\]
``The extremes of the extremes are empty''.
\end{lem}
\begin{proof}
It suffices to check this on a basis element
\[\sigma \colon \Delta^n \to X\]
so
\[\partial_n \sigma = \sum_{i=0}^{n} (-1)^i \sigma \mid[v_0, \ldots, \hat{v}_i,\ldots,v_n]\]
now we apply $\partial_{n-1}$:
\begin{align*}
\partial_{n-1} \partial_n \sigma &= \partial_{n-1} \left(\sum_{i=0}^{n} (-1)^i \sigma \mid[v_0, \ldots, \hat{v}_i,\ldots,v_n] \right) \\
&= \sum_{i=0}^{n} (-1)^i \partial_{n-1} \left(\sigma \mid[v_0, \ldots, \hat{v}_i,\ldots,v_n]\right)\\
&= \sum_{i=0}^{n} (-1)^i \sum_{j= 0}^{n-1} (-1)^j\left(\sigma \mid[v_0, \ldots, \hat{v}_i,\ldots,v_n]\right)\mid[w_0, \ldots, \hat{w}_j,\ldots,w_{n-1}]\\
&= \sum_{i=0}^{n} (-1)^i\left( \sum_{j< i} (-1)^j \sigma \mid[v_0, \ldots,\hat{v}_j, \ldots, \hat{v}_i,\ldots,v_n] \right.\\
&\ +\left. \sum_{j> i} (-1)^{j+1} \sigma \mid[v_0, \ldots, \hat{v}_i,\ldots,\hat{v}_j,\ldots,v_n]\right) \\
&= \sum_{j< i} (-1)^{j+i} \sigma \mid[v_0, \ldots,\hat{v}_j, \ldots, \hat{v}_i,\ldots,v_n] \\
&\ - \sum_{j> i} (-1)^{j+i} \sigma \mid[v_0, \ldots, \hat{v}_i,\ldots,\hat{v}_j,\ldots,v_n] \\
&= 0
\end{align*}
\end{proof}

\subsection{Chain complexes}
\begin{defn}
A sequence $\{C_n\}_{n=0}^{\infty}$ of abelian groups with homomorphisms
\[
\partial_n\colon C_n\to C_{n-1}
\]
such that $\partial^2 = 0$ is called a \emph{chain complex}.
\end{defn}

By convention we take $C_{-1}$ to be 0.

\begin{ex}
\[0\to\ZZ\xrightarrow{\times 2} \ZZ \to 0.\]
\end{ex}

Given two chain complexes we can form the direct sum by taking the direct sum of each of the groups and letting the operators act elementwise.

\paragraph{Terminology}
If $c\in C_n$ we call $c$ an \emph{$n$-chain}.

If $z\in Z_n = \ker(\partial_n)$ we call $z$ an \emph{$n$-cycle}.

If $b\in B_n = \im(\partial_{n-1})$ we call $b$ an \emph{$n$-boundary}.

If $h\in Z_n/B_n = H_n$ we call $h$ a \emph{homology class}.

%TODO diagram

Since $\partial^2 = 0$ we deduce that $B_n \le Z_n$ and $H_n = Z_n/B_n$ makes sense.

\begin{ex}
For
\[0\to\ZZ\xrightarrow{\times 2} \ZZ \to 0\]
we have $H_1 = 0$, $H_0 = \ZZ/2\ZZ$ and $H_k = 0$ for all $k \ge 1$.
\end{ex}

\begin{defn}
If $(X,\sigma)$ is a $\Delta$-complex then set $C_n^\Delta(X) = \ZZ[\{\sigma_\alpha \mid \dim(\Delta_\alpha) = n\}]$ and $\partial_n\colon C_n^\Delta(X) \to C_{n-1}^\Delta(X)$ is the boundary operator.

Then $H_n^\Delta(X)$ are called the \emph{simplicial homology groups} of $X$.
\end{defn}

\begin{thm}
This is independent of the choice of $\Delta$-complex structure on $X$.
\end{thm}

\subsection{Computations}
\begin{enumerate}
\item $X = \{\text{pt}\}$. $C_0^\Delta (X) \cong \ZZ$ and all others are 0, so we have the chain complex:
\[
\cdots \rightarrow 0 \rightarrow 0 \rightarrow \ZZ \rightarrow 0.
\]
So $H_0^\Delta(\text{pt}) \cong \ZZ$ and $H_k^\Delta(\text{pt}) \cong 0$ if $k\ge 1$.

\item $X = S^1$.
$C_0^\Delta (X) \cong \ZZ$, $C_1^\Delta (X) \cong \ZZ$ and all others are 0, so we have the chain complex:
\[
\cdots \rightarrow 0 \rightarrow \ZZ \xrightarrow{\partial} \ZZ \rightarrow 0.
\]
We see that $\partial e = \sum_{i=0}^{1} (-1)^i e|[v_0,\ldots,\hat{v}_i,\ldots,v_1] = e|[v_1] - e|[v_0] = v-v = 0$.
So 
\[H_k^\Delta(S^1) \cong\begin{cases}
\ZZ &\text{ if } k = 0 \text{ or }1, \\
0 &\text{ otherwise}.
\end{cases} \]

%TODO more here?




\end{enumerate}

\begin{exer}
Compute $\H^\Delta (S^1)$ for the $\Delta$-complex structure on $S^1$ with $k$ vertices and $k$ edges.
\end{exer}

\begin{exer}
Compute $\H^\Delta (X)$ for the $X = B^2,\ S^1$ and $K^2$ (the Klein bottle).
\end{exer}

\begin{exer}
Using the fact that $\Delta^n$ is a $\Delta$-complex structure on $B^n$ compute $\H^\Delta (B^n)$.

In general you'll want to make use of the Smith normal form.
\end{exer}

\section{Singular homology}
\begin{defn}
A \emph{singular $n$-simplex} in $X$ is a map $\sigma \colon \Delta^n \to X$.
\end{defn}

\begin{defn}
\[
C_n^{\text{sing}}(X) = \ZZ[\{\sigma\colon \Delta^n \to X\}].\]
\end{defn}

We call $c \in C_n^\text{sing}(X)$ a singular $n$-chain.

\begin{defn}
We define $\partial\colon C_n^\text{sing}(X) \to C_{n-1}^\text{sing}(X)$ exactly as before by
\[\partial \sigma = \sum_{i=0}^{n} (-1)^i \sigma|[v_0,\ldots,\hat{v}_i, \ldots, v_n].\]
\end{defn}

And again we define $Z_n^\text{sing}(X)$ (resp. $B_n^\text{sing}(X)$) exactly as above and call it the group of singular $n$-cycles (resp. $n$-boundaries).

\begin{defn}
Now $H_n^\text{sing}(X) = Z_n^\text{sing}(X)/B_n^\text{sing}(X)$ is the $n$-th \emph{singular homology group}.
\end{defn}

\begin{rmk}
We have that $\partial_{n-1}\circ \partial_n = 0$ exactly as before.
\end{rmk}

\begin{ex}
Suppose $X$ is a single point, then there is a unique $\Delta$-complex structure on $X$.
We say $\sigma^0\colon \ZZ \to X$ is the ``constant map''.
So $0 \to \ZZ \to 0$ is the chain complex $\C^\Delta(X)$.
So
\[
H_n^\Delta =\begin{cases}\ZZ &n=0,\\
0 &n\ge 1.
\end{cases} 
\]
Suppose $X$ is as above again, then we can compute
\[
H_n^\text{sing} =\begin{cases}\ZZ &n=0,\\
0 &n\ge 1.
\end{cases}
\]
This is as in dimension $n$ there is only the constant map
\[
\sigma^n \colon \Delta^n \to X
\]
so $C_n^\text{sing}(X) \cong \ZZ$ and we also have that
\begin{align*}
\partial \sigma^n =& \sum_{i=0}^{n} (-1)^i \sigma|[v_0,\ldots,\hat v_i,\ldots,v_n] = \sum_{i=0}^{n} (-1)^i \sigma^{n-1}\\
=& \left(\sum_{i=0}^{n} (-1)^i\right)\sigma^{n-1} =\begin{cases}
0 &n\text{ odd},\\
\sigma^{n-1} &n\text{ even},
\end{cases}
\end{align*}
except if $n=0$.
So $C_n$ is 
\[
\cdots\to \ZZ \xrightarrow{\times 0}\ZZ \xrightarrow{\times 1}\ZZ \xrightarrow{\times 0}\ZZ \xrightarrow{\times 1}\ZZ \xrightarrow{\times 0}\ZZ \to 0
\]
and so the singular homology groups are as claimed.
\end{ex}

\paragraph{Challenge} Compute $H_n^\text{sing}(S^1)$ from the definitions.

\begin{thm}
If $X$ admits a $\Delta$-complex structure then 
\[
\H^\Delta(X) \cong \H^\text{sing}(X).
\]
The left hand side is generally easier to compute, but the right can be easier to prove theorems with.
\end{thm}

\begin{prop}
If $X = \bigsqcup X_\alpha$ where all $X_\alpha$ are path connected spaces then 
\[
H_n^\text{sing}(X) = \bigoplus_\alpha H_n^\text{sing}(X_\alpha).
\]
\end{prop}
\begin{proof}
\[
C_n^\text{sing}(X) = \bigoplus_\alpha C_n^\text{sing}(X_\alpha).
\]
and $\partial$ respects this ``splitting''.
\end{proof}

\begin{prop}
If $X \ne \emptyset$ and $X$ is path connected then $H_0^\text{sing}(X) \cong \ZZ$.
\end{prop}
\begin{proof}
Define $\epsilon\colon C_0(X) \to \ZZ$ by $\sum n_\alpha v_\alpha \mapsto \sum n_\alpha$, the augmentation map, then $\epsilon$ is surjective.
We claim that $\ker(\epsilon) = \im(\partial_1)$.
Given any $\tau\colon \Delta^1 \to X$ that goes from $v$ to $w$ we have that $\partial\tau = w-v$ so $\epsilon(\partial \tau) = 1-1 =0$ and the image is contained in the kernel.
Now fix $\sum n_\alpha v_\alpha$ s.t. $\epsilon(\sum n_\alpha v_\alpha) = 0$.
Also fix some $u\in X$ and for all $\alpha$ pick $\tau_\alpha\colon\Delta^1 \to X$ a path from $u$ to $v_\alpha$.
Consider $\sum n_\alpha \tau_\alpha\in C_1 (X)$
\begin{align*}
\partial\left(\sum n_\alpha \tau_\alpha\right) &= \sum \partial(n_\alpha \tau_\alpha)\\
&= \sum n_\alpha\partial\tau_\alpha\\
&= \sum n_\alpha(v_\alpha - u) \\
&= \sum n_\alpha v_\alpha - \sum n_\alpha u \\
&= \sum n_\alpha v_\alpha - u\sum n_\alpha \\
&= \sum n_\alpha v_\alpha -u\cdot 0\in \im
\end{align*}
Hence $\im = \ker$ as claimed.

And so $H_0 = \ker(\partial_0)/\im(\partial_1) = \ker(\partial_0)/\ker(\epsilon) = C_0(X) /\ker(\epsilon) \cong \ZZ$.
\end{proof}

\subsection{Reduced Homology}
\begin{defn}
If $X$ has $k$ path components, then $H_0(X) \cong \ZZ^k$ so we define the \emph{augmented chain complex}
\[
\cdots \to C_2(X) \xrightarrow{\partial_2}C_1(X) \xrightarrow{\partial_1}C_0(X) \xrightarrow{\epsilon} \ZZ \to 0,
\]
where $\epsilon$ is the augmentation map from above.
Define the \emph{reduced homology groups} $\tilde \H (X)$ to be the homology groups of this chain complex.
So $\tilde H_n(X) = H_n(X)$ is $n> 0$ and $\tilde H_0(X) = \ker(\epsilon)/\im(\partial_1)$.
Hence if $X$ has $k$ path components
\[
\tilde H_0(X) \cong \ZZ^{k-1}.
\]
\end{defn}

Recall that $\H (X\sqcup Y) = \H(X) \oplus \H(Y)$, the reduced homology groups behave nicely with respect to many operations such as 1-point unions.
In a 1-point union $X\vee Y = X\sqcup Y /x\sim y$ for some designated point $x\in X$ and $y\in Y$.
So $\tilde \H(X\vee Y) = \tilde \H(X) \oplus \tilde \H(Y)$.

\subsection{Functoriality}
\begin{defn}
Suppose $f\colon X \to Y$ is a (continuous) map.
Let $f_n\colon C_n(X) \to C_n(Y)$ by $\sigma \mapsto f\circ \sigma$.
The function $f\circ \sigma$ is again a map from $\Delta^n$ to $Y$ and so still lies in $C_n(Y)$.

The key property of this definition is that $\partial_n\circ f_n = f_{n-1}\circ \partial_n$.
This is saying that the square
\[
\xymatrix{
C_n(X)\ar[d]^{f_{n}} \ar[r]^{\partial_n} & C_{n-1}(X) \ar[d]^{f_{n-1}}\\
C_n(Y) \ar[r]^{\partial_n} & C_{n-1}(Y)\\
}
\]
commutes.
We denote the family of these maps $f_n$ as $f_\#$,
\end{defn}

\begin{exer}
If $X \xrightarrow{f} Y \xrightarrow{g} Z$ then $(g\circ f)_n = g_n \circ f_n$.
\end{exer}

\subsection{Chain maps}
\begin{defn}
If $\C, D_*$ are chain complexes we say that a family of homomorphisms $f_\#\colon \C \to D_*$ is  a \emph{chain map} if 
\[
f_\#\circ \partial = \partial \circ f_\#
\]
\end{defn}

\begin{ex}
If $f\colon X \to Y$ is continuous then $f_\#$ is a chain map from $\C (X) \to \C(Y)$.
\end{ex}
\begin{ex}
Suppose $(X,\sigma)$ is a $\Delta$-complex then 
\[
i\colon C_n^\Delta(X) \to C_n^\text{sing}(X)
\]
is also a chain map.
\end{ex}

\begin{prop}
If $f_\#\colon \C \to D_*$ is a chain map then $f_\#$ induces a homomorphism
\[
f_*\colon \H(C) \to \H(D)
\]
given by
\[
f_*([z]) = [f_\#(z)].
\]
\end{prop}
\begin{proof}
Check that $f_\#(Z_n^C) \le Z_n^D$ (exercise) and that $f_\#(B_n^C) \le B_n^D$.
So $f_\#(b) = f_\#(\partial c )  = \partial f_\#(c)$.
\end{proof}

\begin{rmk}
If $f\colon X \to Y$ is a homomorphism then there exists a continuous inverse $g\colon Y \to X$ such that
\[
f_* \colon \H(X) \to \H(Y)
\]
is inverse to
\[
g_* \colon \H(Y) \to \H(X).
\]
\end{rmk}

\subsection{Homotopic spaces}
\begin{defn}
We say two maps $f$ and $g$ from $X \to Y$ are \emph{homotopic} if there is a map $F\colon X\times [0,1] \to Y$ such that
$f(x) = F(x,0)$ and $g(x) = F(x,1)$.
We write $f\sim g$.

We then say two spaces $X$ and $Y$ are \emph{homotopy equivalent} if there exists maps $f\colon X\to Y$ and $g\colon Y \to X$ such that
\[
(g\circ f) \sim \mathrm{Id}_X \text{ and } (f\circ g) \sim \mathrm{Id}_Y.
\]
\end{defn}

\begin{ex}
\[
S^n \sim \RR^{n+1}\setminus\{0\}
\]
via (for $n=1$)
\begin{align*}
i\colon S^1 &\to \RR^2\setminus\{0\}\\
x&\mapsto x
\end{align*}
and
\begin{align*}
r\colon \RR^2\setminus\{0\}&\to S^1\\
x&\mapsto \frac{x}{\| x\|}.
\end{align*}
We also have
\[
S^n \sim B^{n+1}\setminus\{0\}
\]
\end{ex}

\begin{thm}
If $f\sim g \colon X \to Y$ then 
\[
f_* = g_*\colon \H(X) \to \H(Y).
\]
\end{thm}
\begin{cor}
If $X$ is homotopy equivalent to $Y$ via $f$ then 
\[
f_* \colon \H(X) \to \H(Y)
\]
is an isomorphism.
\end{cor}
\begin{proof}
\[
(\mathrm{Id}_X)_* = \mathrm{Id}_{\H}
\]
\end{proof}

\begin{defn}
Suppose $f_\#,g_\#\colon\C \to D_*$ are chain maps.
A sequence of homomorphisms $P_n\colon C_n \to D_{n+1}$ is called a \emph{chain homotopy} if
\[
\partial_{n-1} P_n + P_{n-1} \partial_n = g_\# - f_\#
\]
in there is a chain homotopy between two chain maps $f_\#,g_\#$ we write $f_\# \sim g_\#$.
\[
\xymatrix{
C_{n+1} \ar[r]^\partial \ar@/_/[d]^f\ar@/^/[d]^g & C_n \ar@/_/[d]^f\ar@/^/[d]^g\ar[r]^\partial & C_{n-1} \ar@/_/[d]^f\ar@/^/[d]^g\\
D_{n+1} \ar[r]^\partial  & D_n \ar[r]^\partial & D_{n-1}
}
\]
\end{defn}

\begin{prop}
If $f_\#\sim g_\#\colon \C \to D_*$ then 
\[
f_* = g_* \colon \H(C) \to \H(D).
\]
\end{prop}
\begin{proof}
Pick any $h\in\H(C)$, we want to show $(g_* - f_*)(h) = 0$.
Choose some $x\in Z_n(C)$ such that $h = [z]$ and compute
\begin{align*}
(g_* - f_*)(h) &= (g_* - f_*)([z])\\
 &= [(g_\#- f_\#)(z)]\\
 &= [(P\partial + \partial P)(z)]\\
 &= [P\partial z + \partial Pz]\\
 &= [P0 + \partial Pz]\\
 &= [\partial(Pz)]\\
 &= 0\text{ (as }B_n = 0\text{ in homology)}.
\end{align*}
\end{proof}

\subsection{Prisms}
\begin{defn}
A \emph{prism} is a copy of $\Delta^n \times I$.
%TODO pic
\end{defn}

We can subdivide $\Delta\times I$ into $n+1$ dimensional simplices of the form
\[
[v_0,v_1,\ldots,v_i,w_i,w_{i+1},\ldots,w_n],
\]
where $v$ are the vertices of the simplex at one end of the interval and $w$ are the vertices at the other.

If we have $F\colon X\times I \to Y$ and $\sigma\colon \Delta^n \to X$ we let
\[
F\sigma = F\circ(\sigma \times \mathrm{Id}_I)\colon \Delta^n \times I \to Y.
\]

\begin{proof}[Proof (of above theorem)]
$f\sim g\colon X \to Y$, let $F\colon X \times I \to Y$ be the homotopy.
Then define
\[
P(\sigma) = \sum_{i=0}^{n} (-1)^i F\sigma|[v_0,\ldots,v_{i},w_i,w_{i+1},\ldots,w_n]
\]
this is the prism operator.
We now claim that $P$ is a chain homotopy from $f_\#$ to $g_\#$.
To see this fix $\sigma\colon\Delta^n\to X$ and compute
\begin{align*}
\partial P\sigma
&= \partial\left(\sum_{i=0}^{n} (-1)^i F\sigma|[v_0,\ldots,v_{i},w_{i+1},\ldots,w_n]\right)\\
&= \sum_{j \le i} (-1)^{i+j} F\sigma|[v_0,\ldots,\hat{v}_{j},\ldots,w_n] + \sum_{i \le j} (-1)^{i+j+1} F\sigma|[v_0,\ldots,\hat{w}_{j},\ldots,w_n]\\
\end{align*}
and
\begin{align*}
P\partial \sigma 
&= P\left(\sum (-1)^j \sigma|[v_0,\ldots,\hat{v}_j,\ldots,v_n]\right)\\
&= \sum_{i<j} (-1)^{i+j} F\sigma|[v_0,\ldots,\hat{w}_j,\ldots,w_n] + \sum_{j<i} (-1)^{i+j-1} F\sigma|[v_0,\ldots,\hat{v}_j,\ldots,w_n].
\end{align*}
So 
\begin{align*}
\partial P \sigma + P \partial \sigma &= \sum (-1)^{2i} F\sigma |[v_0,\ldots,\hat v_i,w_i,\ldots,w_n] + \sum (-1)^{2i+1} F\sigma |[v_0,\ldots,v_i,\hat w_i,\ldots,w_n]\\
&= F\sigma |[\hat v_0,w_0,\ldots,w_n] - F\sigma | [v_0,\ldots,v_n,\hat w_n] \\
&= g_\# \sigma -  f_\# \sigma.
\end{align*}
\end{proof}

\subsection{Exact sequences}
\begin{defn}
We say a complex $\C$ is \emph{exact} if $\H(C) \equiv 0$ (or equivalently if $Z_n = B_n$ for all $n$).

We also say that a sequence is \emph{short} if it has at most 3 non-zero terms.
\end{defn}

\begin{ex}
\[
0 \to \ZZ \to \ZZ^2 \to \ZZ \to 0.
\]
\[
0 \to \ZZ \xrightarrow{\times 2} \ZZ \xrightarrow{\text{mod }2} \ZZ/2\ZZ \to 0.
\]
\end{ex}

\begin{defn}
A short sequence of chain complexes
\[
0 \to A_* \xrightarrow{i_\#} B_* \xrightarrow{j_\#} C_* \to 0
\]
is \emph{exact} if for all $n$
\[
0 \to A_n \xrightarrow{i_n} B_n \xrightarrow{j_n} C_n \to 0
\]
is exact and $i_\#$, $j_\#$ are chain maps.
\end{defn}

\begin{defn}
We say that $(X,A)$ is a \emph{good pair} if $A\subseteq X$ is non-empty, closed and there exists an open $V$ with $X\supset V \supset A$ such that $V$ deformation retracts to $A$.
\end{defn}

\begin{ex}
$(\RR^2,S^1)$ is a good pair.
\end{ex}

We say $f\colon (X,A) \to (Y,B)$ is a \emph{map of pairs} if $f\colon X \to Y$ is a map and $f(A)  \subset B$.
\begin{ex}
\[
f\colon (I, \partial I) \to (\RR^2, S^1).
\]
\end{ex}
Similarly we can define a \emph{homotopy of maps of pairs} to be a function $F\colon X\times I \to Y$ where each $F_t\colon (X,A)\to (Y,B)$ is a map of pairs and $F_0= f$, $F_1 = g$.

\subsection{Relative homology}
Suppose $(X,A)$ is a pair.
Note that 
\[
i_\# \colon \C(A) \to \C(X)
\]
is an inclusion.
Define $\C(X,A) = \C(X)/\C(A)$ and we then have that $C_n(X,A) = C_n(X)/C_n(A)$ and as $\partial^X$ preserves $\C(A)$ it descends to give $\partial^{(X,A)}$.
We have that $\partial^{(X,A)}[c] = [\partial^X c]$.

\begin{exer}
Show $\partial^{(X,A)}$ is well defined and $(\partial^{(X,A)})^2 = 0$.
\end{exer}

Note that 
\[
0 \to \C(A) \to \C(X) \to \C(X,A) \to 0
\]
is a short exact sequence of chain complexes.

\begin{defn}
$H_n(X,A) = Z_n(X,A) / B_n(X,A)$, we also say that $[z] \in Z_n(X,A)$ is a \emph{relative cycle} and $[b]\in B_n(X,A)$ is a \emph{relative boundary}.
\end{defn}

%TODO fill in
If $z \in [z] \in Z_n(X,A)$ then $\partial^{(X,A)}[z] = 0 \in C_n(X,A)$ i.e. $\partial^{(X,A)}[z] = [a][\text{any }a\in C_n(A)]$.
$[\partial^X z] = [a]$ i.e. $\partial^X z = C_n(A)$.

\begin{ex}
\begin{align*}
\H(X,X) = 0,\\
\H(X,\emptyset) = \H(X),\\
\H(X,\{\mathrm{pt}\}) = \tilde{\H(X)} \text{ (exercise)}.
\end{align*}
\end{ex}

\begin{prop}
If $f$ is homotopic, $f\sim g\colon (X,A) \to (Y,B)$ then $f_* = g_*\colon\H(X,A) \to \H(Y,B)$.
\end{prop}
\begin{proof}
The prism operator gives a chain homotopy.
\end{proof}
\begin{cor}
If $A \subset V$ and $V$ deformation retracts to $A$ then $\H(V,A) = 0$.
\end{cor}
\begin{proof}

\end{proof}

\subsection{Long exact sequences}
\begin{thm}
Suppose $ 0\to A_* \xrightarrow{i_\#} B_* \xrightarrow{j_\#} C_* \to 0$ is exact then there is a $\partial_*\colon H_{* + 1}(C) \to H_*(A)$ making the following triangle exact
\[
\xymatrix
{
 \H(A) \ar[rr]^{i_*}& &\H(B)\ar[dl]_{j_*} \\
 &\H(C) \ar[ul]_{\partial_*}&
}
\]
that is
\[
\xymatrix{ \cdots\ar[r] & H_n(A) \ar[r]&
H_n(B) \ar[r] & H_n(C)\ar `r[d] `[l]
`[llld]_{\partial_n} `[dll] [dll]\\
& H_{n-1}(A) \ar[r] & H_{n-1}(B)
\ar[r] & H_{n-1}(C)\ar[r] & {\dots} }
\]
is a long exact sequence of groups (an exact complex).
\end{thm}

\begin{proof}
We define $\delta_*$.
Fix some $[c] \in H_n(C)$, so $c\in Z_n(C) \le C_n$.
The map $j$ is surjective so pick $b \in B_n$ such that $j(b) = c$.
Since $c\in Z_n$ we have $\partial c = 0$ and so $j\partial b = \partial j b  = 0$.
Now since $\ker(j) = \im(i)$ there is some $a \in A_{n-1}$ such that $ia = \partial b$.
We then define $\delta_*[c] = [a]$, we have a few things to check:
\begin{enumerate}
\item $a\in Z_{n-1(A)}$: $i\partial a = 0 \iff \partial a = 0$, $i \partial a = \partial ia = \partial \partial b = 0$ as required.
\item $\delta_*$ is well defined:
\begin{enumerate}[(i)]
\item Suppose we pick $c + \partial c'$ instead of $c$.
Pick any $b'$ such that $jb' = c'$ so that $j(b + \partial b') = c + \partial c'$, however then $\partial(b + \partial b') = \partial b$ is the same.
\item Suppose we picked $b''$ such that $jb'' = c$.
Then $j(b- b'') = 0$ and so there is some $a'\in A_n$ such that $ia' = b - b''$, i.e. $b'' = b - ia'$.
We then have $\partial b'' =\partial b  - \partial ia' = ia - i\partial a' = i(a - \partial a')$ so $b''$ gives $a-\partial a'$ and $b$ gives $a$.
Since $[a ] = [a-\partial a']$ we have that $\delta_*$ is well defined.
\end{enumerate}
\item $\delta_*$ is a homomorphism as $i,j$ and $\partial$ are.
\item To see that the chain complex is as claimed we have some more checks to make:
\begin{enumerate}[(i)]
\item $j_*i_*[a] = 0$: $j_*i_*[a]  = j_*[i_\# a] = [j_\# i_\# a ] = [0]$.
\item $\delta_*j_*[b] = 0$: Set $jb = c$, suppose $\delta_*[c] = [a]$, we know $\partial b= 0$ so $ia = \partial b = 0$ but also that $i$ is injective.
So $a = 0$ and hence $[a] = 0$ as required.
\item $j_*\delta_*[c] = 0$: Set $\delta_*[c] = [a]$ and let $ia = \partial b$ and $jb = c$ as usual.
So $i_*[a] = [ia] = [\partial b] = 0$.
\end{enumerate}
\item To see that the complex is exact we must show the opposite inclusions of images and kernels to the ones demonstrated above, we only show (i) here and (ii) \& (iii) are left as exercises.

$\ker(j_*) \le \im (i_*)$: Pick $[b] \in \ker (j_*)$, suppose $j_*[b] = 0$ i.e. $[jb] = 0$ and if $jb = c$ then there is a $c'$ such that $c = \partial c'$.
We know $j$ is surjective so there is $b'$ such that $jb' = c'$ and therefore $j(b - \partial b') = c - \partial c' = 0$.
So there exists $a$ with $ia = b - \partial b'$ so $i_*[a] = [b-\partial b']  = [b]$.
\end{enumerate}
\end{proof}

\begin{ex}
If $(X,A)$ is a pair then 
\[
\xymatrix
{
 \H(A) \ar[rr]^{i_*}& &\H(X)\ar[dl]_{q_*} \\
 &\H(X,A) \ar[ul]_{\partial_*}&
}
\]
is exact.
\end{ex}

\begin{rmk}
If $A = \emptyset$ then $j_*$ is an isomorphism.
\end{rmk}

\begin{thm}
\[
\xymatrix
{
 \tilde{\H}(A) \ar[rr]^{i_*}& &\tilde{\H}(X)\ar[dl]_{q_*} \\
 &\H(X,A) \ar[ul]_{\partial_*}&
}
\]
is also exact, thus $\tilde{\H}(X) \cong \H(X,X)$.
\end{thm}

\begin{ex}
If $(X,B,A)$ is a triple then
\[
\xymatrix
{
 \H(B,A) \ar[rr]^{i_*}& &\H(X,A)\ar[dl]_{q_*} \\
 &\H(X,B) \ar[ul]_{\partial_*}&
}
\]
is exact.
\end{ex}

\begin{ex}
Set $(X,A) = (B^2, S^1)$ then applying the snake lemma to
\[
0 \to \C(A) \to \C(X)\to \C(X,A) \to 0
\]
gives that
\[
H_k(B^2, S^1) =\begin{cases}\ZZ&\text{ if }k=2,\\
0 &\text{ otherwise}.
\end{cases} 
\]
\end{ex}

\begin{exer}
For $\H^\text{sing}$ suppose that $A\xrightarrow{i} X$ and $X\xrightarrow{r} S$ is a \emph{retraction}, i.e. $r\circ i = \mathrm{id}_A$.
Prove that
\[
\H(X) \cong\H(A) \oplus \H(X,A).
\]
\end{exer}

\subsection{Excision}

\begin{thm}[Excision]~\\
\textbf{Version 1}.
Suppose $Z\subset A \subset X$ with $\operatorname{closure}(A) \subseteq \operatorname{interior}(A)$, then
\[
\H^\text{sing}(X\setminus Z, A\setminus Z) \cong \H^\text{sing}(X, A).
\]
\textbf{Version 2}.
Suppose $A,B\subseteq X$ and $X \subseteq \operatorname{interior}(A) \cup \operatorname{interior}(B)$ then 
\[
\H^\text{sing}(B, B\cap A) \cong \H^\text{sing}(X, A).
\]
\end{thm}

We would like to know that
\[
\H(X,A) = \tilde{\H(X/A)}
\]
and we can prove this for good pairs $(X,A)$ using excision.

Also if $X$ has a $\Delta$-complex structure excision gives that
\[
\H^\Delta \cong \H^s(X).
\]
These isomorphisms are induced by inclusion.
We will use both versions of excision but prove version 2.
\begin{exer}
Show both versions of excision are equivalent.
\end{exer}

\subsubsection{Covers}
\begin{defn}
$\U = \{U_\alpha\}$ is an open cover of $X$ if all $U_\alpha$ are open subsets of $X$ and $X$ is contained in the union of all $U_\alpha$.
\end{defn}

Suppose $A\subset X$ we say $\sigma\colon \Delta \to X$ is subordinate to $A$ if $\sigma(\Delta) \subset A$.
We also say that $\U = \{A_\alpha\}$ is a \emph{cover} of $X$ if $\{\operatorname{interior}A_\alpha\}$ is an open cover of $X$.

\begin{defn}
With the notation as above let
\[
C_n^\U(X) = \left\{ \sum n_\beta \sigma_\beta \in C_n^s(X) \middle| \forall \beta\ \exists \alpha \text{ s.t. } \sigma_\beta \text{ is subordinate to } A_\alpha\right\}.
\]
\end{defn}

Note that $\partial\colon C_n(X) \to C_{n-1}(X)$ respects subordination i.e. restricts to give
\[
\partial\colon C_n^\U (X) \to C_{n-1}^\U(X).
\]
So $\C^\U(X) =\{C_n^\U(X),\partial\}$ is a chain complex.
The map
\[
i_\# \colon \C^\U \to \C^s
\]
is an injective chain map.
And we define $\H^\U$ in the usual way.

\begin{prop}
Suppose $\U = \{A,B\}$ is a cover of $X$, then there is a chain map
\[
p_\# \colon \C(X) \to \C^\U(X)
\]
called subdivision such that $p_\# \circ i_\# = \text{id}_{\C^\U}$, $i_\#\circ p_\# \sim\text{id}_{\C^\U}$ is a chain homotopy equivalence to $\C$ and $\H^\U \cong \H$.
\end{prop}

Assuming the above proposition we now prove excision version 2.
\begin{proof}
$\U = \{A,B\}$ is a cover of $X$ so
\[
i_\# \colon \C^\U \to \C
\]
induces isomorphism on homology by the above proposition.
Also $i$ induces 
\[
\C^\U(X) /\C^s(A) \xrightarrow{i} \C^s(X)/\C^s(A)
\]
so we also get an isomorphism
\[
\C^s(B)/\C^s(B\cap A) \xrightarrow{\cong} \C^\U (X) /\C(A).
\]
Thus 
\[
C_n(B) /C_n(A\cap B) \cong C_n^\U(X)/C_n(A) \xrightarrow{\sim\text{ h.e.}} C_n(X)/C_n(A).
\]
So all three have isomorphisms.
\end{proof}

The subdivision operator, used to prove the above proposition is similar to the prism operator (it breaks a large object into pieces).

\begin{exer}
Suppose $X$ is a $\Delta$-complex and $A\in X$ is a subcomplex ($A\ne \emptyset$) show $(X,A)$ is a good pair.
\end{exer}

\begin{exer}
Show that if
\[
A = \{0\} \cup \left\{\frac{1}{n}\middle| n \in \ZZ_{\ge 0} \right\},\ X=[0,1]
\]
then $(X,A)$ is not a good pair.
\end{exer}

\subsubsection{Coning}
Suppose $Y\subset \RR^m$ is convex, that is if $x,y\in Y$ then the interval $[x,y]\subset Y$.

\begin{defn}
With $Y$ as above we let
\[
C_n^l(Y) =\left\{ \sum n_\alpha \sigma_\alpha \middle| \sigma_\alpha \colon \Delta^n \to Y \text{ is affine linear}\right\}.
\]
\end{defn}

Suppose that $\sigma = [v_0,\ldots,v_n]$ is a simplex in $\RR^m$, i.e. that the $v_i$ do not lie in any affine $(n-1)$-dimensional subspace.
Suppose $b\in \RR^m$ and that $\{b,v_0,\ldots,v_n\}$ is again a set of affinely independent points.

\begin{defn}
\[
b\sigma = [b,v_0,\ldots,v_n]
\]
i.e. $b\sigma$ is the \emph{cone} of $\sigma$ (now called the \emph{base}) to $b$ (the \emph{apex}).

Fixing a $b$ we can then define
\[
b\colon C_n^l (Y) \to C_{n+1}^l(Y)
\text{ by }
\sigma \mapsto b\sigma.
\]
\end{defn}

\begin{exer}
Show that
\[
\partial b + b\partial = 1.
\]
\end{exer}

\subsubsection{Barycenters}
\begin{defn}
Given $\sigma \subset Y$ a simplex ($\sigma = [v_0,\ldots,v_n]$) define the \emph{barycenter}
\[
b_\sigma = \sum_{i=0}^{n} \frac{1}{n+1}v_i.
\]
Now define the \emph{subdivision operator} by $Sv = v$ in dimension 0 and otherwise
\[
S\sigma = b_\sigma S\partial \sigma.
\]
This then gives a map $S\colon C_n^l(Y) \to C_n^l(Y)$ defined by $\sigma \mapsto \sigma_\#\circ S\circ \text{id}_{\Delta^n}$.
\end{defn}
As this definition is recursive all proofs using it will be inductive.

\begin{exer}
If $\sigma$ is an $n$-simplex count the number of $n$-simplices in $S\sigma$.
Challenge: count all of the faces.
\end{exer}

\begin{prop}
The operator $S$ is a chain map.
\end{prop}
\begin{proof}
Inductively
\begin{align*}
\partial S \sigma &= \partial(bS\partial \sigma)\\
&= (\partial b) S\partial \sigma\\
&= (1 - b \partial) S\partial \sigma\\
&= (S\partial - b \partial S \partial) \sigma\\
&= (S\partial - b S \partial^2) \sigma\\
&= S\partial \sigma.
\end{align*}
\end{proof}

\begin{lem}[Fine lemma]
If $\tau$ is a simplex in $S \sigma$ then $\diam(\tau)  \le \frac{n}{n+1}\diam (\sigma)$.
So if $\tau \in S^k \sigma$ then $\diam(\tau) \le \left(\frac{n}{n+1}\right)^k \diam(\sigma)$.
\end{lem}
Note that $\left(\frac{n}{n+1}\right)^k \to 0$ as $k \to \infty$.\\

We want to prove that $S$ is chain homotopic to the identity.
To do this we again subdivide $\Delta^n \times I$ (as for $P$, the prism operator).

\begin{defn}
We recursively define a new prism operator
\[
T\sigma = b_\sigma(\sigma - T\partial \sigma).
\]
Which gives the map
\[
T\colon C_n(X) \to C_{n+1}(X) \text{ defined by } \sigma \mapsto \sigma_\# \circ T \circ \text{id}_{\Delta^n}.
\]

We also define another map $D$ by $D_0 = 0$ and $D_{m+1} = D_m + TS^m$ so
\[
D_m = T\left( \sum_{i=0}^{m-1} S^i\right)
\]
\end{defn}

\begin{prop}
\[
\partial D_m  + D_m \partial = 1 - S^m.
\]
\end{prop}
\begin{proof}
\begin{align*}
\partial D_{m+1} &= \partial (D_m +  TS^m)\\
&= \partial D_m + \partial TS^m \\
&= 1 - S^m -  D_m\partial  + \partial TS^m \\
&= 1 - S^m -  D_m\partial  + (1 - S - T\partial)S^m \\
&= 1 - S^{m+1} - (D_m +  TS^m)\partial \\
&= 1 - S^{m+1} - D_{m+1}\partial.
\end{align*}
\end{proof}

This gives that $1 \sim S^m$ for all $m$.

\begin{lem}[Fine lemma 2]
Fix $\U$ a cover of $X$ and $\sigma \colon \Delta^n \to X$, then there exists $m$ s.t. $S^m \sigma$ is subordinate to $\U$.
\end{lem}
\begin{proof}
Define $\U^\sigma = \{\sigma^{-1}(U) \mid U \in \U\}$ this is a cover of $\Delta^n$.
$\Delta^n$ is compact so by the Lebesgue covering lemma there is some $\epsilon$ for all $x \in \Delta^n$ so that there exists $U \in \U^\sigma$ such that $B_\epsilon(x) \subset \operatorname{interior}(U)$.
Pick $m$ such that $\sqrt{2} \left(\frac{n}{n+1} \right)^m < \frac{\epsilon}{100}$.
Now for all $\tau \in S^m \circ \text{id}_A$ we have some $U \in \U^\sigma$ such that $\tau \subset U$ so $\sigma \circ \tau$ is in some element of $U$ i.e. $S^m\sigma$ is subordinate to $\U$.
Hence $S^m \sigma \in C_n^\U (X)$.
\end{proof}

We now finish proving the main proposition above.

\begin{defn}
We define a function 
\[
m\colon \{ \sigma\colon \Delta^n \to x\} \to \NN \text{ by } \sigma \mapsto m(\sigma).
\]
where $m(\sigma)$ is the least natural number such that $S^m_\sigma \in C_n^\U(X)$.
The map $m$ is well defined by Fine lemma 2.
\end{defn}

\begin{defn}
We now let $D_\sigma = D_{m(\sigma)} \sigma$.
\end{defn}
We want $D$ to be the desired chain homotopy for proving the proposition.
So we want $p\colon \C(X) \to \C^\U(X)$ so that
\[
p\circ i \sim 1^\U \text{ and } i \circ p \sim 1.
\]

\begin{defn}
Define
\[
p(\sigma) = S_{\sigma}^{m(\sigma)} + D_{m(\sigma)} \partial \sigma - D \partial \sigma
\]
and extend linearly.
\end{defn}

\begin{prop}
\[
p\colon \C(X) \to \C^\U(X).
\]
\end{prop}
\begin{proof}
Fix $\sigma \in \C(X)$, $S_\sigma^{m(\sigma)} \in C_n^\U(X)$.
The difference $(D_{m(\sigma)} \partial - D \partial)\sigma$ lies in $\C^\U(X)$ as well, the second term subtracts off insufficiently subdivided simplices.
\end{proof}

\end{document}
