\title{MA4H9 Modular Forms - Lecture Notes}
\author{Based on lectures by Dr Peter Bruin \\ Typeset by Alex J. Best}
\date{\today}

\documentclass{article}
\usepackage{amsmath}
\usepackage{amsfonts}
\usepackage{amssymb}
\usepackage{amsthm}
\usepackage{upgreek}
\usepackage{mathrsfs}
\usepackage{tikz}
\usepackage{hyperref}
\usetikzlibrary{arrows}

\newtheorem*{thm}{Theorem}
\newtheorem*{lem}{Lemma}
\newtheorem*{cor}{Corollary}
\newtheorem*{prop}{Proposition}
\theoremstyle{definition}
\newtheorem*{defn}{Definition}
\newtheorem*{defns}{Definitions}
\newtheorem*{ex}{Example}
\newtheorem*{exer}{Exercise}
\newtheorem*{rem}{Remark}
\newtheorem*{nota}{Notation}
\newtheorem*{alg}{Algorithm}
\newtheorem*{fact}{Fact}

\DeclareMathOperator{\SL}{SL}
\DeclareMathOperator{\PSL}{PSL}
\DeclareMathOperator{\res}{Res}
\DeclareMathOperator{\ord}{ord}

\providecommand{\newoperator}[3]{\newcommand*{#1}{\mathop{#2}#3}}
\providecommand{\renewoperator}[3]{\renewcommand*{#1}{\mathop{#2}#3}}

\renewoperator{\Re}{\mathrm{Re}}{\nolimits}
\renewoperator{\Im}{\mathrm{Im}}{\nolimits} % all about dem standards

\setcounter{tocdepth}{3}

\begin{document}
\maketitle
\tableofcontents

\section{Introduction}
These are lecture notes for the 2013 Modular forms course, taught at the University of Warwick by Peter Bruin.


The recommended books are:
\begin{itemize}
\item F. Diamond \& J. Sherman - A first course in Modular Forms
\item J. S. Milne - Modular functions and Modular forms
\item T. Miyake - Modular Forms
\item J-P. Serre - Cours d'Arithmetique
\end{itemize}
\clearpage

\section{Motivation}
\marginpar{Lecture 1}
The big picture: %TODO put in

\subsection{Historical prologue: Sums of squares}
Famous question: given integers $n \ge 0,\ k\ge 0$, in how many ways can $n$ be written as a sum of $k$ squares (of integers).

In other words, what is
\[
r_k(n) = \#\{(x_1,\ldots,x_n)\in \mathbb{Z}^k \mid x_1^2 + \cdots + x_k^2 = n\}?
\]
Note that sign changes and permutations of $(x_1,\ldots, x_k)$ count separately.

\begin{ex}
$k=5,\ n=2$

$5 = 1^2 + 2^2 = (-1)^2 + 2^2 = 1^2 + (-2)^2 = (-1)^2 + (-2)^2 = 2^2 + 1^2 = \cdots$

So $r_2(5) = 8$.

We can reinterpret this geometrically (using Pythagoras) as the statement that there are 8 points in $\mathbb{Z}^2$ of distance $\sqrt{5}$ from the origin.
\end{ex}

Trivial cases: $k=0$, $k=1$.
\[
r_0(n) = \begin{cases}
1,\ n=0 \\
0,\ \text{otw}
\end{cases},\ 
r_1(n) = \begin{cases}
1,\ n=0 \\
2,\ n\text{ is a square} \\
0,\ \text{otw}
\end{cases}
\]

The case: $k=2$.

Diophantus in the $3^\text{rd}$ century showed that if $m,\ n$ are numbers each of which is the sum of two squares then $mn$ is also a sum of two squares.
In our language this says that if $r_2(m) > 0$ and $r_2(n) > 0$ then $r_2(mn)>0$.

\begin{exer}
Prove this!
\end{exer}

Fermat (1637): If $n$ is a positive odd integer then $n$ is a sum of two squares $\iff$ every prime number $p\mid n$ with $p\equiv 3 \pmod{4}$ occurs an even number of times in the factorisation of $n$.

In particular a prime $p$ is a sum of two squares $\iff p = 2$ or $p\equiv 1\pmod{4}$.

\subsubsection{Formulas for $r_k(n)$}
Jacobi (1829): $r_2(n) = 4\sum_{d\mid n} \chi(d)$, for $n > 0$.
Where
\[\chi(d) = \begin{cases}
0, \text{ if } 2\mid d \\
1, \text{ if } d\equiv 1 \pmod{4} \\
-1, \text{ if } d\equiv 3 \pmod{4}
\end{cases}
\]

Gauss (1801) found a formula for $r_3(n)$, but it is much more complicated.

Jacobi (1829): $r_4(n) = 8\sum_{d\mid n,\ 4\nmid d}d$ for all $n > 0$.
In particular the positivity of this sum implies Lagrange's Theorem.

Jacobi, F. G. Eisenstein, H. J. Smith found that
\[
r_6(n) = \sum_{d\mid n}(16 \chi\left(\frac n d\right) - 4\chi(d))d^2
\text{ and }r_8(n) = \sum_{d\mid n}(-1)^{n-d}d^3.
\]

J. Liouville (1864/65) found 
\[
r_{10}(n) = \frac 4 5\sum_{d\mid n}(16\chi\left(\frac n d\right)  + \chi(d))d^2 + \frac 8 5 \sum_{z\in\mathbb{Z}[i]}.
\]
For even $n$ the formula 
\[
r_{12}(n) = 8\sum_{d\mid n} d^5 - 512\sum_{d\mid\frac n 4}d^5
\]
holds, the second sum is omitted if $4\nmid n$.

\begin{rem}
We note the following about the above formulae:
\begin{itemize}
\item Many are sums over the positive divisors of $n$.
\item The function $\chi(d)$ appears in the formulas for $k \equiv 2 \pmod 4$.
\item When $k=10$ there is an unexpected term involving $\mathbb{Z}[i]$.
\item Formulas exist only for $k\le 12$.
\end{itemize}
\end{rem}

Amazingly, all these facts can be explained using modular forms.

\subsubsection{Generating series for $r_k(n)$}
The generating series of $r_k(n)$ is 
\[\sum_{n=0}^{\infty}r_k(n) q^n \in \mathbb{Z}[[q]]\]

When $k=1$ this is $1 + 2q + 2q^4 + 2q^9 + \ldots = 1 + 2\sum_{n=1}^{\infty}q^{n^2} = \sum_{n\in \mathbb{Z}} q^{n^2}$.
This series is known as Jacobi's $\upvartheta$ series, and is an example of a modular form.

\begin{exer}
Try to show that for all $k \ge 0$ we have $\sum_{n=0}^{\infty}r_k(n)q^n = \upvartheta(q)^k$.

One can show that these power series for $k$ even are examples of ``$q$-expansions of modular forms''.
\end{exer}

\subsection{$\mathbb{H}$ and the group $SL_2(\mathbb{R})$} %TODO make this nice
\marginpar{Lecture 2}

We now introduce some of the basic objects of use when studying modular forms.

The complex upper half plane $\mathbb{H}$ is $\{z\in \mathbb{C} \colon \Im z > 0\}$.
\[\SL_2(\mathbb{R}) =\left\{\begin{pmatrix} a & b\\c&d\end{pmatrix} \colon a,b,c,d\in\mathbb{R},\ ad-bc = 1\right\}.\]

\begin{nota}
For $\gamma\in\SL_2(\mathbb{R})$ and $z\in \mathbb{H}$, write
\[ \gamma z = \frac{az+b}{cz+d} \text{ noting that } cz+d \ne 0.\]
\end{nota}

\begin{prop}
This formula defines a map
\[
\SL_2(\mathbb{R})\times \mathbb{H} \to \mathbb{H},\ (\gamma,z)\mapsto\gamma z
\]
which is a group action.
\end{prop}

\begin{proof}
First note that for all $\gamma \in \SL_2(\mathbb{R})$ and all $z\in \mathbb{H}$
\[\Im(\gamma z) = \Im\left(\frac{az+b}{cz+d}\right) = \Im\left(\frac{(az+b)(c{\bar z} + d)}{|cz+d|^2}\right) = \frac{\Im((az+b)(c{\bar z} +d))}{|az+d|^2}.\]
Now $\Im((az+b)(c{\bar z} + d)) = \Im(ac|z|^2 + bd + adz + bc{\bar z})$, so letting $z= x+iy$ we see this equals
\[\Im(adz + bc{\bar z}) + \Im((ad-bc)iy) = \Im(iy) = y = \Im(z).\]
Therefore
\[\Im(\gamma z) = \frac{\Im(z)}{|cz+d|^2}\]

This shows the action is well defined.
It is a group action as $\begin{pmatrix}1&0\\0&1\end{pmatrix}z = \frac{z+0}{0+1} = z$, and $\gamma(\gamma'z) = (\gamma\gamma')z$.
\end{proof}

\subsection{The modular group}

\begin{defn}
The modular group is the group 
\[\SL_2(\mathbb{Z}) =\left\{\begin{pmatrix} a & b\\c&d\end{pmatrix} \colon a,b,c,d\in\mathbb{Z},\ ad-bc = 1\right\}\subset \SL_2(\mathbb{R}).\]
As a subgroup of $\SL_2(\mathbb{R})$, $\SL_2(\mathbb{Z})$ acts on $\mathbb{H}$.
\end{defn}

\begin{rem}
Apart from $\begin{pmatrix}
1&0\\0&1
\end{pmatrix}$ the matrix $\begin{pmatrix}
-1&0\\0&-1
\end{pmatrix}$ acts trivially on $\mathbb{H}$.


Since $N=\left\{\pm\begin{pmatrix}
1&0\\0&1
\end{pmatrix} \right\}$ is a normal subgroup of $\SL_2(\mathbb{R})$ (and $\SL_2(\mathbb{Z})$), the quotient $\PSL_2(\mathbb{R})= \SL_2(\mathbb{R})/N$ (resp. $\PSL_2(\mathbb{Z})=\ldots$) is again a group with an action on $\mathbb{H}$.

Sometimes it is convenient to work with $\PSL$ rather than $\SL$.
\end{rem}

We now give names to some useful elements of the modular group, let
\[S=\begin{pmatrix}
0&-1\\1&0
\end{pmatrix},\ T= \begin{pmatrix}
1&1\\0&1
\end{pmatrix},\ \text{ then } ST = \begin{pmatrix}
0&-1\\1&1
\end{pmatrix},\ \text{etc.}\]

We can see for all $z\in\mathbb{H}$ we have $Sz = \frac{-1}{z}$ and $Tz = z+1$.

\subsection{A fundamental domain}
Let $D=\{z\in \mathbb{H}\colon |z|\ge 1\text{ and }\frac{-1}{2}\le \Re(z)\le \frac{1}{2}\}$ and let $\rho= e^{2\pi i/3} = -\frac{1}{2} + \frac{\sqrt{3}}{2}i$, then $\rho+1= e^{2\pi i/3} +1= \frac{1}{2} + \frac{\sqrt{3}}{2}i$.


\begin{thm}[Properties of $\SL_2(\mathbb{Z})$ and $\mathbb{H}$]~
\begin{enumerate}
\item For all $z\in \mathbb{H}$ there exists some $\gamma \in \SL_2(\mathbb{Z})$ s.t. $\gamma z\in D$.
\item $z,z'\in D$ are in the same $\SL_2(\mathbb{Z})$ orbit $\iff$ either $z=z'$ or $\Re(z) \pm\frac{1}{2}$ and $z= z' \pm 1$ or $|z|=1$ and $z' = -\frac 1 z$.
\item Let $z\in D$ and let $H_z = \{\gamma \in \SL_2(\mathbb{Z}) \colon \gamma z=z\}$ be that stabilisser of $z$ under the action of $\SL_2(\mathbb{Z})$, then 
\[
H_z \text{ is } \begin{cases} %TODO finish
\text{ cyclic of order 6, generated by } ST \text{ when } z = \rho. \\
\text{ cyclic of order 6, generated by } ? \text{ when } z = \rho + 1. \\
\text{ cyclic of order 4, generated by } ? \text{ when } z = i. \\
\text{ cyclic of order 2, generated by } ? \text{ otherwise.}\\
\end{cases}
\]

\item The group $\SL_2(\mathbb{Z})$ is generated by the two elements $S$ and $T$.
\end{enumerate}
\end{thm}

\marginpar{Lecture 3}
\begin{proof}
\begin{enumerate}
\item Let $z \in \mathbb{H}$.
For $c,d\in \mathbb{Z}$ we have $|cz+d|^2 = |(cx+d) + (cy)i|^2 = (cx+d)^2 + (cy)^2$, so there are only finitely many $c,d$ s.t. $|cz+d|< 1$.
I.e. only finitely many s.t. $\Im(\gamma z) > \Im(z)$ this implies that there is some $\gamma \in \langle S,T\rangle$ s.t. $\Im(\gamma z) \ge \Im(\gamma'z)$ for all $\gamma' \in \langle S,T\rangle$.
By multiplying on the left by an appropriate power of $T$ (i.e. by translating $\gamma z$ by some integer) we may assume that $\gamma$ is chosen s.t. $|\Re(\gamma z)| \le \frac 1 2$.

\paragraph{Claim}
With this $\gamma$ we have $|\gamma z| \ge 1$, and hence $\gamma z \in D$, this proves part 1, with moreover the result that $\gamma$ can be chosen in $\langle S,T\rangle$.
\begin{proof}[Proof of the claim]
By the choice of $\gamma$ we have $\Im(\gamma z) \ge \Im(S\gamma z) = \Im(\frac{-1}{\gamma z}) = \frac{\Im(\gamma z)}{|\gamma z|^2}$ this implies that $|\gamma z| \ge 1$ which proves the claim.
\end{proof}

\item We now let $z,z'\in D$ be in the same $\SL_2(\mathbb{Z})$ orbit.
We may assume $\Im(z') \ge \Im(z)$.
Let $\gamma \in \SL_2(\mathbb{Z})$ be such that $z' = \gamma z$ in particular, 
\[ \Im z' = \frac{\Im z}{|cz+d|^2} \le \frac{\Im z'}{|cz+d|^2}\]
so $|cz+d|^2 \le 1$.
Since $|cz+d|^2 = |cx+d|^2 + |cy|^2$ and $y \ge \frac{\sqrt{3}}{2} > \frac{1}{2}$ this gives $|c| \le 1$, so we deal with the three possible cases for $c\in\{-1,0,1\}$ separately.

\paragraph{Case $c=0$}
\[\gamma =\begin{pmatrix}a&b\\0&d
\end{pmatrix} \implies ad= 1\implies a=d=\pm1\implies z' = \gamma z = \frac{\pm z + b}{\pm 1} = z \pm b.\]
This is only possible if $b\in\{0,\pm 1\}$, and in non-zero cases we must have $|\Re z | = |\Re z'| = \frac 1 2$.

\paragraph{Case $c=1$}
$\gamma =\begin{pmatrix}a&b\\1&d\end{pmatrix},$
\[ 1\ge |cz+d|^2= (x+d)^2 +y^2 = x^2 +y^2 +2xd+d^2 = |z|^2 + 2xd + d^2 \ge 1 + 2xd + d^2\]
So we have $|cz+d| = 1$ and $2xd + d^2 = 0$, hence either $d=0$ or $d = -2x$ which gives $d= \pm 1$, $x = \pm\frac 1 2$.
If $d=-1$, $x = -\frac12$ then $z = \rho$, $z'= \frac{a\rho + b}{\rho +1}$.
\begin{exer}
Show this only lies in $D$ if $(a,b) = (0,-1)$ or $(1,0)$.
\end{exer}
The cases $d= 0,\ d=-1$ are similar, every time we get a small finite number of $\gamma$ and the corresponding possibilities for $z,z'\in D$ s.t. $z' = \gamma z$.

\paragraph{Case $c=-1$} is  completely analogous since $\gamma$ and $-\gamma$ act in the same way on $\mathbb{H}$.

So we are left with a few different possibilities which are summarised in the table.

\begin{table}[h]
\begin{tabular}{|c|c|c|c|}
\hline
$\gamma$ & $z$ & $z'$ & Fixed points of $\gamma$ \\
\hline
$\pm$Id & any $z\in D$ & $z$ & all of $D$ \\
$\pm T$ &i & i& i\\
\hline
\end{tabular}
\caption{\label{}Pairs $(\gamma,z)$ with $z$ and $z' = \gamma z$ both in $D$}
\end{table}

Parts 2. and 3. of the theorem can be read off of this table, it remains to show 4.
\setcounter{enumi}{3}

\item Choose any $z$ in the interior of $D$ (e.g. $z= 2i$) and $\gamma\in \SL_2(\mathbb{z})$.
There exists $\gamma_0\in\langle S,T\rangle$ s.t. $\gamma_0(\gamma z)\in D$ this means that both $z$ and $(\gamma_0 \gamma)z$ are in $D$, and $z$ is not on the boundary, so $\gamma_0\gamma =\pm I$ and $\gamma\in\langle S,T\rangle$.
\end{enumerate}
\end{proof}

\section{Modular forms}
\begin{defn}
Let $f$ be a meromorphic function on $\mathbb{H}$ and let $k$ be an integer.

We say that $f$ is \emph{weakly modular of weight $k$} if
\[f\left(\frac{az+b}{cz+d}\right)= (cz+d)^kf(z)\] for all $z\in\mathbb{H},\ \begin{pmatrix}a&b\\c&d\end{pmatrix} \in\SL_z(\mathbb{Z})$.
\end{defn}

\begin{nota}
\[(f|_k\gamma)(z)=(cz+d)^{-k}f\left(\frac{az+b}{cz+d}\right)\text{, for }\gamma\in\SL_2(\mathbb{R}).\]
\end{nota}

\begin{exer}
Show that if $\mathcal{F}$ is the set of all meromorphic functions on $\mathbb{H}$, then the map $\mathcal{F}\times\SL_2(\mathbb{R})\to \mathcal{F}$ sending $(f,\gamma)$ to $f|_k\gamma$ is a right action of $\SL_2(\mathbb{R})$ on $\mathcal{F}$.
\end{exer}

\begin{rem}
Note that $f$ is weakly modular of weight $k$ $\iff$ $f$ is $\SL_2(\mathbb{Z})$ invariant with respect to the $|_k$ action.

Since $\SL_2(\mathbb{Z})$ is generated by $S$ and $T$ this is equivalent to saying that $f|_k S = f$ and $f|_k T = f$ i.e. $f(z+1) = f(z)$ and $f(\frac{-1}{z}) = z^kf(z)$.
\end{rem}

\marginpar{Lecture 4}
In particular $f$ is weakly modular of weight $k$ $\implies$ $f$ is periodic of period 1.
Consider the exponential map $z\mapsto e^{2\pi i z}$, this is also periodic of period 1.

Consider the diagram
%TODO put in

There is a \emph{unique} function $\tilde f \colon \mathbb{D}^* \to \mathbb{C}\cup\{\infty\}$ s.t. $f(z) = \tilde f(e^{2\pi i z})$.
$\tilde f(q) = f(\frac{\log q}{2\pi i})$ is well defined as $f$ is periodic.

\begin{defn}
Let $f$ be meromorphic on $\mathbb{H}$ and weakly modular of weight $k$.
We say that $f$ is \emph{meromorphic at infinity} if $\tilde f(q)$ can be continued to a meromorphic function on the whole open unit disc $\mathbb{D} = \{q\in \mathbb{C}\colon |q|< 1\}$.
Equivalently, as a Laurent series
\[\tilde f(q) = \sum_{n=-\infty}^{\infty} a_nq ^n,\ a_n\in\mathbb{C}\text{ with }a_n = 0\text{ for all $n$ sufficiently negative},\]
converges in some open neighbourhood of $0\in\mathbb{D}$ (but potentially having a pole at 0.

We say that $f$ is \emph{holomorphic at infinity} if this Laurent series is a power series, i.e.
\[\tilde{f}(z) = \sum_{n=0}^{\infty} a_nq^n.\]
In this case we define $f(\infty) = \tilde f(0) = a_0$.
\end{defn}

\begin{defn}
A \emph{modular form of weight $k$} is a holomorphic function $f\colon\mathbb{H}\to\mathbb{C}$ which is weakly modular of weight $k$ and holomorphic at $\infty$.

A \emph{cusp form} of weight $k$ is a modular form of weight $k$ which vanishes at infinity.
\end{defn}

The reason for this terminology is if we construct the quotient space $\SL_2(\mathbb{Z})\backslash \mathbb{H}$ we can do this by gluing the edges of $D$.
Upon doing this we observe a surface that looks like a raindrop, the squeezed point at infinity is called the cusp.
%TODO diagram

By the above definition a modular form of weight $k$ is a holomorphic function $f\colon\mathbb{H}\to\mathbb{C}$ that can be expressed as a convergent series
\[f(z)=\sum_{n=0}^{\infty}a_nq^n = \sum_{n=0}^{\infty} a_ne^{2\pi i z},\]
with $f\left(\frac{-1}{z}\right) = z^kf(z)$.

It is a cusp form if $a_0 = 0$.

The series $\sum_{n=0}^{\infty}a_nq^n$ is called the $q$-expansion (or Fourier expansion) of $f$.

\section{Eisenstein series}
Let $k$ be an even integer with $k \ge 4$.
Consider the infinite sum 
\[G_k(z) = \sum_{(m,n)\in\mathbb{Z}^2\setminus(0,0)} \frac{1}{(mz+n)^k} = \sideset{}{'}\sum_{m,n}\frac{1}{(mz+n)^k},\ z\in\mathbb{H}.\]

\begin{prop}
This series converges absolutely and uniformly on subsets of $\mathbb{H}$ of the form
\[R_{r,s} = \left\{ x+iy\colon |x|\leq r,\ y\geq s\right\},\ r,s>0\]
\end{prop}

\begin{proof}(Sketch) Given $r,s>0$ one first shows that there exists $c >0$ (depending on $r,s$) such that $|mz+n|^2 \ge c(m^2+n^2)$ for all $m,n\in\mathbb{Z}$, $z\in R_{r,s}$.
This means 
\begin{multline*}
|G_k(z)|\le c^{-k/2}\sideset{}{'}\sum_{m,n}\frac{1}{(mz+n)^{k/2}} = c^{-k/2}\sum_{j=1}^{\infty}\sideset{}{'}\sum_{m,n\text{ in $j$th } \square}\frac{1}{(mz+n)^{k/2}} \\
\le c^{-k/2}\sum_{j=1}^{\infty}8j\frac{1}{j^k} = \frac{8}{c^{k/2}}\sum_{j=1}^{\infty} \frac{1}{j^{k-1}} \le \frac{8}{c^{k/2}}\left(1 + \int_{j=1}^{\infty} \frac{1}{t^{k-1}} dt\right) \\
= \frac{8}{c^{k/2}}\left(1+ \frac{1}{k-2}\right)<\infty
\end{multline*}
\end{proof}

In fact this proposition implies that $G_k(z)$ is a holomorphic function on $\mathbb{H}$.

\begin{thm}
For all even integers $\ge 4$ the function $G_k$ is a modular form of weight $k$.
\end{thm}

\begin{proof}
It is holomorphic on $\mathbb{H}$ by the proposition.
We need to check its invariance under $\SL_2(\mathbb{Z})$, i.e. that
\[G_k(z+1) = G_k(z) \text{ and } G_k\left(\frac{-1}{z}\right) = z^kG_k(z).\]
\begin{proof} of $G_k(\frac{-1}{z}) = z^kG_k(z)$
\[G_k\left(\frac{-1}{z}\right) = \sideset{}{'}\sum_{m,n} \frac{1}{(m\cdot\frac{-1}{z} + n)^k} = z^k\sideset{}{'}\sum_{m,n}\frac{1}{(-m + nz)^k} = G_k(z).\]
The last step is by a couple of variable changes, we can see that the sum will always run over the same values.
\end{proof}

It just remains to check that $G_k$ is holomorphic at infinity, we will do this by calculating the $q$-expansion at infinity, however this is long and requires its own section.
\end{proof}

\section{$q$-expansions of Eisenstein series}
\begin{nota}
\[\sigma_t(n) = \sum_{d|n,\ d > 0} d^t \text{ for } n\ge 1,\ t\ge 0.\]
The Riemann zeta function is
\[\zeta(s) := \sum_{n=1}^{\infty} \frac{1}{n^s},\ s\in\mathbb{C},\ \Re s > 1.\]
\end{nota}

We will only need the zeta function for $s\in \mathbb{N}$ even.

These values can be expressed in terms of the Bernoulli numbers $B_k\in\mathbb{Q}$, defined by the identity
\[\frac{t}{e^t - 1} = \sum_{k=0}^{\infty}\frac{B_k}{k!}t^k\text{ in } \mathbb{Q}[[t]].\]

We have $B_k\ne 0\iff k=1$ or $k\ge 0$ even.
The first few non-zero terms are
\[B_0 = 1,\ B_1= -\frac{1}{2},\ B_2=\frac{1}{6},\ B_4=-\frac{1}{30},\ B_6=\frac{1}{42},\ldots.\]

\begin{fact}
For any even positive integer $k$
\[\zeta(k) = -\frac{(2\pi i)^kB_k}{2k!}\text{ for example } \zeta(2) = \sum_{n=1}^\infty \frac{1}{n^2} = \frac{\pi^2}{6}.\]
\end{fact}

We can rewrite the series defining $G_k$ as follows
\begin{multline*}
G_k(z) = \sum_{n\ne 0}\frac{1}{n^k} + \sum_{m\ne 0}\sum_{n\in\mathbb{Z}}\frac{1}{(mz+n)^k} = 2\sum_{n=1}^{\infty}\frac{1}{n^k} + 2 \sum_{m=1}^{\infty}\sum_{n\in\mathbb{Z}}\frac{1}{(mz+n)^k} \\
= 2\zeta(k) + 2\sum_{m=1}^{\infty}\sum_{n\in\mathbb{Z}}\frac{1}{(mz+n)^k}.\end{multline*}

\begin{fact}
For $k\ge 2$ we have
\[\sum_{n\in\mathbb{Z}}\frac{1}{(z+n)^k} = \frac{(-2\pi i)^k}{(k-1)!}\sum_{d=1}^{\infty}d^{k-1}q^d\text{ where } q= e^{2\pi i z}.\]
\end{fact}

\begin{proof} sketch.
Start with the following product formula for the sine function
\[\sin(\pi z) = \pi z\prod_{n=1}^{\infty}\left(1-\frac{z^2}{n^2}\right)\]
Take logarithmic derivatives ($\frac{d}{dz}\log f(z) = \frac{f'(z)}{f(z)}$) gives
\[\pi \frac{\cos(\pi z)}{\sin(\pi z)} = \frac{1}{z} + \sum_{n=1}^{\infty}\left(\frac{1}{z-n}+ \frac{1}{z+n}\right).\]
On the other hand, using$e^{\pm \pi i z} = \cos\pi z\pm i\sin \pi z$ and $1/1-q = \sum_{d=0}^{\infty}$, one can prove that 
\[\pi\frac{\cos(\pi z)}{\sin(\pi z)} = i\pi i - 2\pi i \sum_{d=1}^{\infty}e^{2 \pi i dz}.\]
Comparing gives
\[\frac{1}{z} + \sum_{n=1}^{\infty}\left(\frac{1}{z-n} + \frac{1}{z+n}\right) = -\pi i -2\pi i\sum_{d=1}^{\infty}e^{2\pi i d z},\]
then taking derivatives gives
\[\sum_{n\in\mathbb{Z}} \frac{1}{(z+n)^2} = (2\pi i)^2\sum_{d=1}^{\infty}de^{2\pi i dz}\]
as required for $k=1$, the general case follows by induction upon taking successive derivatives.
\end{proof}

Applying this fact gives a series for $G_k$, we obtain
\begin{multline*}
G_k(Z) = 2\zeta(k) + 2\frac{(2\pi i)^k}{(k-1)!}\sum_{m=1}^{\infty}\sum_{d=1}^{\infty}d^{k-1}e^{2 \pi i d m z} \\
= 2\zeta(k) + 2\frac{(2\pi i)^k}{(k-1)!}\sum_{n=1}^\infty\sum_{d|n}d^{k-1}e^{2\pi i n z} =2\zeta(k) + 2\frac{(2\pi i)^k}{(k-1)!}\sum_{n=1}^\infty \sigma_{k-1}(n)q^n.\end{multline*}
This gives the $q$-expansion of $G_k$ and shows at the same time that $G_k$ is holomorphic at infinity.
Indeed we have now shown that $G_k$ is a modular form of weight $k$.

It is useful to introduce a rescaled version of $G_k$
\[E_k(z) = \frac{(k-1)!}{2(2\pi i)^k}G_k(z) = \frac{-B_k}{2_k} + \sum_{n=1}^{\infty}\sigma_{k-1}(n)q^n.\]

In particular, note that all $q$-expansion coefficients of $E_k$ are rational numbers.

\begin{rem}
Another common normalisation of $G_k$ is s.t. the constant coefficient becomes 1.
\end{rem}

\begin{ex}
$k=4$:
\[E_4(z) = \frac{1}{240} + \sum_{n=1}^{\infty} \sigma_3(n)q^n = \frac{1}{240} + q + 9q^2 + \ldots.\]
\end{ex}

\section{Motivation: Lattices in $\mathbb{C}$}

\begin{defn}
A \emph{lattice} in the complex plane $\mathbb{C}$ is a subgroup $L\subset \mathbb{C}$ of the form
\[L = \mathbb{Z}\omega_1 + \mathbb{X}\omega_2,\]
where $\omega_1,\omega_2\in\mathbb{C}$ are $\mathbb{R}$ linearly independent.
\end{defn}

\begin{ex}
$\omega_1=z\in\mathbb{H}$, $\omega_2=1$ then we let $L_z = \mathbb{Z}z + \mathbb{Z}\cdot 1$ and we see
\[G_k(z) = \sideset{}{'}\sum_{m,n}\frac{1}{(mz+n)^k} = \sum_{\omega\in L_z\setminus 0} \frac{1}{\omega^k}.\]

For an arbitrary lattice $L$ one can similarly define
\[\mathscr{G}_k(L) = \sum_{\omega\in L\setminus0} \frac{1}{\omega^k}.\]
\end{ex}

Lattices can be scaled by complex numbers $\lambda\in\mathbb{C}^\times$, if $L$ is a lattice we put $\lambda L = \{\lambda\omega\mid \omega\in L\}$.
This is again a lattice.
Scaling defines an equivalence relation on the set $\mathcal{L}$ of all lattices in $\mathbb{C}$, two lattices $L,L'$ are \emph{homothetic} if $L'=  \lambda L$ for some $\lambda\in\mathbb{C}^\times$.



If $\begin{pmatrix}
a&b\\c&d
\end{pmatrix}\in\SL_2(\mathbb{Z})$ then another basis  of $L$ is $(a\omega_1 + b\omega_2, c\omega_1 + d\omega_2)$.
We observe that
\[L_z = \mathbb{Z}z + \mathbb{Z}\cdot 1 = \mathbb{Z}(az+b) + \mathbb{Z}(cz+d),\]
this hints at a connection between lattices and modular forms.

Consider functions $\mathcal{F}\colon\{\text{lattices in }\mathbb{C}\}\to\mathbb{C}$ with the property that $\mathcal{F}(\lambda L) = \lambda^{-k}\mathcal{F}(L)$.
Given such an $\mathcal{F}$ we put $f(z) = \mathcal{F}(L_z)$.
Then we have
\begin{align*}
f(z) &= \mathcal{F}(\mathbb{Z}z+\mathbb{Z}\cdot 1) = \mathcal{F}(\mathbb{Z}(az+b) +\mathbb{Z}(cz+d)) \\ 
&= \mathcal{F}\left((cz+d)\left(\mathbb{Z}\left(\frac{az+b}{cz+d}\right) + \mathbb{Z}\cdot 1\right)\right)=(cz+d)^{-k}\mathcal{F}\left(\mathbb{Z}\frac{az+b}{cz+d} + \mathbb{Z}\cdot 1\right)\\
&= \lambda^{-k}\mathcal{F}\left(L_{\frac{az+b}{cz+d}}\right) = \lambda^{-k}f\left(\frac{az+b}{cz+d}\right) = \left(f|_k\begin{pmatrix}a&b\\c&d\end{pmatrix}\right)(z).\end{align*}

So there is an interpretation of modular forms as functions of \emph{lattices} instead of functions on $\mathbb{H}$.
However in the setting of functions on $\mathbb{H}$ with the property of being holomorphic (both on $\mathbb{H}$ and at infinity) many statements are easier to formulate.

\section{More examples: the modular form $\Delta$ and the modular function $j$}
Recall that
\[E_4 = \frac{1}{240} + q + \ldots,\ E_6 = \frac{1}{540}+ q+ \ldots.\]

\begin{defn}
We define
\[\Delta = \frac{(240E_4)^3 - (-504E_6)^2}{1728}.\]
\end{defn}

Working out the $q$-expansion we see that
\[\Delta = q - 24 q^2 + 252q^3 - 1472q^4+ \ldots\]
is a cusp form of weight 12.
One can in fact show that all the $q$-expansion coefficients are integral.

Moreover we define the function $\tau\colon\mathbb{Z}\to\mathbb{Z}$ through the following equality
\[\Delta = q \prod_{n=1}^\infty(1-q^n)^{24} = \sum_{n=1}^\infty \tau(n)q^n.\]
$\tau$ is known as Ramanujan's $\tau$ function.

$\tau$ has many interesting number-theoretic properties e.g.
\[\tau(n) \equiv \sigma_{11}(n)\pmod{691}.\]

\begin{defn}
The $j$-function is defined as
\[j(z) = \frac{(240E_4)^3}{\Delta} = q^{-1} + 744 +196884q + \ldots.\]
This is not a modular form as it is not holomorphic at infinity, it is however an example of a \emph{modular function}.
\end{defn}

\begin{defn}
A \emph{modular function} is a function $f$ satisfying $f(\gamma z) = f(z)$ for all $z\in\mathbb{H}$, $\gamma\in\SL_2(\mathbb{Z})$ that is also meromorphic on $\mathbb{H}$ and at infinity.
\end{defn}

\section{The valence formula}

The valence formula gives us information about the zeroes and poles of modular and weakly modular functions.

Let $f$ be a meromorphic function on an open subset $U\subseteq\mathbb{C}$, and $w\in U$.
Then $f$ can be expanded in a Laurent series around $w$.
\[f(z) = c_n(z-w)^n + c_{n+1}(z-w)^{n+1} + \ldots\ (n\in\mathbb{Z},\ c_j\in\mathbb{C},\ c_n\ne 0).\]

The \emph{order} or \emph{valuation} of $f$ at $w$ (denoted $\ord_w(f)$ or $\nu_w(f)$) is the $n$ in the above expansion.
The \emph{residue} of $f$ at $w$ is $c_{-1}$ ($ = 0$ if $n\ge 0$).
We denote this by $\res_w(f)$

From this expression we can deduce that
\[\frac{f'}{f}(z) = \frac{n}{z-w} + b_0 + b_1(z-w) + \ldots.\]

In particular $\frac{f'}{f}$ has a simple zero precisely at the zeroes and poles of $f$ and
\[\res_w\left(\frac{f'}{f}\right) = n = \ord_w(f).\]

\begin{thm}[Cauchy's integral formula] Let $g$ be holomorphic on an open subset $U\subseteq\mathbb{C}$.
Let $\mathcal{C}$ be a contour (simple closed curve) in $U$, and let $w\in U$.
Then we have
\[\oint_{\mathcal{C}} \frac{g(z)}{z-w}dz = 2\pi i g(w).\]
\end{thm}

\begin{thm}[Argument principle] Let $f$ be meromorphic on an open subset $U\in\mathbb{C}$, let $\mathcal{C}$ be a contour in $U$.
Assume $f$ has no zeroes or poles on $\mathcal{C}$.
Then
\begin{align*}
\oint_{\mathcal{C}} \frac{f'(z)}{f(z)} \mathrm{d}z &= 2\pi i \sum_{z\in\operatorname{interior}(\mathcal{C})} \ord_z(f) \\
(&=2\pi i \cdot(\textrm{number of zeroes } - \textrm{ number of poles, with multiplicity})).\end{align*}
\end{thm}

There are variants of this statement, such as:

Let $\mathcal{C}$ be an arc around $w\in U$ with angle $\alpha$ and radius $r$.
If $g$ is holomorphic at $w$ we have 
\[\lim_{r\to 0} \oint_{\mathcal{C}}\frac{g(z)}{z-w}\mathrm{d}z = \alpha i g(w).\]
If $f$ is meromorphic at $w$ we have 
\[\lim_{r\to 0} \oint_{\mathcal{C}}\frac{f'(z)}{z-w}\mathrm{d}z = \alpha i \ord_i f.\]

Let $f$ be meromorphic on $\mathbb{H}$ and weakly modular of some weight $k$.
Let $z\in \mathbb{H}$, $\gamma\in\SL_2(\mathbb{Z})$.
Using the transformation formula it is not hard to check that $\ord_{\gamma z} f = \ord_z f$.
Finally if in addition $f$ is meromorphic at infinity and $\tilde{f}(q)$ is defined by
\[f(z) = \tilde{f}(e^{2\pi i z} = \tilde{f}(q)\]
we define $\ord_{z=\infty} f = \ord_{q=0} \tilde{f}$.

\begin{thm}[Valence formula]
Let $f$ be meromorphic on $\mathbb{H}$, weakly modular of weight $k$ and meromorphic at infinity.
Then we have 
\[\ord_\infty f + \frac{1}{2}\ord_i f + \frac{1}{3} \ord_\rho f + \sum_{w\in W}\ord_w f = \frac{k}{12}.\]
Where $W$ is the set of $\SL_2(\mathbb{Z})$-orbits in $\mathbb{H}$ with the orbits of $i$ and $\rho$ omitted.
\end{thm}

\begin{proof}
We may take all orbit representatives in the set $D$.
We assume for simplicity that $f$ has no poles or zeroes on the boundary except for possibly at $i$, $\rho$ or $\rho + 1$.
Let $\mathcal{C}$ be the following contour
%TODO picture.

By the argument principle
\[\oint_{\mathcal{C}}\frac{f'(z)}{f(z)}\mathrm{d}z = 2\pi i \sum_{w\in W}\ord_w f.\]

On  the other hand we can compute the same integral by cutting $\mathcal{C}$ into 8 pieces and using the transformation formula.
We consider the pieces as follows
\begin{itemize}
\item We have
\[\int_{D'}^{E}\frac{f'(z)}{f(z)}\mathrm{d}z = \int_{B}^A \frac{f'}{f}(z' + 1)\mathrm{d}z' = \int_B^A \frac{f'}{f}(z)\mathrm{d}z = \int_A^B \frac{f'}{f}(z)\mathrm{d}z.\]
I.e. the integrals $\int_B^A$ and $\int_E^{D'}$ cancel.
\item From the equation $f(-1/z) = z^kf(z)$ we deduce
\[z^{-2}f'\left(\frac{-1}{z}\right) = kz^{k-1}f(z) + z^k f'(z) \implies z^{-2}\frac{f'}{f}\left(\frac{-1}{z}\right) = \frac{k}{2} + \frac{f'}{f}(z).\]
So
\begin{align*}
\int_{C'}^D \frac{f'}{f}(z)\mathrm{d}z &= \int_C^{B'}\frac{f'}{f}\left(\frac{-1}{z}\right)(z')^{-2}\mathrm{d} z' \text{ where } z' = \frac{-1}{z} \\
&=\int_C^{B'} \frac{k}{z'} + \frac{f'}{f}(z)\mathrm{d}z' \\
&=k\int_C^{B'}\frac{1}{z}\mathrm{d}z - \int_{B'}^C \frac{f'}{f}(z)\mathrm{d}z.
\end{align*}

This implies 
\[\int_{B'}^C\frac{f'}{f}(z)\mathrm{d}z + \int_{C'}^D \to k^{\frac{\pi i}{6}} \text{ as } r\to \infty.\]

\item As $r\to 0$, we have (by a result quoted above).
\begin{align*}
\int_B^{B'} \frac{f'}{f}(z)\mathrm{d}z &\to -\frac{\pi i}{3}\ord_\rho f, \\
\int_C^{C'} \frac{f'}{f}(z)\mathrm{d}z &\to -\pi i\ord_i f, \\
\int_D^{D'} \frac{f'}{f}(z)\mathrm{d}z &\to -\frac{\pi i}{3}\ord_{\rho+1} f = -\frac{\pi i}{3}\ord_\rho f.
\end{align*}

\item To calculate the integral from $E$ to $A$ we make the change of variables $q = e^{2 \pi i z}$.
\end{itemize}
\end{proof}
\end{document}
